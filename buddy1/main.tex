\PassOptionsToPackage{dvipsnames}{xcolor}
\PassOptionsToPackage{unicode=true,colorlinks=true,urlcolor=blue}{hyperref} % options for packages loaded elsewhere
\PassOptionsToPackage{hyphens}{url}
\documentclass[a4paper,12pt,notitlepage,twoside,openright]{article}

\usepackage{ifxetex}
\ifxetex{}
\else
\errmessage{Must be built with XeLaTeX}
\fi

\usepackage{amsmath}
\usepackage{fontspec}
\usepackage{fourier-otf} % erewhon-math
\setmonofont{iosevka-type-slab-regular}[
  Path=../common/iosevka-type-slab/,
  Extension=.ttf,
  BoldFont=iosevka-type-slab-bold,
  ItalicFont=iosevka-type-slab-italic,
  BoldItalicFont=iosevka-type-slab-bolditalic,
  Scale=MatchLowercase,
]

% Math
\usepackage[binary-units]{siunitx}

\usepackage{caption}
\usepackage{authblk}
\usepackage{enumitem}
\usepackage{footnote}
\usepackage{bookmark}

% Table
\usepackage{tabu}
\usepackage{longtable}
\usepackage{booktabs}
\usepackage{multirow}

% Verbatim & Source code
\usepackage{fancyvrb}
\usepackage{minted}

% Beauty
\usepackage[protrusion]{microtype}
\usepackage[defaultlines=3]{nowidow}
\usepackage{upquote}
\usepackage{parskip}
\usepackage[strict]{changepage}

\usepackage{hyperref}

% Graph
\usepackage{graphicx}
\usepackage{grffile}
\usepackage{tikz}

\usepackage{endnotes}
\setlist{noitemsep}

\hypersetup{
	bookmarksnumbered,
	colorlinks, % hidelinks,
	pdfpagemode=UseNone,
	pdfstartview=FitH
}

\usetikzlibrary{calc,shapes.multipart,chains,arrows,matrix}
\tikzset{>=latex'}
\tikzstyle{list}=[
	 rectangle split,
	 rectangle split parts=2, draw,
	 rectangle split horizontal,
	 rectangle split part align=base,
	 text width=3ex, text centered
    ]
    
\newcommand{\pFF}[2]{\draw[->] let \p1 = (#1.one), \p2 = (#1.center), \p3 = (#2.one), \p4 = (#2.north) in (\x1,\y2) -- (\x3,\y4)}
\newcommand{\pSF}[2]{\draw[->] let \p1 = (#1.two), \p2 = (#1.center), \p3 = (#2.west), \p4 = (#2.center) in (\x1,\y2) -- (\x3,\y4)}

\newcolumntype{L}{>{\RaggedRight}X}

\newcommand{\SpecialChapter}[1]{%
  \chapter*{#1}%
  \addcontentsline{toc}{chapter}{#1}%
  \markboth{\MakeUppercase{#1}}{\MakeUppercase{#1}}
}

\title{Dynamic Storage Allocation: A Survey and Critical Review}
\author{Paul R. Wilson, Mark S. Johnstone, Michael Neely, and David Boles}
\date{October 1999}

\begin{document}

\maketitle{}

\section*{3.7 Buddy Systems}

Buddy systems [Kno65, PN77] are a variant of segregated lists that supports
a limited but efficient kind of splitting and coalescing. In the simple buddy
schemes, the entire heap area is conceptually split into two large areas, and
those areas are further split into two smaller areas, and so on. This hierarchical
division of memory is used to constrain where objects are allocated, what their
allowable sizes are, and how they may be coalesced into larger free areas. For each
allowable size, a separate free list is maintained, in an array of free lists. Buddy
systems are therefore actually \emph{a special case of segregated fits}, using size classes
with rounding, and a peculiar limited technique for splitting and coalescing.

Buddy systems therefore implement an approximation of a best fit policy, but
with potentially serious variations due to peculiarities in splitting and coalescing.

(In practical terms, buddy systems appear to be distinctly inferior to more
general schemes supporting arbitrary coalescing; without heroic efforts at optimization
and/or hybridization, their cost in internal fragmentation alone seems
to be higher than the total fragmentation costs of better schemes.)

A free block may only be merged with its \emph{buddy}, which is its unique neighbor
at the same level in the binary hierarchical division. The resulting free block is
therefore always one of the free areas at the next higher level in the memory division
hierarchy --- at any level, the first block may only be merged with the
following block, which follows it in memory; conversely, the second block may
only be merged with the first, which precedes it in memory. This constraint on
coalescing ensures that the resulting merged free area will always be aligned on
one of the boundaries of the hierarchical splitting.

(This is perhaps best understood by example; the reader may wish to skip
ahead to the description of binary buddies, which are the simplest kind of buddy
systems.)

The purpose of the buddy allocation constraint is to ensure that when a
block is freed, its (unique) buddy can always be found by a simple address
computation, and its buddy will always be either a whole, entirely free block,
or an unavailable block. An unavailable block may be entirely allocated, or may
have been split and have some of its sub-parts allocated but not others. Either
way, the address computation will always be able to locate the beginning of the
buddy --- it will never find the middle of an allocated object. The buddy will be
either a whole (allocated or free) block of a determinate size, or the beginning
of a block of that size that has been split in a determinate way. If (and only
if) it turns out to be the header of a free block, and the block is the whole
buddy, the buddies can be merged. If the buddy is entirely or partly allocated,
the buddies cannot be merged --- even if there is an adjacent free area within the
(split) buddy.

Buddy coalescing is relatively fast, but perhaps the biggest advantage in
some contexts is that it requires little space overhead per object --- only one bit
is required per buddy, to indicate whether the buddy is a contiguous free area.
This can be implemented with a single-bit header per object or free block. Unfortunately,
for this to work, \emph{the size of the block being freed must be known} --- the
buddy mechanism itself does not record the sizes of the blocks. This is workable
in some statically-typed languages, where object sizes are known statically and
the compiler can supply the size argument to the freeing routine. In most current
languages and implementations, however, this is not the case, due to the presence
of variable-sized objects and/or because of the way libraries are typically
linked. Even in some languages where the sizes of objects are known, the ``single''
bit ends up up costing an entire word per object, because a single bit cannot
be ``stolen'' from the space for an allocated object --- objects must be aligned on
word boundaries for architectural reasons, and there is no provision for stealing
a bit from the space allocated to an object.\footnote{In some implementations of some languages, this is less of a problem, because all
objects have headers that encode type information, and one bit can be reserved for
use by the allocator and ignored by the language implementation. This complicates
the language implementation, but may be worthwhile if a buddy system is used.} Stealing a bit from each object
can be avoided, however, by keeping the bits in a separate table ``off to the side''
[IGK71], but this is fairly awkward, and such a bit table could probably be put
to better use with an entirely different basic allocation mechanism.

In practical terms, therefore, buddy systems usually require a header word
per object, to record the type and/or size. Other, less restrictive schemes can
get by with a word per object as well. Since buddy systems also incur internal
fragmentation, this apparently makes buddy systems unattractive relative to
more general coalescing schemes such as segregated fits.\footnote{Of course, buddy systems could become more attractive if it were to turn out that
the buddy policy has significant beneficial interactions with actual program behavior,
and unexpectedly reduced external fragmentation or increased locality. At present,
this does not appear to be the case.}

In experiments using both real and synthetic traces, buddy systems generally
exhibit significantly more fragmentation than segregated fits and indexed fits
schemes using boundary tags to support general coalescing. (Most of these results
come from synthetic trace studies, however; it appears that only two buddy
systems have ever been studied using real traces [WJNB95].)

Several significant variations on buddy systems have been devised:

\emph{Binary buddies}. Binary buddies are the simplest and best-known kind of buddy
system [Kno65]. In this scheme, all buddy sizes are a power of two, and each size
is divided into two equal parts. This makes address computations simple, because
all buddies are aligned on a power-of-two boundary offset from the beginning of
the heap area, and each bit in the offset of a block represents one level in the
buddy system's hierarchical splitting of memory --- if the bit is 0, it is the first of
a pair of buddies, and if the bit is 1, it is the second. These operations can be
implemented efficiently with bitwise logical operations.

On the other hand, systems based on closer size class spacings may be similarly
efficient if lookup tables are used to perform size class mappings quickly.

A major problem with binary buddies is that internal fragmentation is usually
relatively high--the expected case is (very roughly) about 28\% [Knu73, PN77],\footnote{
This figure varies somewhat depending on the expected range and skew of the size
distribution [PN77].
}
because any object size must be rounded up to the nearest power of two (minus
a word for the header, if the size field is stored).

\emph{Fibonacci buddies}. This variant of the buddy scheme uses a more closely-spaced
set of size classes, based on a Fibonacci series, to reduce internal fragmentation
[Hir73]. Since each number in the Fibonacci series is the sum of the two previous
numbers, a block can always be split (unevenly) to yield two blocks whose sizes
are also in the series. This limits the number of free lists required.

A further refinement, called generalized Fibonacci buddies [Hir73, Bur76,
PN77] uses a Fibonacci-like number series that starts with a larger number
and generates a somewhat more closely-spaced set of sizes.

A possible disadvantage of Fibonacci buddies is that when a block is split to
satisfy a request for a particular size, the remaining block is of a different size,
which is less likely to be useful if the program allocates many objects of the same
size [Wis78].

\emph{Weighted buddies}. Weighted buddy systems [SP74] use a different kind of size
class series than either binary or Fibonacci buddy systems. Some size classes can
be split only one way, while other size classes can be split in two ways. The size
classes include the powers of two, but in between each pair of successive sizes,
there is also a size that is three times a power of two. The series is thus 2, 3, 4,
6, 8, 12... (words). (Often, the series actually starts at 4 words.)

Sizes that are powers of two may only be split evenly in two, as in the binary
buddy system. This always yields another size in the series, namely the next
lower power of two.

Sizes that are three times a power of two can be split in two ways. They may
be split evenly in two, yielding a size that is another three-times-a-power-of-two
size. (E.g., a six may be split into two threes.) They may also be split unevenly
into two sizes that are one third and two thirds of the original size; these sizes
are always a power of two. (E.g., six may be split into two and four.).

Double buddies. Double buddy systems [Wis78, PH86, WJNB95] use a different
technique to allow a closer spacing of size classes. They use two different binary
buddy systems, with staggered sizes. For example, one buddy system may use
powers-of-two sizes (2, 4, 8, 16...) while another uses a powers-of-two spacing
starting at a different size, such as 3. (The resulting sizes are 3, 6, 12, 24 ...).
This is the same set of sizes used in weighted buddies, but the splitting rule is
quite different. Blocks may only be split in half, as in the binary buddy system,
so the resulting blocks are always in the same binary buddy series.

Request sizes are rounded up to the nearest size class in either series. This
reduces the internal fragmentation by about half, but means that space used for
blocks in one size series can only coalesced or split into sizes in that series. That
is, splitting a size whose place in the combined series is odd always produces
another size whose place is odd; likewise, splitting an even-numbered size always
produces an even-numbered size. (E.g., a block of size 16 can be split into 8's
and 4's, and a block of size 24 can be split into 12's and 6's, but not 8's or 4's.)

This may cause external fragmentation if blocks in one size series are freed,
and blocks in the other are requested. As an optimization, free areas of a relatively
large size (e.g., a whole free page) may be made available to the other size
series and split according to that size series' rules. (This complicates the treatment
of large objects, which could be treated entirely differently, or by another
buddy system for large units of free storage such as pages.)

Naturally, more than two buddy systems could be combined, to decrease
internal fragmentation at a possible cost in external fragmentation due to limitations
on sharing free memory between the different buddy systems.

As with simple segregated storage, it is possible to keep per-page counts of live
objects, and notice when an entire page is empty. Empty pages can be transferred
from one buddy series to another. To our knowledge, such an optimization has
never been implemented for a double buddy scheme.

Buddy systems can easily be enhanced with deferred coalescing techniques,
as in ``recombination delaying'' buddy systems [Kau84]. Another optimization
is to tailor a buddy system's size class series to a particular program, picking a
series that produces little internal fragmentation for the object sizes the program
uses heavily.

\end{document}
