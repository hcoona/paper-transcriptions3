\PassOptionsToPackage{dvipsnames}{xcolor}
\PassOptionsToPackage{unicode=true,colorlinks=true,urlcolor=blue}{hyperref} % options for packages loaded elsewhere
\PassOptionsToPackage{hyphens}{url}
\documentclass[a4paper,12pt,notitlepage,twoside,openright]{article}

\usepackage{ifxetex}
\ifxetex{}
\else
\errmessage{Must be built with XeLaTeX}
\fi

\usepackage{amsmath}
\usepackage{fontspec}
\usepackage{fourier-otf} % erewhon-math
\setmonofont{iosevka-type-slab-regular}[
  Path=../common/iosevka-type-slab/,
  Extension=.ttf,
  BoldFont=iosevka-type-slab-bold,
  ItalicFont=iosevka-type-slab-italic,
  BoldItalicFont=iosevka-type-slab-bolditalic,
  Scale=MatchLowercase,
]

% Math
\usepackage[binary-units]{siunitx}

\usepackage{caption}
\usepackage{authblk}
\usepackage{enumitem}
\usepackage{footnote}

% Table
\usepackage{tabu}
\usepackage{longtable}
\usepackage{booktabs}
\usepackage{multirow}

% Verbatim & Source code
\usepackage{fancyvrb}
\usepackage{minted}

% Beauty
\usepackage[protrusion]{microtype}
\usepackage[defaultlines=3]{nowidow}
\usepackage{upquote}
\usepackage{parskip}
\usepackage[strict]{changepage}

\usepackage{hyperref}

% Graph
\usepackage{graphicx}
\usepackage{grffile}
\usepackage{tikz}

\usepackage{endnotes}
\hypersetup{
  bookmarksnumbered,
  pdfborder={0 0 0},
  pdfpagemode=UseNone,
  pdfstartview=FitH,
  breaklinks=true}
\urlstyle{same}  % don't use monospace font for urls

\usetikzlibrary{arrows.meta,calc,shapes.geometric,shapes.misc}

\setminted{
  autogobble,
  breakbytokenanywhere,
  breaklines,
  fontsize=\footnotesize,
}
\setmintedinline{
  autogobble,
  breakbytokenanywhere,
  breaklines,
  fontsize=\footnotesize,
}

\makeatletter
\def\maxwidth{\ifdim\Gin@nat@width>\linewidth\linewidth\else\Gin@nat@width\fi}
\def\maxheight{\ifdim\Gin@nat@height>\textheight\textheight\else\Gin@nat@height\fi}
\makeatother

% Scale images if necessary, so that they will not overflow the page
% margins by default, and it is still possible to overwrite the defaults
% using explicit options in \includegraphics[width, height, ...]{}
\setkeys{Gin}{width=\maxwidth,height=\maxheight,keepaspectratio}
\setlength{\emergencystretch}{3em}  % prevent overfull lines
\setcounter{secnumdepth}{3}

% Redefines (sub)paragraphs to behave more like sections
\ifx\paragraph\undefined\else
\let\oldparagraph\paragraph{}
\renewcommand{\paragraph}[1]{\oldparagraph{#1}\mbox{}}
\fi
\ifx\subparagraph\undefined\else
\let\oldsubparagraph\subparagraph{}
\renewcommand{\subparagraph}[1]{\oldsubparagraph{#1}\mbox{}}
\fi

% set default figure placement to htbp
\makeatletter
\def\fps@figure{htbp}
\makeatother


\newminted{csharp}{}

\title{ConfigureAwait FAQ}
\author{Stephen Toub}
\date{December 11th, 2019}

\begin{document}
\maketitle

.NET added \texttt{async}/\texttt{await} to the languages and libraries
over seven years ago. In that time, it's caught on like wildfire, not
only across the .NET ecosystem, but also being replicated in a myriad of
other languages and frameworks. It's also seen a ton of improvements in
.NET, in terms of additional language constructs that utilize
asynchrony, APIs offering async support, and fundamental improvements in
the infrastructure that makes \texttt{async}/\texttt{await} tick (in
particular performance and diagnostic-enabling improvements in .NET
Core).

However, one aspect of \texttt{async}/\texttt{await} that continues to
draw questions is \texttt{ConfigureAwait}. In this post, I hope to
answer many of them. I intend for this post to be both readable from
start to finish as well as being a list of Frequently Asked Questions
(FAQ) that can be used as future reference.

To really understand \texttt{ConfigureAwait}, we need to start a bit
earlier\ldots{}

\hypertarget{what-is-a-synchronizationcontext}{%
\section{What is a
SynchronizationContext?}\label{what-is-a-synchronizationcontext}}

The
\href{https://docs.microsoft.com/en-us/dotnet/api/system.threading.synchronizationcontext}{\texttt{System.Threading.SynchronizationContext}
docs} state that it ``Provides the basic functionality for propagating a
synchronization context in various synchronization models.'' Not an
entirely obvious description.

For the 99.9\% use case, \texttt{SynchronizationContext} is just a type
that provides a virtual \texttt{Post} method, which takes a delegate to
be executed asynchronously (there are a variety of other virtual members
on \texttt{SynchronizationContext}, but they're much less used and are
irrelevant for this discussion). The base type's \texttt{Post} literally
\href{https://github.com/dotnet/runtime/blob/5e67c2480d8b9361923566243c1395a3d1a5d617/src/libraries/System.Private.CoreLib/src/System/Threading/SynchronizationContext.cs\#L25}{just
calls} \texttt{ThreadPool.QueueUserWorkItem} to asynchronously invoke
the supplied delegate. However, derived types override \texttt{Post} to
enable that delegate to be executed in the most appropriate place and at
the most appropriate time.

For example, Windows Forms has a
\href{https://github.com/dotnet/winforms/blob/94ce4a2e52bf5d0d07d3d067297d60c8a17dc6b4/src/System.Windows.Forms/src/System/Windows/Forms/WindowsFormsSynchronizationContext.cs}{\texttt{SynchronizationContext}-derived
type} that overrides \texttt{Post} to do the equivalent of
\texttt{Control.BeginInvoke}; that means any calls to its \texttt{Post}
method will cause the delegate to be invoked at some later point on the
thread associated with that relevant Control, aka ``the UI thread''.
Windows Forms relies on Win32 message handling and has a ``message
loop'' running on the UI thread, which simply sits waiting for new
messages to arrive to process. Those messages could be for mouse
movements and clicks, for keyboard typing, for system events, for
delegates being available to invoke, etc. So, given a
\texttt{SynchronizationContext} instance for the UI thread of a Windows
Forms application, to get a delegate to execute on that UI thread, one
simply needs to pass it to \texttt{Post}.

The same goes for Windows Presentation Foundation (WPF). It has its own
\href{https://github.com/dotnet/wpf/blob/ac9d1b7a6b0ee7c44fd2875a1174b820b3940619/src/Microsoft.DotNet.Wpf/src/WindowsBase/System/Windows/Threading/DispatcherSynchronizationContext.cs}{\texttt{SynchronizationContext}-derived
type} with a \texttt{Post} override that similarly ``marshals'' a
delegate to the UI thread (via \texttt{Dispatcher.BeginInvoke}), in this
case managed by a WPF Dispatcher rather than a Windows Forms Control.

And for Windows RunTime (WinRT). It has its own
\href{https://github.com/dotnet/runtime/blob/60d1224ddd68d8ac0320f439bb60ac1f0e9cdb27/src/libraries/System.Runtime.WindowsRuntime/src/System/Threading/WindowsRuntimeSynchronizationContext.cs}{\texttt{SynchronizationContext}-derived
type} with a \texttt{Post} override that also queues the delegate to the
UI thread via its \texttt{CoreDispatcher}.

This goes beyond just ``run this delegate on the UI thread''. Anyone can
implement a \texttt{SynchronizationContext} with a \texttt{Post} that
does anything. For example, I may not care what thread a delegate runs
on, but I want to make sure that any delegates \texttt{Post}'d to my
\texttt{SynchronizationContext} are executed with some limited degree of
concurrency. I can achieve that with a custom
\texttt{SynchronizationContext} like this:

\begin{csharpcode}
internal sealed class MaxConcurrencySynchronizationContext : SynchronizationContext
{
    private readonly SemaphoreSlim _semaphore;

    public MaxConcurrencySynchronizationContext(int maxConcurrencyLevel) =>
        _semaphore = new SemaphoreSlim(maxConcurrencyLevel);

    public override void Post(SendOrPostCallback d, object state) =>
        _semaphore.WaitAsync().ContinueWith(delegate
        {
            try { d(state); } finally { _semaphore.Release(); }
        }, default, TaskContinuationOptions.None, TaskScheduler.Default);

    public override void Send(SendOrPostCallback d, object state)
    {
        _semaphore.Wait();
        try { d(state); } finally { _semaphore.Release(); }
    }
}
\end{csharpcode}

In fact, the unit testing framework xunit
\href{https://github.com/xunit/xunit/blob/d81613bf752bb4b8774e9d4e77b2b62133b0d333/src/xunit.execution/Sdk/MaxConcurrencySyncContext.cs}{provides
a \texttt{SynchronizationContext}} very similar to this, which it uses
to limit the amount of code associated with tests that can be run
concurrently.

The benefit of all of this is the same as with any abstraction: it
provides a single API that can be used to queue a delegate for handling
however the creator of the implementation desires, without needing to
know the details of that implementation. So, if I'm writing a library,
and I want to go off and do some work, and then queue a delegate back to
the original location's ``context'', I just need to grab their
\texttt{SynchronizationContext}, hold on to it, and then when I'm done
with my work, call \texttt{Post} on that context to hand off the
delegate I want invoked. I don't need to know that for Windows Forms I
should grab a \texttt{Control} and use its \texttt{BeginInvoke}, or for
WPF I should grab a \texttt{Dispatcher} and uses its
\texttt{BeginInvoke}, or for xunit I should somehow acquire its context
and queue to it; I simply need to grab the current
\texttt{SynchronizationContext} and use that later on. To achieve that,
\texttt{SynchronizationContext} provides a \texttt{Current} property,
such that to achieve the aforementioned objective I might write code
like this:

\begin{csharpcode}
public void DoWork(Action worker, Action completion)
{
    SynchronizationContext sc = SynchronizationContext.Current;
    ThreadPool.QueueUserWorkItem(_ =>
    {
        try { worker(); }
        finally { sc.Post(_ => completion(), null); }
    });
}
\end{csharpcode}

A framework that wants to expose a custom context from \texttt{Current}
uses the \texttt{SynchronizationContext.SetSynchronizationContext}
method.

\hypertarget{what-is-a-taskscheduler}{%
\section{What is a TaskScheduler?}\label{what-is-a-taskscheduler}}

\texttt{SynchronizationContext} is a general abstraction for a
``scheduler''. Individual frameworks sometimes have their own
abstractions for a scheduler, and \texttt{System.Threading.Tasks} is no
exception. When \texttt{Task}s are backed by a delegate such that they
can be queued and executed, they're associated with a
\texttt{System.Threading.Tasks.TaskScheduler}. Just as
\texttt{SynchronizationContext} provides a virtual \texttt{Post} method
to queue a delegate's invocation (with the implementation later invoking
the delegate via typical delegate invocation mechanisms),
\texttt{TaskScheduler} provides an abstract \texttt{QueueTask} method
(with the implementation later invoking that \texttt{Task} via the
\texttt{ExecuteTask} method).

The default scheduler as returned by \texttt{TaskScheduler.Default} is
the thread pool, but it's possible to derive from \texttt{TaskScheduler}
and override the relevant methods to achieve arbitrary behaviors for
when and where a \texttt{Task} is invoked. For example, the core
libraries include the
\texttt{System.Threading.Tasks.ConcurrentExclusiveSchedulerPair} type.
An instance of this class exposes two \texttt{TaskScheduler} properties,
one called \texttt{ExclusiveScheduler} and one called
\texttt{ConcurrentScheduler}. Tasks scheduled to the
\texttt{ConcurrentScheduler} may run concurrently, but subject to a
limit supplied to \texttt{ConcurrentExclusiveSchedulerPair} when it was
constructed (similar to the
\texttt{MaxConcurrencySynchronizationContext} shown earlier), and no
\texttt{ConcurrentScheduler} \texttt{Task}s will run when a
\texttt{Task} scheduled to \texttt{ExclusiveScheduler} is running, with
only one exclusive \texttt{Task} allowed to run at a time\ldots{} in
this way, it behaves very much like a reader/writer-lock.

Like \texttt{SynchronizationContext}, \texttt{TaskScheduler} also has a
\texttt{Current} property, which returns the ``current''
\texttt{TaskScheduler}. Unlike \texttt{SynchronizationContext}, however,
there's no method for setting the current scheduler. Instead, the
current scheduler is the one associated with the currently running
\texttt{Task}, and a scheduler is provided to the system as part of
starting a \texttt{Task}. So, for example, this program will output
``True'', as the lambda used with \texttt{StartNew} is executed on the
\texttt{ConcurrentExclusiveSchedulerPair}'s \texttt{ExclusiveScheduler}
and will see \texttt{TaskScheduler.Current} set to that scheduler:

\begin{csharpcode}
using System;
using System.Threading.Tasks;

class Program
{
    static void Main()
    {
        var cesp = new ConcurrentExclusiveSchedulerPair();
        Task.Factory.StartNew(() =>
        {
            Console.WriteLine(TaskScheduler.Current == cesp.ExclusiveScheduler);
        }, default, TaskCreationOptions.None, cesp.ExclusiveScheduler).Wait();
    }
}
\end{csharpcode}

Interestingly, \texttt{TaskScheduler} provides a static
\texttt{FromCurrentSynchronizationContext} method, which creates a new
\texttt{TaskScheduler} that queues \texttt{Task}s to run on whatever
\texttt{SynchronizationContext.Current} returned, using its
\texttt{Post} method for queueing tasks.

\hypertarget{how-do-synchronizationcontext-and-taskscheduler-relate-to-await}{%
\section{How do SynchronizationContext and TaskScheduler relate to
await?}\label{how-do-synchronizationcontext-and-taskscheduler-relate-to-await}}

Consider writing a UI app with a \texttt{Button}. Upon clicking the
\texttt{Button}, we want to download some text from a web site and set
it as the \texttt{Button}'s \texttt{Content}. The \texttt{Button} should
only be accessed from the UI thread that owns it, so when we've
successfully downloaded the new date and time text and want to store it
back into the \texttt{Button}'s \texttt{Content}, we need to do so from
the thread that owns the control. If we don't, we get an exception like:

\begin{verbatim}
System.InvalidOperationException:
  'The calling thread cannot access this object because a different thread owns it.'
\end{verbatim}

If we were writing this out manually, we could use
\texttt{SynchronizationContext} as shown earlier to marshal the setting
of the \texttt{Content} back to the original context, such as via a
\texttt{TaskScheduler}:

\begin{csharpcode}
private static readonly HttpClient s_httpClient = new HttpClient();

private void downloadBtn_Click(object sender, RoutedEventArgs e)
{
    s_httpClient.GetStringAsync("http://example.com/currenttime").ContinueWith(downloadTask =>
    {
        downloadBtn.Content = downloadTask.Result;
    }, TaskScheduler.FromCurrentSynchronizationContext());
}
\end{csharpcode}

or using \texttt{SynchronizationContext} directly:

\begin{csharpcode}
private static readonly HttpClient s_httpClient = new HttpClient();

private void downloadBtn_Click(object sender, RoutedEventArgs e)
{
    SynchronizationContext sc = SynchronizationContext.Current;
    s_httpClient.GetStringAsync("http://example.com/currenttime").ContinueWith(downloadTask =>
    {
        sc.Post(delegate
        {
            downloadBtn.Content = downloadTask.Result;
        }, null);
    });
}
\end{csharpcode}

Both of these approaches, though, explicitly uses callbacks. We would
instead like to write the code naturally with
\texttt{async}/\texttt{await}:

\begin{csharpcode}
private static readonly HttpClient s_httpClient = new HttpClient();

private async void downloadBtn_Click(object sender, RoutedEventArgs e)
{
    string text = await s_httpClient.GetStringAsync("http://example.com/currenttime");
    downloadBtn.Content = text;
}
\end{csharpcode}

This ``just works'', successfully setting \texttt{Content} on the UI
thread, because just as with the manually implemented version above,
\texttt{await}ing a \texttt{Task} pays attention by default to
\texttt{SynchronizationContext.Current}, as well as to
\texttt{TaskScheduler.Current}. When you \texttt{await} anything in C\#,
the compiler transforms the code to ask (via calling
\texttt{GetAwaiter}) the ``awaitable'' (in this case, the \texttt{Task})
for an ``awaiter'' (in this case, a
\texttt{TaskAwaiter\textless{}string\textgreater{}}). That awaiter is
responsible for hooking up the callback (often referred to as the
``continuation'') that will call back into the state machine when the
awaited object completes, and it does so using whatever
context/scheduler it captured at the time the callback was registered.
While not exactly the code used (there are additional optimizations and
tweaks employed), it's something like this:

\begin{csharpcode}
object scheduler = SynchronizationContext.Current;
if (scheduler is null && TaskScheduler.Current != TaskScheduler.Default)
{
    scheduler = TaskScheduler.Current;
}
\end{csharpcode}

In other words, it first checks whether there's a
\texttt{SynchronizationContext} set, and if there isn't, whether there's
a non-default \texttt{TaskScheduler} in play. If it finds one, when the
callback is ready to be invoked, it'll use the captured scheduler;
otherwise, it'll generally just execute the callback on as part of the
operation completing the awaited task.

\hypertarget{what-does-configureawaitfalse-do}{%
\section{What does ConfigureAwait(false)
do?}\label{what-does-configureawaitfalse-do}}

The \texttt{ConfigureAwait} method isn't special: it's not recognized in
any special way by the compiler or by the runtime. It is simply a method
that returns a struct (a \texttt{ConfiguredTaskAwaitable}) that wraps
the original task it was called on as well as the specified Boolean
value. Remember that \texttt{await} can be used with any type that
exposes the right pattern. By returning a different type, it means that
when the compiler accesses the instances \texttt{GetAwaiter} method
(part of the pattern), it's doing so off of the type returned from
\texttt{ConfigureAwait} rather than off of the task directly, and that
provides a hook to change the behavior of how the \texttt{await} behaves
via this custom awaiter.

Specifically, awaiting the type returned from
\texttt{ConfigureAwait(continueOnCapturedContext:\ false)} instead of
awaiting the \texttt{Task} directly ends up impacting the logic shown
earlier for how the target context/scheduler is captured. It effectively
makes the previously shown logic more like this:

\begin{csharpcode}
object scheduler = null;
if (continueOnCapturedContext)
{
    scheduler = SynchronizationContext.Current;
    if (scheduler is null && TaskScheduler.Current != TaskScheduler.Default)
    {
        scheduler = TaskScheduler.Current;
    }
}
\end{csharpcode}

In other words, by specifying \texttt{false}, even if there is a current
context or scheduler to call back to, it pretends as if there isn't.

\hypertarget{why-would-i-want-to-use-configureawaitfalse}{%
\section{Why would I want to use
ConfigureAwait(false)?}\label{why-would-i-want-to-use-configureawaitfalse}}

\texttt{ConfigureAwait(continueOnCapturedContext:\ false)} is used to
avoid forcing the callback to be invoked on the original context or
scheduler. This has a few benefits:

\textbf{Improving performance.} There is a cost to queueing the callback
rather than just invoking it, both because there's extra work (and
typically extra allocation) involved, but also because it means certain
optimizations we'd otherwise like to employ in the runtime can't be used
(we can do more optimization when we know exactly how the callback will
be invoked, but if it's handed off to an arbitrary implementation of an
abstraction, we can sometimes be limited). For very hot paths, even the
extra costs of checking for the current \texttt{SynchronizationContext}
and the current \texttt{TaskScheduler} (both of which involve accessing
thread statics) can add measurable overhead. If the code after an
\texttt{await} doesn't actually require running in the original context,
using \texttt{ConfigureAwait(false)} can avoid all these costs: it won't
need to queue unnecessarily, it can utilize all the optimizations it can
muster, and it can avoid the unnecessary thread static accesses.

\textbf{Avoiding deadlocks.} Consider a library method that uses
\texttt{await} on the result of some network download. You invoke this
method and synchronously block waiting for it to complete, such as by
using \texttt{.Wait()} or \texttt{.Result} or
\texttt{.GetAwaiter().GetResult()} off of the returned \texttt{Task}
object. Now consider what happens if your invocation of it happens when
the current \texttt{SynchronizationContext} is one that limits the
number of operations that can be running on it to 1, whether explicitly
via something like the \texttt{MaxConcurrencySynchronizationContext}
shown earlier, or implicitly by this being a context that only has one
thread that can be used, e.g.~a UI thread. So you invoke the method on
that one thread and then block it waiting for the operation to complete.
The operation kicks off the network download and awaits it. Since by
default awaiting a \texttt{Task} will capture the current
\texttt{SynchronizationContext}, it does so, and when the network
download completes, it queues back to the
\texttt{SynchronizationContext} the callback that will invoke the
remainder of the operation. But the only thread that can process the
queued callback is currently blocked by your code blocking waiting on
the operation to complete. And that operation won't complete until the
callback is processed. Deadlock! This can apply even when the context
doesn't limit the concurrency to just 1, but when the resources are
limited in any fashion. Imagine the same situation, except using the
\texttt{MaxConcurrencySynchronizationContext} with a limit of 4. And
instead of making just one call to the operation, we queue to that
context 4 invocations, each of which makes the call and blocks waiting
for it to complete. We've now still blocked all of the resources while
waiting for the async methods to complete, and the only thing that will
allow those async methods to complete is if their callbacks can be
processed by this context that's already entirely consumed. Again,
deadlock! If instead the library method had used
\texttt{ConfigureAwait(false)}, it would not queue the callback back to
the original context, avoiding the deadlock scenarios.

\hypertarget{why-would-i-want-to-use-configureawaittrue}{%
\section{Why would I want to use
ConfigureAwait(true)?}\label{why-would-i-want-to-use-configureawaittrue}}

You wouldn't, unless you were using it purely as an indication that you
were purposefully not using \texttt{ConfigureAwait(false)} (e.g.~to
silence static analysis warnings or the like).
\texttt{ConfigureAwait(true)} does nothing meaningful. When comparing
\texttt{await\ task} with \texttt{await\ task.ConfigureAwait(true)},
they're functionally identical. If you see \texttt{ConfigureAwait(true)}
in production code, you can delete it without ill effect.

The \texttt{ConfigureAwait} method accepts a Boolean because there are
some niche situations in which you want to pass in a variable to control
the configuration. But the 99\% use case is with a hardcoded false
argument value, \texttt{ConfigureAwait(false)}.

\hypertarget{when-should-i-use-configureawaitfalse}{%
\section{When should I use
ConfigureAwait(false)?}\label{when-should-i-use-configureawaitfalse}}

It depends: are you implementing application-level code or
general-purpose library code?

When writing applications, you generally want the default behavior
(which is why it is the default behavior). If an app model / environment
(e.g.~Windows Forms, WPF, ASP.NET Core, etc.) publishes a custom
\texttt{SynchronizationContext}, there's almost certainly a really good
reason it does: it's providing a way for code that cares about
synchronization context to interact with the app model / environment
appropriately. So if you're writing an event handler in a Windows Forms
app, writing a unit test in xunit, writing code in an ASP.NET MVC
controller, whether or not the app model did in fact publish a
\texttt{SynchronizationContext}, you want to use that
\texttt{SynchronizationContext} if it exists. And that means the default
/ \texttt{ConfigureAwait(true)}. You make simple use of \texttt{await},
and the right things happen with regards to callbacks/continuations
being posted back to the original context if one existed. This leads to
the general guidance of: \textbf{if you're writing app-level code,
\emph{do not} use \texttt{ConfigureAwait(false)}}. If you think back to
the Click event handler code example earlier in this post:

\begin{csharpcode}
private static readonly HttpClient s_httpClient = new HttpClient();

private async void downloadBtn_Click(object sender, RoutedEventArgs e)
{
    string text = await s_httpClient.GetStringAsync("http://example.com/currenttime");
    downloadBtn.Content = text;
}
\end{csharpcode}

the setting of \texttt{downloadBtn.Content\ =\ text} needs to be done
back in the original context. If the code had violated this guideline
and instead used \texttt{ConfigureAwait(false)} when it shouldn't have:

\begin{csharpcode}
private static readonly HttpClient s_httpClient = new HttpClient();

private async void downloadBtn_Click(object sender, RoutedEventArgs e)
{
    string text = await s_httpClient.GetStringAsync("http://example.com/currenttime").ConfigureAwait(false); // bug
    downloadBtn.Content = text;
}
\end{csharpcode}

bad behavior will result. The same would go for code in a classic
ASP.NET app reliant on \texttt{HttpContext.Current}; using
\texttt{ConfigureAwait(false)} and then trying to use
\texttt{HttpContext.Current} is likely going to result in problems.

In contrast, general-purpose libraries are ``general purpose'' in part
because they don't care about the environment in which they're used. You
can use them from a web app or from a client app or from a test, it
doesn't matter, as the library code is agnostic to the app model it
might be used in. Being agnostic then also means that it's not going to
be doing anything that needs to interact with the app model in a
particular way, e.g.~it won't be accessing UI controls, because a
general-purpose library knows nothing about UI controls. Since we then
don't need to be running the code in any particular environment, we can
avoid forcing continuations/callbacks back to the original context, and
we do that by using \texttt{ConfigureAwait(false)} and gaining both the
performance and reliability benefits it brings. This leads to the
general guidance of: \textbf{if you're writing general-purpose library
code, use \texttt{ConfigureAwait(false)}}. This is why, for example,
you'll see every (or almost every) \texttt{await} in the .NET Core
runtime libraries using \texttt{ConfigureAwait(false)} on every
\texttt{await}; with a few exceptions, in cases where it doesn't it's
very likely a bug to be fixed. For example,
\href{https://github.com/dotnet/corefx/pull/38610}{this PR} fixed a
missing \texttt{ConfigureAwait(false)} call in \texttt{HttpClient}.

As with all guidance, of course, there can be exceptions, places where
it doesn't make sense. For example, one of the larger exemptions (or at
least categories that requires thought) in general-purpose libraries is
when those libraries have APIs that take delegates to be invoked. In
such cases, the caller of the library is passing potentially app-level
code to be invoked by the library, which then effectively renders those
``general purpose'' assumptions of the library moot. Consider, for
example, an asynchronous version of LINQ's Where method, e.g.
\texttt{public\ static\ async\ IAsyncEnumerable\textless{}T\textgreater{}\ WhereAsync(this\ IAsyncEnumerable\textless{}T\textgreater{}\ source,\ Func\textless{}T,\ bool\textgreater{}\ predicate)}.
Does \texttt{predicate} here need to be invoked back on the original
\texttt{SynchronizationContext} of the caller? That's up to the
implementation of \texttt{WhereAsync} to decide, and it's a reason it
may choose not to use \texttt{ConfigureAwait(false)}.

Even with these special cases, the general guidance stands and is a very
good starting point: use \texttt{ConfigureAwait(false)} if you're
writing general-purpose library / app-model-agnostic code, and otherwise
don't.

\hypertarget{does-configureawaitfalse-guarantee-the-callback-wont-be-run-in-the-original-context}{%
\section{Does ConfigureAwait(false) guarantee the callback won't be
run in the original
context?}\label{does-configureawaitfalse-guarantee-the-callback-wont-be-run-in-the-original-context}}

No.~It guarantees it won't be queued back to the original
context\ldots{} but that doesn't mean the code after an
\texttt{await\ task.ConfigureAwait(false)} won't still run in the
original context. That's because awaits on already-completed awaitables
just keep running past the \texttt{await} synchronously rather than
forcing anything to be queued back. So, if you \texttt{await} a task
that's already completed by the time it's awaited, regardless of whether
you used \texttt{ConfigureAwait(false)}, the code immediately after this
will continue to execute on the current thread in whatever context is
still current.

\hypertarget{is-it-ok-to-use-configureawaitfalse-only-on-the-first-await-in-my-method-and-not-on-the-rest}{%
\section{Is it ok to use ConfigureAwait(false) only on the first
await in my method and not on the
rest?}\label{is-it-ok-to-use-configureawaitfalse-only-on-the-first-await-in-my-method-and-not-on-the-rest}}

In general, no. See the previous FAQ. If the
\texttt{await\ task.ConfigureAwait(false)} involves a task that's
already completed by the time it's awaited (which is actually incredibly
common), then the \texttt{ConfigureAwait(false)} will be meaningless, as
the thread continues to execute code in the method after this and still
in the same context that was there previously.

One notable exception to this is if you know that the first
\texttt{await} will always complete asynchronously and the thing being
awaited will invoke its callback in an environment free of a custom
SynchronizationContext or a TaskScheduler. For example,
\texttt{CryptoStream} in the .NET runtime libraries wants to ensure that
its potentially computationally-intensive code doesn't run as part of
the caller's synchronous invocation, so it
\href{https://github.com/dotnet/runtime/blob/4f9ae42d861fcb4be2fcd5d3d55d5f227d30e723/src/libraries/System.Security.Cryptography.Primitives/src/System/Security/Cryptography/CryptoStream.cs\#L205}{uses
a custom awaiter} to ensure that everything after the first
\texttt{await} runs on a thread pool thread. However, even in that case
you'll notice that the next \texttt{await} still uses
\texttt{ConfigureAwait(false)}; technically that's not necessary, but it
makes code review a lot easier, as otherwise every time this code is
looked at it doesn't require an analysis to understand why
\texttt{ConfigureAwait(false)} was left off.

\hypertarget{can-i-use-task.run-to-avoid-using-configureawaitfalse}{%
\section{Can I use Task.Run to avoid using
ConfigureAwait(false)?}\label{can-i-use-task.run-to-avoid-using-configureawaitfalse}}

Yes. If you write:

\begin{csharpcode}
Task.Run(async delegate
{
    await SomethingAsync(); // won't see the original context
});
\end{csharpcode}

then a \texttt{ConfigureAwait(false)} on that \texttt{SomethingAsync()}
call will be a nop, because the delegate passed to \texttt{Task.Run} is
going to be executed on a thread pool thread, with no user code higher
on the stack, such that \texttt{SynchronizationContext.Current} will
return \texttt{null}. Further, \texttt{Task.Run} implicitly uses
\texttt{TaskScheduler.Default}, which means querying
\texttt{TaskScheduler.Current} inside of the delegate will also return
\texttt{Default}. That means the \texttt{await} will exhibit the same
behavior regardless of whether \texttt{ConfigureAwait(false)} was used.
It also doesn't make any guarantees about what code inside of this
lambda might do. If you have the code:

\begin{csharpcode}
Task.Run(async delegate
{
    SynchronizationContext.SetSynchronizationContext(new SomeCoolSyncCtx());
    await SomethingAsync(); // will target SomeCoolSyncCtx
});
\end{csharpcode}

then the code inside \texttt{SomethingAsync} will in fact see
\texttt{SynchronizationContext.Current} as that \texttt{SomeCoolSyncCtx}
instance, and both this \texttt{await} and any non-configured awaits
inside \texttt{SomethingAsync} will post back to it. So to use this
approach, you need to understand what all of the code you're queueing
may or may not do and whether its actions could thwart yours.

This approach also comes at the expense of needing to create/queue an
additional task object. That may or may not matter to your app or
library depending on your performance sensitivity.

Also keep in mind that such tricks may cause more problems than they're
worth and have other unintended consequences. For example, static
analysis tools (e.g.~Roslyn analyzers) have been written to flag awaits
that don't use \texttt{ConfigureAwait(false)}, such as
\href{https://docs.microsoft.com/en-us/visualstudio/code-quality/ca2007?view=vs-2019}{CA2007}.
If you enable such an analyzer but then employ a trick like this just to
avoid using \texttt{ConfigureAwait}, there's a good chance the analyzer
will flag it, and actually cause more work for you. So maybe you then
disable the analyzer because of its noisiness, and now you end up
missing other places in the codebase where you actually should have been
using \texttt{ConfigureAwait(false)}.

\hypertarget{can-i-use-synchronizationcontext.setsynchronizationcontext-to-avoid-using-configureawaitfalse}{%
\section{Can I use SynchronizationContext.SetSynchronizationContext
to avoid using
ConfigureAwait(false)?}\label{can-i-use-synchronizationcontext.setsynchronizationcontext-to-avoid-using-configureawaitfalse}}

No.~Well, maybe. It depends on the involved code.

Some developers write code like this:

\begin{csharpcode}
Task t;
SynchronizationContext old = SynchronizationContext.Current;
SynchronizationContext.SetSynchronizationContext(null);
try
{
    t = CallCodeThatUsesAwaitAsync(); // awaits in here won't see the original context
}
finally { SynchronizationContext.SetSynchronizationContext(old); }
await t; // will still target the original context
\end{csharpcode}

in hopes that it'll make the code inside
\texttt{CallCodeThatUsesAwaitAsync} see the current context as
\texttt{null}. And it will. However, the above will do nothing to affect
what the \texttt{await} sees for \texttt{TaskScheduler.Current}, so if
this code is running on some custom \texttt{TaskScheduler},
\texttt{await}s inside \texttt{CallCodeThatUsesAwaitAsync} (and that
don't use \texttt{ConfigureAwait(false)}) will still see and queue back
to that custom \texttt{TaskScheduler}.

All of the same caveats also apply as in the previous
\texttt{Task.Run}-related FAQ: there are perf implications of such a
workaround, and the code inside the try could also thwart these attempts
by setting a different context (or invoking code with a non-default
\texttt{TaskScheduler}).

With such a pattern, you also need to be careful about a slight
variation:

\begin{csharpcode}
SynchronizationContext old = SynchronizationContext.Current;
SynchronizationContext.SetSynchronizationContext(null);
try
{
    await t;
}
finally { SynchronizationContext.SetSynchronizationContext(old); }
\end{csharpcode}

See the problem? It's a bit hard to see but also potentially very
impactful. There's no guarantee that the \texttt{await} will end up
invoking the callback/continuation on the original thread, which means
the resetting of the \texttt{SynchronizationContext} back to the
original may not actually happen on the original thread, which could
lead subsequent work items on that thread to see the wrong context (to
counteract this, well-written app models that set a custom context
generally add code to manually reset it before invoking any further user
code). And even if it does happen to run on the same thread, it may be a
while before it does, such that the context won't be appropriately
restored for a while. And if it runs on a different thread, it could end
up setting the wrong context onto that thread. And so on. Very far from
ideal.

\hypertarget{im-using-getawaiter.getresult.-do-i-need-to-use-configureawaitfalse}{%
\section{I'm using GetAwaiter().GetResult(). Do I need to use
ConfigureAwait(false)?}\label{im-using-getawaiter.getresult.-do-i-need-to-use-configureawaitfalse}}

No.~\texttt{ConfigureAwait} only affects the callbacks. Specifically,
the awaiter pattern requires awaiters to expose an \texttt{IsCompleted}
property, a \texttt{GetResult} method, and an \texttt{OnCompleted}
method (optionally with an \texttt{UnsafeOnCompleted} method).
\texttt{ConfigureAwait} only affects the behavior of
\texttt{\{Unsafe\}OnCompleted}, so if you're just directly calling to
the awaiter's \texttt{GetResult()} method, whether you're doing it on
the \texttt{TaskAwaiter} or the
\texttt{ConfiguredTaskAwaitable.ConfiguredTaskAwaiter} makes zero
behavior difference. So, if you see
\texttt{task.ConfigureAwait(false).GetAwaiter().GetResult()} in code,
you can replace it with \texttt{task.GetAwaiter().GetResult()} (and also
consider whether you really want to be blocking like that).

\hypertarget{i-know-im-running-in-an-environment-that-will-never-have-a-custom-synchronizationcontext-or-custom-taskscheduler.-can-i-skip-using-configureawaitfalse}{%
\section{I know I'm running in an environment that will never have a
custom SynchronizationContext or custom TaskScheduler. Can I skip using
ConfigureAwait(false)?}\label{i-know-im-running-in-an-environment-that-will-never-have-a-custom-synchronizationcontext-or-custom-taskscheduler.-can-i-skip-using-configureawaitfalse}}

Maybe. It depends on how sure you are of the ``never'' part. As
mentioned in previous FAQs, just because the app model you're working in
doesn't set a custom \texttt{SynchronizationContext} and doesn't invoke
your code on a custom \texttt{TaskScheduler} doesn't mean that some
other user or library code doesn't. So you need to be sure that's not
the case, or at least recognize the risk if it may be.

\hypertarget{ive-heard-configureawaitfalse-is-no-longer-necessary-in-.net-core.-true}{%
\section{I've heard ConfigureAwait(false) is no longer necessary in
.NET Core.
True?}\label{ive-heard-configureawaitfalse-is-no-longer-necessary-in-.net-core.-true}}

False. It's needed when running on .NET Core for exactly the same
reasons it's needed when running on .NET Framework. Nothing's changed in
that regard.

What has changed, however, is whether certain environments publish their
own \texttt{SynchronizationContext}. In particular, whereas the classic
ASP.NET on .NET Framework has
\href{https://github.com/microsoft/referencesource/blob/3b1eaf5203992df69de44c783a3eda37d3d4cd10/System.Web/AspNetSynchronizationContextBase.cs}{its
own \texttt{SynchronizationContext}}, in contrast ASP.NET Core does not.
That means that code running in an ASP.NET Core app by default won't see
a custom \texttt{SynchronizationContext}, which lessens the need for
\texttt{ConfigureAwait(false)} running in such an environment.

It doesn't mean, however, that there will never be a custom
\texttt{SynchronizationContext} or \texttt{TaskScheduler} present. If
some user code (or other library code your app is using) sets a custom
context and calls your code, or invokes your code in a \texttt{Task}
scheduled to a custom \texttt{TaskScheduler}, then even in ASP.NET Core
your awaits may see a non-default context or scheduler that would lead
you to want to use \texttt{ConfigureAwait(false)}. Of course, in such
situations, if you avoid synchronously blocking (which you should avoid
doing in web apps regardless) and if you don't mind the small
performance overheads in such limited occurrences, you can probably get
away without using \texttt{ConfigureAwait(false)}.

\hypertarget{can-i-use-configureawait-when-await-foreaching-an-iasyncenumerable}{%
\section{Can I use ConfigureAwait when 'await foreach'ing an
IAsyncEnumerable?}\label{can-i-use-configureawait-when-await-foreaching-an-iasyncenumerable}}

Yes. See this
\href{https://docs.microsoft.com/en-us/archive/msdn-magazine/2019/november/csharp-iterating-with-async-enumerables-in-csharp-8}{MSDN
Magazine article} for an example.

\texttt{await\ foreach} binds to a pattern, and so while it can be used
to enumerate an \texttt{IAsyncEnumerable\textless{}T\textgreater{}}, it
can also be used to enumerate something that exposes the right API
surface area. The .NET runtime libraries include a
\texttt{ConfigureAwait}
\href{https://github.com/dotnet/runtime/blob/91a717450bf5faa44d9295c01f4204dc5010e95c/src/libraries/System.Private.CoreLib/src/System/Threading/Tasks/TaskAsyncEnumerableExtensions.cs\#L25-L26}{extension
method} on \texttt{IAsyncEnumerable\textless{}T\textgreater{}} that
returns a custom type that wraps the
\texttt{IAsyncEnumerable\textless{}T\textgreater{}} and a
\texttt{Boolean} and exposes the right pattern. When the compiler
generates calls to the enumerator's \texttt{MoveNextAsync} and
\texttt{DisposeAsync} methods, those calls are to the returned
configured enumerator struct type, and it in turns performs the awaits
in the desired configured way.

\hypertarget{can-i-use-configureawait-when-await-using-an-iasyncdisposable}{%
\section{Can I use ConfigureAwait when `await using' an
IAsyncDisposable?}\label{can-i-use-configureawait-when-await-using-an-iasyncdisposable}}

Yes, though with a minor complication.

As with \texttt{IAsyncEnumerable\textless{}T\textgreater{}} described in
the previous FAQ, the .NET runtime libraries expose a
\texttt{ConfigureAwait} extension method on \texttt{IAsyncDisposable},
and \texttt{await\ using} will happily work with this as it implements
the appropriate pattern (namely exposing an appropriate
\texttt{DisposeAsync} method):

\begin{csharpcode}
await using (var c = new MyAsyncDisposableClass().ConfigureAwait(false))
{
    ...
}
\end{csharpcode}

The problem here is that the type of \texttt{c} is now not
\texttt{MyAsyncDisposableClass} but rather a
\texttt{System.Runtime.CompilerServices.ConfiguredAsyncDisposable},
which is the type returned from that \texttt{ConfigureAwait} extension
method on \texttt{IAsyncDisposable}.

To get around that, you need to write one extra line:

\begin{csharpcode}
var c = new MyAsyncDisposableClass();
await using (c.ConfigureAwait(false))
{
    ...
}
\end{csharpcode}

Now the type of \texttt{c} is again the desired
\texttt{MyAsyncDisposableClass}. This also has the effect of increasing
the scope of \texttt{c}; if that's impactful, you can wrap the whole
thing in braces.

\hypertarget{i-used-configureawaitfalse-but-my-asynclocal-still-flowed-to-code-after-the-await.-is-that-a-bug}{%
\section{I used ConfigureAwait(false), but my AsyncLocal still flowed
to code after the await. Is that a
bug?}\label{i-used-configureawaitfalse-but-my-asynclocal-still-flowed-to-code-after-the-await.-is-that-a-bug}}

No, that is expected. \texttt{AsyncLocal\textless{}T\textgreater{}} data
flows as part of \texttt{ExecutionContext}, which is separate from
\texttt{SynchronizationContext}. Unless you've explicitly disabled
\texttt{ExecutionContext} flow with
\texttt{ExecutionContext.SuppressFlow()}, \texttt{ExecutionContext} (and
thus \texttt{AsyncLocal\textless{}T\textgreater{}} data) will always
flow across \texttt{await}s, regardless of whether
\texttt{ConfigureAwait} is used to avoid capturing the original
\texttt{SynchronizationContext}. For more information, see this
\href{https://devblogs.microsoft.com/pfxteam/executioncontext-vs-synchronizationcontext/}{blog
post}.

\hypertarget{could-the-language-help-me-avoid-needing-to-use-configureawaitfalse-explicitly-in-my-library}{%
\section{Could the language help me avoid needing to use
ConfigureAwait(false) explicitly in my
library?}\label{could-the-language-help-me-avoid-needing-to-use-configureawaitfalse-explicitly-in-my-library}}

Library developers sometimes express their frustration with needing to
use \texttt{ConfigureAwait(false)} and ask for less invasive
alternatives.

Currently there aren't any, at least not built into the language /
compiler / runtime. There are however numerous proposals for what such a
solution might look like, e.g.
\url{https://github.com/dotnet/csharplang/issues/645},
\url{https://github.com/dotnet/csharplang/issues/2542},
\url{https://github.com/dotnet/csharplang/issues/2649}, and
\url{https://github.com/dotnet/csharplang/issues/2746}.

If this is important to you, or if you feel like you have new and
interesting ideas here, I encourage you to contribute your thoughts to
those or new discussions.

\end{document}
