\PassOptionsToPackage{dvipsnames}{xcolor}
\PassOptionsToPackage{unicode=true,colorlinks=true,urlcolor=blue}{hyperref} % options for packages loaded elsewhere
\PassOptionsToPackage{hyphens}{url}
\documentclass[a4paper,12pt,notitlepage,twoside,openright]{article}

\usepackage{ifxetex}
\ifxetex{}
\else
\errmessage{Must be built with XeLaTeX}
\fi

\usepackage{amsmath}
\usepackage{fontspec}
\usepackage{fourier-otf} % erewhon-math
\setmonofont{iosevka-type-slab-regular}[
  Path=../common/iosevka-type-slab/,
  Extension=.ttf,
  BoldFont=iosevka-type-slab-bold,
  ItalicFont=iosevka-type-slab-italic,
  BoldItalicFont=iosevka-type-slab-bolditalic,
  Scale=MatchLowercase,
]

% Math
\usepackage[binary-units]{siunitx}

\usepackage{caption}
\usepackage{authblk}
\usepackage{enumitem}
\usepackage{footnote}
\usepackage{bookmark}

% Table
\usepackage{tabu}
\usepackage{longtable}
\usepackage{booktabs}
\usepackage{multirow}

% Verbatim & Source code
\usepackage{fancyvrb}
\usepackage{minted}

% Beauty
\usepackage[protrusion]{microtype}
\usepackage[defaultlines=3]{nowidow}
\usepackage{upquote}
\usepackage{parskip}
\usepackage[strict]{changepage}

\usepackage{hyperref}

% Graph
\usepackage{graphicx}
\usepackage{grffile}
\usepackage{tikz}

\usepackage{endnotes}
\setlist{noitemsep}

\hypersetup{
	bookmarksnumbered,
	colorlinks, % hidelinks,
	pdfpagemode=UseNone,
	pdfstartview=FitH
}

\usetikzlibrary{calc,shapes.multipart,chains,arrows,matrix}
\tikzset{>=latex'}
\tikzstyle{list}=[
	 rectangle split,
	 rectangle split parts=2, draw,
	 rectangle split horizontal,
	 rectangle split part align=base,
	 text width=3ex, text centered
    ]
    
\newcommand{\pFF}[2]{\draw[->] let \p1 = (#1.one), \p2 = (#1.center), \p3 = (#2.one), \p4 = (#2.north) in (\x1,\y2) -- (\x3,\y4)}
\newcommand{\pSF}[2]{\draw[->] let \p1 = (#1.two), \p2 = (#1.center), \p3 = (#2.west), \p4 = (#2.center) in (\x1,\y2) -- (\x3,\y4)}

\newcolumntype{L}{>{\RaggedRight}X}

\newcommand{\SpecialChapter}[1]{%
  \chapter*{#1}%
  \addcontentsline{toc}{chapter}{#1}%
  \markboth{\MakeUppercase{#1}}{\MakeUppercase{#1}}
}

\title{Parquet}

\begin{document}

\maketitle

Parquet is a columnar storage format that supports nested data.

Parquet metadata is encoded using Apache Thrift.

The \texttt{Parquet-format} project contains all Thrift definitions that
are necessary to create readers and writers for Parquet files.

\hypertarget{motivation}{%
\section{Motivation}\label{motivation}}

We created Parquet to make the advantages of compressed, efficient
columnar data representation available to any project in the Hadoop
ecosystem.

Parquet is built from the ground up with complex nested data structures
in mind, and uses the
\href{https://github.com/julienledem/redelm/wiki/The-striping-and-assembly-algorithms-from-the-Dremel-paper}{record
shredding and assembly algorithm} described in the Dremel paper. We
believe this approach is superior to simple flattening of nested name
spaces.

Parquet is built to support very efficient compression and encoding
schemes. Multiple projects have demonstrated the performance impact of
applying the right compression and encoding scheme to the data. Parquet
allows compression schemes to be specified on a per-column level, and is
future-proofed to allow adding more encodings as they are invented and
implemented.

Parquet is built to be used by anyone. The Hadoop ecosystem is rich with
data processing frameworks, and we are not interested in playing
favorites. We believe that an efficient, well-implemented columnar
storage substrate should be useful to all frameworks without the cost of
extensive and difficult to set up dependencies.

\hypertarget{modules}{%
\section{Modules}\label{modules}}

The \texttt{parquet-format} project contains format specifications and
Thrift definitions of metadata required to properly read Parquet files.

The \texttt{parquet-mr} project contains multiple sub-modules, which
implement the core components of reading and writing a nested,
column-oriented data stream, map this core onto the parquet format, and
provide Hadoop Input/Output Formats, Pig loaders, and other java-based
utilities for interacting with Parquet.

The \texttt{parquet-compatibility} project contains compatibility tests
that can be used to verify that implementations in different languages
can read and write each other's files.

\hypertarget{building}{%
\section{Building}\label{building}}

Java resources can be built using \texttt{mvn\ package}. The current
stable version should always be available from Maven Central.

C++ thrift resources can be generated via make.

Thrift can be also code-generated into any other thrift-supported
language.

\hypertarget{glossary}{%
\section{Glossary}\label{glossary}}

\begin{itemize}
\item
  Block (HDFS block): This means a block in HDFS and the meaning is
  unchanged for describing this file format. The file format is designed
  to work well on top of HDFS.
\item
  File: A HDFS file that must include the metadata for the file. It does
  not need to actually contain the data.
\item
  Row group: A logical horizontal partitioning of the data into rows.
  There is no physical structure that is guaranteed for a row group. A
  row group consists of a column chunk for each column in the dataset.
\item
  Column chunk: A chunk of the data for a particular column. They live
  in a particular row group and are guaranteed to be contiguous in the
  file.
\item
  Page: Column chunks are divided up into pages. A page is conceptually
  an indivisible unit (in terms of compression and encoding). There can
  be multiple page types which are interleaved in a column chunk.
\end{itemize}

Hierarchically, a file consists of one or more row groups. A row group
contains exactly one column chunk per column. Column chunks contain one
or more pages.

\hypertarget{unit-of-parallelization}{%
\section{Unit of parallelization}\label{unit-of-parallelization}}

\begin{itemize}

\item
  MapReduce - File/Row Group
\item
  IO - Column chunk
\item
  Encoding/Compression - Page
\end{itemize}

\hypertarget{file-format}{%
\section{File format}\label{file-format}}

This file and the \href{src/main/thrift/parquet.thrift}{thrift
definition} should be read together to understand the format.

\begin{verbatim}
4-byte magic number "PAR1"
<Column 1 Chunk 1 + Column Metadata>
<Column 2 Chunk 1 + Column Metadata>
...
<Column N Chunk 1 + Column Metadata>
<Column 1 Chunk 2 + Column Metadata>
<Column 2 Chunk 2 + Column Metadata>
...
<Column N Chunk 2 + Column Metadata>
...
<Column 1 Chunk M + Column Metadata>
<Column 2 Chunk M + Column Metadata>
...
<Column N Chunk M + Column Metadata>
File Metadata
4-byte length in bytes of file metadata (little endian)
4-byte magic number "PAR1"
\end{verbatim}

In the above example, there are N columns in this table, split into M
row groups. The file metadata contains the locations of all the column
metadata start locations. More details on what is contained in the
metadata can be found in the thrift definition.

Metadata is written after the data to allow for single pass writing.

Readers are expected to first read the file metadata to find all the
column chunks they are interested in. The columns chunks should then be
read sequentially.

\begin{figure}
\centering
\includegraphics{./3553e76845a6ef453c1495ddbe8643693d4aa213.png}
\caption{File Layout}
\end{figure}

\hypertarget{metadata}{%
\section{Metadata}\label{metadata}}

There are three types of metadata: file metadata, column (chunk)
metadata and page header metadata. All thrift structures are serialized
using the TCompactProtocol.

\begin{figure}
\centering
\includegraphics{./7b4cc4d01103e697021ddc668cbfcc6ab1411d13.png}
\caption{Metadata diagram}
\end{figure}

\hypertarget{types}{%
\section{Types}\label{types}}

The types supported by the file format are intended to be as minimal as
possible, with a focus on how the types effect on disk storage. For
example, 16-bit ints are not explicitly supported in the storage format
since they are covered by 32-bit ints with an efficient encoding. This
reduces the complexity of implementing readers and writers for the
format. The types are:

\begin{itemize}

\item
  BOOLEAN: 1 bit boolean
\item
  INT32: 32 bit signed ints
\item
  INT64: 64 bit signed ints
\item
  INT96: 96 bit signed ints
\item
  FLOAT: IEEE 32-bit floating point values
\item
  DOUBLE: IEEE 64-bit floating point values
\item
  BYTE\_ARRAY: arbitrarily long byte arrays.
\end{itemize}

\hypertarget{logical-types}{%
\subsection{Logical Types}\label{logical-types}}

Logical types are used to extend the types that parquet can be used to
store, by specifying how the primitive types should be interpreted. This
keeps the set of primitive types to a minimum and reuses parquet's
efficient encodings. For example, strings are stored as byte arrays
(binary) with a UTF8 annotation. These annotations define how to further
decode and interpret the data. Annotations are stored as
\texttt{ConvertedType} fields in the file metadata and are documented in
\url{LogicalTypes.md}.

\hypertarget{nested-encoding}{%
\section{Nested Encoding}\label{nested-encoding}}

To encode nested columns, Parquet uses the Dremel encoding with
definition and repetition levels. Definition levels specify how many
optional fields in the path for the column are defined. Repetition
levels specify at what repeated field in the path has the value
repeated. The max definition and repetition levels can be computed from
the schema (i.e.~how much nesting there is). This defines the maximum
number of bits required to store the levels (levels are defined for all
values in the column).

Two encodings for the levels are supported BIT\_PACKED and RLE. Only RLE
is now used as it supersedes BIT\_PACKED.

\hypertarget{nulls}{%
\section{Nulls}\label{nulls}}

Nullity is encoded in the definition levels (which is run-length
encoded). NULL values are not encoded in the data. For example, in a
non-nested schema, a column with 1000 NULLs would be encoded with
run-length encoding (0, 1000 times) for the definition levels and
nothing else.

\hypertarget{data-pages}{%
\section{Data Pages}\label{data-pages}}

For data pages, the 3 pieces of information are encoded back to back,
after the page header.

In order we have:

\begin{enumerate}
\def\labelenumi{\arabic{enumi}.}

\item
  repetition levels data
\item
  definition levels data
\item
  encoded values
\end{enumerate}

The value of \texttt{uncompressed\_page\_size} specified in the header
is for all the 3 pieces combined.

The encoded values for the data page is always required. The definition
and repetition levels are optional, based on the schema definition. If
the column is not nested (i.e. the path to the column has length 1), we
do not encode the repetition levels (it would always have the value 1).
For data that is required, the definition levels are skipped (if
encoded, it will always have the value of the max definition level).

For example, in the case where the column is non-nested and required,
the data in the page is only the encoded values.

The supported encodings are described in
\href{https://github.com/apache/parquet-format/blob/master/Encodings.md}{Encodings.md}

\hypertarget{column-chunks}{%
\section{Column chunks}\label{column-chunks}}

Column chunks are composed of pages written back to back. The pages
share a common header and readers can skip over pages they are not
interested in. The data for the page follows the header and can be
compressed and/or encoded. The compression and encoding is specified in
the page metadata.

Additionally, files can contain an optional column index to allow
readers to skip pages more efficiently. See \url{PageIndex.md} for
details and the reasoning behind adding these to the format.

\hypertarget{checksumming}{%
\section{Checksumming}\label{checksumming}}

Data pages can be individually checksummed. This allows disabling of
checksums at the HDFS file level, to better support single row lookups.
Data page checksums are calculated using the standard CRC32 algorithm on
the compressed data of a page (not including the page header itself).

\hypertarget{error-recovery}{%
\section{Error recovery}\label{error-recovery}}

If the file metadata is corrupt, the file is lost. If the column
metadata is corrupt, that column chunk is lost (but column chunks for
this column in other row groups are okay). If a page header is corrupt,
the remaining pages in that chunk are lost. If the data within a page is
corrupt, that page is lost. The file will be more resilient to
corruption with smaller row groups.

Potential extension: With smaller row groups, the biggest issue is
placing the file metadata at the end. If an error happens while writing
the file metadata, all the data written will be unreadable. This can be
fixed by writing the file metadata every Nth row group. Each file
metadata would be cumulative and include all the row groups written so
far. Combining this with the strategy used for rc or avro files using
sync markers, a reader could recover partially written files.

\hypertarget{separating-metadata-and-column-data.}{%
\section{Separating metadata and column
data.}\label{separating-metadata-and-column-data.}}

The format is explicitly designed to separate the metadata from the
data. This allows splitting columns into multiple files, as well as
having a single metadata file reference multiple parquet files.

\hypertarget{configurations}{%
\section{Configurations}\label{configurations}}

\begin{itemize}

\item
  Row group size: Larger row groups allow for larger column chunks which
  makes it possible to do larger sequential IO. Larger groups also
  require more buffering in the write path (or a two pass write). We
  recommend large row groups (512MB - 1GB). Since an entire row group
  might need to be read, we want it to completely fit on one HDFS block.
  Therefore, HDFS block sizes should also be set to be larger. An
  optimized read setup would be: 1GB row groups, 1GB HDFS block size, 1
  HDFS block per HDFS file.
\item
  Data page size: Data pages should be considered indivisible so smaller
  data pages allow for more fine grained reading (e.g.~single row
  lookup). Larger page sizes incur less space overhead (less page
  headers) and potentially less parsing overhead (processing headers).
  Note: for sequential scans, it is not expected to read a page at a
  time; this is not the IO chunk. We recommend 8KB for page sizes.
\end{itemize}

\hypertarget{extensibility}{%
\section{Extensibility}\label{extensibility}}

There are many places in the format for compatible extensions:

\begin{itemize}

\item
  File Version: The file metadata contains a version.
\item
  Encodings: Encodings are specified by enum and more can be added in
  the future.
\item
  Page types: Additional page types can be added and safely skipped.
\end{itemize}

\hypertarget{contributing}{%
\section{Contributing}\label{contributing}}

Comment on the issue and/or contact
\href{http://mail-archives.apache.org/mod_mbox/parquet-dev/}{the
parquet-dev mailing list} with your questions and ideas.

Changes to this core format definition are proposed and discussed in
depth on the mailing list. You may also be interested in contributing to
the Parquet-MR subproject, which contains all the Java-side
implementation and APIs. See the ``How To Contribute'' section of the
\href{https://github.com/apache/parquet-mr\#how-to-contribute}{Parquet-MR
project}

\hypertarget{code-of-conduct}{%
\section{Code of Conduct}\label{code-of-conduct}}

We hold ourselves and the Parquet developer community to a code of
conduct as described by
\href{https://engineering.twitter.com/opensource}{Twitter OSS}:
\url{https://github.com/twitter/code-of-conduct/blob/master/code-of-conduct.md}.

\hypertarget{license}{%
\section{License}\label{license}}

Copyright 2013 Twitter, Cloudera and other contributors.

Licensed under the Apache License, Version 2.0:
\url{http://www.apache.org/licenses/LICENSE-2.0}

\end{document}
