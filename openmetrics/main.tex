\PassOptionsToPackage{dvipsnames}{xcolor}
\PassOptionsToPackage{unicode=true,colorlinks=true,urlcolor=blue}{hyperref} % options for packages loaded elsewhere
\PassOptionsToPackage{hyphens}{url}
\documentclass[a4paper,12pt,notitlepage,twoside,openright]{article}

\usepackage{ifxetex}
\ifxetex{}
\else
\errmessage{Must be built with XeLaTeX}
\fi

\usepackage{amsmath}
\usepackage{fontspec}
\usepackage{fourier-otf} % erewhon-math
\setmonofont{iosevka-type-slab-regular}[
  Path=../common/iosevka-type-slab/,
  Extension=.ttf,
  BoldFont=iosevka-type-slab-bold,
  ItalicFont=iosevka-type-slab-italic,
  BoldItalicFont=iosevka-type-slab-bolditalic,
  Scale=MatchLowercase,
]

% Math
\usepackage[binary-units]{siunitx}

\usepackage{caption}
\usepackage{authblk}
\usepackage{enumitem}
\usepackage{footnote}

% Table
\usepackage{tabu}
\usepackage{longtable}
\usepackage{booktabs}
\usepackage{multirow}

% Verbatim & Source code
\usepackage{fancyvrb}
\usepackage{minted}

% Beauty
\usepackage[protrusion]{microtype}
\usepackage[defaultlines=3]{nowidow}
\usepackage{upquote}
\usepackage{parskip}
\usepackage[strict]{changepage}

\usepackage{hyperref}

% Graph
\usepackage{graphicx}
\usepackage{grffile}
\usepackage{tikz}

\usepackage{endnotes}
\hypersetup{
  bookmarksnumbered,
  pdfborder={0 0 0},
  pdfpagemode=UseNone,
  pdfstartview=FitH,
  breaklinks=true}
\urlstyle{same}  % don't use monospace font for urls

\usetikzlibrary{arrows.meta,calc,shapes.geometric,shapes.misc}

\setminted{
  autogobble,
  breakbytokenanywhere,
  breaklines,
  fontsize=\footnotesize,
}
\setmintedinline{
  autogobble,
  breakbytokenanywhere,
  breaklines,
  fontsize=\footnotesize,
}

\makeatletter
\def\maxwidth{\ifdim\Gin@nat@width>\linewidth\linewidth\else\Gin@nat@width\fi}
\def\maxheight{\ifdim\Gin@nat@height>\textheight\textheight\else\Gin@nat@height\fi}
\makeatother

% Scale images if necessary, so that they will not overflow the page
% margins by default, and it is still possible to overwrite the defaults
% using explicit options in \includegraphics[width, height, ...]{}
\setkeys{Gin}{width=\maxwidth,height=\maxheight,keepaspectratio}
\setlength{\emergencystretch}{3em}  % prevent overfull lines
\setcounter{secnumdepth}{3}

% Redefines (sub)paragraphs to behave more like sections
\ifx\paragraph\undefined\else
\let\oldparagraph\paragraph{}
\renewcommand{\paragraph}[1]{\oldparagraph{#1}\mbox{}}
\fi
\ifx\subparagraph\undefined\else
\let\oldsubparagraph\subparagraph{}
\renewcommand{\subparagraph}[1]{\oldsubparagraph{#1}\mbox{}}
\fi

% set default figure placement to htbp
\makeatletter
\def\fps@figure{htbp}
\makeatother


\title{OpenMetrics, a cloud-native, highly scalable metrics protocol}
\author{Richard Hartmann \and Ben Kochie \and Brian Brazil \and Rob Skillington}
\date{Nov 25 2020}

\begin{document}
\maketitle

\begin{abstract}

OpenMetrics specifies today's de-facto standard for transmitting
cloud-native metrics at scale, with support for both text representation
and Protocol Buffers and brings it into IETF. It supports both pull and
push-based data collection.

\end{abstract}

\hypertarget{introduction}{%
\section{Introduction}\label{introduction}}

Created in 2012, Prometheus has been the default for cloud-native
observability since 2015. A central part of Prometheus' design is its
text metric exposition format, called the Prometheus exposition format
0.0.4, stable since 2014. In this format, special care has been taken to
make it easy to generate, to ingest, and to understand by humans. As of
2020, there are more than 700 publicly listed exporters, an unknown
number of unlisted exporters, and thousands of native library
integrations using this format. Dozens of ingestors from various
projects and companies support consuming it.

With OpenMetrics, we are cleaning up and tightening the specification
with the express purpose of bringing it into IETF. We are documenting a
working standard with wide and organic adoption while introducing
minimal, largely backwards-compatible, and well-considered changes. As
of 2020, dozens of exporters, integrations, and ingestors use and
preferentially negotiate OpenMetrics already.

Given the wide adoption and significant coordination requirements in the
ecosystem, sweeping changes to either the Prometheus exposition format
0.0.4 or OpenMetrics 1.0 are considered out of scope.

\hypertarget{requirements-language}{%
\subsection{Requirements Language}\label{requirements-language}}

\{::boilerplate bcp14\}

\hypertarget{overview}{%
\section{Overview}\label{overview}}

Metrics are a specific kind of telemetry data. They represent a snapshot
of the current state for a set of data. They are distinct from logs or
events, which focus on records or information about individual events.

OpenMetrics is primarily a wire format, independent of any particular
transport for that format. The format is expected to be consumed on a
regular basis and to be meaningful over successive expositions.

Implementers MUST expose metrics in the OpenMetrics text format in
response to a simple HTTP GET request to a documented URL for a given
process or device. This endpoint SHOULD be called ``/metrics''.
Implementers MAY also expose OpenMetrics formatted metrics in other
ways, such as by regularly pushing metric sets to an operator-configured
endpoint over HTTP.

\hypertarget{metrics-and-time-series}{%
\subsection{Metrics and Time Series}\label{metrics-and-time-series}}

This standard expresses all system states as numerical values; counts,
current values, enumerations, and boolean states being common examples.
Contrary to metrics, singular events occur at a specific time. Metrics
tend to aggregate data temporally. While this can lose information, the
reduction in overhead is an engineering trade-off commonly chosen in
many modern monitoring systems.

Time series are a record of changing information over time. While time
series can support arbitrary strings or binary data, only numeric data
is in scope for this RFC.

Common examples of metric time series would be network interface
counters, device temperatures, BGP connection states, and alert states.

\hypertarget{data-model}{%
\section{Data Model}\label{data-model}}

This section MUST be read together with the ABNF section. In case of
disagreements between the two, the ABNF's restrictions MUST take
precedence. This reduces repetition as the text wire format MUST be
supported.

\hypertarget{data-types}{%
\subsection{Data Types}\label{data-types}}

\hypertarget{values}{%
\subsubsection{Values}\label{values}}

Metric values in OpenMetrics MUST be either floating points or integers.
Note that ingestors of the format MAY only support float64. The non-real
values NaN, +Inf and -Inf MUST be supported. NaN MUST NOT be considered
a missing value, but it MAY be used to signal a division by zero.

\hypertarget{booleans}{%
\paragraph{Booleans}\label{booleans}}

Boolean values MUST follow 1==true, 0==false.

\hypertarget{timestamps}{%
\subsubsection{Timestamps}\label{timestamps}}

Timestamps MUST be Unix Epoch in seconds. Negative timestamps MAY be
used.

\hypertarget{strings}{%
\subsubsection{Strings}\label{strings}}

Strings MUST only consist of valid UTF-8 characters and MAY be zero
length. NULL (ASCII 0x0) MUST be supported.

\hypertarget{label}{%
\subsubsection{Label}\label{label}}

Labels are key-value pairs consisting of strings.

Label names beginning with underscores are RESERVED and MUST NOT be used
unless specified by this standard. Label names MUST follow the
restrictions in the ABNF section.

Empty label values SHOULD be treated as if the label was not present.

\hypertarget{labelset}{%
\subsubsection{LabelSet}\label{labelset}}

A LabelSet MUST consist of Labels and MAY be empty. Label names MUST be
unique within a LabelSet.

\hypertarget{metricpoint}{%
\subsubsection{MetricPoint}\label{metricpoint}}

Each MetricPoint consists of a set of values, depending on the
MetricFamily type.

\hypertarget{exemplars}{%
\subsubsection{Exemplars}\label{exemplars}}

Exemplars are references to data outside of the MetricSet. A common use
case are IDs of program traces.

Exemplars MUST consist of a LabelSet and a value, and MAY have a
timestamp. They MAY each be different from the MetricPoints' LabelSet
and timestamp.

The combined length of the label names and values of an Exemplar's
LabelSet MUST NOT exceed 128 UTF-8 characters. Other characters in the
text rendering of an exemplar such as ",= are not included in this limit
for implementation simplicity and for consistency between the text and
proto formats.

Ingestors MAY discard exemplars.

\hypertarget{metric}{%
\subsubsection{Metric}\label{metric}}

Metrics are defined by a unique LabelSet within a MetricFamily. Metrics
MUST contain a list of one or more MetricPoints. Metrics with the same
name for a given MetricFamily SHOULD have the same set of label names in
their LabelSet.

MetricPoints SHOULD NOT have explicit timestamps.

If more than one MetricPoint is exposed for a Metric, then its
MetricPoints MUST have monotonically increasing timestamps.

\hypertarget{metricfamily}{%
\subsubsection{MetricFamily}\label{metricfamily}}

A MetricFamily MAY have zero or more Metrics. A MetricFamily MUST have a
name, HELP, TYPE, and UNIT metadata. Every Metric within a MetricFamily
MUST have a unique LabelSet.

\hypertarget{name}{%
\paragraph{Name}\label{name}}

MetricFamily names are a string and MUST be unique within a MetricSet.
Names SHOULD be in snake\_case. Metric names MUST follow the
restrictions in the ABNF section.

Colons in MetricFamily names are RESERVED to signal that the
MetricFamily is the result of a calculation or aggregation of a general
purpose monitoring system.

MetricFamily names beginning with underscores are RESERVED and MUST NOT
be used unless specified by this standard.

\hypertarget{suffixes}{%
\subparagraph{Suffixes}\label{suffixes}}

The name of a MetricFamily MUST NOT result in a potential clash for
sample metric names as per the ABNF with another MetricFamily in the
Text Format within a MetricSet. An example would be a gauge called
``foo\_created'' as a counter called ``foo'' could create a
``foo\_created'' in the text format.

Exposers SHOULD avoid names that could be confused with the suffixes
that text format sample metric names use.

\begin{itemize}
\item
  Suffixes for the respective types are:
\item
  Counter: '\_total', '\_created'
\item
  Summary: '\_count', '\_sum', '\_created', '\,' (empty)
\item
  Histogram: '\_count', '\_sum', '\_bucket', '\_created'
\item
  GaugeHistogram: '\_gcount', '\_gsum', '\_bucket'
\item
  Info: '\_info'
\item
  Gauge: '\,' (empty)
\item
  StateSet: '\,' (empty)
\item
  Unknown: '\,' (empty)
\end{itemize}

\hypertarget{type}{%
\paragraph{Type}\label{type}}

Type specifies the MetricFamily type. Valid values are ``unknown'',
``gauge'', ``counter'', ``stateset'', ``info'', ``histogram'',
``gaugehistogram'', and ``summary''.

\hypertarget{unit}{%
\paragraph{Unit}\label{unit}}

Unit specifies MetricFamily units. If non-empty, it MUST be a suffix of
the MetricFamily name separated by an underscore. Be aware that further
generation rules might make it an infix in the text format.

\hypertarget{help}{%
\paragraph{Help}\label{help}}

Help is a string and SHOULD non-empty. It is used to give a brief
description of the MetricFamily for human consumption and SHOULD be
short enough to be used as a tooltip.

\hypertarget{metricset}{%
\paragraph{MetricSet}\label{metricset}}

A MetricSet is the top level object exposed by OpenMetrics. It MUST
consist of MetricFamilies and MAY be empty.

Each MetricFamily name MUST be unique. The same label name and value
SHOULD NOT appear on every Metric within a MetricSet.

There is no specific ordering of MetricFamilies required within a
MetricSet. An exposer MAY make an exposition easier to read for humans,
for example sort alphabetically if the performance tradeoff makes sense.

If present, an Info MetricFamily called ``target'' per the ``Supporting
target metadata in both push-based and pull-based systems'' section
below SHOULD be first.

\hypertarget{metric-types}{%
\subsection{Metric Types}\label{metric-types}}

\hypertarget{gauge}{%
\subsubsection{Gauge}\label{gauge}}

Gauges are current measurements, such as bytes of memory currently used
or the number of items in a queue. For gauges the absolute value is what
is of interest to a user.

A MetricPoint in a Metric with the type gauge MUST have a single value.

Gauges MAY increase, decrease, or stay constant over time. Even if they
only ever go in one direction, they might still be gauges and not
counters. The size of a log file would usually only increase, a resource
might decrease, and the limit of a queue size may be constant.

A gauge MAY be used to encode an enum where the enum has many states and
changes over time, it is the most efficient but least user friendly.

\hypertarget{counter}{%
\subsubsection{Counter}\label{counter}}

Counters measure discrete events. Common examples are the number of HTTP
requests received, CPU seconds spent, or bytes sent. For counters how
quickly they are increasing over time is what is of interest to a user.

A MetricPoint in a Metric with the type Counter MUST have one value
called Total. A Total is a non-NaN and MUST be monotonically
non-decreasing over time, starting from 0.

A MetricPoint in a Metric with the type Counter SHOULD have a Timestamp
value called Created. This can help ingestors discern between new
metrics and long-running ones it did not see before.

A MetricPoint in a Metric's Counter's Total MAY reset to 0. If present,
the corresponding Created time MUST also be set to the timestamp of the
reset.

A MetricPoint in a Metric's Counter's Total MAY have an exemplar.

\hypertarget{stateset}{%
\subsubsection{StateSet}\label{stateset}}

StateSets represent a series of related boolean values, also called a
bitset. If ENUMs need to be encoded this MAY be done via StateSet.

A point of a StateSet metric MAY contain multiple states and MUST
contain one boolean per State. States have a name which are Strings.

A StateSet Metric's LabelSet MUST NOT have a label name which is the
same as the name of its MetricFamily.

If encoded as a StateSet, ENUMs MUST have exactly one Boolean which is
true within a MetricPoint.

This is suitable where the enum value changes over time, and the number
of States isn't much more than a handful.

EDITOR'S NOTE: This might be better as Consideration

MetricFamilies of type StateSets MUST have an empty Unit string.

\hypertarget{info}{%
\subsubsection{Info}\label{info}}

Info metrics are used to expose textual information which SHOULD NOT
change during process lifetime. Common examples are an application's
version, revision control commit, and the version of a compiler.

A MetricPoint of an Info Metric contains a LabelSet. An Info
MetricPoint's LabelSet MUST NOT have a label name which is the same as
the name of a label of the LabelSet of its Metric.

Info MAY be used to encode ENUMs whose values do not change over time,
such as the type of a network interface.

MetricFamilies of type Info MUST have an empty Unit string.

\hypertarget{histogram}{%
\subsubsection{Histogram}\label{histogram}}

Histograms measure distributions of discrete events. Common examples are
the latency of HTTP requests, function runtimes, or I/O request sizes.

A Histogram MetricPoint MUST contain at least one bucket, and SHOULD
contain Sum, and Created values. Every bucket MUST have a threshold and
a value.

Histogram MetricPoints MUST have at least a bucket with an +Inf
threshold. Buckets MUST be cumulative. As an example for a metric
representing request latency in seconds its values for buckets with
thresholds 1, 2, 3, and +Inf MUST follow value\_1 \textless= value\_2
\textless= value\_3 \textless= value\_+Inf. If ten requests took 1
second each, the values of the 1, 2, 3, and +Inf buckets MUST equal 10.

The +Inf bucket counts all requests. If present, the Sum value MUST
equal the Sum of all the measured event values. Bucket thresholds within
a MetricPoint MUST be unique.

Semantically, Sum, and buckets values are counters so MUST NOT be NaN or
negative. Negative threshold buckets MAY be used, but then the Histogram
MetricPoint MUST NOT contain a sum value as it would no longer be a
counter semantically. Bucket thresholds MUST NOT equal NaN. Count and
bucket values MUST be integers.

A Histogram MetricPoint SHOULD have a Timestamp value called Created.
This can help ingestors discern between new metrics and long-running
ones it did not see before.

A Histogram's Metric's LabelSet MUST NOT have a ``le'' label name.

Bucket values MAY have exemplars. Buckets are cumulative to allow
monitoring systems to drop any non-+Inf bucket for
performance/anti-denial-of-service reasons in a way that loses
granularity but is still a valid Histogram.

EDITOR'S NOTE: The second sentence is a consideration, it can be moved
if needed

Each bucket covers the values less and or equal to it, and the value of
the exemplar MUST be within this range. Exemplars SHOULD be put into the
bucket with the highest value. A bucket MUST NOT have more than one
exemplar.

\hypertarget{gaugehistogram}{%
\subsubsection{GaugeHistogram}\label{gaugehistogram}}

GaugeHistograms measure current distributions. Common examples are how
long items have been waiting in a queue, or size of the requests in a
queue.

A GaugeHistogram MetricPoint MUST have at least one bucket with an +Inf
threshold, and SHOULD contain a Gsum value. Every bucket MUST have a
threshold and a value.

The buckets for a GaugeHistogram follow all the same rules as for a
Histogram.

The bucket and Gsum of a GaugeHistogram are conceptually gauges, however
bucket values MUST NOT be negative or NaN. If negative threshold buckets
are present, then sum MAY be negative. Gsum MUST NOT be NaN. Bucket
values MUST be integers.

A GaugeHistogram's Metric's LabelSet MUST NOT have a ``le'' label name.

Bucket values can have exemplars.

Each bucket covers the values less and or equal to it, and the value of
the exemplar MUST be within this range. Exemplars SHOULD be put into the
bucket with the highest value. A bucket MUST NOT have more than one
exemplar.

\hypertarget{summary}{%
\subsubsection{Summary}\label{summary}}

Summaries also measure distributions of discrete events and MAY be used
when Histograms are too expensive and/or an average event size is
sufficient.

They MAY also be used for backwards compatibility, because some existing
instrumentation libraries expose precomputed quantiles and do not
support Histograms. Precomputed quantiles SHOULD NOT be used, because
quantiles are not aggregatable and the user often can not deduce what
timeframe they cover.

A Summary MetricPoint MAY consist of a Count, Sum, Created, and a set of
quantiles.

Semantically, Count and Sum values are counters so MUST NOT be NaN or
negative. Count MUST be an integer.

A MetricPoint in a Metric with the type Summary which contains Count or
Sum values SHOULD have a Timestamp value called Created. This can help
ingestors discern between new metrics and long-running ones it did not
see before. Created MUST NOT relate to the collection period of quantile
values.

Quantiles are a map from a quantile to a value. An example is a quantile
0.95 with value 0.2 in a metric called
myapp\_http\_request\_duration\_seconds which means that the 95th
percentile latency is 200ms over an unknown timeframe. If there are no
events in the relevant timeframe, the value for a quantile MUST be NaN.
A Quantile's Metric's LabelSet MUST NOT have ``quantile'' label name.
Quantiles MUST be between 0 and 1 inclusive. Quantile values MUST NOT be
negative. Quantile values SHOULD represent the recent values. Commonly
this would be over the last 5-10 minutes.

\hypertarget{unknown}{%
\subsubsection{Unknown}\label{unknown}}

Unknown SHOULD NOT be used. Unknown MAY be used when it is impossible to
determine the types of individual metrics from 3rd party systems.

A point in a metric with the unknown type MUST have a single value.

\hypertarget{data-transmission-wire-formats}{%
\section{Data transmission \& wire
formats}\label{data-transmission-wire-formats}}

The text wire format MUST be supported and is the default. The protobuf
wire format MAY be supported and MUST ONLY be used after negotiation.

The OpenMetrics formats are Regular Chomsky Grammars, making writing
quick and small parsers possible. The text format compresses well, and
protobuf is already binary and efficiently encoded.

Partial or invalid expositions MUST be considered erroneous in their
entirety.

\hypertarget{protocol-negotiation}{%
\subsection{Protocol Negotiation}\label{protocol-negotiation}}

All ingestor implementations MUST be able to ingest data secured with
TLS 1.2 or later. All exposers SHOULD be able to emit data secured with
TLS 1.2 or later. ingestor implementations SHOULD be able to ingest data
from HTTP without TLS. All implementations SHOULD use TLS to transmit
data.

Negotiation of what version of the OpenMetrics format to use is
out-of-band. For example for pull-based exposition over HTTP standard
HTTP content type negotiation is used, and MUST default to the oldest
version of the standard (i.e.~1.0.0) if no newer version is requested.

Push-based negotiation is inherently more complex, as the exposer
typically initiates the connection. Producers MUST use the oldest
version of the standard (i.e.~1.0.0) unless requested otherwise by the
ingestor. Text format

\hypertarget{abnf}{%
\subsubsection{ABNF}\label{abnf}}

ABNF as per RFC 5234

EDITOR'S NOTE: Should we update to RFC 7405, in particular the case
insensitive bits?

``exposition'' is the top level token of the ABNF.

\begin{verbatim}
exposition = metricset HASH SP eof [ LF ]

metricset = *metricfamily

metricfamily = *metric-descriptor *metric

metric-descriptor = HASH SP type SP metricname SP metric-type LF
metric-descriptor =/ HASH SP help SP metricname SP escaped-string LF
metric-descriptor =/ HASH SP unit SP metricname SP 1*metricname-char LF

metric = *sample

metric-type = counter / gauge / histogram / gaugehistogram / stateset
metric-type =/ info / summary / unknown

sample = metricname [labels] SP number [SP timestamp] [exemplar] LF

exemplar = SP HASH SP labels SP number [SP timestamp]

labels = "{" [label *(COMMA label)] "}"

label = label-name EQ DQUOTE escaped-string DQUOTE

number = realnumber
; Case insensitive
number =/ [SIGN] ("inf" / "infinity")
number =/ "nan"

timestamp = realnumber

; Not 100% sure this captures all float corner cases.
; Leading 0s explicitly okay
realnumber = [SIGN] 1*DIGIT
realnumber =/ [SIGN] 1*DIGIT ["." *DIGIT] [ "e" [SIGN] 1*DIGIT ]
realnumber =/ [SIGN] *DIGIT "." 1*DIGIT [ "e" [SIGN] 1*DIGIT ]


; RFC 5234 is case insensitive.
; Uppercase
eof = %d69.79.70
type = %d84.89.80.69
help = %d72.69.76.80
unit = %d85.78.73.84
; Lowercase
counter = %d99.111.117.110.116.101.114
gauge = %d103.97.117.103.101
histogram = %d104.105.115.116.111.103.114.97.109
gaugehistogram = gauge histogram
stateset = %d115.116.97.116.101.115.101.116
info = %d105.110.102.111
summary = %d115.117.109.109.97.114.121
unknown = %d117.110.107.110.111.119.110

BS = "\"
EQ = "="
COMMA = ","
HASH = "#"
SIGN = "-" / "+"

metricname = metricname-initial-char 0*metricname-char

metricname-char = metricname-initial-char / DIGIT
metricname-initial-char = ALPHA / "_" / ":"

label-name = label-name-initial-char *label-name-char

label-name-char = label-name-initial-char / DIGIT
label-name-initial-char = ALPHA / "_"

escaped-string = *escaped-char

escaped-char = normal-char
escaped-char =/ BS ("n" / DQUOTE / BS)

; Any unicode character, except newline, double quote, and backslash
normal-char = %x00-09 / %x0B-21 / %x23-5B / %x5D-D7FF / %xE000-10FFFF
\end{verbatim}

\hypertarget{overall-structure}{%
\subsubsection{Overall Structure}\label{overall-structure}}

UTF-8 MUST be used. Byte order markers (BOMs) MUST NOT be used. As an
important reminder for implementers, byte 0 is valid UTF-8 while, for
example, byte 255 is not.

The content type MUST be: application/openmetrics-text; version=1.0.0;
charset=utf-8

Line endings MUST be signalled with line feed (\\n) and MUST NOT contain
carriage returns (\\r). Expositions MUST end with EOF and SHOULD end with
`EOF\\n'.

An example of a complete exposition:
\textasciitilde\textasciitilde\textasciitilde\textasciitilde{} \# TYPE
acme\_http\_router\_request\_seconds summary \# UNIT
acme\_http\_router\_request\_seconds seconds \# HELP
acme\_http\_router\_request\_seconds Latency though all of ACME's HTTP
request router.
acme\_http\_router\_request\_seconds\_sum\{path=``/api/v1'',method=``GET''\}
9036.32
acme\_http\_router\_request\_seconds\_count\{path=``/api/v1'',method=``GET''\}
807283.0
acme\_http\_router\_request\_seconds\_created\{path=``/api/v1'',method=``GET''\}
1605281325.0
acme\_http\_router\_request\_seconds\_sum\{path=``/api/v2'',method=``POST''\}
479.3
acme\_http\_router\_request\_seconds\_count\{path=``/api/v2'',method=``POST''\}
34.0
acme\_http\_router\_request\_seconds\_created\{path=``/api/v2'',method=``POST''\}
1605281325.0 \# TYPE go\_goroutines gauge \# HELP go\_goroutines Number
of goroutines that currently exist. go\_goroutines 69 \# TYPE
process\_cpu\_seconds counter \# UNIT process\_cpu\_seconds seconds \#
HELP process\_cpu\_seconds Total user and system CPU time spent in
seconds. process\_cpu\_seconds\_total 4.20072246e+06 \# EOF
\textasciitilde\textasciitilde\textasciitilde\textasciitilde{}

\hypertarget{escaping}{%
\paragraph{Escaping}\label{escaping}}

Where the ABNF notes escaping, the following escaping MUST be applied
Line feed, `\\n' (0x0A) -\textgreater{} literally `\\n' (Bytecode 0x5c
0x6e) Double quotes -\textgreater{} `"' (Bytecode 0x5c 0x22) Backslash
-\textgreater{} `\textbackslash{}' (Bytecode 0x5c 0x5c)

\hypertarget{numbers}{%
\paragraph{Numbers}\label{numbers}}

Integer numbers MUST NOT have a decimal point. Examples are ``23'',
``0042'', and ``1341298465647914''.

Floating point numbers MUST be represented either with a decimal point
or using scientific notation. Examples are ``8903.123421'' and
``1.89e-7''. Floating point numbers MUST fit within the range of a
64-bit floating point value as defined by IEEE 754, but MAY require so
many bits in the mantissa that results in lost precision. This MAY be
used to encode nanosecond resolution timestamps.

Arbitrary integer and floating point rendering of numbers MUST NOT be
used for ``quantile'' and ``le'' label values as in section ``Canonical
Numbers''. They MAY be used anywhere else numbers are used.

\hypertarget{considerations-canonical-numbers}{%
\subparagraph{Considerations: Canonical
Numbers}\label{considerations-canonical-numbers}}

Numbers in the ``le'' label values of histograms and ``quantile'' label
values of summary metrics are special in that they're label values, and
label values are intended to be opaque. As end users will likely
directly interact with these string values, and as many monitoring
systems lack the ability to deal with them as first-class numbers, it
would be beneficial if a given number had the exact same text
representation.

Consistency is highly desirable, but real world implementations of
languages and their runtimes make mandating this impractical. The most
important common quantiles are 0.5, 0.95, 0.9, 0.99, 0.999 and bucket
values representing values from a millisecond up to 10.0 seconds,
because those cover cases like latency SLAs and Apdex for typical web
services. Powers of ten are covered to try to ensure that the switch
between fixed point and exponential rendering is consistent as this
varies across runtimes. The target rendering is equivalent to the
default Go rendering of float64 values (i.e.~\%g), with a .0 appended in
case there is no decimal point or exponent to make clear that they are
floats.

Exposers MUST produce output for positive infinity as +Inf.

Exposers SHOULD produce output for the values 0.0 up to 10.0 in 0.001
increments in line with the following examples: 0.0 0.001 0.002 0.01 0.1
0.9 0.95 0.99 0.999 1.0 1.7 10.0

Exposers SHOULD produce output for the values 1e-10 up to 1e+10 in
powers of ten in line with the following examples: 1e-10 1e-09 1e-05
0.0001 0.1 1.0 100000.0 1e+06 1e+10

Parsers MUST NOT reject inputs which are outside of the canonical values
merely because they are not consistent with the canonical values. For
example 1.1e-4 must not be rejected, even though it is not the
consistent rendering of 0.00011.

Exposers SHOULD follow these patterns for non-canonical numbers, and the
intention is by adjusting the rendering algorithm to be consistent for
these values that the vast majority of other values will also have
consistent rendering. Exposers using only a few particular le/quantile
values could also hardcode. In languages such as C where a minimal
floating point rendering algorithm such as Grisu3 such as Grisu3 is not
readily available, exposers MAY use a different rendering.

A warning to implementers in C and other languages that share its printf
implementation: The standard precision of \%f, \%e and \%g is only six
significant digits. 17 significant digits are required for full
precision, e.g.~\texttt{printf("\%.17g",\ d)}.

\hypertarget{timestamps-1}{%
\paragraph{Timestamps}\label{timestamps-1}}

Timestamps SHOULD NOT use exponential float rendering for timestamps if
nanosecond precision is needed as rendering of a float64 does not have
sufficient precision, e.g.~1604676851.123456789.

\hypertarget{metricfamily-1}{%
\subsubsection{MetricFamily}\label{metricfamily-1}}

There MUST NOT be an explicit separator between MetricFamilies. The next
MetricFamily MUST be signalled with either metadata or a new sample
metric name which cannot be part of the previous MetricFamily.

MetricFamilies MUST NOT be interleaved.

\hypertarget{metricfamily-metadata}{%
\paragraph{MetricFamily metadata}\label{metricfamily-metadata}}

There are four pieces of metadata: The MetricFamily name, TYPE, UNIT and
HELP. An example of the metadata for a counter Metric called foo is:

\begin{verbatim}
# TYPE foo counter
\end{verbatim}

If no TYPE is exposed, the MetricFamily MUST be of type Unknown.

If a unit is specified it MUST be provided in a UNIT metadata line. In
addition, an underscore and the unit MUST be the suffix of the
MetricFamily name.

A valid example for a foo\_seconds metric with a unit of ``seconds'':
\textasciitilde\textasciitilde\textasciitilde\textasciitilde{} \# TYPE
foo\_seconds counter \# UNIT foo\_seconds seconds
\textasciitilde\textasciitilde\textasciitilde\textasciitilde{}

An invalid example, where the unit is not a suffix on the name:
\textasciitilde\textasciitilde\textasciitilde\textasciitilde{} \# TYPE
foo counter \# UNIT foo seconds
\textasciitilde\textasciitilde\textasciitilde\textasciitilde{}

It is also valid to have:
\textasciitilde\textasciitilde\textasciitilde\textasciitilde{} \# TYPE
foo\_seconds counter
\textasciitilde\textasciitilde\textasciitilde\textasciitilde{}

If the unit is known it SHOULD be provided.

The value of a UNIT or HELP line MAY be empty. This MUST be treated as
if no metadata line for the MetricFamily existed.

\begin{verbatim}
# TYPE foo_seconds counter
# UNIT foo_seconds seconds
# HELP foo_seconds Some text and \\n some \" escaping
\end{verbatim}

There MUST NOT be more than one of each type of metadata line for a
MetricFamily. The ordering SHOULD be TYPE, UNIT, HELP.

Aside from this metadata and the EOF line at the end of the message, you
MUST NOT expose lines beginning with a \#.

\hypertarget{metric-1}{%
\paragraph{Metric}\label{metric-1}}

Metrics MUST NOT be interleaved.

See the example in ``Text format -\textgreater{} MetricPoint''. Labels A
sample without labels or a timestamp and the value 0 MUST be rendered
either like:

\begin{verbatim}
bar_seconds_count 0
\end{verbatim}

or like

\begin{verbatim}
bar_seconds_count{} 0
\end{verbatim}

Label values MAY be any valid UTF-8 value, so escaping MUST be applied
as per the ABNF. A valid example with two labels:
\textasciitilde\textasciitilde\textasciitilde\textasciitilde{}
bar\_seconds\_count\{a=``x'',b=``escaping" example \\n ''\} 0
\textasciitilde\textasciitilde\textasciitilde\textasciitilde{}

The rendering of values for a MetricPoint can include additional labels
(e.g.~the ``le'' label for a Histogram type), which MUST be rendered in
the same way as a Metric's own LabelSet.

\hypertarget{metricpoint-1}{%
\subsubsection{MetricPoint}\label{metricpoint-1}}

MetricPoints MUST NOT be interleaved.

A correct example where there were multiple MetricPoints and Samples
within a MetricFamily would be:

\begin{verbatim}
# TYPE foo_seconds summary
# UNIT foo_seconds seconds
foo_seconds_count{a="bb"} 0 123
foo_seconds_sum{a="bb"} 0 123
foo_seconds_count{a="bb"} 0 456
foo_seconds_sum{a="bb"} 0 456
foo_seconds_count{a="ccc"} 0 123
foo_seconds_sum{a="ccc"} 0 123
foo_seconds_count{a="ccc"} 0 456
foo_seconds_sum{a="ccc"} 0 456
\end{verbatim}

An incorrect example where Metrics are interleaved:

\begin{verbatim}
# TYPE foo_seconds summary
# UNIT foo_seconds seconds
foo_seconds_count{a="bb"} 0 123
foo_seconds_count{a="ccc"} 0 123
foo_seconds_count{a="bb"} 0 456
foo_seconds_count{a="ccc"} 0 456
\end{verbatim}

An incorrect example where MetricPoints are interleaved:

\begin{verbatim}
# TYPE foo_seconds summary
# UNIT foo_seconds seconds
foo_seconds_count{a="bb"} 0 123
foo_seconds_count{a="bb"} 0 456
foo_seconds_sum{a="bb"} 0 123
foo_seconds_sum{a="bb"} 0 456
\end{verbatim}

\hypertarget{metric-types-1}{%
\subsubsection{Metric types}\label{metric-types-1}}

\hypertarget{gauge-1}{%
\paragraph{Gauge}\label{gauge-1}}

The Sample MetricName for the value of a MetricPoint for a MetricFamily
of type Gauge MUST NOT have a suffix.

An example MetricFamily with a Metric with no labels and a MetricPoint
with no timestamp:
\textasciitilde\textasciitilde\textasciitilde\textasciitilde{} \# TYPE
foo gauge foo 17.0
\textasciitilde\textasciitilde\textasciitilde\textasciitilde{}

An example of a MetricFamily with two Metrics with a label and
MetricPoints with no timestamp:
\textasciitilde\textasciitilde\textasciitilde\textasciitilde{} \# TYPE
foo gauge foo\{a=``bb''\} 17.0 foo\{a=``ccc''\} 17.0
\textasciitilde\textasciitilde\textasciitilde\textasciitilde{}

An example of a MetricFamily with no Metrics:
\textasciitilde\textasciitilde\textasciitilde\textasciitilde{} \# TYPE
foo gauge \textasciitilde\textasciitilde\textasciitilde\textasciitilde{}

An example with a Metric with a label and a MetricPoint with a
timestamp:
\textasciitilde\textasciitilde\textasciitilde\textasciitilde{} \# TYPE
foo gauge foo\{a=``b''\} 17.0 1520879607.789
\textasciitilde\textasciitilde\textasciitilde\textasciitilde{}

An example with a Metric with no labels and MetricPoint with a
timestamp:
\textasciitilde\textasciitilde\textasciitilde\textasciitilde{} \# TYPE
foo gauge foo 17.0 1520879607.789
\textasciitilde\textasciitilde\textasciitilde\textasciitilde{}

An example with a Metric with no labels and two MetricPoints with
timestamps:
\textasciitilde\textasciitilde\textasciitilde\textasciitilde{} \# TYPE
foo gauge foo 17.0 123 foo 18.0 456
\textasciitilde\textasciitilde\textasciitilde\textasciitilde{}

\hypertarget{counter-1}{%
\paragraph{Counter}\label{counter-1}}

The MetricPoint's Total Value Sample MetricName MUST have the suffix
"\_total``. If present the MetricPoint's Created Value Sample MetricName
MUST have the suffix''\_created".

An example with a Metric with no labels, and a MetricPoint with no
timestamp and no created:
\textasciitilde\textasciitilde\textasciitilde\textasciitilde{} \# TYPE
foo counter foo\_total 17.0
\textasciitilde\textasciitilde\textasciitilde\textasciitilde{}

An example with a Metric with no labels, and a MetricPoint with a
timestamp and no created:
\textasciitilde\textasciitilde\textasciitilde\textasciitilde{} \# TYPE
foo counter foo\_total 17.0 1520879607.789
\textasciitilde\textasciitilde\textasciitilde\textasciitilde{}

An example with a Metric with no labels, and a MetricPoint with no
timestamp and a created:
\textasciitilde\textasciitilde\textasciitilde\textasciitilde{} \# TYPE
foo counter foo\_total 17.0 foo\_created 1520430000.123
\textasciitilde\textasciitilde\textasciitilde\textasciitilde{}

An example with a Metric with no labels, and a MetricPoint with a
timestamp and a created:
\textasciitilde\textasciitilde\textasciitilde\textasciitilde{} \# TYPE
foo counter foo\_total 17.0 1520879607.789 foo\_created 1520430000.123
1520879607.789
\textasciitilde\textasciitilde\textasciitilde\textasciitilde{}

Exemplars MAY be attached to the MetricPoint's Total sample.

\hypertarget{stateset-1}{%
\paragraph{StateSet}\label{stateset-1}}

The Sample MetricName for the value of a MetricPoint for a MetricFamily
of type StateSet MUST NOT have a suffix.

StateSets MUST have one sample per State in the MetricPoint. Each
State's sample MUST have a label with the MetricFamily name as the label
name and the State name as the label value. The State sample's value
MUST be 1 if the State is true and MUST be 0 if the State is false.

An example with the states ``a'', ``bb'', and ``ccc'' in which only the
value b is enabled and the metric name is foo:

\begin{verbatim}
# TYPE foo stateset
foo{foo="a"} 0
foo{foo="bb"} 1
foo{foo="ccc"} 0
\end{verbatim}

An example of an ``entity'' label on the Metric:
\textasciitilde\textasciitilde\textasciitilde\textasciitilde{} \# TYPE
foo stateset foo\{entity=``controller'',foo=``a''\} 1.0
foo\{entity=``controller'',foo=``bb''\} 0.0
foo\{entity=``controller'',foo=``ccc''\} 0.0
foo\{entity=``replica'',foo=``a''\} 1.0
foo\{entity=``replica'',foo=``bb''\} 0.0
foo\{entity=``replica'',foo=``ccc''\} 1.0
\textasciitilde\textasciitilde\textasciitilde\textasciitilde{}

\hypertarget{info-1}{%
\paragraph{Info}\label{info-1}}

The Sample MetricName for the value of a MetricPoint for a MetricFamily
of type Info MUST have the suffix "\_info". The Sample value MUST always
be 1.

An example of a Metric with no labels, and one MetricPoint value with
``name'' and ``version'' labels:
\textasciitilde\textasciitilde\textasciitilde\textasciitilde{} \# TYPE
foo info foo\_info\{name=``pretty name'',version=``8.2.7''\} 1
\textasciitilde\textasciitilde\textasciitilde\textasciitilde{}

An example of a Metric with label ``entity'' and one MetricPoint value
with ``name'' and ``version'' labels:
\textasciitilde\textasciitilde\textasciitilde\textasciitilde{} \# TYPE
foo info foo\_info\{entity=``controller'',name=``pretty
name'',version=``8.2.7''\} 1.0
foo\_info\{entity=``replica'',name=``prettier name'',version=``8.1.9''\}
1.0 \textasciitilde\textasciitilde\textasciitilde\textasciitilde{}

Metric labels and MetricPoint value labels MAY be in any order.

\hypertarget{summary-1}{%
\paragraph{Summary}\label{summary-1}}

If present, the MetricPoint's Sum Value Sample MetricName MUST have the
suffix "\_sum``. If present, the MetricPoint's Count Value Sample
MetricName MUST have the suffix''\_count``. If present, the
MetricPoint's Created Value Sample MetricName MUST have the
suffix''\_created``. If present, the MetricPoint's Quantile Values MUST
specify the quantile measured using a label with a label name
of''quantile" and with a label value of the quantile measured.

An example of a Metric with no labels and a MetricPoint with Sum, Count
and Created values:
\textasciitilde\textasciitilde\textasciitilde\textasciitilde{} \# TYPE
foo summary foo\_count 17.0 foo\_sum 324789.3 foo\_created
1520430000.123
\textasciitilde\textasciitilde\textasciitilde\textasciitilde{}

An example of a Metric with no labels and a MetricPoint with two
quantiles:
\textasciitilde\textasciitilde\textasciitilde\textasciitilde{} \# TYPE
foo summary foo\{quantile=``0.95''\} 123.7 foo\{quantile=``0.99''\}
150.0 \textasciitilde\textasciitilde\textasciitilde\textasciitilde{}

Quantiles MAY be in any order.

\hypertarget{histogram-1}{%
\paragraph{Histogram}\label{histogram-1}}

The MetricPoint's Bucket Values Sample MetricNames MUST have the suffix
"\_bucket``. If present, the MetricPoint's Sum Value Sample MetricName
MUST have the suffix''\_sum``. If present, the MetricPoint's Created
Value Sample MetricName MUST have the suffix''\_created``. If and only
if a Sum Value is present in a MetricPoint, then the MetricPoint's +Inf
Bucket value MUST also appear in a Sample with a MetricName with the
suffix''\_count".

Buckets MUST be sorted in number increasing order of ``le'', and the
value of the ``le'' label MUST follow the rules for Canonical Numbers.

An example of a Metric with no labels and a MetricPoint with Sum, Count,
and Created values, and with 12 buckets. A wide and atypical but valid
variety of ``le'' values is shown on purpose:
\textasciitilde\textasciitilde\textasciitilde\textasciitilde{} \# TYPE
foo histogram foo\_bucket\{le=``0.0''\} 0 foo\_bucket\{le=``1e-05''\} 0
foo\_bucket\{le=``0.0001''\} 5 foo\_bucket\{le=``0.1''\} 8
foo\_bucket\{le=``1.0''\} 10 foo\_bucket\{le=``10.0''\} 11
foo\_bucket\{le=``100000.0''\} 11 foo\_bucket\{le=``1e+06''\} 15
foo\_bucket\{le=``1e+23''\} 16 foo\_bucket\{le=``1.1e+23''\} 17
foo\_bucket\{le=``+Inf''\} 17 foo\_count 17 foo\_sum 324789.3
foo\_created 1520430000.123
\textasciitilde\textasciitilde\textasciitilde\textasciitilde{}

\hypertarget{exemplars-1}{%
\subparagraph{Exemplars}\label{exemplars-1}}

Exemplars without Labels MUST represent an empty LabelSet as \{\}.

An example of Exemplars showcasing several valid cases: The ``0.01''
bucket has no Exemplar. The 0.1 bucket has an Exemplar with no Labels.
The 1 bucket has an Exemplar with one Label. The 10 bucket has an
Exemplar with a Label and a timestamp. In practice all buckets SHOULD
have the same style of Exemplars.
\textasciitilde\textasciitilde\textasciitilde\textasciitilde{} \# TYPE
foo histogram foo\_bucket\{le=``0.01''\} 0 foo\_bucket\{le=``0.1''\} 8
\# \{\} 0.054 foo\_bucket\{le=``1''\} 11 \#
\{trace\_id=``KOO5S4vxi0o''\} 0.67 foo\_bucket\{le=``10''\} 17 \#
\{trace\_id=``oHg5SJYRHA0''\} 9.8 1520879607.789
foo\_bucket\{le=``+Inf''\} 17 foo\_count 17 foo\_sum 324789.3
foo\_created 1520430000.123
\textasciitilde\textasciitilde\textasciitilde\textasciitilde{}

\hypertarget{gaugehistogram-1}{%
\paragraph{GaugeHistogram}\label{gaugehistogram-1}}

The MetricPoint's Bucket Values Sample MetricNames MUST have the suffix
"\_bucket``. If present, the MetricPoint's Sum Value Sample MetricName
MUST have the suffix''\_gsum``. If present, the MetricPoint's Created
Value Sample MetricName MUST have the suffix''\_created``. If and only
if a Sum Value is present in a MetricPoint, then the MetricPoint's +Inf
Bucket value MUST also appear in a Sample with a MetricName with the
suffix''\_gcount".

Buckets MUST be sorted in number increasing order of ``le'', and the
value of the ``le'' label MUST follow the rules for Canonical Numbers.

An example of a Metric with no labels, and one MetricPoint value with no
Exemplar with no Exemplars in the buckets:
\textasciitilde\textasciitilde\textasciitilde\textasciitilde{} \# TYPE
foo gaugehistogram foo\_bucket\{le=``0.01''\} 20.0
foo\_bucket\{le=``0.1''\} 25.0 foo\_bucket\{le=``1''\} 34.0
foo\_bucket\{le=``10''\} 34.0 foo\_bucket\{le=``+Inf''\} 42.0
foo\_gcount 42.0 foo\_gsum 3289.3 foo\_created 1520430000.123
\textasciitilde\textasciitilde\textasciitilde\textasciitilde{}

\hypertarget{unknown-1}{%
\paragraph{Unknown}\label{unknown-1}}

The sample metric name for the value of the MetricPoint for a
MetricFamily of type Unknown MUST NOT have a suffix.

An example with a Metric with no labels and a MetricPoint with no
timestamp:
\textasciitilde\textasciitilde\textasciitilde\textasciitilde{} \# TYPE
foo unknown foo 42.23
\textasciitilde\textasciitilde\textasciitilde\textasciitilde{}

\hypertarget{protobuf-format}{%
\subsection{Protobuf format}\label{protobuf-format}}

\hypertarget{overall-structure-1}{%
\subsubsection{Overall Structure}\label{overall-structure-1}}

Protobuf messages MUST be encoded in binary and MUST have
``application/openmetrics-protobuf; version=1.0.0'' as their content
type.

All payloads MUST be a single binary encoded MetricSet message, as
defined by the OpenMetrics protobuf schema.

\hypertarget{version}{%
\paragraph{Version}\label{version}}

The protobuf format MUST follow the proto3 version of the protocol
buffer language.

\hypertarget{strings-1}{%
\paragraph{Strings}\label{strings-1}}

All string fields MUST be UTF-8 encoded.

\hypertarget{timestamps-2}{%
\paragraph{Timestamps}\label{timestamps-2}}

Timestamp representations in the OpenMetrics protobuf schema MUST follow
the published google.protobuf.Timestamp {[}timestamp{]} message. The
timestamp message MUST be in Unix epoch seconds as an int64 and a
non-negative fraction of a second at nanosecond resolution as an int32
that counts forward from the seconds timestamp component. It MUST be
within 0 to 999,999,999 inclusive.

\hypertarget{protobuf-schema}{%
\subsubsection{Protobuf schema}\label{protobuf-schema}}

\begin{verbatim}
syntax = "proto3";

// The OpenMetrics protobuf schema which defines the protobuf wire
// format.
// Ensure to interpret "required" as semantically required for a valid
// message.
// All string fields MUST be UTF-8 encoded strings.
package openmetrics;

import "google/protobuf/timestamp.proto";

// The top-level container type that is encoded and sent over the wire.
message MetricSet {
  // Each MetricFamily has one or more MetricPoints for a single Metric.
  repeated MetricFamily metric_families = 1;
}

// One or more Metrics for a single MetricFamily, where each Metric
// has one or more MetricPoints.
message MetricFamily {
  // Required.
  string name = 1;

  // Optional.
  MetricType type = 2;

  // Optional.
  string unit = 3;

  // Optional.
  string help = 4;

  // Optional.
  repeated Metric metrics = 5;
}

// The type of a Metric.
enum MetricType {
  // Unknown must use unknown MetricPoint values.
  UNKNOWN = 0;
  // Gauge must use gauge MetricPoint values.
  GAUGE = 1;
  // Counter must use counter MetricPoint values.
  COUNTER = 2;
  // State set must use state set MetricPoint values.
  STATE_SET = 3;
  // Info must use info MetricPoint values.
  INFO = 4;
  // Histogram must use histogram value MetricPoint values.
  HISTOGRAM = 5;
  // Gauge histogram must use histogram value MetricPoint values.
  GAUGE_HISTOGRAM = 6;
  // Summary quantiles must use summary value MetricPoint values.
  SUMMARY = 7;
}

// A single metric with a unique set of labels within a metric family.
message Metric {
  // Optional.
  repeated Label labels = 1;

  // Optional.
  repeated MetricPoint metric_points = 2;
}

// A name-value pair. These are used in multiple places: identifying
// timeseries, value of INFO metrics, and exemplars in Histograms.
message Label {
  // Required.
  string name = 1;

  // Required.
  string value = 2;
}

// A MetricPoint in a Metric.
message MetricPoint {
  // Required.
  oneof value {
    UnknownValue unknown_value = 1;
    GaugeValue gauge_value = 2;
    CounterValue counter_value = 3;
    HistogramValue histogram_value = 4;
    StateSetValue state_set_value = 5;
    InfoValue info_value = 6;
    SummaryValue summary_value = 7;
  }

  // Optional.
  google.protobuf.Timestamp timestamp = 8;
}

// Value for UNKNOWN MetricPoint.
message UnknownValue {
  // Required.
  oneof value {
    double double_value = 1;
    int64 int_value = 2;
  }
}

// Value for GAUGE MetricPoint.
message GaugeValue {
  // Required.
  oneof value {
    double double_value = 1;
    int64 int_value = 2;
  }
}

// Value for COUNTER MetricPoint.
message CounterValue {
  // Required.
  oneof total {
    double double_value = 1;
    uint64 int_value = 2;
  }

  // The time values began being collected for this counter.
  // Optional.
  google.protobuf.Timestamp created = 3;

  // Optional.
  Exemplar exemplar = 4;
}

// Value for HISTOGRAM or GAUGE_HISTOGRAM MetricPoint.
message HistogramValue {
  // Optional.
  oneof sum {
    double double_value = 1;
    int64 int_value = 2;
  }

  // Optional.
  uint64 count = 3;

  // The time values began being collected for this histogram.
  // Optional.
  google.protobuf.Timestamp created = 4;

  // Optional.
  repeated Bucket buckets = 5;

  // Bucket is the number of values for a bucket in the histogram
  // with an optional exemplar.
  message Bucket {
    // Required.
    uint64 count = 1;

    // Optional.
    double upper_bound = 2;

    // Optional.
    Exemplar exemplar = 3;
  }
}

message Exemplar {
  // Required.
  double value = 1;

  // Optional.
  google.protobuf.Timestamp timestamp = 2;

  // Labels are additional information about the exemplar value
  // (e.g. trace id).
  // Optional.
  repeated Label label = 3;
}

// Value for STATE_SET MetricPoint.
message StateSetValue {
  // Optional.
  repeated State states = 1;

  message State {
    // Required.
    bool enabled = 1;

    // Required.
    string name = 2;
  }
}

// Value for INFO MetricPoint.
message InfoValue {
  // Optional.
  repeated Label info = 1;
}

// Value for SUMMARY MetricPoint.
message SummaryValue {
  // Optional.
  oneof sum {
    double double_value = 1;
    int64 int_value = 2;
  }

  // Optional.
  uint64 count = 2;

  // The time sum and count values began being collected for this
  // summary.
  // Optional.
  google.protobuf.Timestamp created = 3;

  // Optional.
  repeated Quantile quantile = 4;

  message Quantile {
    // Required.
    double quantile = 1;

    // Required.
    double value = 2;
  }
}
\end{verbatim}

\hypertarget{design-considerations}{%
\section{Design Considerations}\label{design-considerations}}

\hypertarget{scope}{%
\subsection{Scope}\label{scope}}

OpenMetrics is intended to provide telemetry for online systems. It runs
over protocols which do not provide hard or soft real time guarantees,
so it can not make any real time guarantees itself. Latency and jitter
properties of OpenMetrics are as imprecise as the underlying network,
operating systems, CPUs, and the like. It is sufficiently accurate for
aggregations to be used as a basis for decision-making, but not to
reflect individual events.

Systems of all sizes should be supported, from applications that receive
a few requests an hour up to monitoring bandwidth usage on a 400Gb
network port. Aggregation and analysis of transmitted telemetry should
be possible over arbitrary time periods.

It is intended to transport snapshots of state at the time of data
transmission at a regular cadence.

\hypertarget{out-of-scope}{%
\subsubsection{Out of scope}\label{out-of-scope}}

How ingestors discover which exposers exist, and vice-versa, is out of
scope for and thus not defined in this standard.

\hypertarget{extensions-and-improvements}{%
\subsection{Extensions and
Improvements}\label{extensions-and-improvements}}

This first version of OpenMetrics is based upon well established and de
facto standard Prometheus text format 0.0.4, deliberately without adding
major syntactic or semantic extensions, or optimisations on top of it.
For example no attempt has been made to make the text representation of
Histogram buckets more compact, relying on compression in the underlying
stack to deal with their repetitive nature.

This is a deliberate choice, so that the standard can take advantage of
the adoption and momentum of the existing user base. This ensures a
relatively easy transition from the Prometheus text format 0.0.4.

It also ensures that there is a basic standard which is easy to
implement. This can be built upon in future versions of the standard.
The intention is that future versions of the standard will always
require support for this 1.0 version, both syntactically and
semantically.

We want to allow monitoring systems to get usable information from an
OpenMetrics exposition without undue burden. If one were to strip away
all metadata and structure and just look at an OpenMetrics exposition as
an unordered set of samples that should be usable on its own. As such,
there are also no opaque binary types, such as sketches or t-digests
which could not be expressed as a mix of gauges and counters as they
would require custom parsing and handling.

This principle is applied consistently throughout the standard. For
example a MetricFamily's unit is duplicated in the name so that the unit
is available for systems that don't understand the unit metadata. The
``le'' label is a normal label value, rather than getting its own
special syntax, so that ingestors don't have to add special histogram
handling code to ingest them. As a further example, there are no
composite data types. For example, there is no geolocation type for
latitude/longitude as this can be done with separate gauge metrics.

\hypertarget{units-and-base-units}{%
\subsection{Units and Base Units}\label{units-and-base-units}}

For consistency across systems and to avoid confusion, units are largely
based on SI base units. Base units include seconds, bytes, joules,
grams, meters, ratios, volts, amperes, and celsius. Units should be
provided where they are applicable.

For example, having all duration metrics in seconds, there is no risk of
having to guess whether a given metric is nanoseconds, microseconds,
milliseconds, seconds, minutes, hours, days or weeks nor having to deal
with mixed units. By choosing unprefixed units, we avoid situations like
ones in which kilomilliseconds were the result of emergent behaviour of
complex systems.

As values can be floating point, sub-base-unit precision is built into
the standard.

Similarly, mixing bits and bytes is confusing, so bytes are chosen as
the base. While Kelvin is a better base unit in theory, in practice most
existing hardware exposes Celsius. Kilograms are the SI base unit,
however the kilo prefix is problematic so grams are chosen as the base
unit.

While base units SHOULD be used in all possible cases, Kelvin is a
well-established unit which MAY be used instead of Celsius for use cases
such as color or black body temperatures where a comparison between a
Celsius and Kelvin metric are unlikely.

Ratios are the base unit, not percentages. Where possible, raw data in
the form of gauges or counters for the given numerator and denominator
should be exposed. This has better mathematical properties for analysis
and aggregation in the ingestors.

Decibels are not a base unit as firstly, deci is a SI prefix and
secondly, bels are logarithmic. To expose signal/energy/power ratios
exposing the ratio directly would be better, or better still the raw
power/energy if possible. Floating point exponents are more than
sufficient to cover even extreme scientific uses. An electron volt
(\textasciitilde1e-19 J) all the way up to the energy emitted by a
supernova (\textasciitilde1e44 J) is 63 orders of magnitude, and a
64-bit floating point number can cover over 2000 orders of magnitude.

If non-base units can not be avoided and conversion is not feasible, the
actual unit should still be included in the metric name for clarity. For
example, joule is the base unit for both energy and power, as watts can
be expressed as a counter with a joule unit. In practice a given 3rd
party system may only expose watts, so a gauge expressed in watts would
be the only realistic choice in that case.

Not all MetricFamilies have units. For example a count of HTTP requests
wouldn't have a unit. Technically the unit would be HTTP requests, but
in that sense the entire MetricFamily name is the unit. Going to that
extreme would not be useful. The possibility of having good axes on
graphs in downstream systems for human consumption should always be kept
in mind.

\hypertarget{statelessness}{%
\subsection{Statelessness}\label{statelessness}}

The wire format defined by OpenMetrics is stateless across expositions.
What information has been exposed before MUST have no impact on future
expositions. Each exposition is a self-contained snapshot of the current
state of the exposer.

The same self-contained exposition MUST be provided to existing and new
ingestors.

A core design choice is that exposers MUST NOT exclude a metric merely
because it has had no recent changes, or observations. An exposer must
not make any assumptions about how often ingestors are consuming
expositions.

\hypertarget{exposition-across-time-and-metric-evolution}{%
\subsection{Exposition Across Time and Metric
Evolution}\label{exposition-across-time-and-metric-evolution}}

Metrics are most useful when their evolution over time can be analysed,
so accordingly expositions must make sense over time. Thus, it is not
sufficient for one single exposition on its own to be useful and valid.
Some changes to metric semantics can also break downstream users.

Parsers commonly optimize by caching previous results. Thus, changing
the order in which labels are exposed across expositions SHOULD be
avoided even though it is technically not breaking This also tends to
make writing unit tests for exposition easier.

Metrics and samples SHOULD NOT appear and disappear from exposition to
exposition, for example a counter is only useful if it has history. In
principle, a given Metric should be present in exposition from when the
process starts until the process terminates. It is often not possible to
know in advance what Metrics a MetricFamily will have over the lifetime
of a given process (e.g.~a label value of a latency histogram is a HTTP
path, which is provided by an end user at runtime), but once a
counter-like Metric is exposed it should continue to be exposed until
the process terminates. That a counter is not getting increments doesn't
invalidate that it still has its current value. There are cases where it
may make sense to stop exposing a given Metric; see the section on
Missing Data.

In general changing a MetricFamily's type, or adding or removing a label
from its Metrics will be breaking to ingestors.

A notable exception is that adding a label to the value of an Info
MetricPoints is not breaking. This is so that you can add additional
information to an existing Info MetricFamily where it makes sense to be,
rather than being forced to create a brand new info metric with an
additional label value. ingestor systems should ensure that they are
resilient to such additions.

Changing a MetricFamily's Help is not breaking. For values where it is
possible, switching between floats and ints is not breaking. Adding a
new state to a stateset is not breaking. Adding unit metadata where it
doesn't change the metric name is not breaking.

Histogram buckets SHOULD NOT change from exposition to exposition, as
this is likely to both cause performance issues and break ingestors and
cause. Similarly all expositions from any consistent binary and
environment of an application SHOULD have the same buckets for a given
Histogram MetricFamily, so that they can be aggregated by all ingestors
without ingestors having to implement histogram merging logic for
heterogeneous buckets. An exception might be occasional manual changes
to buckets which are considered breaking, but may be a valid tradeoff
when performance characteristics change due to a new software release.

Even if changes are not technically breaking, they still carry a cost.
For example frequent changes may cause performance issues for ingestors.
A Help string that varies from exposition to exposition may cause each
Help value to be stored. Frequently switching between int and float
values could prevent efficient compression.

\hypertarget{nan}{%
\subsection{NaN}\label{nan}}

NaN is a number like any other in OpenMetrics, usually resulting from a
division by zero such as for a summary quantile if there have been no
observations recently. NaN does not have any special meaning in
OpenMetrics, and in particular MUST NOT be used as a marker for missing
or otherwise bad data.

\hypertarget{missing-data}{%
\subsection{Missing Data}\label{missing-data}}

There are valid cases when data stops being present. For example a
filesystem can be unmounted and thus its Gauge Metric for free disk
space no longer exists. There is no special marker or signal for this
situation. Subsequent expositions simply do not include this Metric.

\hypertarget{exposition-performance}{%
\subsection{Exposition Performance}\label{exposition-performance}}

Metrics are only useful if they can be collected in reasonable time
frames. Metrics that take minutes to expose are not considered useful.

As a rule of thumb, exposition SHOULD take no more than a second.

Metrics from legacy systems serialized through OpenMetrics may take
longer. For this reason, no hard performance assumptions can be made.

Exposition SHOULD be of the most recent state. For example, a thread
serving the exposition request SHOULD NOT rely on cached values, to the
extent it is able to bypass any such caching

\hypertarget{concurrency}{%
\subsection{Concurrency}\label{concurrency}}

For high availability and ad-hoc access a common approach is to have
multiple ingestors. To support this, concurrent expositions MUST be
supported. All BCPs for concurrent systems SHOULD be followed, common
pitfalls include deadlocks, race conditions, and overly-coarse grained
locking preventing expositions progressing concurrently.

\hypertarget{metric-naming-and-namespaces}{%
\subsection{Metric Naming and
Namespaces}\label{metric-naming-and-namespaces}}

EDITOR'S NOTE: This section might be good for a BCP paper.

We aim for a balance between understandability, avoiding clashes, and
succinctness in the naming of metrics and label names. Names are
separated through underscores, so metric names end up being in
``snake\_case''.

To take an example ``http\_request\_seconds'' is succinct but would
clash between large numbers of applications, and it's also unclear
exactly what this metric is measuring. For example, it might be before
or after auth middleware in a complex system.

Metric names should indicate what piece of code they come from. So a
company called A Company Manufacturing Everything might prefix all
metrics in their code with ``acme\_'', and if they had a HTTP router
library measuring latency it might have a metric such as
``acme\_http\_router\_request\_seconds'' with a Help string indicating
that it is the overall latency.

It is not the aim to prevent all potential clashes across all
applications, as that would require heavy handed solutions such as a
global registry of metric namespaces or very long namespaces based on
DNS. Rather the aim is to keep to a lightweight informal approach, so
that for a given application that it is very unlikely that there is
clash across its constituent libraries.

Across a given deployment of a monitoring system as a whole the aim is
that clashes where the same metric name means different things are
uncommon. For example acme\_http\_router\_request\_seconds might end up
in hundreds of different applications developed by A Company
Manufacturing Everything, which is normal. If Another Corporation Making
Entities also used the metric name acme\_http\_router\_request\_seconds
in their HTTP router that's also fine. If applications from both
companies were being monitored by the same monitoring system the clash
is undesirable, but acceptable as no application is trying to expose
both names and no one target is trying to (incorrectly) expose the same
metric name twice. If an application wished to contain both My Example
Company's and Mega Exciting Company's HTTP router libraries that would
be a problem, and one of the metric names would need to be changed
somehow.

As a corollary, the more public a library is the better namespaced its
metric names should be to reduce the risk of such scenarios arising.
acme\_ is not a bad choice for internal use within a company, but these
companies might for example choose the prefixes acmeverything\_ or
acorpme\_ for code shared outside their company.

After namespacing by company or organisation, namespacing and naming
should continue by library/subsystem/application fractally as needed
such as the http\_router library above. The goal is that if you are
familiar with the overall structure of a codebase, you could make a good
guess at where the instrumentation for a given metric is given its
metric name.

For a common very well known existing piece of software, the name of the
software itself may be sufficiently distinguishing. For example bind\_
is probably sufficient for the DNS software, even though isc\_bind\_
would be the more usual naming.

Metric names prefixed by scrape\_ are used by ingestors to attach
information related to individual expositions, so should not be exposed
by applications directly. Metrics that have already been consumed and
passed through a general purpose monitoring system may include such
metric names on subsequent expositions. If an exposer wishes to provide
information about an individual exposition, a metric prefix such as
myexposer\_scrape\_ may be used. A common example is a gauge
myexposer\_scrape\_duration\_seconds for how long that exposition took
from the exposer's standpoint.

Within the Prometheus ecosystem a set of per-process metrics has emerged
that are consistent across all implementations, prefixed with process\_.
For example for open file ulimits the MetricFamiles process\_open\_fds
and process\_max\_fds gauges provide both the current and maximum value.
(These names are legacy, if such metrics were defined today they would
be more likely called process\_fds\_open and process\_fds\_limit). In
general it is very challengings to get names with identical semantics
like this, which is why different instrumentation should use different
names.

Avoid redundancy in metric names. Avoid substrings like ``metric'',
``timer'', ``stats'', ``counter'', ``total'', ``float64'' and so on - by
virtue of being a metric with a given type (and possibly unit) exposed
via OpenMetrics information like this is already implied so should not
be included explicitly. You should not include label names of a metric
in the metric name for the same reasons, and in addition subsequent
aggregation of the metric by a monitoring system could make such
information incorrect.

Avoid including implementation details from other layers of your
monitoring system in the metric names contained in your instrumentation.
For example a MetricFamily name should not contain the string
``openmetrics'' merely because it happens to be currently exposed via
OpenMetrics somewhere, or ``prometheus'' merely because your current
monitoring system is Prometheus.

\hypertarget{label-namespacing}{%
\subsection{Label Namespacing}\label{label-namespacing}}

For label names no explicit namespacing by company or library is
recommended, namespacing from the metric name is sufficient for this
when considered against the length increase of the label name. However
some minimal care to avoid common clashes is recommended.

There are label names such as region, zone, cluster, availability\_zone,
az, datacenter, dc, owner, customer, stage, service, team, job,
instance, environment, and env which are highly likely to clash with
labels used to identify targets which a general purpose monitoring
system may add. Try to avoid them, adding minimal namespacing may be
appropriate in these cases.

The label name ``type'' is highly generic and should be avoided. For
example for HTTP-related metrics ``method'' would be a better label name
if you were distinguishing between GET, POST, and PUT requests.

While there is metadata about metric names such as HELP, TYPE and UNIT
there is no metadata for label names. This is as it would be bloating
the format for little gain. Out-of-band documentation is one way for
exposers could present this their ingestors.

\hypertarget{metric-names-versus-labels}{%
\subsection{Metric Names versus
Labels}\label{metric-names-versus-labels}}

There are situations in which both using multiple Metrics within a
MetricFamily or multiple MetricFamilies seem to make sense. Summing or
averaging aMetricFamily should be meaningful even if it's not always
useful. For example, mixing voltage and fan speed is not meaningful.

As a reminder, OpenMetrics is built with the assumption that ingestors
can process and perform aggregations on data.

Exposing a total sum alongside other metrics is wrong, as this would
result in double-counting upon aggregation in downstream ingestors.
\textasciitilde\textasciitilde\textasciitilde\textasciitilde{}
wrong\_metric\{label=``a''\} 1 wrong\_metric\{label=``b''\} 6
wrong\_metric\{label=``total''\} 7
\textasciitilde\textasciitilde\textasciitilde\textasciitilde{}

Labels of a Metric should be to the minimum needed to ensure uniqueness
as every extra label is one more that users need to consider when
determining what Labels to work with downstream. Labels which could be
applied many MetricFamilies are candidates for being moved into \_info
metrics similar to database \{\{normalization\}\}. If virtually all
users of a Metric could be expected to want the additional label, it may
be a better trade-off to add it to all MetricFamilies. For example if
you had a MetricFamily relating to different SQL statements where
uniqueness was provided by a label containing a hash of the full SQL
statements, it would be okay to have another label with the first 500
characters of the SQL statement for human readability.

Experience has shown that downstream ingestors find it easier to work
with separate total and failure MetricFamiles rather than using
\{result=``success''\} and \{result=``failure''\} Labels within one
MetricFamily. Also it is usually better to expose separate read \& write
and send \& receive MetricFamiles as full duplex systems are common and
downstream ingestors are more likely to care about those values
separately than in aggregate.

All of this is not as easy as it may sound. It's an area where
experience and engineering trade-offs by domain-specific experts in both
exposition and the exposed system are required to find a good balance.
Metric and Label Name Characters

OpenMetrics builds on the existing widely adopted Prometheus text
exposition format and the ecosystem which formed around it. Backwards
compatibility is a core design goal. Expanding or contracting the set of
characters that are supported by the Prometheus text format would work
against that goal. Breaking backwards compatibility would have wider
implications than just the wire format. In particular, the query
languages created or adopted to work with data transmitted within the
Prometheus ecosystem rely on these precise character sets. Label values
support full UTF-8, so the format can represent multi-lingual metrics.

\hypertarget{types-of-metadata}{%
\subsection{Types of Metadata}\label{types-of-metadata}}

Metadata can come from different sources. Over the years, two main
sources have emerged. While they are often functionally the same, it
helps in understanding to talk about their conceptual differences.

``Target metadata'' is metadata commonly external to an exposer. Common
examples would be data coming from service discovery, a CMDB, or
similar, like information about a datacenter region, if a service is
part of a particular deployment, or production or testing. This can be
achieved by either the exposer or the ingestor adding labels to all
Metrics that capture this metadata. Doing this through the ingestor is
preferred as it is more flexible and carries less overhead. On
flexibility, the hardware maintenance team might care about which server
rack a machine is located in, whereas the database team using that same
machine might care that it contains replica number 2 of the production
database. On overhead, hardcoding or configuring this information needs
an additional distribution path.

``Exposer metadata'' is coming from within an exposer. Common examples
would be software version, compiler version, or Git commit SHA.

\hypertarget{supporting-target-metadata-in-both-push-based-and-pull-based-systems}{%
\subsubsection{Supporting Target Metadata in both Push-based and
Pull-based
Systems}\label{supporting-target-metadata-in-both-push-based-and-pull-based-systems}}

In push-based consumption, it is typical for the exposer to provide the
relevant target metadata to the ingestor. In pull-based consumption the
push-based approach could be taken, but more typically the ingestor
already knows the metadata of the target a-priori such as from a machine
database or service discovery system, and associates it with the metrics
as it consumes the exposition.

OpenMetrics is stateless and provides the same exposition to all
ingestors, which is in conflict with the push-style approach. In
addition the push-style approach would break pull-style ingestors, as
unwanted metadata would be exposed.

One approach would be for push-style ingestors to provide target
metadata based on operator configuration out-of-band, for example as a
HTTP header. While this would transport target metadata for push-style
ingestors, and is not precluded by this standard, it has the
disadvantage that even though pull-style ingestors should use their own
target metadata, it is still often useful to have access to the metadata
the exposer itself is aware of.

The preferred solution is to provide this target metadata as part of the
exposition, but in a way that does not impact on the exposition as a
whole. Info MetricFamilies are designed for this. An exposer may include
an Info MetricFamily called ``target'' with a single Metric with no
labels with the metadata. An example in the text format might be:
\textasciitilde\textasciitilde\textasciitilde\textasciitilde{} \# TYPE
target info \# HELP target Target metadata
target\_info\{env=``prod'',hostname=``myhost'',datacenter=``sdc'',region=``europe'',owner=``frontend''\}
1 \textasciitilde\textasciitilde\textasciitilde\textasciitilde{}

When an exposer is providing this metric for this purpose it SHOULD be
first in the exposition. This is for efficiency, so that ingestors
relying on it for target metadata don't have to buffer up the rest of
the exposition before applying business logic based on its content.

Exposers MUST NOT add target metadata labels to all Metrics from an
exposition, unless explicitly configured for a specific ingestor.
Exposers MUST NOT prefix MetricFamily names or otherwise vary
MetricFamily names based on target metadata. Generally, the same Label
should not appear on every Metric of an exposition, but there are rare
cases where this can be the result of emergent behaviour. Similarly all
MetricFamily names from an exposer may happen to share a prefix in very
small expositions. For example an application written in the Go language
by A Company Manufacturing Everything would likely include metrics with
prefixes of acme\_, go\_, process\_, and metric prefixes from any 3rd
party libraries in use.

Exposers can expose exposer metadata as Info MetricFamilies.

The above discussion is in the context of individual exposers. An
exposition from a general purpose monitoring system may contain metrics
from many individual targets, and thus may expose multiple target info
Metrics. The metrics may already have had target metadata added to them
as labels as part of ingestion. The metric names MUST NOT be varied
based on target metadata. For example it would be incorrect for all
metrics to end up being prefixed with staging\_ even if they all
originated from targets in a staging environment).

\hypertarget{client-calculations-and-derived-metrics}{%
\subsection{Client Calculations and Derived
Metrics}\label{client-calculations-and-derived-metrics}}

Exposers should leave any math or calculation up to ingestors. A notable
exception is the Summary quantile which is unfortunately required for
backwards compatibility. Exposition should be of raw values which are
useful over arbitrary time periods.

As an example, you should not expose a gauge with the average rate of
increase of a counter over the last 5 minutes. Letting the ingestor
calculate the increase over the data points they have consumed across
expositions has better mathematical properties and is more resilient to
scrape failures.

Another example is the average event size of a histogram/summary.
Exposing the average rate of increase of a counter since an application
started or since a Metric was created has the problems from the earlier
example and it also prevents aggregation.

Standard deviation also falls into this category. Exposing a sum of
squares as a counter would be the correct approach. It was not included
in this standard as a Histogram value because 64bit floating point
precision is not sufficient for this to work in practice. Due to the
squaring only half the 53bit mantissa would be available in terms of
precision. As an example a histogram observing 10k events per second
would lose precision within 2 hours. Using 64bit integers would be no
better due to the loss of the floating decimal point because a
nanosecond resolution integer typically tracking events of a second in
length would overflow after 19 observations. This design decision can be
revisited when 128bit floating point numbers become common.

Another example is to avoid exposing a request failure ratio, exposing
separate counters for failed requests and total requests instead.

\hypertarget{number-types}{%
\subsection{Number Types}\label{number-types}}

For a counter that was incremented a million times per second it would
take over a century to begin to lose precision with a float64 as it has
a 53 bit mantissa. Yet a 100 Gbps network interface's octet throughput
precision could begin to be lost with a float64 within around 20 hours.
While losing 1KB of precision over the course of years for a 100Gbps
network interface is unlikely to be a problem in practice, int64s are an
option for integral data with such a high throughput.

Summary quantiles must be float64, as they are estimates and thus
fundamentally inaccurate.

\hypertarget{exposing-timestamps}{%
\subsection{Exposing Timestamps}\label{exposing-timestamps}}

One of the core assumptions of OpenMetrics is that exposers expose the
most up to date snapshot of what they're exposing.

While there are limited use cases for attaching timestamps to exposed
data, these are very uncommon. Data which had timestamps previously
attached, in particular data which has been ingested into a general
purpose monitoring system may carry timestamps. Live or raw data should
not carry timestamps. It is valid to expose the same metric MetricPoint
value with the same timestamp across expositions, however it is invalid
to do so if the underlying metric is now missing.

Time synchronization is a hard problem and data should be internally
consistent in each system. As such, ingestors should be able to attach
the current timestamp from their perspective to data rather than based
on the system time of the exposer device.

With timestamped metrics it is not generally possible to detect the time
when a Metric went missing across expositions. However with
non-timestamped metrics the ingestor can use its own timestamp from the
exposition where the Metric is no longer present.

All of this is to say that, in general, MetricPoint timestamps should
not be exposed, as it should be up to the ingestor to apply their own
timestamps to samples they ingest.

\hypertarget{tracking-when-metrics-last-changed}{%
\subsubsection{Tracking When Metrics Last
Changed}\label{tracking-when-metrics-last-changed}}

Presume you had a counter my\_counter which was initialized, and then
later incremented by 1 at time 123. This would be a correct way to
expose it in the text format:
\textasciitilde\textasciitilde\textasciitilde\textasciitilde{} \# HELP
my\_counter Good increment example \# TYPE my\_counter counter
my\_counter\_total 1
\textasciitilde\textasciitilde\textasciitilde\textasciitilde{} As per
the parent section, ingestors should be free to attach their own
timestamps, so this would be incorrect:
\textasciitilde\textasciitilde\textasciitilde\textasciitilde{} \# HELP
my\_counter Bad increment example \# TYPE my\_counter counter
my\_counter\_total 1 123
\textasciitilde\textasciitilde\textasciitilde\textasciitilde{}

In case the specific time of the last change of a counter matters, this
would be the correct way:
\textasciitilde\textasciitilde\textasciitilde\textasciitilde{} \# HELP
my\_counter Good increment example \# TYPE my\_counter counter
my\_counter\_total 1 \# HELP
my\_counter\_last\_increment\_timestamp\_seconds When my\_counter was
last incremented \# TYPE
my\_counter\_last\_increment\_timestamp\_seconds gauge \# UNIT
my\_counter\_last\_increment\_timestamp\_seconds seconds
my\_counter\_last\_increment\_timestamp\_seconds 123
\textasciitilde\textasciitilde\textasciitilde\textasciitilde{}

By putting the timestamp of last change into its own Gauge as a value,
ingestors are free to attach their own timestamp to both Metrics.

Experience has shown that exposing absolute timestamps (epoch is
considered absolute here) is more robust than time elapsed, seconds
since, or similar. In either case, they would be gauges. For example
\textasciitilde\textasciitilde\textasciitilde\textasciitilde{} \# TYPE
my\_boot\_time\_seconds gauge \# HELP my\_boot\_time\_seconds Boot time
of the machine \# UNIT my\_boot\_time\_seconds seconds
my\_boot\_time\_seconds 1256060124
\textasciitilde\textasciitilde\textasciitilde\textasciitilde{}

Is better than
\textasciitilde\textasciitilde\textasciitilde\textasciitilde{} \# TYPE
my\_time\_since\_boot\_seconds gauge \# HELP
my\_time\_since\_boot\_seconds Time elapsed since machine booted \# UNIT
my\_time\_since\_boot\_seconds seconds my\_time\_since\_boot\_seconds
123 \textasciitilde\textasciitilde\textasciitilde\textasciitilde{}

Conversely, there are no best practice restrictions on exemplars
timestamps. Keep in mind that due to race conditions or time not being
perfectly synced across devices, that an exemplar timestamp may appear
to be slightly in the future relative to a ingestor's system clock or
other metrics from the same exposition. Similarly it is possible that a
"\_created" for a MetricPoint could appear to be slightly after an
exemplar or sample timestamp for that same MetricPoint.

Keep in mind that there are monitoring systems in common use which
support everything from nanosecond to second resolution, so having two
MetricPoints that have the same timestamp when truncated to second
resolution may cause an apparent duplicate in the ingestor. In this case
the MetricPoint with the earliest timestamp MUST be used.

\hypertarget{thresholds}{%
\subsection{Thresholds}\label{thresholds}}

Exposing desired bounds for a system can make sense, but proper care
needs to be taken. For values which are universally true, it can make
sense to emit Gauge metrics for such thresholds. For example, a data
center HVAC system knows the current measurements, the setpoints, and
the alert setpoints. It has a globally valid and correct view of the
desired system state. As a counter example, some thresholds can change
with scale, deployment model, or over time. A certain amount of CPU
usage may be acceptable in one setting and undesirable in another.
Aggregation of values can further change acceptable values. In such a
system, exposing bounds could be counter-productive.

For example a the maximum size of a queue may be exposed alongside the
number of items currently in the queue like:
\textasciitilde\textasciitilde\textasciitilde\textasciitilde{} \# HELP
acme\_notifications\_queue\_capacity The capacity of the notifications
queue. \# TYPE acme\_notifications\_queue\_capacity gauge
acme\_notifications\_queue\_capacity 10000 \# HELP
acme\_notifications\_queue\_length The number of notifications in the
queue. \# TYPE acme\_notifications\_queue\_length gauge
acme\_notifications\_queue\_length 42
\textasciitilde\textasciitilde\textasciitilde\textasciitilde{}

\hypertarget{size-limits}{%
\subsection{Size Limits}\label{size-limits}}

This standard does not prescribe any particular limits on the number of
samples exposed by a single exposition, the number of labels that may be
present, the number of states a stateset may have, the number of labels
in an info value, or metric name/label name/label value/help character
limits.

Specific limits run the risk of preventing reasonable use cases, for
example while a given exposition may have an appropriate number of
labels after passing through a general purpose monitoring system a few
target labels may have been added that would push it over the limit.
Specific limits on numbers such as these would also not capture where
the real costs are for general purpose monitoring systems. These
guidelines are thus both to aid exposers and ingestors in understanding
what is reasonable.

On the other hand, an exposition which is too large in some dimension
could cause significant performance problems compared to the benefit of
the metrics exposed. Thus some guidelines on the size of any single
exposition would be useful.

ingestors may choose to impose limits themselves, for in particular to
prevent attacks or outages. Still, ingestors need to consider reasonable
use cases and try not to disproportionately impact them. If any single
value/metric/exposition exceeds such limits then the whole exposition
must be rejected.

In general there are three things which impact the performance of a
general purpose monitoring system ingestion time series data: the number
of unique time series, the number of samples over time in those series,
and the number of unique strings such as metric names, label names,
label values, and HELP. ingestors can control how often they ingest, so
that aspect does not need further consideration.

The number of unique time series is roughly equivalent to the number of
non-comment lines in the text format. As of 2020, 10 million time series
in total is considered a large amount and is commonly the order of
magnitude of the upper bound of any single-instance ingestor. Any single
exposition should not go above 10k time series without due diligence.
One common consideration is horizontal scaling: What happens if you
scale your instance count by 1-2 orders of magnitude? Having a thousand
top-of-rack switches in a single deployment would have been hard to
imagine 30 years ago. If a target was a singleton (e.g.~exposing metrics
relating to an entire cluster) then several hundred thousand time series
may be reasonable. It is not the number of unique MetricFamilies or the
cardinality of individual labels/buckets/statesets that matters, it is
the total order of magnitude of the time series. 1,000 gauges with one
Metric each are as costly as a single gauge with 1,000 Metrics.

If all targets of a particular type are exposing the same set of time
series, then each additional targets' strings poses no incremental cost
to most reasonably modern monitoring systems. If however each target has
unique strings, there is such a cost. As an extreme example, a single
10k character metric name used by many targets is on its own very
unlikely to be a problem in practice. To the contrary, a thousand
targets each exposing a unique 36 character UUID is over three times as
expensive as that single 10k character metric name in terms of strings
to be stored assuming modern approaches. In addition, if these strings
change over time older strings will still need to be stored for at least
some time, incurring extra cost. Assuming the 10 million times series
from the last paragraph, 100MB of unique strings per hour might indicate
a use case for then the use case may be more like event logging, not
metric time series.

There is a hard 128 UTF-8 character limit on exemplar length, to prevent
misuse of the feature for tracing span data and other event logging.

\hypertarget{IANA}{%
\section{Security Considerations}\label{IANA}}

Implementors MAY choose to offer authentication, authorization, and
accounting; if they so choose, this SHOULD be handled outside of
OpenMetrics.

All exposer implementations SHOULD be able to secure their HTTP traffic
with TLS 1.2 or later. If an exposer implementation does not support
encryption, operators SHOULD use reverse proxies, firewalling, and/or
ACLs where feasible.

Metric exposition should be independent of production services exposed
to end users; as such, having a /metrics endpoint on ports like TCP/80,
TCP/443, TCP/8080, and TCP/8443 is generally discouraged for publicly
exposed services using OpenMetrics.

\hypertarget{Security}{%
\section{IANA Considerations}\label{Security}}

While currently most implementations of the Prometheus exposition format
are using non-IANA-registered ports from an informal registry at
\{\{PrometheusPorts\}\}, OpenMetrics can be found on a well-defined
port.

The port assigned by IANA for clients exposing data is \textless9099
requested for historical consistency\textgreater.

If more than one metric endpoint needs to be reachable at a common IP
address and port, operators might consider using a reverse proxy that
communicates with exposers over localhost addresses. To ease
multiplexing, endpoints SHOULD carry their own name in their path,
i.e.~\texttt{/node\_exporter/metrics}. Expositions SHOULD NOT be
combined into one exposition, for the reasons covered under ``Supporting
target metadata in both push-based and pull-based systems'' and to allow
for independent ingestion without a single point of failure.

OpenMetrics would like to register two MIME types,
\texttt{application/openmetrics-text} and
\texttt{application/openmetrics-proto}.

EDITOR'S NOTE: \texttt{application/openmetrics-text} is in active use
since 2018, \texttt{application/openmetrics-proto} is not yet in active
use.

EDITOR'S NOTE: We would like to thank Sumeer Bhola, but kramdown 2.x
does not support \texttt{Contributor:} any more so we will add this by
hand once consensus has been achieved.

\end{document}
