\PassOptionsToPackage{unicode=true,colorlinks=true,urlcolor=blue}{hyperref} % options for packages loaded elsewhere
\PassOptionsToPackage{hyphens}{url}
\documentclass[a4paper,12pt,notitlepage,twoside,openright]{article}

\usepackage{ifxetex}
\ifxetex{}
\else
\errmessage{Must be built with XeLaTeX}
\fi

\usepackage{amsmath}
\usepackage{fontspec}
\usepackage{fourier-otf} % erewhon-math
\setmonofont{iosevka-type-slab-regular}[
  Path=../common/iosevka-type-slab/,
  Extension=.ttf,
  BoldFont=iosevka-type-slab-bold,
  ItalicFont=iosevka-type-slab-italic,
  BoldItalicFont=iosevka-type-slab-bolditalic,
  Scale=MatchLowercase,
]

% Math
\usepackage[binary-units]{siunitx}

\usepackage{caption}
\usepackage{authblk}
\usepackage{enumitem}
\usepackage{footnote}

% Table
\usepackage{tabu}
\usepackage{longtable}
\usepackage{booktabs}
\usepackage{multirow}

% Verbatim & Source code
\usepackage{fancyvrb}
\usepackage{minted}

% Beauty
\usepackage[protrusion]{microtype}
\usepackage[defaultlines=3]{nowidow}
\usepackage{upquote}
\usepackage{parskip}
\usepackage[strict]{changepage}

\usepackage{hyperref}

% Graph
\usepackage{graphicx}
\usepackage{grffile}
\usepackage{tikz}

\hypersetup{
  bookmarksnumbered,
  pdfborder={0 0 0},
  pdfpagemode=UseNone,
  pdfstartview=FitH,
  breaklinks=true}
\urlstyle{same}  % don't use monospace font for urls

\usetikzlibrary{arrows.meta,calc,shapes.geometric,shapes.misc}

\setminted{
  autogobble,
  breakbytokenanywhere,
  breaklines,
  fontsize=\footnotesize,
}
\setmintedinline{
  autogobble,
  breakbytokenanywhere,
  breaklines,
  fontsize=\footnotesize,
}

\makeatletter
\def\maxwidth{\ifdim\Gin@nat@width>\linewidth\linewidth\else\Gin@nat@width\fi}
\def\maxheight{\ifdim\Gin@nat@height>\textheight\textheight\else\Gin@nat@height\fi}
\makeatother

% Scale images if necessary, so that they will not overflow the page
% margins by default, and it is still possible to overwrite the defaults
% using explicit options in \includegraphics[width, height, ...]{}
\setkeys{Gin}{width=\maxwidth,height=\maxheight,keepaspectratio}
\setlength{\emergencystretch}{3em}  % prevent overfull lines
\setcounter{secnumdepth}{3}

% Redefines (sub)paragraphs to behave more like sections
\ifx\paragraph\undefined\else
\let\oldparagraph\paragraph{}
\renewcommand{\paragraph}[1]{\oldparagraph{#1}\mbox{}}
\fi
\ifx\subparagraph\undefined\else
\let\oldsubparagraph\subparagraph{}
\renewcommand{\subparagraph}[1]{\oldsubparagraph{#1}\mbox{}}
\fi

% set default figure placement to htbp
\makeatletter
\def\fps@figure{htbp}
\makeatother


\title{The advantages of an LSM vs a B-Tree\\{\footnotesize \url{https://smalldatum.blogspot.com/2016/01/summary-of-advantages-of-lsm-vs-b-tree.html}}}
\author{Mark Callaghan }
\date{January 19, 2016}

\begin{document}
\maketitle

The \href{https://en.wikipedia.org/wiki/Log-structured_merge-tree}{log
structured merge tree} (LSM) is an interesting algorithm. It was
designed for disks yet has been shown to be effective on SSD. Not all
algorithms grow better with age. A long time ago I met one of the LSM
co-inventors, \href{http://www.cs.umb.edu/~poneil/}{Patrick O'Neil}, at
the first job I had after graduate school. He was advising my team on
\href{https://en.wikipedia.org/wiki/Bitmap_index}{bitmap indexes}. He
did early and interesting work on both topics. I went on to maintain
bitmap index code in the Oracle RDBMS for a few years. Patrick O'Neil
made my career more interesting.

Performance evaluations are hard. It took me a long time to get
expertise in InnoDB, then I repeated that for
\href{http://rocksdb.org/}{RocksDB}. Along the way I made many mistakes.
Advice on doing benchmarks for RocksDB is
\href{http://smalldatum.blogspot.com/2014/07/benchmarking-leveldb-family.html}{here}
and
\href{http://smalldatum.blogspot.com/2015/04/comparing-leveldb-and-rocksdb-take-2.html}{here}.

tl;dr --- the MyRocks advantage is better compression and less
write-amplification

\hypertarget{the-myrocks-advantage}{%
\section{The MyRocks Advantage}\label{the-myrocks-advantage}}

There are many benefits of the MyRocks LSM relative to a B-Tree. If you
want to try MyRocks the source is
\href{https://github.com/facebook/mysql-5.6}{on github}, there is
\href{https://github.com/facebook/mysql-5.6/wiki}{a wiki}~with notes on
\href{https://github.com/facebook/mysql-5.6/wiki/Getting-Started-with-MyRocks}{building},~\href{https://github.com/facebook/mysql-5.6/wiki/my.cnf-tuning}{my.cnf}~and
more details on the
\href{https://github.com/facebook/mysql-5.6/wiki/MyRocks-advantages-over-InnoDB}{MyRocks
advantage}. Track down Yoshinori at the
\href{https://www.percona.com/live/data-performance-conference-2016/sessions/myrocks-deep-dive-flash-optimized-lsm-database-mysql-and-its-use-case-facebook}{MyRocks
tutorial} at Percona Live or his
\href{https://fosdem.org/2016/schedule/event/facebook/}{talk at FOSDEM}
to learn more. The advantages include:

\begin{description}
\item
  [better compression] on real and synthetic workloads we measure 2X
  better compression with MyRocks compared to compressed InnoDB. A
  B-Tree wastes space when pages fragment. An LSM doesn't fragment.
  While an LSM can waste space from old versions of rows, with leveled
  compaction the overhead is \textasciitilde10\% of the database size
  compared to between 33\% and 50\% for a fragmented B-Tree and I have
  confirmed such fragmentation in production. MyRocks also uses less
  space for
  \href{http://dev.mysql.com/doc/refman/5.7/en/innodb-multi-versioning.html}{per-row
  metadata than InnoDB}. Finally, InnoDB disk pages have a fixed size
  and more space is lost from rounding up the compressed page output
  (maybe 5KB) to the fixed page size (maybe 8KB).
\item
  [no page reads for secondary index maintenance] MyRocks does not read
  the old version of a secondary index entry during insert, update or
  delete maintenance for a non-unique secondary index. So this is a
  write-only operation compared to read-page, modify-page, write-page
  for a B-Tree. This greatly reduces the random IO overhead for some
  workloads (benchmark results to be shared soon).
\item
  [less IO capacity used for persisting changes] a B-Tree does more \&
  smaller writes to persist a database while the MyRocks LSM does fewer
  \& larger writes. Not only is less random IO capacity used, but the
  MyRocks LSM writes less data to storage per transaction compared to
  InnoDB
  \href{http://smalldatum.blogspot.com/2016/01/rocksdb-vs-innodb-via-linkbench.html}{as
  seen in the Linkbench results} (look for wKB). In that case the
  difference was \textasciitilde10X.
\item
  [less write amplification from flash GC] the write pattern from the
  MyRocks LSM is friendlier to
  \href{https://en.wikipedia.org/wiki/Write_amplification\#BG-GC}{flash
  GC} compared to InnoDB. This leads to lower write-amplification so
  that InnoDB was writing up to 18X more data per transaction compared
  to MyRocks
  \href{http://smalldatum.blogspot.com/2016/01/even-more-write-amplification-when.html}{as
  seen in the Linkbench results}.
\item
  [simpler code] RocksDB is much simpler than InnoDB. I have spent years
  with both --- debugging and reading code. MyRocks still has mutex
  contention, but the problem is easier to fix because the engine is
  simpler. New features are easier to add because the engine is simpler.
  We are moving fast to make MyRocks better.
\end{description}

\hypertarget{the-myrocks-disadvantage}{%
\section{The MyRocks Disadvantage}\label{the-myrocks-disadvantage}}

There is no free lunch with database algorithms. From theory
\href{http://smalldatum.blogspot.com/2015/11/read-write-space-amplification-b-tree.html}{we
can expect} better efficiency on writes and worse efficiency from reads
with an LSM compared to a B-Tree. But what does that mean in practice?
This will take time to figure out and I try to include IO efficiency
metrics in my benchmark results that include disk reads, disk KB read
and disk KB written per transaction to understand the differences. But
at this point the IO efficiency improvements I see from MyRocks are much
greater than the IO efficiency losses.

MyRocks is not as feature complete as InnoDB. While the team is moving
fast the MySQL storage engine API is hard and getting harder. Missing
features include text, geo, native partitioning, online DDL and XA
recovery on the master (keep binlog \& RocksDB in sync on master). I am
not sure that text \& geo will ever get implemented. I hope to see XA
recovery and partial support for online DDL this year.

MyRocks is not proven like InnoDB. This takes time, a variety of
workloads and robust QA. It is getting done but I am not aware of
anybody running MyRocks in production today.

An LSM might use more CPU per query because more data structures can be
checked to satisfy a query. This can increase the number of CPU cache
misses and key comparisons done per query. It is likely to be a bigger
deal on CPU-bound workloads and my focus has been on IO-bound. One
example of the difference is the tuning
\href{http://smalldatum.blogspot.com/2016/01/faster-loads-for-myrocks.html}{that
was required} to make the MyRocks load performance match InnoDB when
using fast storage. Although in that case the problem might be more of
an implementation artifact than something fundamental to the LSM
algorithm.

An LSM like MyRocks can also suffer from the range read penalty. An LSM
might have to check many files to get data for a point or range query.
Most LSMs, including MyRocks, use bloom filters to reduce the number of
files to be checked during a point query. Bloom filters can eliminate
all disk reads for queries of keys that don't exist. We have had some
workloads on InnoDB for which 30\% of the disk reads were for keys that
don't exist.

Bloom filters can't be used on pure range queries. The most frequent
query in Linkbench, and the real workload it models, is a short range
query. Fortunately this query includes an equality predicate on the
leading two columns of a covering secondary index
(\href{https://github.com/mdcallag/linkbench}{see the id1\_type index in
the Linkbench schema}) and RocksDB has a feature to create the bloom
filter on a prefix of the key (innovation!).

Even when the prefix bloom filter can't be used the range read penalty
can be overstated. It is important to distinguish between logical and
physical reads. A logical read means that data is read from a file. A
physical read means that a logical read had to use the storage device to
get the data. Otherwise a logical read is satisfied from cache. A range
query with an LSM can require more logical reads than a B-Tree. But the
important question is whether it requires more physical reads.

\end{document}
