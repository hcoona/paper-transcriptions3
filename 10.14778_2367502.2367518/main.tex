\PassOptionsToPackage{unicode=true}{hyperref} % options for packages loaded elsewhere
\PassOptionsToPackage{hyphens}{url}
\documentclass[a4paper,12pt,notitlepage,twoside,openright]{article}

\usepackage{ifxetex}
\ifxetex{}
\else
\errmessage{Must be built with XeLaTeX}
\fi

\usepackage{amsmath}
\usepackage{fontspec}
\usepackage{fourier-otf} % erewhon-math
\setmonofont{iosevka-type-slab-regular}[
  Path=../common/iosevka-type-slab/,
  Extension=.ttf,
  BoldFont=iosevka-type-slab-bold,
  ItalicFont=iosevka-type-slab-italic,
  BoldItalicFont=iosevka-type-slab-bolditalic,
  Scale=MatchLowercase,
]

% Math
\usepackage[binary-units]{siunitx}

\usepackage{caption}
\usepackage{authblk}
\usepackage{enumitem}
\usepackage{footnote}

% Table
\usepackage{tabu}
\usepackage{longtable}
\usepackage{booktabs}
\usepackage{multirow}

% Verbatim & Source code
\usepackage{fancyvrb}
\usepackage{minted}

% Beauty
\usepackage[protrusion]{microtype}
\usepackage[defaultlines=3]{nowidow}
\usepackage{upquote}
\usepackage{parskip}
\usepackage[strict]{changepage}

\usepackage{hyperref}

% Graph
\usepackage{graphicx}
\usepackage{grffile}
\usepackage{tikz}

\hypersetup{
  bookmarksnumbered,
  pdfborder={0 0 0},
  pdfpagemode=UseNone,
  pdfstartview=FitH,
  breaklinks=true}
\urlstyle{same}  % don't use monospace font for urls

\usetikzlibrary{arrows.meta,calc,shapes.geometric,shapes.misc}

\setminted{
  autogobble,
  breakbytokenanywhere,
  breaklines,
  fontsize=\footnotesize,
}
\setmintedinline{
  autogobble,
  breakbytokenanywhere,
  breaklines,
  fontsize=\footnotesize,
}

\makeatletter
\def\maxwidth{\ifdim\Gin@nat@width>\linewidth\linewidth\else\Gin@nat@width\fi}
\def\maxheight{\ifdim\Gin@nat@height>\textheight\textheight\else\Gin@nat@height\fi}
\makeatother

% Scale images if necessary, so that they will not overflow the page
% margins by default, and it is still possible to overwrite the defaults
% using explicit options in \includegraphics[width, height, ...]{}
\setkeys{Gin}{width=\maxwidth,height=\maxheight,keepaspectratio}
\setlength{\emergencystretch}{3em}  % prevent overfull lines
\setcounter{secnumdepth}{3}

% Redefines (sub)paragraphs to behave more like sections
\ifx\paragraph\undefined\else
\let\oldparagraph\paragraph{}
\renewcommand{\paragraph}[1]{\oldparagraph{#1}\mbox{}}
\fi
\ifx\subparagraph\undefined\else
\let\oldsubparagraph\subparagraph{}
\renewcommand{\subparagraph}[1]{\oldsubparagraph{#1}\mbox{}}
\fi

% set default figure placement to htbp
\makeatletter
\def\fps@figure{htbp}
\makeatother


\title{The Vertica Analytic Database: C-Store 7 Years Later}
\author{Andrew Lamb, Matt Fuller, Ramakrishna Varadarajan,
Nga Tran, Ben Vandiver, Lyric Doshi, Chuck Bear}
\date{August 2012}

\begin{document}
\maketitle

\begin{abstract}

This paper describes the system architecture of the Vertica Analytic
Database (Vertica), a commercialization of the design of the C-Store
research prototype. Vertica demonstrates a modern commercial RDBMS
system that presents a classical relational interface while at the same
time achieving the high performance expected from modern ``web scale''
analytic systems by making appropriate architectural choices. Vertica is
also an instructive lesson in how academic systems research can be
directly commercialized into a successful product.

\end{abstract}

\hypertarget{introduction}{%
\section{INTRODUCTION}\label{introduction}}

The Vertica Analytic Database (Vertica) is a distributed\footnote{We use
  the term \emph{distributed database} to mean a shared-nothing,
  scale-out system as opposed to a set of locally autonomous (own
  catalog, security settings, etc.) RDBMS systems.}, massively parallel
RDBMS system that commercializes the ideas of the C-Store{[}21{]}
project. It is one of the few new commercial relational database systems
that is widely used in business critical systems. At the time of this
writing, there are over 500 production deployments of Vertica, at least
three of which are substantially over a petabyte in size. Despite the
recent interest in academia and industry about so called ``NoSQL''
systems {[}13, 19, 12{]}, the C-Store project anticipated the need for
web scale distributed processing and these new NoSQL systems use many of
the same techniques found in C-Store and other relational systems. Like
any language or system, SQL is not perfect, but it has been a
transformational abstraction for application developers, freeing them
from many implementation details of storing and finding their data to
focus their efforts on using the information effectively.

Vertica's experience in the marketplace and the emergence of other
technologies such as Hive {[}7{]} and Tenzing {[}9{]} validate that the
problem is \textbf{not} SQL. Rather, the unsuitability of legacy RDBMs
systems for massive analytic workloads is that they were designed for
transactional workloads on late-model computer hardware 40 years ago.
Vertica is designed for analytic workloads on modern hardware and its
success proves the commercial and technical viability of large scale
distributed databases which offer fully ACID transactions yet
efficiently process petabytes of structured data. This main
contributions of this paper are:

\begin{enumerate}
\def\labelenumi{\arabic{enumi}.}
\item
  An overview of the architecture of the Vertica Analytic Database,
  focusing on deviations from C-Store.
\item
  Implementation and deployment lessons that led to those differences.
\item
  Observations on real-world experiences that can inform future research
  directions for large scale analytic systems.
\end{enumerate}

We hope that this paper contributes a perspective on commercializing
research projects and emphasizes the contributions of the database
research community towards large scale distributed computing.

\hypertarget{background}{%
\section{BACKGROUND}\label{background}}

Vertica was the direct result of commercializing the C-Store research
system. Vertica Systems was founded in 2005 by several of the C-Store
authors and was acquired in 2011 by Hewlett-Packard (HP) after several
years of commercial development. {[}2{]}. Significant research and
development efforts continue on the Vertica Analytic Database.

\hypertarget{design-overview}{%
\subsection{Design Overview}\label{design-overview}}

\hypertarget{design-goals}{%
\subsubsection{Design Goals}\label{design-goals}}

Vertica utilizes many (but not all) of the ideas of C-Store, but none of
the code from the research prototype. Vertica was explicitly designed
for analytic workloads rather than for transactional workloads.

\textbf{Transactional} workloads are characterized by a large number of
transactions per second (e.g. thousands) where each transaction involves
a handful of tuples. Most of the transactions take the form of single
row insertions or modifications to existing rows. Examples are inserting
a new sales record and updating a bank account balance.

\textbf{Analytic} workloads are characterized by smaller transaction
volume (e.g. tens per second), but each transaction examines a
significant fraction of the tuples in a table. Examples are aggregating
sales data across time and geography dimensions and analyzing the
behavior of distinct users on a web site.

As typical table sizes, even for small companies, have grown to millions
and billions of rows, the difference between the transactional and
analytic workloads has been increasing. As others have pointed out
{[}26{]}, it is possible to exceed the performance of existing
one-size-fits-all systems by orders of magnitudes by focusing
specifically on analytic workloads.

Vertica is a distributed system designed for modern commodity hardware.
In 2012, this means x86 64 servers, Linux and commodity gigabit Ethernet
interconnects. Like C-Store, Vertica is designed from the ground up to
be a distributed database. When nodes are added to the database, the
system's performance should scale linearly. To achieve such scaling,
using a shared disk (often referred to as network-attached storage) is
not acceptable as it almost immediately becomes a bottleneck. Also, the
storage system's data placement, the optimizer and execution engine
should avoid consuming large amounts of network bandwidth to prevent the
interconnect from becoming the bottleneck.

In analytic workloads, while transactions per second is relatively low
by Online-Transaction-Processing (OLTP) standards, rows processed per
second is incredibly high. This applies not only to querying but also to
loading the data into the database. Special care must be taken to
support high ingest rates. If it takes days to load your data, a
superfast analytic query engine will be of limited use. Bulk load must
be fast and must not prevent or unduly slow down queries ongoing in
parallel.

For a real production system, all operations must be ``online''. Vertica
can not require stopping or suspending queries for storage management or
maintenance tasks. Vertica also aims to ease the management burden by
making ease of use an explicit goal. We trade CPU cycles (which are
cheap) for human wizard cycles (which are expensive) whenever possible.
This takes many forms such as minimizing complex networking and disk
setup, limiting performance tuning required, and automating physical
design and management. All vendors claim management ease, though success
in the real world is mixed.

Vertica was written entirely from scratch with the following exceptions,
which are based on the PostgreSQL {[}5{]} implementation:

\begin{enumerate}
\def\labelenumi{\arabic{enumi}.}
\item
  The SQL parser, semantic analyzer, and standard SQL rewrites.
\item
  Early versions of the standard client libraries, such as JDBC and ODBC
  and the command line interface.
\end{enumerate}

All other components were custom written from the ground up. While this
choice required significant engineering effort and delayed the initial
introduction of Vertica to the market, it means Vertica is positioned to
take full advantage of its architecture.

\hypertarget{data-model}{%
\section{DATA MODEL}\label{data-model}}

Like all SQL based systems, Vertica models user data as tables of
columns (attributes), though the data is not physically arranged in this
manner. Vertica supports the full range of standard INSERT, UPDATE,
DELETE constructs for logically inserting and modifying data as well as
a bulk loader and full SQL support for querying.

\hypertarget{projections}{%
\subsection{Projections}\label{projections}}

Like C-Store, Vertica physically organizes table data into
\emph{projections}, which are sorted subsets of the attributes of a
table. Any number of projections with different sort orders and subsets
of the table columns are allowed. Because Vertica is a column store and
has been optimized so heavily for performance, it is \textbf{NOT}
required to have one projection for each predicate that a user might
restrict. In practice, most customers have one super projection
(described below) and between zero and three narrow, non-super
projections.

Each projection has a specific sort order on which the data is totally
sorted as shown in Figure 1. Projections may be thought of as a
restricted form of materialized view {[}11, 25{]}. They differ from
standard materialized views because they are the only physical data
structure in Vertica, rather than auxiliary indexes. Classical
materialized views also contain aggregation, joins and other query
constructs that Vertica projections do not. Experience has shown that
the maintenance cost and additional implementation complexity of
maintaining materialized views with aggregation and filtering is not
practical in real world distributed systems. Vertica does support a
special case to physically denormalize certain joins within prejoin
projections as described below.

\hypertarget{join-indexes}{%
\subsection{Join Indexes}\label{join-indexes}}

C-Store uses a data structure called a \emph{join index} to reconstitute
tuples from the original table using different partial projections.
While the authors expected only a few join indices in practice, Vertica
does not implement join indices at all, instead requiring at least one
\emph{super projection} containing every column of the anchoring table.
In practice and experiments with early prototypes, we found that the
costs of using join indices far outweighed their benefits. Join indices
were complex to implement and the runtime cost of reconstructing full
tuples during distributed query execution was very high. In addition,
explicitly storing row ids consumed significant disk space for large
tables. The excellent compression achieved by our columnar design helped
keep the cost of super projections to a minimum and we have no plans to
lift the super projection requirement.

\hypertarget{prejoin-projections}{%
\subsection{Prejoin Projections}\label{prejoin-projections}}

Like C-Store, Vertica supports \emph{prejoin} projections which permit
joining the projection's anchor table with any number of dimension
tables via N:1 joins. This permits a normalized logical schema, while
allowing the physical storage to be denormalized. The cost of storing
physically denormalized data is much less than in traditional systems
because of the available encoding and compression. Prejoin projections
are not used as often in practice as we expected. This is because
Vertica's execution engine handles joins with small dimension tables
very well (using highly optimized hash and merge join algorithms), so
the benefits of a prejoin for query execution are not as significant as
we initially predicted. In the case of joins involving a fact and a
large dimension table or two large fact tables where the join cost is
high, most customers are unwilling to slow down bulk loads to optimize
such joins. In addition, joins during load offer fewer optimization
opportunities than joins during query because the database knows nothing
apriori about the data in the load stream.

\textbf{Figure 1: Relationship between tables and projections. The}
\emph{sales} \textbf{tables has 2 projections: (1) A super projection,
sorted by date, segmented by} \emph{HASH}(\emph{sale id}) \textbf{and
(2) A non-super projection containing only} (\emph{cust,price})
\textbf{attributes, sorted by} \emph{cust}\textbf{, segmented by}
\emph{HASH}(\emph{cust})\textbf{.}

\hypertarget{encoding-and-compression}{%
\subsection{Encoding and Compression}\label{encoding-and-compression}}

Each column in each projection has a specific encoding scheme. Vertica
implements a different set of encoding schemes than C-store, some of
which are enumerated in Section 3.4.1. Different columns in a projection
may have different encodings, and the same column may have a different
encoding in each projection in which it appears. The same encoding
schemes in Vertica are often far more effective than in other systems
because of Vertica's sorted physical storage. A comparative illustration
can be found in Section 8.2.

\hypertarget{encoding-types}{%
\subsubsection{Encoding Types}\label{encoding-types}}

\begin{enumerate}
\def\labelenumi{\arabic{enumi}.}
\item
  \textbf{Auto}: The system automatically picks the most advantageous
  encoding type based on properties of the data itself. This type is the
  default and is used when insufficient usage examples are known.
\item
  \textbf{RLE}: Replaces sequences of identical values with a single
  pair that contains the value and number of occurrences. This type is
  best for low cardinality columns that are sorted.
\item
  \textbf{Delta Value}: Data is recorded as a difference from the
  smallest value in a data block. This type is best used for
  many-valued, unsorted integer or integer-based columns.
\item
  \textbf{Block Dictionary}: Within a data block, distinct column values
  are stored in a dictionary and actual values are replaced with
  references to the dictionary. This type is best for few-valued,
  unsorted columns such as stock prices.
\item
  \textbf{Compressed Delta Range}: Stores each value as a delta from the
  previous one. This type is ideal for many-valued float columns that
  are either sorted or confined to a range.
\item
  \textbf{Compressed Common Delta}: Builds a dictionary of all the
  deltas in the block and then stores indexes into the dictionary using
  entropy coding. This type is best for sorted data with predictable
  sequences and occasional sequence breaks. For example, timestamps
  recorded at periodic intervals or primary keys.
\end{enumerate}

\hypertarget{partitioning}{%
\subsection{Partitioning}\label{partitioning}}

C-Store mentions intra-node ``Horizontal Partitioning'' as a way to
improve performance by increasing parallelism within a single node. In
contrast, Vertica's execution engine, as described in Section 6.1,
obtains intra-node parallelism without requiring separation of the
on-disk physical structures. It does so by dividing each on-disk
structure into logical regions at runtime and processing the regions in
parallel. Despite the automatic parallelization, Vertica does provide a
way to keep data segregated in physical structures based on value
through a simple syntax:

CREATE TABLE ... PARTITION BY \textless expr\textgreater.

This instructs Vertica to maintain physical storage so that all tuples
within a ROS container\footnote{ROS and ROS containers are explained in
  section 3.7} evaluate to the same distinct value of the partition
expression. Partition expressions are most often date related such as
extracting the month and year from a timestamp.

The first reason for partitioning, as in other RDBMS systems, is fast
bulk deletion. It is common to keep data separated into files based on a
combination of month and year, so removing a specific month of data from
the system is as simple as deleting files from a filesystem. This
arrangement is very fast and reclaims storage immediately. The
alternative, if the data is not pre-separated, requires searching all
physical files for rows matching the delete predicate and adding delete
vectors (further explained in Section 3.7.1) for deleted records. It is
much slower to find and mark deleted records than deleting files, and
this procedure actually increases storage requirements and degrades
query performance until the tuple mover's next merge-out operation is
performed (see Section 4). Because bulk deletion is only fast if all
projections are partitioned the same way, partitioning is specified at
the table level and not the projection level.

The second way Vertica takes advantage of physical storage separation is
improving query performance. As described here {[}22{]}, Vertica stores
the minimum and maximum values of the column data in each ROS to quickly
prune containers at plan time that can not possibly pass query
predicates. Partitioning makes this technique more effective by
preventing intermixed column values in the same ROS.

\hypertarget{segmentation-cluster-distribution}{%
\subsection{Segmentation: Cluster
Distribution}\label{segmentation-cluster-distribution}}

C-Store separates physical storage into segments based on the first
column in the sort order of a projection and the authors briefly mention
their plan to design a storage allocator for assigning segments to
nodes. Vertica has a fully implemented distributed storage system that
assigns tuples to specific computation nodes. We call this internode
(splitting tuples among nodes) horizontal partitioning
\emph{segmentation} to distinguish it from the intra-node (segregating
tuples within nodes) partitioning described in Section 3.5. Segmentation
is specified for each projection, which can be (and most often is)
different from the sort order. Projection segmentation provides a
deterministic mapping of tuple value to node and thus enables many
important optimizations. For example, Vertica uses segmentation to
perform fully local distributed joins and efficient distributed
aggregations, which is particularly effective for the computation of
high-cardinality distinct aggregates

Projections can either be \emph{replicated} or \emph{segmented} on some
or all cluster nodes. As the name implies, a replicated projection
stores a copy of each tuple on every projection node. Segmented
projections store each tuple on exactly one specific projection node.
The node on which the tuple is stored is determined by a segmentation
clause in the projection definition: CREATE PROJECTION ... SEGMENTED BY
\textless expr\textgreater{} where \textless expr\textgreater{} is an
arbitrary \footnote{While it is possible to manually specify
  segmentation, most users let the Database Designer determine an
  appropriate segmentation expression for projections.} integral
expression.

Nodes are assigned to store ranges of segmentation expression values,
starting with the following mapping where \emph{C\textsubscript{MAX }}is
the maximum integral value (2\textsuperscript{64} in Vertica).

This is a classic ring style segmentation scheme. The most common choice
is
\emph{HASH}(\emph{col}\textsubscript{1}\emph{..col\textsubscript{n}}),
where \emph{col\textsubscript{i }}is some suitably high cardinality
column with relatively even value distributions, commonly a primary key
column. Within each node, in addition to the specified partitioning,
Vertica keeps tuples physically segregated into ``local segments'' to
facilitate online expansion and contractions of the cluster. When nodes
are added or removed, data is quickly transferred by assigning one or
more of the existing local segments to a new node and transferring the
segment data wholesale in its native format, without any rearrangement
or splitting necessary.

\hypertarget{read-and-write-optimized-stores}{%
\subsection{Read and Write Optimized
Stores}\label{read-and-write-optimized-stores}}

Like C-Store, Vertica has a Read Optimized Store (ROS) and a Write
Optimized Store (WOS). Data in the ROS is physically stored in multiple
\emph{ROS containers} on a standard file system. Each ROS container
logically contains some number of complete tuples sorted by the
projection's sort order, stored as a pair of files per column. Vertica
is a true column store --- column files may be independently retrieved as
the storage is physically separate. Vertica stores two files per column
within a ROS container: one with the actual column data, and one with a
\emph{position index}. Data is identified within each ROS container by a
\emph{position} which is simply its ordinal position within the file.
Positions are implicit and are never stored explicitly. The position
index is approximately \(\frac{1}{1000}\) the
size of the raw column data and stores metadata per disk block such as
start position, minimum value and maximum value that improve the speed
of the execution engine and permits fast tuple reconstruction. Unlike
C-Store, this index structure does not utilize a B-Tree as the ROS
containers are never modified. Complete tuples are reconstructed by
fetching values with the same position from each column file within a
ROS container. Vertica also supports grouping multiple columns together
into the same file when writing to a ROS container. This hybrid
row-column storage mode is very rarely used in practice because of the
performance and compression penalty it exacts.

Data in the WOS is solely in memory, where column or row orientation
doesn't matter. The WOS's primary purpose is to buffer small data
inserts, deletes and updates so that writes to physical structures
contain a sufficient numbers of rows to amortize the cost of the
writing. The WOS has changed over time from row orientation to column
orientation and back again. We did not find any significant performance
differences between these approaches and the changes were driven
primarily by software engineering considerations. Data is not encoded or
compressed when it is in the WOS. However, it \emph{is} segmented
according to the projection's segmentation expression.

\hypertarget{data-modifications-and-delete-vectors}{%
\subsubsection{Data Modifications and Delete
Vectors}\label{data-modifications-and-delete-vectors}}

Data in Vertica is never modified in place. When a tuple is deleted or
updated from either the WOS or ROS, Vertica creates a \emph{delete
vector}. A delete vector is a list of positions of rows that have been
deleted. Delete vectors are stored in the same format as user data: they
are first written to a \emph{DVWOS} in memory, then moved to \emph{DVROS
containers} on disk by the tuple mover (further explained in section 4)
and stored using efficient compression mechanisms. There may be multiple
delete vectors for the WOS and multiple delete vectors for any
particular ROS container. SQL UPDATE is supported by deleting the row
being updated and then inserting a row containing the updated column
values.

\textbf{Figure 2: Physical storage layout within a node. This figure
illustrates how columns are stored in projections using files on disk.
The table is partitioned by} EXTRACT MONTH, YEAR FROM TIMESTAMP
\textbf{and segmented by} HASH(cid)\textbf{. There are 14 ROS
containers, each with two columns. Each column's data within its ROS
container is stored as a single file for a total of 28 files of user
data. The data has four partition keys: 3/2012, 4/2012, 5/2012 and
6/2012. As the projection is segmented by} HASH(cid)\textbf{, this node
is responsible for storing all data that satisfies the condition}
\emph{C\textsubscript{nmin} \textless{} hash}(\emph{cid}) ≤
\emph{C\textsubscript{nmax}}\textbf{, for some value of}
\emph{C\textsubscript{nmin }}\textbf{and}
\emph{C\textsubscript{nmax}}\textbf{. This node has divided the data
into three local segments such that: Local Segment}

\textbf{1 has , Local Segment 2 has and Local Segment}

\textbf{3 has .}

\hypertarget{tuple-mover}{%
\section{TUPLE MOVER}\label{tuple-mover}}

The \emph{tuple mover} is an automatic system which oversees and
rearranges the physical data files to increase data storage and ingest
efficiency during query processing. Its work can be grouped into two
main functions:

\begin{enumerate}
\def\labelenumi{\arabic{enumi}.}
\item
  \textbf{Moveout}: asynchronously moves data from the WOS to the ROS
\item
  \textbf{Mergeout}: merges multiple ROS files together into larger
  ones.
\end{enumerate}

As the WOS fills up, the tuple mover automatically executes a
\emph{moveout} operation to move data from WOS to ROS. In the event that
the WOS becomes saturated before moveout is complete, subsequently
loaded data is written directly to new ROS Containers until the WOS
regains sufficient capacity. The tuple mover must balance its moveout
work so that it is not overzealous (creating too many little ROS
containers) but also not too lazy (resulting in WOS overflow which also
creates too many little files).

\emph{Mergeout} decreases the number of ROS containers on disk. Numerous
small ROS containers decrease compression opportunities and slow query
processing. Many files require more file handles, more seeks, and more
merges of the sorted files. The tuple mover merges smaller files
together into larger ones, and it reclaims storage by filtering out
tuples which were deleted prior to the Ancient History Mark (further
explained in section 5.1) as there is no way a user can query them.
Unlike C-Store, the tuple mover does not intermix data from the WOS and
ROS in order to strongly bound the number of times a tuple is
(re)merged. When a tuple is part of a mergeout operation, it is read
from disk once and written to disk once.

The tuple mover periodically quantizes the ROS containers into several
exponential sized \emph{strata} based on file size. The output ROS
container from a mergeout operation are planned such that the resulting
ROS container is in at least one strata larger than any of the input ROS
containers. Vertica does not impose any size restrictions on ROS
containers, but the tuple mover will not create ROS containers greater
than some maximum (currently 2TB) so as to strongly bound the number of
strata and thus the number of merges. The maximum ROS container size is
chosen to be sufficiently large that any per-file overhead is amortized
to irrelevance and yet the files are not too unwieldy to manage. By
choosing strata sizes exponentially, the number of times any tuple is
rewritten is bounded to the number of strata.

The tuple mover takes care to preserve partition and local segment
boundaries when choosing merge candidates. It has also been tuned to
maximize the system's tuple ingest rate while preventing an explosion in
the number of ROS containers. An important design point of the tuple
mover is that operations are not centrally coordinated across the
cluster. The specific ROS container layouts are private to every node,
and while two nodes might contain the same tuples, it is common for them
to have a different layout of ROS containers due to factors such as
different patterns of merging, available resources, node failure and
recovery.


\textbf{Table 1: Lock Compatibility Matrix}


\hypertarget{updates-and-transactions}{%
\section{UPDATES AND TRANSACTIONS}\label{updates-and-transactions}}

Every tuple in Vertica is timestamped with the logical time at which it
was committed. Each delete marker is paired with the logical time the
row was deleted. These logical timestamps are called epochs and are
implemented as implicit 64-bit integral columns on the projection or
delete vector. All nodes agree on the epoch in which each transaction
commits, thus an epoch boundary represents a globally consistent
snapshot. In concert with Vertica's policy of never modifying storage, a
query executing in the recent past needs no locks and is assured of a
consistent snapshot. The default transaction isolation in Vertica is
READ COMMITTED, where each query targets the latest epoch (the current
epoch - 1).

Because most queries, as explained above, do not require any locks,
Vertica has an analytic-workload appropriate table locking model. Lock
compatibility and conversion matrices are shown in Table 1 and Table 2
respectively, both adapted from {[}15{]}.

\begin{itemize}
\item

  \textbf{S}hared lock: while held, prevents concurrent modification of
  the table. Used to implement SERIALIZABLE isolation.

\item

  \textbf{I}nsert lock: required to insert data into a table. An
  \textbf{I}nsert lock is compatible with itself, enabling multiple
  inserts and bulk loads to occur simultaneously which is critical to
  maintain high ingest rates and parallel loads yet still offer
  transactional semantics.

\item

  \textbf{S}hared\textbf{I}nsert lock: required for read and insert, but
  not update or delete.

\item

  E\textbf{X}clusive lock: required for deletes and updates.

\item

  \textbf{T}uple mover lock: required for certain tuple mover
  operations. This lock is compatible with every lock except \textbf{X}
  and is used by the tuple mover during certain short operations on
  delete vectors.

\item

  \textbf{U}sage lock: required for parts of moveout and mergeout
  operations.

\item

  \textbf{O}wner lock: required for significant DDL such as dropping
  partitions and adding columns.

\end{itemize}

Vertica employs a distributed agreement and group membership protocol to
coordinate actions between nodes in the cluster. The messaging protocol
uses broadcast and point-to-point delivery to ensure that any control
message is successfully received by every node. Failure to receive a
message will cause a node to be ejected from the cluster and the
remaining nodes will be informed of the loss. Like C-Store,
\textbf{Table 2: Lock Conversion Matrix}

Vertica does not employ traditional two-phase commit{[}15{]}. Rather,
once a cluster transaction commit message is sent, nodes either
successfully complete the commit or are ejected from the cluster. A
commit succeeds on the cluster if it succeeds on a quorum of nodes. Any
ROS or WOS created by the committing transaction becomes visible to
other transactions when the commit completes. Nodes that fail during the
commit process leave the cluster and rejoin the cluster in a consistent
state via the recovery mechanism described in section 5.2. Transaction
rollback simply entails discarding any ROS container or WOS data created
by the transaction.

\hypertarget{epoch-management}{%
\subsection{Epoch Management}\label{epoch-management}}

Originally, Vertica followed the C-store epoch model: epochs contained
all transactions committed in a given time window. However, users
running in READ COMMITTED were often confused because their commits did
not become ``visible'' until the epoch advanced. Now Vertica
automatically advances the epoch as part of commit when the committing
transaction includes DML or certain data-modifying DDL. In addition to
reducing user confusion, automatic epoch advancement simplifies many of
the internal management processes (like the tuple mover).

Vertica tracks two epoch values worthy of mention: the \emph{Last Good
Epoch} (LGE) and the \emph{Ancient History Mark} (AHM). The LGE for a
node is the epoch for which all data has been successfully moved out of
the WOS and into ROS containers on disk. The LGE is tracked per
projection because data that exists only in the WOS is lost in the event
of a node failure. The AHM is an analogue of C-store's low water mark
where Vertica discards historical information prior to the AHM when data
reorganization occurs. Whenever the tuple mover observes a row deleted
prior to the AHM, it elides the row from the output of the operation.
The AHM advances automatically according to a user-specified policy. The
AHM normally does not advance when nodes are down so as to preserve the
history necessary to incrementally replay DML operations during recovery
(described in Section 5.2).

\hypertarget{tolerating-failures}{%
\subsection{Tolerating Failures}\label{tolerating-failures}}

Vertica replicates data to provide fault tolerance by employing the
projection segmentation mechanism explained in section 3.6. Each
projection must have at least one \emph{buddy projection} containing the
same columns and a segmentation method that ensures that no row is
stored on the same node by both projections. When a node is down, the
buddy projection is employed to source the missing rows from the down
node. Like any distributed database, Vertica must gracefully handle
failed nodes rejoining the cluster. In Vertica, this process is called
\emph{recovery}. Vertica has no need of traditional transaction logs
because the data+epoch itself serves as a log of past system activity.
Vertica implements efficient incremental recovery by utilizing this
historical record to replay DML the down node has missed. When a node
rejoins the cluster after a failure, it recovers each projection segment
from a corresponding buddy projection segment. First, the node truncates
all tuples that were inserted after its LGE, ensuring that it starts at
a consistent state. Then recovery proceeds in two phases to minimize
operational disruption.

\begin{itemize}
\item

  \textbf{Historical Phase}: recovers committed data from the LGE to
  some previous epoch \emph{E\textsubscript{h}}. No locks are held while
  data between the recovering node's LGE and \emph{E\textsubscript{h
  }}is copied from the buddy projection. When complete, the projection's
  LGE is advanced to \emph{E\textsubscript{h }}and either the historical
  phase continues or the current phase is entered, depending on the
  amount of data between the new LGE and the current epoch.

\item

  \textbf{Current Phase}: recovers committed data from the LGE until the
  current epoch. The current phase takes a \textbf{S}hared lock on the
  projection's tables and copies any remaining data. After the current
  phase, recovery is complete and the projection participates in all
  future DML transactions.

\end{itemize}

If the projection and its buddy have matching sort orders, recovery
simply copies whole ROS containers and their delete vectors from one
node to another. Otherwise, an execution plan similar to INSERT ...
SELECT ... is used to move rows (including deleted rows) to the
recovering node. A separate plan is used to move delete vectors. The
\emph{refresh} and \emph{rebalance} operations are similar to the
recovery mechanism. Refresh is used to populate new projections which
were created after the table was loaded with data. Rebalance
redistributes segments between nodes to rebalance storage as nodes are
added and removed. Both have a historical phase where older data is
copied and a current phase where a \textbf{S}hared lock is held while
any remaining data is transferred.

Backup is handled completely differently by taking advantage of
Vertica's read-only storage system. A backup operation takes a snapshot
of the database catalog and creates hard-links for each Vertica data
file on the file system. The hard-links ensure that the data files are
not removed while the backup image is copied off the cluster to the
backup location. Afterwards, the hard-links are removed, ensuring that
storage used by any files artificially preserved by the backup is
reclaimed. The backup mechanism supports both full and incremental
backup.

Recovery, refresh, rebalance and backup are all online operations;
Vertica continues to load and query data while they are running. They
impact ongoing operations only to the extent that they require
computational and bandwidth resources to complete.

\hypertarget{cluster-integrity}{%
\subsection{Cluster Integrity}\label{cluster-integrity}}

The primary state managed between the nodes is the metadata catalog,
which records information about tables, users, nodes, epochs, etc.
Unlike other databases, the catalog is not stored in database tables, as
Vertica's table design is inappropriate for catalog access and update.
Instead, the catalog is implemented using a custom memory resident data
structure and transactionally persisted to disk via its own mechanism,
both of which are beyond the scope of this paper.

As in C-Store, Vertica provides the notion of \emph{K-safety}: With
\emph{K} or fewer nodes down, the cluster is guaranteed to remain
available. To achieve K-Safety, the database projection design must
ensure at least \emph{K}+1 copies of each segment are present on
different nodes such that a failure of \textbf{any} \emph{K} nodes
leaves at least one copy available. The failure of \emph{K}+1 nodes does
not guarantee a database shutdown. Only when node failures actually
cause data to become unavailable will the database shutdown until the
failures can be repaired and consistency restored via recovery. A
Vertica cluster will also perform a safety shutdown if \(\frac{N}{2}\) nodes
are lost where \emph{N} is the number of nodes in the cluster. The
agreement protocol requires a \(\frac{N}{2} + 1\)
quorum to protect against network partitions and avoid a split brain
effect where two halves of the cluster continue to operate
independently.

\hypertarget{query-execution}{%
\section{QUERY EXECUTION}\label{query-execution}}

Vertica supports the standard SQL declarative query language along with
its own proprietary extensions. Vertica's extensions are designed for
cases where easily querying timeseries and log style data in SQL was
overly cumbersome or impossible. Users submit SQL queries using an
interactive \texttt{vsql} command prompt or via standard JDBC, ODBC, or ADO .net
drivers. Rather than continuing to add more proprietary extensions,
Vertica has chosen to add an SDK with hooks for users to extend various
parts of the execution engine.

\hypertarget{query-operators-and-plan-format}{%
\subsection{Query Operators and Plan
Format}\label{query-operators-and-plan-format}}

The data processing of the plan is performed by the Vertica
\emph{Execution Engine} (EE). A Vertica query plan is a standard tree of
operators where each operator is responsible for performing a certain
algorithm. The output of one operator serves as the input to the
following operator. A simple single node plan is illustrated in figure
3. Vertica's execution engine is multi-threaded and pipelined: more than
one operator can be running at any time and more than one thread can be
executing the code for any individual operator. As in C-store, the EE is
fully vectorized and makes requests for blocks of rows at a time instead
of requesting single rows at a time. Vertica's operators use a pull
processing model: the most downstream operator requests rows from the
next operator upstream in the processing pipeline. This operator does
the same until a request is made of an operator that reads data from
disk or the network. The available operator types in the EE are
enumerated below. Each operator can use one of several possible
algorithms which are automatically chosen by the query optimizer.

\begin{enumerate}
\def\labelenumi{\arabic{enumi}.}
\item
  \textbf{Scan}: Reads data from a particular projection's ROS
  containers, and applies predicates in the most advantageous manner
  possible.
\item
  \textbf{GroupBy}: Groups and aggregates data. We have several
  different hash based algorithms depending on what is needed for
  maximal performance, how much memory is allotted, and if the operator
  must produce unique groups. Vertica also implements classic pipelined
  (one-pass) aggregates, with a choice to keep the incoming data encoded
  or not.
\end{enumerate}

\begin{enumerate}
\def\labelenumi{\arabic{enumi}.}
\setcounter{enumi}{2}
\item
  \textbf{Join}: Performs classic relational join. Vertica supports both
  hash join and merge join algorithms which are capable of externalizing
  if necessary. All flavors of INNER, LEFT OUTER, RIGHT OUTER, FULL
  OUTER, SEMI, and ANTI joins are supported.
\item
  \textbf{ExprEval}: Evaluate an expression
\item
  \textbf{Sort}: Sorts incoming data, externalizing if needed.
\item
  \textbf{Analytic}: Computes SQL-99 Analytics style windowed aggregates
\item
  \textbf{Send/Recv}: Sends tuples from one node to another. Both
  broadcast and sending to nodes based on segmentation expression
  evaluation is supported. Each Send and Recv pair is capable of
  retaining the sortedness of the input stream.
\end{enumerate}

\textbf{Figure 3: Plan representing a SQL query. The query plan contains
a scan operator for reading data followed by operators for grouping and
aggregation finally followed by a filter operation. The StorageUnion
dispatches threads for processing data on a set of ROS containers. The
StorageUnion also locally resegments the data for the above GroupBys.
The ParallelUnion dispatches threads for processing the GroupBys And
Filters in parallel.}

Vertica's operators are optimized for the sorted data that the storage
system maintains. Like C-Store, significant care has been taken and
implementation complexity has been added to ensure operators can operate
directly on encoded data, which is especially important for scans, joins
and certain low level aggregates.

The EE has several techniques to achieve high performance. Sideways
Information Passing (SIP) has been effective in improving join
performance by filtering data as early as possible in the plan. It can
be thought of as an advanced variation of predicate push down since the
join is being used to do filtering {[}27{]}. For example, consider a
HashJoin that joins two tables using simple equality predicates. The
HashJoin will first create a hash table from the inner input before it
starts reading data from the outer input to do the join. Special SIP
filters are built during optimizer planning and placed in the Scan
operator. At run time, the Scan has access to the Join's hash table and
the SIP filters are used to evaluate whether the outer key values exist
in the hash table. Rows that do not pass these filters are not output by
the Scan thus increasing performance since we are not unnecessarily
bringing the data through the plan only to be filtered away later by the
join. Depending on the join type, we are not always able to push the SIP
filter to the Scan, but we do push the filters down as far as possible.
We can also perform SIP for merge joins with a slightly different type
of SIP filter beyond the scope of this paper.

The EE also switches algorithms during runtime as it observes data
flowing through the system. For example, if Vertica determines at
runtime the hash table for a hash join will not fit into memory, we will
perform a sort-merge join instead. We also institute several ``prepass''
operators to compute partial results in parallel but which are not
required for correctness (see Figure 3). The results of prepass
operators are fed into the final operator to compute the complete
result. For example, the query optimizer plans grouping operations in
several stages for maximal performance. In the first stage, it attempts
to aggregate immediately after fetching columns off the disk using an L1
cache sized hash table. When the hash table fills up, the operator
outputs its current contents, clears the hash table, and starts
aggregating afresh with the next input. The idea is to cheaply reduce
the amount of data before sending it through other operators in the
pipeline. Since there is still a small, but non-zero cost to run the
prepass operator, the EE will decide at runtime to stop if it is not
actually reducing the number of rows which pass.

During query compile time, each operator is given a memory budget based
on the resources available given a user defined workload policy and what
each operator is going to do. All operators are capable of handling
arbitrary sized inputs, regardless of the memory allocated, by
externalizing their buffers to disk. This is critical for a production
database to ensure users queries are always answered. One challenge of a
fully pipelined execution engine such as Vertica's is that all operators
must share common resources, potentially causing unnecessary spills to
disk. In Vertica, the plan is separated into multiple zones that can not
be executing at the same time\footnote{Separated by operators such as
  Sort}. Downstream operators are able to reclaim resources previously
used by upstream operators, allowing each operator more memory than if
we pessimistically to assumed all operators would need their resources
at the same time.

Many computations are data type dependent and require the code to branch
to type specific implementations at query runtime. To improve
performance and reduce control flow overhead, Vertica uses just in time
compilation of certain expression evaluations to avoid branching by
compiling the necessary assembly code on the fly.

Although the simplest implementation of a pull execution engine is a
single thread, Vertica uses multiple threads for processing the same
plan. For example, multiple worker threads are dispatched to fetch data
from disk and perform initial aggregations on non overlapping sections
of ROS containers. The Optimizer and EE work together to combine the
data from each pipeline at the required locations to get correct
answers. It is necessary to combine partial results because alike values
are not co-located in the same pipeline. The Send and Recv operators
ship data to the nodes in the cluster. The send operator is capable of
segmenting the data in such as way that all alike values are sent to the
same node in the cluster. This allows each node's operator to compute
the full results independently of the other nodes. In the same way we
fully utilize the cluster of nodes by dividing the data in advantageous
ways, we can divide the data locally on each node to process data in
parallel and keep the all the cores fully utilized. As shown in Figure
3, multiple GroupBy operators are run in parallel requesting data from
the StorageUnion which resegments the data such that the GroupBy is able
to compute complete results.

\hypertarget{query-optimization}{%
\subsection{Query Optimization}\label{query-optimization}}

C-Store has a minimal optimizer, in which the projections it reaches
first are chosen for tables in the query, and the join order of the
projections is completely random. The Vertica Optimizer has evolved
through three generations: StarOpt, StarifiedOpt, and V2Opt.

\emph{StarOpt}, the initial Vertica optimizer, was a Kimball-style
optimizer{[}18{]} which assumed that any interesting warehouse schema
could be modeled as a classic star or snowflake. A star schema
classifies attributes of an event into fact tables and descriptive
attributes into dimension tables. Usually, a fact table is much larger
than a dimension table and has a many-to-one relationship with its
associated descriptive dimension tables. A snowflake schema is an
extension of a star schema, where one or more dimension tables has
many-to-one relationships with further descriptive dimension tables.
This paper uses the term star to represent both star and snowflake
designs. The information of a star schema is often requested through
(star) queries that join fact tables with their dimensions. Efficient
plans for star queries are to join a fact table with its most highly
selective dimensions first. Thus, the most important process in planning
a Vertica StarOpt query is choosing and joining projections with highly
compressed and sorted predicate and join columns, to make sure that not
only fast scans and merge joins on compressed columns are applied first,
but also that the cardinality of the data for later joins is reduced.

Besides the described StarOpt and columnar-specific techniques described
above, StartOpt and two other Vertica optimizers described later also
employ other techniques to take advantage of the specifics of sorted
columnar storage and compression, such as late materialization{[}8{]},
compression-aware costing and planning, stream aggregation, sort
elimination, and merge joins.

Although Vertica has been a distributed system since the beginning,
StarOpt was designed only to handle queries whose tables have data with
co-located projections. In other words, projections of different tables
in the query must be either replicated on all nodes, or segmented on the
same range of data on their join keys, so the plan can be executed
locally on each node and the results sent to the node that the client is
connected to. Even with this limitation, StarOpt still works well with
star schemas because only the data of large fact tables needs to be
segmented throughout the cluster. Data of small dimension tables can be
replicated everywhere without performance degradation. As many Vertica
customers demonstrated their increasing need for non-star queries,
Vertica developed its second generation optimizer,
\emph{StarfiedOpt}\footnote{US patent 8,086,598, Query Optimizer with
  Schema Conversion 6} as a modification to StarOpt. By forcing non-star
queries to look like a star, Vertica could run the StarOpt algorithm on
the query to optimize it. StarifiedOpt is far more effective for
non-star queries than we could have reasonably hoped, but, more
importunately, it bridged the gap to optimize both star and non-star
queries while we designed and implemented the third generation
optimizer: the custom built V2Opt.

The distribution aware \emph{V2Opt}\footnote{Pending patent, Modular
  Query Optimizer}, which allows data to be transferred on-the-fly
between nodes of the cluster during query execution, is designed from
the start as a set of extensible modules. In this way, the brains of the
optimizer can be changed without rewriting lots of the code. In fact,
due to the inherent extensible design, knowledge gleaned from end-user
experiences has already been incorporated into the V2Opt optimizer
without a lot of additional engineering effort. V2Opt plans a query by
categorizing and classifying the query's physical-properties, such as
column selectivity, projection column sort order, projection data
segmentation, prejoin projection availability, and integrity constraint
availability. These physical-property heuristics, combined with a
pruning strategy using a cost-model, based on compression aware I/O, CPU
and Network transfer costs, help the optimizer (1) control the explosion
in search space while continuing to explore optimal plans and (2)
account for data distribution and bushy plans during the join order
enumeration phase. While innovating on the V2Opt core algorithms, we
also incorporated many of the best practices developed over the past 30
years of optimizer research such as using equi-height histograms to
calculate selectivity, applying sample-based estimates of the number of
distinct values {[}16{]}, introducing transitive predicates based on
join keys, converting outer joins to inner joins, subquery
de-correlation, subquery flattening {[}17{]} , view flattening,
optimizing queries to favor co-located joins where possible, and
automatically pruning out unnecessary parts of the query.

The Vertica Database Designer described in Section 6.3 works
hand-in-glove with the optimizer to produce a physical design that takes
advantage of the numerous optimization techniques available to the
optimizer. Furthermore, when one or more nodes in the database cluster
goes down, the optimizer replans the query by replacing and then
recosting the projections on unavailable nodes with their corresponding
buddy projections on working nodes. This can lead to a new plan with a
different join order from the original one.

\hypertarget{automatic-physical-design}{%
\subsection{Automatic Physical Design}\label{automatic-physical-design}}

Vertica features an automatic physical design tool called the
\emph{Database Designer} (DBD). The physical design problem in Vertica
is to determine sets of projections that optimize a representative query
workload for a given schema and sample data while remaining within a
certain space budget. The major tensions to resolve during projection
design are optimizing query performance while reducing data load
overhead and minimizing storage footprint.


The DBD design algorithm has two sequential phases:


\begin{enumerate}
\def\labelenumi{\arabic{enumi}.}
\item
  \textbf{Query Optimization}: Chooses projection sort order and
  segmentation to optimize the performance of the query workload. During
  this phase, the DBD enumerates candidate projections based on
  heuristics such as predicates, group by columns, order by columns,
  aggregate columns, and join predicates. The optimizer is invoked for
  each input query and given a choice of the candidate projections. The
  resulting plan is used to choose the best projections from amongst the
  candidates. The DBD's system to resolve conflicts when different
  queries are optimized by different projections is important, but
  beyond the scope of this paper. The DBD's direct use of the optimizer
  and cost model guarantees that it remains synchronized as the
  optimizer evolves over time.
\item
  \textbf{Storage Optimization}: Finds the best encoding schemes for the
  designed projections via a series of empirical encoding experiments on
  the sample data, given the sort orders chosen in the query
  optimization phase.
\end{enumerate}

The DBD provides different design policies so users can trade off query
optimization and storage footprint: (a) load-optimized, (b)
query-optimized and (c) balanced. These policies indirectly control the
number of projections proposed to achieve the desired balance between
query performance and storage/load constraints. Other design challenges
include monitoring changes in query workload, schema, and cluster layout
and determining the incremental impact on the design.

As our user base has expanded, the DBD is now universally used for a
baseline physical design. Users can then manually modify the proposed
design before deployment. Especially in the case of the largest (and
thus most important) tables, expert users sometimes make minor changes
to projection-segmentation, select-list or sort-list based on their
specific knowledge of their data or use cases which may be unavailable
to the DBD. It is extremely rare for any user to override the column
encoding choices of the DBD, which we credit to the empirical
measurement during the storage-optimization phase.

\hypertarget{user-experience}{%
\section{USER EXPERIENCE}\label{user-experience}}

In this section we highlight some of the features of our system which
have led to its wide adoption and commercial success, as well as the
observations which led us to those features.

\begin{itemize}
\item

  \textbf{SQL}: First and foremost, standard SQL support was critical
  for commercial success as most customer organizations have large skill
  and tool investments in the language. Despite the temptation to invent
  new languages or dialects to avoid pet peeves, \footnote{Which at
    least one author admits having done in the past} standard SQL
  provides a data management system of much greater reach than a new
  language that people must learn.

\item

  \textbf{Resource Management}: Specifying how a cluster's resources are
  to be shared and reporting on the current resource allocation with
  many concurrent users is critical to real world deployments. We
  initially under appreciated this point early in Vertica's lifetime and
  we believe it is still an understudied problem in academic data
  management research.

\item

  \textbf{Automated Tuning}: Database users by and large wish to remain
  ignorant of a database's inner workings and focus on their application
  logic. Legacy RDBMS systems often require heroic tuning efforts, which
  Vertica has largely avoided by significant engineering effort and
  focus. For example, performance of early beta versions was a function
  of the physical storage layout and required users to learn how to tune
  and control the storage system. Automating storage layout management
  required Vertica to make significant and interrelated changes to the
  storage system, execution engine and tuple mover.

\item

  \textbf{Predictability vs. Special Case Optimizations}: It was
  tempting to pick low hanging performance optimization fruit that could
  be delivered quickly, such as transitive predicate creation for INNER
  but not OUTER joins or specialized filter predicates for Hash joins
  but not Merge joins. To our surprise, such special case optimizations
  caused almost as many problems as they solved because certain user
  queries would go super fast and some would not in hard to predict
  ways, often due to some incredibly low level implementation detail. To
  our surprise, users didn't accept the rationale that it was better
  that some queries got faster even though not all did.

\item
  \textbf{Direct Loading to the ROS}: While appealing in theory,
  directing all newly-inserted data to the WOS wastefully consumes
  memory. Especially while initially loading a system, the amount of
  data in a single bulk load operation was likely to be many tens of
  gigabytes in size and thus not memory resident. Users are more than
  happy to explicitly tag such loads to target the ROS in exchange for
  improved resource usage.
\item

  \textbf{Bulk Loading and Rejected Records}: Handling input data from
  the bulk loader that did not conform to the defined schema in a large
  distributed system turned out to be important and complex to
  implement.

\end{itemize}

\textbf{Table 3: Performance of Vertica compared with CStore on single
node Pentium 4 hardware using the queries and test harness of the
C-Store paper.}

\hypertarget{performance-measurements}{%
\section{PERFORMANCE MEASUREMENTS}\label{performance-measurements}}

\hypertarget{c-store}{%
\subsection{C-Store}\label{c-store}}

One of the early concerns of the Vertica investors was that the demands
of a product-grade feature set would degrade performance, or that the
performance claims of the C-Store prototype would otherwise not
generalize to a full commercial database implementation. In fact, there
were many features to be added, any of which could have degraded
performance such as support for: (1) multiple data types, such as FLOAT
and VARCHAR, where C-Store only supported INTEGER, (2) processing SQL
NULLs, which often have to be special cased, (3) updating/deleting data,
(4) multiple ROS and WOS stores, (5) ACID transactions, query
optimization, resource management, and other overheads, and (6) 64-bit
instead of 32-bit for integral data types.

Vertica reclaims any performance loss using software engineering methods
such as vectorized execution and more sophisticated compression
algorithms. Any remaining overhead is amortized across the query, or
across all rows in a data block, and turns out to be negligible. Hence,
Vertica is roughly twice as fast as C-Store on a single-core machine, as
shown in table 3. \footnote{Comparison on a cluster of modern multicore
  machines was deemed unfair, as the C-Store prototype is a
  single-threaded program and cannot take advantage of MPP hardware.}

\hypertarget{compression}{%
\subsection{Compression}\label{compression}}

This section describes experiments that show Vertica's storage engine
achieves significant compression with both contrived and real customer
data. Table 4 summarizes our results which were first presented here
{[}6{]}.

\hypertarget{m-random-integers}{%
\subsubsection{1M Random Integers}\label{m-random-integers}}

In this experiment, we took a text file containing a million random
integers between 1 and 10 million. The raw data is 7.5 MB because each
line is on average 7 digits plus a newline. Applying gzip, the data
compresses to about 3.6

MB, because the numbers are made of digits, which are a subset of all
byte representations. Sorting the data before applying gzip makes it
much more compressible resulting in a compressed size of 2.2 MB.
However, by avoiding strings and using a suitable encoding, Vertica
stores the same data in 0.6 MB.

\begin{longtable}[]{@{}llll@{}}
\toprule
& Size (MB) & Comp. & Bytes Per\tabularnewline
\midrule
\endhead
\begin{minipage}[t]{0.22\columnwidth}\raggedright
\strut
\end{minipage} & \begin{minipage}[t]{0.22\columnwidth}\raggedright
\strut
\end{minipage} & \begin{minipage}[t]{0.22\columnwidth}\raggedright

Ratio
\strut
\end{minipage} & \begin{minipage}[t]{0.22\columnwidth}\raggedright
Row\strut
\end{minipage}\tabularnewline
\begin{minipage}[t]{0.22\columnwidth}\raggedright

\textbf{Rand. Integers}
\strut
\end{minipage} & \begin{minipage}[t]{0.22\columnwidth}\raggedright
\strut
\end{minipage} & \begin{minipage}[t]{0.22\columnwidth}\raggedright
\strut
\end{minipage} & \begin{minipage}[t]{0.22\columnwidth}\raggedright
\strut
\end{minipage}\tabularnewline
Raw & 7.5 & 1 & 7.9\tabularnewline
gzip & 3.6 & 2.1 & 3.7\tabularnewline
gzip+sort & 2.3 & 3.3 & 2.4\tabularnewline
Vertica & 0.6 & 12.5 & 0.6\tabularnewline
\textbf{Customer Data} & & &\tabularnewline
Raw CSV & 6200 & 1 & 32.5\tabularnewline
gzip & 1050 & 5.9 & 5.5\tabularnewline
Vertica & 418 & 14.8 & 2.2\tabularnewline
\bottomrule
\end{longtable}

\textbf{Table 4: Compression achieved in Vertica for 1M Random Integers
and Customer Data.}

\hypertarget{m-customer-records}{%
\subsubsection{200M Customer Records}\label{m-customer-records}}

Vertica has a customer that collects metrics from some meters. There are
4 columns in the schema: \textbf{Metric}: There are a few hundred
metrics collected. \textbf{Meter}: There are a couple of thousand
meters. \textbf{Collection Time Stamp}: Each meter spits out metrics
every 5 minutes, 10 minutes, hour, etc., depending on the metric.
\textbf{Metric Value}: A

64-bit floating point value.

A baseline file of 200 million comma separated values (CSV) of the
meter/metric/time/value rows takes 6200 MB, for 32 bytes per row.
Compressing with gzip reduces this to 1050 MB. By sorting the data on
metric, meter, and collection time, Vertica not only optimizes common
query predicates (which specify the metric or a time range), but exposes
great compression opportunities for each column. The total size for all
the columns in Vertica is 418MB (slightly over 2 bytes per row).
\textbf{Metric}: There aren't many. With RLE, it is as if there are only
a few hundred rows. Vertica compressed this column to 5 KB.
\textbf{Meter}: There are quite a few, and there is one record for each
meter for each metric. With RLE, Vertica brings this down to a mere 35
MB. \textbf{Collection Time Stamp}: The regular collection intervals
present a great compression opportunity. Vertica compressed this column
to 20 MB. \textbf{Metric Value}: Some metrics have trends (like lots of
0 values when nothing happens). Others change gradually with time. Some
are much more random, and less compressible. However, Vertica compressed
the data to only 363MB.

\hypertarget{related-work}{%
\section{RELATED WORK}\label{related-work}}

The contributions of Vertica and C-Store are their unique combination of
previously documented design features applied to a specific workload.
The related work section in {[}21{]} provides a good overview of the
research roots of both CStore and Vertica prior to 2005. Since 2005,
several other research projects have been or are being commercialized
such as InfoBright {[}3{]}, Brighthouse {[}24{]}, Vectorwise {[}1{]},
and MonetDB/X100 {[}10{]}. These systems apply techniques similar

to those of Vertica such as column oriented storage, multicore execution
and automatic storage pruning for analytical workloads. The SAP HANA
{[}14{]} system takes a different approach to analytic workloads and
focuses on columnar inmemory storage and tight integration with other
business applications. Blink {[}23{]} also focuses on in-memory
execution as well as being a distributed shared-nothing system. In
addition, the success of Vertica and other native column stores has led
legacy RDBMS vendors to add columnar storage options {[}20, 4{]} to
their existing engines.

\hypertarget{conclusions}{%
\section{CONCLUSIONS}\label{conclusions}}

In this paper, we described the system architecture of the Vertica
Analytic Database, pointing out where our design differs or extends that
of C-Store. We have also shown some quantitative and qualitative
advantages afforded by that architecture.

Vertica is positive proof that modern RDBMS systems can continue to
present a familiar relational interface yet still achieve the high
performance expected from modern analytic systems. This performance is
achieved with appropriate architectural choices drawing on the rich
database research of the last 30 years.

Vertica would not have been possible except for new innovations from the
research community since the last major commercial RBDMs were designed.
We emphatically believe that database research is not and should not be
about incremental changes to existing paradigms. Rather, the community
should focus on transformational and innovative engine designs to
support the ever expanding requirements placed on such systems. It is an
exciting time to be a database implementer and researcher.

\hypertarget{acknowledgments}{%
\section{ACKNOWLEDGMENTS}\label{acknowledgments}}

The Vertica Analytic Database is the product of the hard work of many
great engineers. Special thanks to Goetz Graefe, Kanti Mann, Pratibha
Rana, Jaimin Dave, Stephen Walkauskas, and Sreenath Bodagala who helped
review this paper and contributed many interesting ideas.

\hypertarget{references}{%
\section{REFERENCES}\label{references}}

\begin{enumerate}
\def\labelenumi{\arabic{enumi}.}
\item
  Actian Vectorwise. \url{http://www.actian.com/products/vectorwise}.
\item
  HP Completes Acquisition of Vertica Systems, Inc.
  \url{http://www.hp.com/hpinfo/newsroom/press/2011/110322c.html}.
\item
  Infobright. \url{http://www.infobright.com/}.
\item
  Oracle Hybrid Columnar Compression on Exadata.
  \url{http://www.oracle.com/technetwork/middleware/bifoundation/ehcc-twp-131254.pdf}.
\item
  PostgreSQL. \url{http://www.postgresql.org/}.
\item Why Verticas Compression is Better. \url{http://www.vertica.com/2010/05/26/why-verticas-compression-is-better}.
\item
  A. Thusoo, J.S. Sarma, N. Jain, Z. Shao, P. Chakka,
  S. Anthony, H. Liu, P. Wyckoff and R. Murthy. Hive --- A Warehousing
  Solution Over a MapReduce Framework. \emph{PVLDB}, 2(2):1626--1629,
  2009.
\item
  D. J. Abadi, D. S. Myers, D. J. Dewitt, and S. R. Madden.
  Materialization Strategies in a Column-Oriented DBMS. In \emph{ICDE},
  pages 466--475, 2007.
\item
  B. Chattopadhyay, L. Lin, W. Liu, S. Mittal, P.Aragonda, V. Lychagina,
  Y. Kwon and M. Wong. Tenzing: A SQL Implementation On The MapReduce
  framework. \emph{PVLDB}, 4(12):1318--1327, 2011.
\item
  P. A. Boncz, M. Zukowski, and N. Nes.
  MonetDB/X100: Hyper-Pipelining Query Execution. In \emph{CIDR}, pages
  225--237, 2005.
\item
  S. Ceri and J. Widom. Deriving Production Rules for
  Incremental View Maintenance. In \emph{VLDB}, pages 577--589, 1991.
\item
  J. Dean and S. Ghemawat. MapReduce: Simplified Data Processing on
  Large Clusters. In \emph{OSDI}, pages 137--150, 2004.
\item
  G. DeCandia, D. Hastorun, M. Jampani,
  G. Kakulapati, A. Lakshman, A. Pilchin,
  S. Sivasubramanian, P. Vosshall, and W. Vogels.
  Dynamo: Amazon's Highly Available Key-value Store.
  In \emph{SOSP}, pages 205--220, 2007.
\item
  F. F\"{a}rber, S. K. Cha, J. Primsch, C. Bornh\"{o}vd,
  S. Sigg, and W. Lehner. SAP HANA Database: DataManagement for Modern
  Business Applications. \emph{ACM SIGMOD Record}, 40(4):45--51, 2012.
  J. Gray and A. Reuter. \emph{Transaction Processing: Concepts and Techniques}. Morgan Kaufmann Publishers Inc., 1992.
\item
  P. J. Haas, J. F. Naughton, S. Seshadri, and L. Stokes. Sampling-Based
  Estimation of the Number of Distinct Values of an Attribute. In
  \emph{VLDB}, pages 311--322, 1995.
\item
  W. Kim. On Optimizing a SQL-like Nested Query.
  \emph{ACM TODS}, 7(3):443--469, 1982.
\item
  R. Kimball and M. Ross. \emph{The Data Warehouse Toolkit: The Complete Guide to Dimensional Modeling}. Wiley, John
  \& Sons, Inc., 2002.
\item
  A. Lakshman and P. Malik. Cassandra: A
  Decentralized Structured Storage System. \emph{SIGOPS Operating Systems
Review}, 44(2):35--40, 2010.
\item
  P.-\r{A}. Larson, E. N. Hanson, and S. L. Price. Columnar Storage in SQL
  Server 2012. \emph{IEEE Data Engineering Bulletin}, 35(1):15--20,
  2012.
\item
  M. Stonebraker, D. J. Abadi, A. Batkin, X. Chen, M. Cherniack, M.
  Ferreira, E. Lau, A. Lin, S. Madden and E. J. O'Neil et.al. C-Store: A
  Column-oriented DBMS. In \emph{VLDB}, pages 553--564, 2005.
\item
  G. Moerkotte. Small Materialized Aggregates: A Light Weight Index
  Structure for data warehousing. In \emph{VLDB}, pages 476--487, 1998.
\item
  R. Barber, P. Bendel, M. Czech, O. Draese, F. Ho, N. Hrle, S. Idreos,
  M.S. Kim, O. Koeth and J.G. Lee et.al. Business Analytics in (a)
  Blink. \emph{IEEE Data Engineering Bulletin}, 35(1):9--14, 2012.
\item
  D. Slezak, J. Wroblewski, V. Eastwood, and P. Synak. Brighthouse: An
  Analytic Data Warehouse for Ad-hoc Queries. \emph{PVLDB},
  1(2):1337--1345, 2008.
\item
  M. Staudt and M. Jarke. Incremental Maintenance of Externally
  Materialized Views. In \emph{VLDB}, pages 75--86, 1996.
\item
  M. Stonebraker. One Size Fits All: An Idea Whose Time has Come and
  Gone. In \emph{ICDE}, pages 2--11, 2005.
\item
  J. D. Ullman. \emph{Principles of Database and Knowledge-Base Systems,
  Volume II}. Computer Science Press, 1989.
\end{enumerate}

\end{document}
