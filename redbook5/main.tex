\PassOptionsToPackage{unicode=true}{hyperref} % options for packages loaded elsewhere
\PassOptionsToPackage{hyphens}{url}
\documentclass[b5paper,11pt,twoside,openright]{book}

\usepackage{ifxetex}
\ifxetex{}
\else
\errmessage{Must be built with XeLaTeX}
\fi

\usepackage{amsmath}
\usepackage{fontspec}
\usepackage{fourier-otf} % erewhon-math
\setmonofont{iosevka-type-slab-regular}[
  Path=../common/iosevka-type-slab/,
  Extension=.ttf,
  BoldFont=iosevka-type-slab-bold,
  ItalicFont=iosevka-type-slab-italic,
  BoldItalicFont=iosevka-type-slab-bolditalic,
  Scale=MatchLowercase,
]

% Math
\usepackage[binary-units]{siunitx}

\usepackage{caption}
\usepackage{authblk}
\usepackage{enumitem}
\usepackage{footnote}
\usepackage{bookmark}

% Table
\usepackage{tabu}
\usepackage{longtable}
\usepackage{booktabs}
\usepackage{multirow}

% Verbatim & Source code
\usepackage{fancyvrb}
\usepackage{minted}

% Beauty
\usepackage[protrusion]{microtype}
\usepackage[defaultlines=3]{nowidow}
\usepackage{upquote}
\usepackage{parskip}
\usepackage[strict]{changepage}

\usepackage{hyperref}

% Graph
\usepackage{graphicx}
\usepackage{grffile}
\usepackage{tikz}

\setlist{noitemsep}

\hypersetup{
	bookmarksnumbered,
	colorlinks, % hidelinks,
	pdfpagemode=UseNone,
	pdfstartview=FitH
}

\usetikzlibrary{calc,shapes.multipart,chains,arrows,matrix}
\tikzset{>=latex'}
\tikzstyle{list}=[
	 rectangle split,
	 rectangle split parts=2, draw,
	 rectangle split horizontal,
	 rectangle split part align=base,
	 text width=3ex, text centered
    ]
    
\newcommand{\pFF}[2]{\draw[->] let \p1 = (#1.one), \p2 = (#1.center), \p3 = (#2.one), \p4 = (#2.north) in (\x1,\y2) -- (\x3,\y4)}
\newcommand{\pSF}[2]{\draw[->] let \p1 = (#1.two), \p2 = (#1.center), \p3 = (#2.west), \p4 = (#2.center) in (\x1,\y2) -- (\x3,\y4)}

\newcolumntype{L}{>{\RaggedRight}X}

\newcommand{\SpecialChapter}[1]{%
  \chapter*{#1}%
  \addcontentsline{toc}{chapter}{#1}%
  \markboth{\MakeUppercase{#1}}{\MakeUppercase{#1}}
}

\pagestyle{headings}
\setcounter{secnumdepth}{3}

\title{Readings in Database Systems, 5th Edition}
\author{}
\date{}

\usepackage{xparse}

% Renew \section* for:
% 1. hypertarget & label
% 2. toc
% 3. markboth
\let\oldsection\section
\makeatletter
\RenewDocumentCommand{\section}{s o m}{%
  \IfBooleanTF{#1}
  {% \section*
    \IfValueTF{#2}% \section
    {% \section*[#2]{#3}
      \hypertarget{#2}{\oldsection*{#3}\label{#2}}
      \addcontentsline{toc}{section}{#2}
      \markright{\MakeUppercase{#2}}
    }
    {% \section*{#3}
      \hypertarget{#3}{\oldsection*{#3}\label{#3}}
      \addcontentsline{toc}{section}{#3}
      \markright{\MakeUppercase{#3}}
    }
  }
  {% \section
    \IfValueTF{#2}% \section
    {% \section[#2]{#3}
      \oldsection[#2]{#3}
    }
    {% \section{#3}
      \oldsection{#3}
    }
  }
}
\makeatother

\newcommand\Chapter[3]{
  \hypertarget{#1}{
    \chapter[#2]{#2\\{\Large #3}}\label{#1}
  }
}
\newcommand\Section[2]{
  \hypertarget{#1}{
    \section{#2}\label{#1}
  }
}


\begin{document}
\maketitle
\thispagestyle{empty}

\cleardoublepage
\pagenumbering{roman}
\tableofcontents

\hypertarget{preface}{\chapter*{Preface}\label{preface}}
\addcontentsline{toc}{chapter}{Preface}
\markboth{}{\MakeUppercase{Preface}}

In the ten years since the previous edition of \emph{Readings in
Database Systems}, the field of data management has exploded. Database
and data-intensive systems today operate over unprecedented volumes of
data, fueled in large part by the rise of ``Big Data'' and massive
decreases in the cost of storage and computation. Cloud computing and
microarchitectural trends have made distribution and parallelism nearly
ubiquitous concerns. Data is collected from an increasing variety of
heterogeneous formats and sources in increasing volume, and utilized for
an ever increasing range of tasks. As a result, commodity database
systems have evolved considerably along several dimensions, from the use
of new storage media and processor designs, up through query processing
architectures, programming interfaces, and emerging application
requirements in both transaction processing and analytics. It is an
exciting time, with considerable churn in the marketplace and many new
ideas from research.

In this time of rapid change, our update to the traditional ``Red Book''
is intended to provide both a grounding in the core concepts of the
field as well as a commentary on selected trends. Some new technologies
bear striking resemblance to predecessors of decades past, and we think
it's useful for our readers to be familiar with the primary sources. At
the same time, technology trends are necessitating a re-evaluation of
almost all dimensions of database systems, and many classic designs are
in need of revision. Our goal in this collection is to surface important
long-term lessons and foundational designs, and highlight the new ideas
we believe are most novel and relevant.

Accordingly, we have chosen a mix of classic, traditional papers from
the early database literature as well as papers that have been most
influential in recent developments, including transaction processing,
query processing, advanced analytics, Web data, and language design.
Along with each chapter, we have included a short commentary introducing
the papers and describing why we selected each. Each commentary is
authored by one of the editors, but all editors provided input; we hope
the commentaries do not lack for opinion.

When selecting readings, we sought topics and papers that met a core set
of criteria. First, each selection represents a major trend in data
management, as evidenced by both research interest and market demand.
Second, each selection is canonical or near-canonical; we sought the
most representative paper for each topic. Third, each selection is a
primary source. There are good surveys on many of the topics in this
collection, which we reference in commentaries. However, reading primary
sources provides historical context, gives the reader exposure to the
thinking that shaped influential solutions, and helps ensure that our
readers are well-grounded in the field. Finally, this collection
represents our current tastes about what is ``most important''; we
expect our readers to view this collection with a critical eye.

One major departure from previous editions of the Red Book is the way we
have treated the final two sections on Analytics and Data Integration.
It's clear in both research and the marketplace that these are two of
the biggest problems in data management today. They are also
quickly-evolving topics in both research and in practice. Given this
state of flux, we found that we had a hard time agreeing on
``canonical'' readings for these topics. Under the circumstances, we
decided to omit official readings but instead offer commentary. This
obviously results in a highly biased view of what's happening in the
field. So we do not recommend these sections as the kind of ``required
reading'' that the Red Book has traditionally tried to offer. Instead,
we are treating these as optional end-matter: ``Biased Views on Moving
Targets''. Readers are cautioned to take these two sections with a grain
of salt (even larger that the one used for the rest of the book.)

We are releasing this edition of the Red Book free of charge, with a
permissive license on our text that allows unlimited non-commercial
re-distribution, in multiple formats. Rather than secure rights to the
recommended papers, we have simply provided links to Google Scholar
searches that should help the reader locate the relevant papers. We
expect this electronic format to allow more frequent editions of the
``book.'' We plan to evolve the collection as appropriate.

A final note: this collection has been alive since 1988, and we expect
it to have a long future life. Accordingly, we have added a modicum of
``young blood'' to the gray beard editors. As appropriate, the editors
of this collection may further evolve over time.

\enlargethispage{2\baselineskip}
Peter Bailis\\
Joseph M. Hellerstein\\
Michael Stonebraker\\


\Chapter{ch1-background}{Background}{%
  Introduced by Michael Stonebraker}
\pagenumbering{arabic}

\begin{framed}
Readings:

\begin{itemize}
  \item
  \href{https://scholar.google.com/scholar?cluster=7366182905777149494}{Joseph
    M. Hellerstein and Michael Stonebraker. {What Goes Around Comes Around}.
    {Readings in Database Systems}, 4th Edition (2005).}
  \item
  \href{https://scholar.google.com/scholar?cluster=11466590537214723805}{Joseph
    M. Hellerstein, Michael Stonebraker, James Hamilton. {Architecture of a
      Database System}. {Foundations and Trends in Databases}, 1, 2 (2007).}
\end{itemize}
\end{framed}

I am amazed that these two papers were written a mere decade ago! My
amazement about the anatomy paper is that the details have changed a lot
- just a few years later. My amazement about the data model paper is
that nobody ever seems to learn anything from history. Lets talk about
the data model paper first.

A decade ago, the buzz was all XML. Vendors were intent on adding XML to
their relational engines. Industry analysts (and more than a few
researchers) were touting XML as ``the next big thing''. A decade later
it is a niche product, and the field has moved on. In my opinion, (as
predicted in the paper) it succumbed to a combination of:

\begin{itemize}
  \item
  excessive complexity (which nobody could understand)
  \item
  complex extensions of relational engines, which did not seem to
  perform all that well and
  \item
  no compelling use case where it was wildly accepted
\end{itemize}

It is a bit ironic that a prediction was made in the paper that X would
win the Turing Award by successfully simplifying XML. That prediction
turned out to be totally wrong! The net-net was that relational won and
XML lost.

Of course, that has not stopped ``newbies'' from reinventing the wheel.
Now it is JSON, which can be viewed in one of three ways:

\begin{itemize}
  \item
  A general purpose hierarchical data format. Anybody who thinks this is
  a good idea should read the section of the data model paper on IMS.
  \item
  A representation for sparse data. Consider attributes about an
  employee, and suppose we wish to record hobbies data. For each hobby,
  the data we record will be different and hobbies are fundamentally
  sparse. This is straightforward to model in a relational DBMS but it
  leads to very wide, very sparse tables. This is disasterous for
  disk-based row stores but works fine in column stores. In the former
  case, JSON is a reasonable encoding format for the ``hobbies'' column,
  and several RDBMSs have recently added support for a JSON data type.
  \item
  As a mechanism for ``schema on read''. In effect, the schema is very
  wide and very sparse, and essentially all users will want some
  projection of this schema. When reading from a wide, sparse schema, a
  user can say what he wants to see at run time. Conceptually, this is
  nothing but a projection operation. Hence, 'schema on read'' is just a
  relational operation on JSON-encoded data.
\end{itemize}

In summary, JSON is a reasonable choice for sparse data. In this
context, I expect it to have a fair amount of ``legs''. On the other
hand, it is a disaster in the making as a general hierarchical data
format. I fully expect RDBMSs to subsume JSON as merely a data type
(among many) in their systems. In other words, it is a reasonable way to
encode spare relational data.

No doubt the next version of the Red Book will trash some new
hierarchical format invented by people who stand on the toes of their
predecessors, not on their shoulders.

The other data model generating a lot of buzz in the last decade is
Map-Reduce, which was purpose-built by Google to support their web crawl
data base. A few years later, Google stopped using Map-Reduce for that
application, moving instead to Big Table. Now, the rest of the world is
seeing what Google figured out earlier; Map-Reduce is not an
architecture with any broad scale applicability. Instead the Map-Reduce
market has morphed into an HDFS market, and seems poised to become a
relational SQL market. For example, Cloudera has recently introduced
Impala, which is a SQL engine, built on top of HDFS, not using
Map-Reduce.

More recently, there has been another thrust in HDFS land which merit
discussion, namely ``data lakes''. A reasonable use of an HDFS cluster
(which by now most enterprises have invested in and want to find
something useful for them to do) is as a queue of data files which have
been ingested. Over time, the enterprise will figure out which ones are
worth spending the effort to clean up (data curation; covered in
\hyperref[ch12-dataintegration]{Chapter 12} of this book). Hence, the
data lake is just a ``junk drawer'' for files in the meantime. Also, we
will have more to say about HDFS, Spark and Hadoop in
\hyperref[ch5-dataflow]{Chapter 5}.

In summary, in the last decade nobody seems to have heeded the lessons
in ``comes around''. New data models have been invented, only to morph
into SQL on tables. Hierarchical structures have been reinvented with
failure as the predicted result. I would not be surprised to see the
next decade to be more of the same. People seemed doomed to reinvent the
wheel!

With regard to the Anatomy paper; a mere decade later, we can note
substantial changes in how DBMSs are constructed. Hence, the details
have changed a lot, but the overall architecture described in the paper
is still pretty much true. The paper describes how most of the legacy
DBMSs (e.g. Oracle, DB2) work, and a decade ago, this was the prevalent
implementation. Now, these systems are historical artifacts; not very
good at anything. For example, in the data warehouse market column
stores have replaced the row stores described in this paper, because
they are 1-2 orders of magnitude faster. In the OLTP world, main-memory
SQL engines with very lightweight transaction management are fast
becoming the norm. These new developments are chronicled in
\hyperref[ch4-newdbms]{Chapter 4} of this book. It is now hard to find
an application area where legacy row stores are competitive. As such,
they deserve to be sent to the ``home for retired software''.

It is hard to imagine that ``one size fits all'' will ever be the
dominant architecture again. Hence, the ``elephants'' have a bad
``innovators dilemma'' problem. In the classic book by Clayton
Christiansen, he argues that it is difficult for the vendors of legacy
technology to morph to new constructs without losing their customer
base. However, it is already obvious how the elephants are going to try.
For example, SQLServer 14 is at least two engines (Hekaton - a main
memory OLTP system and conventional SQLServer --- a legacy row store)
united underneath a common parser. Hence, the Microsoft strategy is
clearly to add new engines under their legacy parser, and then support
moving data from a tired engine to more modern ones, without disturbing
applications. It remains to be seen how successful this will be.

However, the basic architecture of these new systems continues to follow
the parsing/optimizer/executor structure described in the paper. Also,
the threading model and process structure is as relevant today as a
decade ago. As such, the reader should note that the details of
concurrency control, crash recovery, optimization, data structures and
indexing are in a state of rapid change, but the basic architecture of
DBMSs remains intact.

In addition, it will take a long time for these legacy systems to die.
In fact, there is still an enormous amount of IMS data in production
use. As such, any student of the field is well advised to understand the
architecture of the (dominant for a while) systems.

Furthermore, it is possible that aspects of this paper may become more
relevant in the future as computing architectures evolve. For example,
the impending arrival of NVRAM may provide an opportunity for new
architectural concepts, or a reemergence of old ones.


\Chapter{ch2-importantdbms}{Traditional RDBMS Systems}{%
Introduced by Michael Stonebraker
}

\begin{framed}
Readings:

\begin{itemize}
\item
\href{https://scholar.google.com/scholar?cluster=15466550502837111601}{Morton
  M. Astrahan, Mike W. Blasgen, Donald D. Chamberlin, Kapali P. Eswaran,
  Jim Gray, Patricia P. Griffiths, W. Frank King III, Raymond A. Lorie,
  Paul R. McJones, James W. Mehl, Gianfranco R. Putzolu, Irving L.
  Traiger, Bradford W. Wade, Vera Watson. {System R: Relational Approach
    to Database Management}. {ACM Transactions on Database Systems}, 1(2),
  1976, 97-137.}
\item
\href{https://scholar.google.com/scholar?cluster=7945977557090027847}{Michael
  Stonebraker and Lawrence A. Rowe. {The design of POSTGRES}. {SIGMOD},
  1986.}
\item
\href{https://scholar.google.com/scholar?cluster=8912521541627865753}{David
  J. DeWitt, Shahram Ghandeharizadeh, Donovan Schneider, Allan Bricker,
  Hui-I Hsiao, Rick Rasmussen. {The Gamma Database Machine Project}. {IEEE
    Transactions on Knowledge and Data Engineering}, 2(1), 1990, 44-62.}
\end{itemize}
\end{framed}

In this section are papers on (arguably) the three most important real
DBMS systems. We will discuss them chronologically in this introduction.

The System R project started under the direction of Frank King at IBM
Research - probably around 1972. By then Ted Codd's pioneering paper was
18 months old, and it was obvious to a lot of people that one should
build a prototype to test out his ideas. Unfortunately, Ted was not a
permitted to lead this effort, and he went off to consider natural
language interfaces to DBMSs. System R quickly decided to implement SQL,
which morphed from a clean block structured language in
1972~{{[}\protect\hyperlink{ref-chamberlin-sequel}{2}{]}} to a much more
complex structure described in the paper
here~{{[}\protect\hyperlink{ref-chamberlin-history}{1}{]}}.
See~{{[}\protect\hyperlink{ref-date}{3}{]}} for a commentary on the
design of the SQL language, written a decade later.

System R was structured into two groups, the ``lower half'' and the
``upper half''. They were not totally synchronized, as the lower half
implemented links, which were not supported by the upper half. In
defense of the decision by the lower half team, it was clear they were
competing against IMS, which had this sort of construct, so it was
natural to include it. The upper half simply didn't get the optimizer to
work for this construct.

The transaction manager is probably the biggest legacy of the project,
and it is clearly the work of the late Jim Gray. Much of his design
endures to this day in commercial systems. Second place goes to the
System R optimizer. The dynamic programming cost-based approach is still
the gold standard for optimizer technology.

My biggest complaint about System R is that the team never stopped to
clean up SQL. Hence, when the ``upper half'' was simply glued onto VSAM
to form DB2, the language level was left intact. All the annoying
features of the language have endured to this day. SQL will be the COBOL
of 2020, a language we are stuck with that everybody will complain
about.

My second biggest complaint is that System R used a subroutine call
interface (now ODBC) to couple a client application to the DBMS. I
consider ODBC among the worst interfaces on the planet. To issue a
single query, one has to open a data base, open a cursor, bind it to a
query and then issue individual fetches for data records. It takes a
page of fairly inscrutable code just to run one query. Both
Ingres~{{[}\protect\hyperlink{ref-ingres}{11}{]}} and Chris
Date~{{[}\protect\hyperlink{ref-date-language}{4}{]}} had much cleaner
language embeddings. Moreover,
Pascal-R~{{[}\protect\hyperlink{ref-pascalr}{9}{]}} and
Rigel~{{[}\protect\hyperlink{ref-rigel}{7}{]}} were also elegant ways to
include DBMS functionality in a programming language. Only recently with
the advent of Linq~{{[}\protect\hyperlink{ref-linq}{6}{]}} and Ruby on
Rails~{{[}\protect\hyperlink{ref-rails}{8}{]}} are we seeing a
resurgence of cleaner language-specific enbeddings.

After System R, Jim Gray went off to Tandem to work on Non-stop SQL and
Kapali Eswaren did a relational startup. Most of the remainder of the
team remained at IBM and moved on to work on various other projects,
include R*.

The second paper concerns Postgres. This project started in 1984 when it
was obvious that continuing to prototype using the academic Ingres code
base made no sense. A recounting of the history of Postgres appears
in~{{[}\protect\hyperlink{ref-stonebraker-turing}{10}{]}}, and the
reader is directed there for a full blow-by-blow recap of the ups and
downs in the development process.

However, in my opinion the important legacy of Postgres is its abstract
data type (ADT) system. User-defined types and functions have been added
to most mainstream relational DBMSs, using the Postgres model. Hence,
that design feature endures to this day. The project also experimented
with time-travel, but it did not work very well. I think no-overwrite
storage will have its day in the sun as faster storage technology alters
the economics of data management.

It should also be noted that much of the importance of Postgres should
be accredited to the availability of a robust and performant open-source
code line. This is an example of the open-source community model of
development and maintenance at its best. A pickup team of volunteers
took the Berkeley code line in the mid 1990's and has been shepherding
its development ever since. Both Postgres and 4BSD
Unix~{{[}\protect\hyperlink{ref-bsdunix}{5}{]}} were instrumental in
making open source code the preferred mechanism for code development.

The Postgres project continued at Berkeley until 1992, when the
commercial company Illustra was formed to support a commercial code
line. See~{{[}\protect\hyperlink{ref-stonebraker-turing}{10}{]}} for a
description of the ups and downs experienced by Illustra in the
marketplace.

Besides the ADT system and open source distribution model, a key legacy
of the Postgres project was a generation of highly trained DBMS
implementers, who have gone on to be instrumental in building several
other commercial systems

The third system in this section is Gamma, built at Wisconsin between
1984 and 1990. In my opinion, Gamma popularized the shared-nothing
partitioned table approach to multi-node data management. Although
Teradata had the same ideas in parallel, it was Gamma that popularized
the concepts. In addition, prior to Gamma, nobody talked about
hash-joins so Gamma should be credited (along with Kitsuregawa Masaru)
with coming up with this class of algorithms.

Essentially all data warehouse systems use a Gamma-style architecture.
Any thought of using a shared disk or shared memory system have all but
disappeared. Unless network latency and bandwidth get to be comparable
to disk bandwidth, I expect the current shared-nothing architecture to
continue.

\section*{References}

\leavevmode\hypertarget{ref-chamberlin-history}{}%
{[}1{]} Chamberlin, D.D. Early history of sQL. \emph{Annals of the
  History of Computing, IEEE}. 34, 4 (2012), 78-82.

\leavevmode\hypertarget{ref-chamberlin-sequel}{}%
{[}2{]} Chamberlin, D.D. and Boyce, R.F. SEQUEL: A structured english
query language. \emph{Proceedings of the 1974 aCM sIGFIDET (now sIGMOD)
  workshop on data description, access and control}, 1974, 249-264.

\leavevmode\hypertarget{ref-date}{}%
{[}3{]} Date, C.J. A critique of the SQL database language. \emph{ACM
  SIGMOD Record}. 14, 3 (Nov. 1984).

\leavevmode\hypertarget{ref-date-language}{}%
{[}4{]} Date, C.J. An architecture for high-level language database
extensions. \emph{SIGMOD}, 1976.

\leavevmode\hypertarget{ref-bsdunix}{}%
{[}5{]} McKusick, M.K., Bostic, K., Karels, M.J. and Quarterman, J.S.
The design and implementation of the 4.4 BSD operating system. Pearson
Education. 1996.

\leavevmode\hypertarget{ref-linq}{}%
{[}6{]} Meijer, E., Beckman, B. and Bierman, G. Linq: Reconciling
object, relations and XML in the .NET framework. \emph{SIGMOD}, 2006.

\leavevmode\hypertarget{ref-rigel}{}%
{[}7{]} Rowe, L.A. and Shoens, K.A. Data abstraction, views and updates
in RIGEL. \emph{SIGMOD}, 1979.

\leavevmode\hypertarget{ref-rails}{}%
{[}8{]} Ruby on rails: Ruby on rails. \url{http://www.rubyonrails.org}.

\leavevmode\hypertarget{ref-pascalr}{}%
{[}9{]} Schmidt, J.W. Some high level language constructs for data of
type relation. \emph{ACM Trans. Database Syst.} 2, 3 (Sep. 1977).

\leavevmode\hypertarget{ref-stonebraker-turing}{}%
{[}10{]} Stonebraker, M. The land sharks are on the squawk box.
\emph{Communications of the ACM}.

\leavevmode\hypertarget{ref-ingres}{}%
{[}11{]} Stonebraker, M., Held, G., Wong, E. and Kreps, P. The design
and implementation of iNGRES. \emph{ACM Transactions on Database Systems
  (TODS)}. 1, 3 (1976), 189-222.


\Chapter{ch3-techniques}{%
Techniques Everyone Should Know}{%
Introduced by Peter Bailis
}

\begin{framed}
Readings:

\begin{itemize}
\item
\href{https://scholar.google.com/scholar?cluster=102545501597608314}{Patricia
  G. Selinger, Morton M. Astrahan, Donald D. Chamberlin, Raymond A. Lorie,
  Thomas G. Price. {Access path selection in a relational database
    management system}. {SIGMOD}, 1979.}
\item
\href{https://scholar.google.com/scholar?cluster=2142924814045750364}{C.
  Mohan, Donald J. Haderle, Bruce G. Lindsay, Hamid Pirahesh, Peter M.
  Schwarz. {ARIES: A Transaction Recovery Method Supporting
    Fine-Granularity Locking and Partial Rollbacks Using Write-Ahead
    Logging}. {ACM Transactions on Database Systems}, 17(1), 1992, 94-162.}
\item
\href{https://scholar.google.com/scholar?cluster=15730220590995320737}{Jim
  Gray, Raymond A. Lorie, Gianfranco R. Putzolu, Irving L. Traiger.
  {Granularity of Locks and Degrees of Consistency in a Shared Data Base}.
  {}, IBM, September, 1975.}
\item
\href{https://scholar.google.com/scholar?cluster=9784855600346107276}{Rakesh
  Agrawal, Michael J. Carey, Miron Livny. {Concurrency Control Performance
    Modeling: Alternatives and Implications}. {ACM Transactions on Database
    Systems}, 12(4), 1987, 609-654.}
\item
\href{https://scholar.google.com/scholar?cluster=6135007404184895390}{C.
  Mohan, Bruce G. Lindsay, Ron Obermarck. {Transaction Management in the
    R* Distributed Database Management System}. {ACM Transactions on
    Database Systems}, 11(4), 1986, 378-396.}
\end{itemize}
\end{framed}

In this chapter, we present primary and near-primary sources for several
of the most important core concepts in database system design: query
planning, concurrency control, database recovery, and distribution. The
ideas in this chapter are so fundamental to modern database systems that
nearly every mature database system implementation contains them. Three
of the papers in this chapter are far and away the canonical references
on their respective topics. Moreover, in contrast with the prior
chapter, this chapter focuses on broadly applicable techniques and
algorithms rather than whole systems.

\Section{query-optimization}{Query Optimization}

Query optimization is important in relational database architecture
because it is core to enabling data-independent query processing.
Selinger et al.'s foundational paper on System R enables practical query
optimization by decomposing the problem into three distinct subproblems:
cost estimation, relational equivalences that define a search space, and
cost-based search.

The optimizer provides an estimate for the cost of executing each
component of the query, measured in terms of I/O and CPU costs. To do
so, the optimizer relies on both pre-computed statistics about the
contents of each relation (stored in the system catalog) as well as a
set of heuristics for determining the cardinality (size) of the query
output (e.g., based on estimated predicate selectivity). As an exercise,
consider these heuristics in detail: when do they make sense, and on
what inputs will they fail? How might they be improved?

Using these cost estimates, the optimizer uses a dynamic programming
algorithm to construct a plan for the query. The optimizer defines a set
of physical operators that implement a given logical operator (e.g.,
looking up a tuple using a full 'segment' scan versus an index). Using
this set, the optimizer iteratively constructs a ``left-deep'' tree of
operators that in turn uses the cost heuristics to minimize the total
amount of estimated work required to run the operators, accounting for
``interesting orders'' required by upstream consumers. This avoids
having to consider all possible orderings of operators but is still
exponential in the plan size; as we discuss in
\hyperref[ch7-queryoptimization]{Chapter 7}, modern query optimizers
still struggle with large plans (e.g., many-way joins). Additionally,
while the Selinger et al. optimizer performs compilation in advance,
other early systems, like
Ingres~{{[}\protect\hyperlink{ref-ingres}{25}{]}} interpreted the query
plan - in effect, on a tuple-by-tuple basis.

Like almost all query optimizers, the Selinger et al. optimizer is not
actually ``optimal'' - there is no guarantee that the plan that the
optimizer chooses will be the fastest or cheapest. The relational
optimizer is closer in spirit to code optimization routines within
modern language compilers (i.e., will perform a best-effort search)
rather than mathematical optimization routines (i.e., will find the best
solution). However, many of today's relational engines adopt the basic
methodology from the paper, including the use of binary operators and
cost estimation.

\Section{concurrency-control}{%
Concurrency Control
}

Our first paper on transactions, from Gray et al., introduces two
classic ideas: multi-granularity locking and multiple lock modes. The
paper in fact reads as two separate papers.

First, the paper presents the concept of multi-granularity locking. The
problem here is simple: given a database with a hierarchical structure,
how should we perform mutual exclusion? When should we lock at a coarse
granularity (e.g., the whole database) versus a finer granularity (e.g.,
a single record), and how can we support concurrent access to different
portions of the hierarchy at once? While Gray et al.'s hierarchical
layout (consisting of databases, areas, files, indexes, and records)
differs slightly from that of a modern database system, all but the most
rudimentary database locking systems adapt their proposals today.

Second, the paper develops the concept of multiple degrees of isolation.
As Gray et al. remind us, a goal of concurrency control is to maintain
data that is ``consistent'' in that it obeys some logical assertions.
Classically, database systems used serializable transactions as a means
of enforcing consistency: if individual transactions each leave the
database in a ``consistent'' state, then a serializable execution
(equivalent to some serial execution of the transactions) will guarantee
that all transactions observe a ``consistent'' state of the
database~{{[}\protect\hyperlink{ref-eswaran76}{5}{]}}. Gray et al.'s
``Degree 3'' protocol describes the classic (strict) ``two-phase
locking'' (2PL), which guarantees serializable execution and is a major
concept in transaction processing.

However, serializability is often considered too expensive to enforce.
To improve performance, database systems often instead execute
transactions using non-serializable isolation. In the paper here,
holding locks is expensive: waiting for a lock in the case of a conflict
takes time, and, in the event of a deadlock, might take forever (or
cause aborts). Therefore, as early as 1973, database systems such as IMS
and System R began to experiment with non-serializable policies. In a
lock-based concurrency control system, these policies are implemented by
holding locks for shorter durations. This allows greater concurrency,
may lead to fewer deadlocks and system-induced aborts, and, in a
distributed setting, may permit greater availability of operation.

In the second half of this paper, Gray et al. provide a rudimentary
formalization of the behavior of these lock-based policies. Today, they
are prevalent; as we discuss in \hyperref[ch6-isolation]{Chapter 6},
non-serializable isolation is the default in a majority of commercial
and open source RDBMSs, and some RDBMSs do not offer serializability at
all. Degree 2 is now typically called Repeatable Read isolation and
Degree 1 is now called Read Committed isolation, while Degree 0 is
infrequently used~{{[}\protect\hyperlink{ref-ansi-critique}{1}{]}}. The
paper also discusses the important notion of recoverability: policies
under which a transaction can be aborted (or ``undone'') without
affecting other transactions. All but Degree 0 transactions satisfy this
property.

A wide range of alternative concurrency control mechanisms followed Gray
et al.'s pioneering work on lock-based serializability. As hardware,
application demands, and access patterns have changed, so have
concurrency control subsystems. However, one property of concurrency
control remains a near certainty: there is no unilateral ``best''
mechanism in concurrency control. The optimal strategy is
workload-dependent. To illustrate this point, we've included a study
from Agrawal, Carey, and Livny. Although dated, this paper's methodology
and broad conclusions remain on target. It's a great example of
thoughtful, implementation-agnostic performance analysis work that can
provide valuable lessons over time.

Methodologically, the ability to perform so-called ``back of the
envelope'' calculations is a valuable skill: quickly estimating a metric
of interest using crude arithmetic to arrive at an answer within an
order of magnitude of the correct value can save hours or even years of
systems implementation and performance analysis. This is a long and
useful tradition in database systems, from the ``Five Minute
Rule''~{{[}\protect\hyperlink{ref-fiveminute}{12}{]}} to Google's
``Numbers Everyone Should
Know''~{{[}\protect\hyperlink{ref-google-numbers}{4}{]}}. While some of
the lessons drawn from these estimates are
transient~{{[}\protect\hyperlink{ref-five-twenty}{8},
  \protect\hyperlink{ref-five-ten}{10}{]}}, often the conclusions provide
long-term lessons.

However, for analysis of complex systems such as concurrency control,
simulation can be a valuable intermediate step between back of the
envelope and full-blown systems benchmarking. The Agrawal study is an
example of this approach: the authors use a carefully designed system
and user model to simulate locking, restart-based, and optimistic
concurrency control.

Several aspects of the evaluation are particularly valuable. First,
there is a ``crossover'' point in almost every graph: there aren't clear
winners, as the best-performing mechanism depends on the workload and
system configuration. In contrast, virtually every performance study
without a crossover point is likely to be uninteresting. If a scheme
``always wins,'' the study should contain an analytical analysis, or,
ideally, a proof of why this is the case. Second, the authors consider a
wide range of system configurations; they investigate and discuss almost
all parameters of their model. Third, many of the graphs exhibit
non-monotonicity (i.e., don't always go up and to the right); this a
product of thrashing and resource limitations. As the authors
illustrate, an assumption of infinite resources leads to dramatically
different conclusions. A less careful model that made this assumption
implicit would be much less useful.

Finally, the study's conclusions are sensible. The primary cost of
restart-based methods is ``wasted'' work in the event of conflicts. When
resources are plentiful, speculation makes sense: wasted work is less
expensive, and, in the event of infinite resources, it is free. However,
in the event of more limited resources, blocking strategies will consume
fewer resources and offer better overall performance. Again, there is no
unilaterally optimal choice. However, the paper's concluding remarks
have proven prescient: computing resources are still scarce, and, in
fact, few commodity systems today employ entirely restart-based methods.
However, as technology ratios - disk, network, CPU speeds - continue to
change, re-visiting this trade-off is valuable.

\Section{database-recovery}{%
Database Recovery
}

Another major problem in transaction processing is maintaining
durability: the effects of transaction processing should survive system
failures. A near-ubiquitous technique for maintaining durability is to
perform logging: during transaction execution, transaction operations
are stored on fault-tolerant media (e.g., hard drives or SSDs) in a log.
Everyone working in data systems should understand how write-ahead
logging works, preferably in some detail.

The canonical algorithm for implementing a ``No Force, Steal'' WAL-based
recovery manager is IBM's ARIES algorithm, the subject of our next
paper. (Senior database researchers may tell you that very similar ideas
were invented at the same time at places like Tandem and Oracle.) In
ARIES, the database need not write dirty pages to disk at commit time
(``No Force''), and the database can flush dirty pages to disk at any
time (``Steal'')~{{[}\protect\hyperlink{ref-haerder-reuter}{15}{]}};
these policies allow high performance and are present in almost every
commercial RDBMS offering but in turn add complexity to the database.
The basic idea in ARIES is to perform crash recovery in three stages.
First, ARIES performs an analysis phase by replaying the log forwards in
order to determine which transactions were in progress at the time of
the crash. Second, ARIES performs a redo stage by (again) replaying the
log and (this time) performing the effects of any transactions that were
in progress at the time of the crash. Third, ARIES performs an undo
stage by playing the log backwards and undoing the effect of uncommitted
transactions. Thus, the key idea in ARIES is to ``repeat history'' to
perform recovery; in fact, the undo phase can execute the same logic
that is used to abort a transaction during normal operation.

ARIES should be a fairly simple paper but it is perhaps the most
complicated paper in this collection. In graduate database courses, this
paper is a rite of passage. However, this material is fundamental, so it
is important to understand. Fortunately, Ramakrishnan and Gehrke's
undergraduate textbook~{{[}\protect\hyperlink{ref-cowbook}{22}{]}} and a
survey paper by Michael
Franklin~{{[}\protect\hyperlink{ref-franklin-aries}{7}{]}} each provide
a milder treatment. The full ARIES paper we have included here is
complicated significantly by its diversionary discussions of the
drawbacks of alternative design decisions along the way. On the first
pass, we encourage readers to ignore this material and focus solely on
the ARIES approach. The drawbacks of alternatives are important but
should be saved for a more careful second or third read. Aside from its
organization, the discussion of ARIES protocols is further complicated
by discussions of managing internal state like indexes (i.e., nested top
actions and logical undo logging --- the latter of which is also used in
exotic schemes like Escrow
transactions~{{[}\protect\hyperlink{ref-escrow}{21}{]}}) and techniques
to minimize downtime during recovery. In practice, it is important for
recovery time to appear as short as possible; this is tricky to achieve.

\Section{distribution}{%
Distribution
}

Our final paper in this chapter concerns transaction execution in a
distributed environment. This topic is especially important today, as an
increasing number of databases are distributed - either replicated, with
multiple copies of data on different servers, or partitioned, with data
items stored on disjoint servers (or both). Despite offering benefits to
capacity, durability, and availability, distribution introduces a new
set of concerns. Servers may fail and network links may be unreliable.
In the absence of failures, network communication may be costly.

We concentrate on one of the core techniques in distributed transaction
processing: atomic commitment (AC). Very informally, given a transaction
that executes on multiple servers (whether multiple replicas, multiple
partitions, or both), AC ensures that the transaction either commits or
aborts on all of them. The classic algorithm for achieving AC dates to
the mid-1970s and is called Two-Phase Commit (2PC; not to be confused
with 2PL above!)~{{[}\protect\hyperlink{ref-gray-2pc}{9},
  \protect\hyperlink{ref-lampson-2pc}{18}{]}}. In addition to providing a
good overview of 2PC and interactions between the commit protocol and
the WAL, the paper here contains two variants of AC that improve its
performance. The Presumed Abort variant allows processes to avoid
forcing an abort decision to disk or acknowledge aborts, reducing disk
utilization and network traffic. The Presumed Commit optimization is
similar, optimizing space and network traffic when more transactions
commit. Note the complexity of the interactions between the 2PC
protocol, local storage, and the local transaction manager; the details
are important, and correct implementation of these protocols can be
challenging.

The possibility of failures substantially complicates AC (and most
problems in distributed computing). For example, in 2PC, what happens if
a coordinator and participant both fail after all participants have sent
their votes but before the coordinator has heard from the failed
participant? The remaining participants will not know whether or to
commit or abort the transaction: did the failed participant vote YES or
vote NO? The participants cannot safely continue. In fact, \emph{any}
implementation of AC may block, or fail to make progress, when operating
over an unreliable
network~{{[}\protect\hyperlink{ref-bernstein-book}{2}{]}}. Coupled with
a serializable concurrency control mechanism, blocking AC means
throughput may stall. As a result, a related set of AC algorithms
examined AC under relaxed assumptions regarding both the network (e.g.,
by assuming a synchronous
network)~{{[}\protect\hyperlink{ref-3pc}{24}{]}} and the information
available to servers (e.g., by making use of a ``failure detector'' that
determines when nodes fail)~{{[}\protect\hyperlink{ref-ac-srds}{14}{]}}.

Finally, many readers may be familiar with the closely related problem
of consensus or may have heard of consensus implementations such as the
Paxos algorithm. In consensus, any proposal can be chosen, as long as
all processes eventually will agree on it. (In contrast, in AC, any
individual participant can vote NO, after which all participants must
abort.) This makes consensus an ``easier'' problem than
AC~{{[}\protect\hyperlink{ref-ac-consensus}{13}{]}}, but, like AC, any
implementation of consensus can also block in certain
scenarios~{{[}\protect\hyperlink{ref-flp}{6}{]}}. In modern distributed
databases, consensus is often used as the basis for replication, to
ensure replicas apply updates in the same order, an instance of
state-machine replication (see Schneider's
tutorial~{{[}\protect\hyperlink{ref-statemachine}{23}{]}}). AC is often
used to execute transactions that span multiple partitions. Paxos by
Lamport~{{[}\protect\hyperlink{ref-paxos}{17}{]}} is one of the earliest
(and most famous, due in part to a presentation that rivals ARIES in
complexity) implementations of consensus. However, the Viewstamped
Replication~{{[}\protect\hyperlink{ref-vrr}{19}{]}} and
Raft~{{[}\protect\hyperlink{ref-raft}{20}{]}},
ZAB~{{[}\protect\hyperlink{ref-zab}{16}{]}}, and
Multi-Paxos~{{[}\protect\hyperlink{ref-multipaxos}{3}{]}} algorithms may
be more helpful in practice. This is because these algorithms implement
a distributed log abstraction (rather than a 'consensus object' as in
the original Paxos paper).

Unfortunately, the database and distributed computing communities are
somewhat separate. Despite shared interests in replicated data, transfer
of ideas between the two were limited for many years. In the era of
cloud and Internet-scale data management, this gap has shrunk. For
example, Gray and Lamport collaborated in 2006 on Paxos
Commit~{{[}\protect\hyperlink{ref-paxoscommit}{11}{]}}, an interesting
algorithm combining AC and Lamport's Paxos. There is still much to do in
this intersection, and the number of ``techniques everyone should know''
in this space has grown.

\section*{References}

\leavevmode\hypertarget{ref-ansi-critique}{}%
{[}1{]} Berenson, H., Bernstein, P., Gray, J., Melton, J., O'Neil, E.
and O'Neil, P. A critique of ANSI SQL isolation levels. \emph{SIGMOD},
1995.

\leavevmode\hypertarget{ref-bernstein-book}{}%
{[}2{]} Bernstein, P., Hadzilacos, V. and Goodman, N. Concurrency
control and recovery in database systems. Addison-Wesley New York. 1987.

\leavevmode\hypertarget{ref-multipaxos}{}%
{[}3{]} Chandra, T.D., Griesemer, R. and Redstone, J. Paxos made live:
An engineering perspective. \emph{PODC}, 2007.

\leavevmode\hypertarget{ref-google-numbers}{}%
{[}4{]} Dean, J. Designs, lessons and advice from building large
distributed systems (keynote). \emph{LADIS}, 2009.

\leavevmode\hypertarget{ref-eswaran76}{}%
{[}5{]} Eswaran, K.P., Gray, J.N., Lorie, R.A. and Traiger, I.L. The
notions of consistency and predicate locks in a database system.
\emph{Communications of the ACM}. 19, 11 (1976), 624-633.

\leavevmode\hypertarget{ref-flp}{}%
{[}6{]} Fischer, M.J., Lynch, N.A. and Paterson, M.S. Impossibility of
distributed consensus with one faulty process. \emph{Journal of the ACM
  (JACM)}. 32, 2 (1985), 374-382.

\leavevmode\hypertarget{ref-franklin-aries}{}%
{[}7{]} Franklin, M.J. Concurrency control and recovery. \emph{The
  Computer Science and Engineering Handbook}. (1997), 1-58-1077.

\leavevmode\hypertarget{ref-five-twenty}{}%
{[}8{]} Graefe, G. The five-minute rule twenty years later, and how
flash memory changes the rules. \emph{DaMoN}, 2007.

\leavevmode\hypertarget{ref-gray-2pc}{}%
{[}9{]} Gray, J. Notes on data base operating systems. \emph{Operating
  systems: An advanced course}. Springer Berlin Heidelberg. 393-481.

\leavevmode\hypertarget{ref-five-ten}{}%
{[}10{]} Gray, J. and Graefe, G. The five-minute rule ten years later,
and other computer storage rules of thumb. \emph{ACM SIGMOD Record}. 26,
4 (1997), 63-68.

\leavevmode\hypertarget{ref-paxoscommit}{}%
{[}11{]} Gray, J. and Lamport, L. Consensus on transaction commit.
\emph{ACM Transactions on Database Systems (TODS)}. 31, 1 (Mar. 2006),
133-160.

\leavevmode\hypertarget{ref-fiveminute}{}%
{[}12{]} Gray, J. and Putzolu, F. The 5 minute rule for trading memory
for disc accesses and the 10 byte rule for trading memory for cPU time.
\emph{SIGMOD}, 1987.

\leavevmode\hypertarget{ref-ac-consensus}{}%
{[}13{]} Guerraoui, R. Revisiting the relationship between non-blocking
atomic commitment and consensus. \emph{WDAG}, 1995.

\leavevmode\hypertarget{ref-ac-srds}{}%
{[}14{]} Guerraoui, R., Larrea, M. and Schiper, A. Non blocking atomic
commitment with an unreliable failure detector. \emph{SRDS}, 1995.

\leavevmode\hypertarget{ref-haerder-reuter}{}%
{[}15{]} Haerder, T. and Reuter, A. Principles of transaction-oriented
database recovery. \emph{ACM Computing Surveys (CSUR)}. 15, 4 (1983),
287-317.

\leavevmode\hypertarget{ref-zab}{}%
{[}16{]} Junqueira, F.P., Reed, B.C. and Serafini, M. Zab:
High-performance broadcast for primary-backup systems. \emph{DSN}, 2011.

\leavevmode\hypertarget{ref-paxos}{}%
{[}17{]} Lamport, L. The part-time parliament. \emph{ACM Transactions on
  Computer Systems (TOCS)}. 16, 2 (1998), 133-169.

\leavevmode\hypertarget{ref-lampson-2pc}{}%
{[}18{]} Lampson, B. and Sturgis, H. Crash recovery in a distributed
data storage system. Xerox PARC Technical Report. 1979.

\leavevmode\hypertarget{ref-vrr}{}%
{[}19{]} Liskov, B. and Cowling, J. Viewstamped replication revisited.
MIT; MIT-CSAIL-TR-2012-021. 2012.

\leavevmode\hypertarget{ref-raft}{}%
{[}20{]} Ongaro, D. and Ousterhout, J. In search of an understandable
consensus algorithm. \emph{USENIX ATC}, 2014.

\leavevmode\hypertarget{ref-escrow}{}%
{[}21{]} O'Neil, P.E. The escrow transactional method. \emph{ACM
  Transactions on Database Systems}. 11, 4 (1986), 405-430.

\leavevmode\hypertarget{ref-cowbook}{}%
{[}22{]} Ramakrishnan, R. and Gehrke, J. Database management systems.
McGraw Hill. 2000.

\leavevmode\hypertarget{ref-statemachine}{}%
{[}23{]} Schneider, F.B. Implementing fault-tolerant services using the
state machine approach: A tutorial. \emph{ACM Computing Surveys (CSUR)}.
22, 4 (1990), 299-319.

\leavevmode\hypertarget{ref-3pc}{}%
{[}24{]} Skeen, D. Nonblocking commit protocols. \emph{SIGMOD}, 1981.

\leavevmode\hypertarget{ref-ingres}{}%
{[}25{]} Stonebraker, M., Held, G., Wong, E. and Kreps, P. The design
and implementation of iNGRES. \emph{ACM Transactions on Database Systems
  (TODS)}. 1, 3 (1976), 189-222.


\Chapter{ch4-newdbms}{%
New DBMS Architectures
}{%
Introduced by Michael Stonebraker
}

\begin{framed}
Readings:
\begin{itemize}
\item
\href{https://scholar.google.com/scholar?cluster=12924804892742402591}{Mike
  Stonebraker, Daniel J. Abadi, Adam Batkin, Xuedong Chen, Mitch
  Cherniack, Miguel Ferreira, Edmond Lau, Amerson Lin, Sam Madden,
  Elizabeth O'Neil, Pat O'Neil, Alex Rasin, Nga Tran, Stan Zdonik.
  {C-store: A Column-oriented DBMS}. {SIGMOD}, 2005.}
\item
\href{https://scholar.google.com/scholar?cluster=14161764654889427045}{Cristian
  Diaconu, Craig Freedman, Erik Ismert, Per-Ake Larson, Pravin Mittal,
  Ryan Stonecipher, Nitin Verma, Mike Zwilling. {Hekaton: SQL Server's
    Memory-optimized OLTP Engine}. {SIGMOD}, 2013.}
\item
\href{https://scholar.google.com/scholar?cluster=12931776946707721868}{Stavros
  Harizopoulos, Daniel J. Abadi, Samuel Madden, Michael Stonebraker. {OLTP
    Through the Looking Glass, and What We Found There}. {SIGMOD}, 2008.}
\end{itemize}
\end{framed}

Probably the most important thing that has happened in the DBMS
landscape is the death of ``one size fits all''. Until the early 2000's
the traditional disk-based row-store architecture was omni-present. In
effect, the commercial vendors had a hammer and everything was a nail.

In the last fifteen years, there have been several major upheavals,
which we discuss in turn.

First, the community realized that column stores are dramatically
superior to row stores in the data warehouse marketplace. Data
warehouses found early acceptance in customer facing retail environments
and quickly spread to customer facing data in general. Warehouses
recorded historical information on customer transactions. In effect,
this is the who-what-why-when-where of each customer interaction.

The conventional wisdom is to structure a data warehouse around a
central Fact table in which this transactional information is recorded.
Surrounding this are dimension tables which record information that can
be factored out of the Fact table. In a retail scenario one has
dimension tables for Stores, Customers, Products and Time. The result is
a so-called star schema~{{[}\protect\hyperlink{ref-kimball-book}{3}{]}}.
If stores are grouped into regions, then there may be multiple levels of
dimension tables and a snowflake schema results.

The key observation is that Fact tables are generally ``fat'' and often
contain a hundred or more attributes. Obviously, they also ``long''
since there are many, many facts to record. In general, the queries to a
data warehouse are a mix of repeated requests (produce a monthy sales
report by store) and ``ad hoc'' ones. In a retail warehouse, for
example, one might want to know what is selling in the Northeast when a
snowstorm occurs and what is selling along the Atlantic seaboard during
hurricanes.

Moreover, nobody runs a select * query to fetch all of the rows in the
Fact table. Instead, they invariably specify an aggregate, which
retrieves a half-dozen attributes out of the 100 in the table. The next
query retrieves a different set, and there is little-to-no locality
among the filtering criteria.

In this use case, it is clear that a column store will move a factor of
16 less data from the disk to main memory than a row store will (6
columns versus 100). Hence, it has an unfair advantage. Furthermore,
consider a storage block. In a column store, there is a single attribute
on that block, while a row store will have 100. Compression will clearly
work better on one attribute than on 100. In addition, row stores have a
header on the front of each record (in SQLServer it is apparently 16
bytes). In contrast, column stores are careful to have no such header.

Lastly, a row-based executor has an inner loop whereby a record is
examined for validity in the output. Hence, the overhead of the inner
loop, which is considerable, is paid per record examined. In contrast,
the fundamental operation of a column store is to retrieve a column and
pick out the qualifying items. As such, the inner-loop overhead is paid
once per column examined and not once per row examined. As such a column
executor is way more efficient in CPU time and retrieves way less data
from the disk. In most real-world environments, column stores are 50-100
times faster than row stores.

Early column stores included Sybase
IQ~{{[}\protect\hyperlink{ref-sybase}{5}{]}}, which appeared in the
1990s, and MonetDB~{{[}\protect\hyperlink{ref-monetdb}{2}{]}}. However,
the technology dates to the
1970s~{{[}\protect\hyperlink{ref-earlycolumn1}{1},
  \protect\hyperlink{ref-earlycolumn2}{4}{]}}. In the 2000's
C-Store/Vertica appeared as well-financed startup with a high
performance implementation. Over the next decade the entire data
warehouse market morphed from a row-store world to a column store world.
Arguably, Sybase IQ could have done the same thing somewhat earlier, if
Sybase had invested more aggressively in the technology and done a
multi-node implementation. The advantages of a column executor are
persuasively discussed in~{{[}\protect\hyperlink{ref-monetdb}{2}{]}},
although it is ``down in the weeds'' and hard to read.

The second major sea change was the precipitous decline in main memory
prices. At the present time, one can buy a 1Terabyte for perhaps
\$25,000, and a high performance computing cluster with a few terabytes
can be bought for maybe \$100K. The key insight is that OLTP data bases
are just not that big. One terabyte is a very big OLTP data base, and is
a candidate for main memory deployment. As noted in the looking glass
paper in this section, one does not want to run a disk-based row store
when data fits in main memory - the overhead is just way too high.

In effect, the OLTP marketplace is now becoming a main memory DBMS
marketplace. Again, traditional disk-based row stores are just not
competitive. To work well, new solutions are needed for concurrency
control, crash recovery, and multi-threading, and I expect OLTP
architectures to evolve over the next few years.

My current best guess is that nobody will use traditional two phase
locking. Techniques based on timestamp ordering or multiple versions are
likely to prevail. The third paper in this section discusses Hekaton,
which implements a state-of-the art MVCC scheme.

Crash recovery must also be dealt with. In general, the solution
proposed is replication, and on-line failover, which was pioneered by
Tandem two decades ago. The traditional wisdom is to write a log, move
the log over the network, and then roll forward at the backup site. This
active-passive architecture has been shown
in~{{[}\protect\hyperlink{ref-commandlogging}{6}{]}} to be a factor of 3
inferior to an active-active scheme where the transactions is simply run
at each replica. If one runs an active-active scheme, then one must
ensure that transactions are run in the same order at each replica.
Unfortunately, MVCC does not do this. This has led to interest in
deterministic concurrency control schemes, which are likely to be wildly
faster in an end-to-end system that MVCC.

In any case, OLTP is going to move to main memory deployment, and a new
class of main memory DBMSs is unfolding to support this use case.

The third phenomenon that has unfolded is the ``no SQL'' movement. In
essence, there are 100 or so DBMSs, which support a variety of data
models and have the following two characteristics:

\begin{enumerate}
  \item
  ``Out of box'' experience. They are easy for a programmer to get going
  and do something productive. RDBMSs, in contrast, are very
  heavyweight, requiring a schema up front.
  \item
  Support for semi-structured data. If every record can have values for
  different attributes, then a traditional row store will have very,
  very wide tuples, and be very sparse, and therefore inefficient.
\end{enumerate}

鈥═his is a wake-up call to the commercial vendors to make systems that
are easier to use and support semi-structured data types, such as JSON.
In general, I expect the No SQL market to merge with the SQL market over
time as RDBMSs react to the two points noted above.

The fourth sea change is the emergence of the Hadoop/HDFS/Spark
environment, which is discussed in \hyperref[ch6-isolation]{Chapter 6}.

\section*{References}

\leavevmode\hypertarget{ref-earlycolumn1}{}%
{[}1{]} Batory, D.S. On searching transposed files. \emph{ACM
  Transactions on Database Systems (TODS)}. 4, 4 (Dec. 1979).

\leavevmode\hypertarget{ref-monetdb}{}%
{[}2{]} Boncz, P.A., Zukowski, M. and Nes, N. MonetDB/X100:
Hyper-pipelining query execution. \emph{CIDR}, 2005.

\leavevmode\hypertarget{ref-kimball-book}{}%
{[}3{]} Kimball, R. and Ross, M. The data warehouse toolkit: The
complete guide to dimensional modeling. John Wiley \& Sons. 2011.

\leavevmode\hypertarget{ref-earlycolumn2}{}%
{[}4{]} Lorie, R. and Symonds, A. A relational access method for
interactive applications. \emph{Courant Computer Science Symposia, Vol.
  6: Data Base Systems}. (1971).

\leavevmode\hypertarget{ref-sybase}{}%
{[}5{]} MacNicol, R. and French, B. Sybase iQ multiplex-designed for
analytics. \emph{VLDB}, 2004.

\leavevmode\hypertarget{ref-commandlogging}{}%
{[}6{]} Malviya, N., Weisberg, A., Madden, S. and Stonebraker, M.
Rethinking main memory OLTP recovery. \emph{ICDE}, 2014.



\Chapter{ch5-dataflow}{%
Large-Scale Dataflow Engines
}{%
Introduced by Peter Bailis
}

\begin{framed}
Readings:
\begin{itemize}
\item
\href{https://scholar.google.com/scholar?cluster=10940266603640308767}{Jeff
  Dean and Sanjay Ghemawat. {MapReduce: Simplified Data Processing on
    Large Clusters}. {OSDI}, 2004.}
\item
\href{https://scholar.google.com/scholar?cluster=3662601067977846800}{Yuan
  Yu, Michael Isard, Dennis Fetterly, Mihai Budiu. {DryadLINQ: A System
    for General-Purpose Distributed Data-Parallel Computing Using a
    High-Level Language}. {OSDI}, 2008.}
\end{itemize}
\end{framed}

Of the many developments in data management over the past decade,
MapReduce and subsequent large-scale data processing systems have been
among the most disruptive and the most controversial. Cheap commodity
storage and rising data volumes led many Internet service vendors to
discard conventional database systems and data warehouses and build
custom, home-grown engines instead. Google's string of publications on
their large-scale systems, including Google File
System~{{[}\protect\hyperlink{ref-gfs}{11}{]}}, MapReduce,
Chubby~{{[}\protect\hyperlink{ref-chubby}{6}{]}}, and
BigTable~{{[}\protect\hyperlink{ref-bigtable}{7}{]}}, are perhaps the
most famous and influential in the market. In almost all cases, these
new, homegrown systems implemented a small subset of the features found
in conventional databases, including high-level languages, query
optimizers, and efficient execution strategies. However, these systems
and the resulting open source Hadoop ecosystem proved highly popular
with many developers. This led to considerable investment, marketing,
research interest, and development on these platforms, which, today are
in flux, but, as an ecosystem, have come to resemble traditional data
warehouses---with some important modifications. We reflect on these
trends here.

\Section{history-and-successors}{%
History and Successors
}

Our first reading is the original Google MapReduce paper from 2004.
MapReduce was a library built for simplifying parallel, distributed
computation over distributed data at Google's scale---particularly, the
batch rebuild of web search indexes from crawled pages. It is unlikely
that, at the time, a traditional data warehouse could have handled this
workload. However, compared to a conventional data warehouse, MapReduce
provides a very low-level interface (two-stage dataflow) that is closely
tied to a fault-tolerant execution strategy (intermediate
materialization between two-stage dataflow). Equally importantly,
MapReduce was designed as a library for parallel programming rather than
an end-to-end data warehousing solution; for example, MapReduce
delegates storage to Google File System. At the time, members of the
database community decried the architecture as simplistic, inefficient,
and of limited use~{{[}\protect\hyperlink{ref-mr-majorstep}{8}{]}}.

While the original MapReduce paper was released in 2003, there was
relatively little additional activity external to Google until 2006,
when Yahoo! open-sourced the Hadoop MapReduce implementation.
Subsequently, there was an explosion of interest: within a year, a range
of projects including Dryad
(Microsoft)~{{[}\protect\hyperlink{ref-dryad}{15}{]}}, Hive
(Facebook)~{{[}\protect\hyperlink{ref-hive}{26}{]}}, Pig
(Yahoo)~{{[}\protect\hyperlink{ref-pig}{22}{]}} were all under
development. These systems, which we will call post-MapReduce systems,
gained considerable traction with developers---who were largely
concentrated in Silicon Valley---as well as serious VC investment. A
multitude of research spanning the systems, databases, and networking
communities investigated issues including scheduling, straggler
mitigation, fault tolerance, UDF query optimization, and alternative
programming models~{{[}\protect\hyperlink{ref-fnt-mr}{5}{]}}.

Almost immediately, post-MapReduce systems expanded their interface and
functionality to include more sophisticated declarative interfaces,
query optimization strategies, and efficient runtimes. Today's
post-MapReduce systems have come to implement a growing proportion of
the feature set of conventional RDBMSs. The latest generation of data
processing engines such as
Spark~{{[}\protect\hyperlink{ref-spark}{27}{]}},
F1~{{[}\protect\hyperlink{ref-f1}{24}{]}},
Impala~{{[}\protect\hyperlink{ref-impala}{16}{]}},
Tez~{{[}\protect\hyperlink{ref-tez}{3}{]}},
Naiad~{{[}\protect\hyperlink{ref-naiad}{21}{]}},
Flink/Stratosphere~{{[}\protect\hyperlink{ref-stratosphere}{1}{]}},
AsterixDB~{{[}\protect\hyperlink{ref-asterixdb}{2}{]}}, and
Drill~{{[}\protect\hyperlink{ref-drill}{14}{]}} frequently i) expose
higher-level query languages such as SQL, ii) more advanced execution
strategies, including the ability to process general graphs of
operators, and iii) use indexes and other functionality of structured
input data sources when possible. In the Hadoop ecosystem, dataflow
engines have become the substrate for a suite of higher-level
functionality and declarative interfaces, including
SQL~{{[}\protect\hyperlink{ref-sparksql}{4},
  \protect\hyperlink{ref-hive}{26}{]}}, graph
processing~{{[}\protect\hyperlink{ref-graphx}{13},
  \protect\hyperlink{ref-pregel}{19}{]}}, and machine
learning~{{[}\protect\hyperlink{ref-systemml}{12},
  \protect\hyperlink{ref-mllib}{25}{]}}. There is also increasing interest
in stream processing functionality, revisiting many of the concepts
pioneered in the database community in the 2000s. A growing commercial
and open source ecosystem has developed ``connectors'' to various
structured and semi-structured data sources, catalog functionality
(e.g., HCatalog), and data serving and limited transactional
capabilities (e.g., HBase). Much of this functionality, such as the
typical query optimizers in these frameworks, is rudimentary compared to
many mature commercial databases but is quickly evolving.

DryadLINQ, our second selected reading for this section, is perhaps most
interesting for its interface: a set of embedded language bindings for
data processing that integrates seamlessly with Microsoft's .NET LINQ to
provide a parallelized collections library. DryadLINQ executes queries
via the earlier Dryad system~{{[}\protect\hyperlink{ref-dryad}{15}{]}},
which implemented a runtime for arbitrary dataflow graphs using a
replay-based fault tolerance. While DryadLINQ still restricts
programmers to a set of side-effect free dataset transformations
(including ``SQL-like'' operations), it presents a considerably
higher-level interface than Map Reduce. DryadLINQ's language
integration, lightweight fault tolerance, and basic query optimization
techniques proved influential in later dataflow systems, including
Apache Spark~{{[}\protect\hyperlink{ref-spark}{27}{]}} and Microsoft's
Naiad~{{[}\protect\hyperlink{ref-naiad}{21}{]}}.

\Section{impact-and-legacy}{%
Impact and Legacy
}

There are at least three lasting impacts of the MapReduce phenomenon
that might not have occurred otherwise. These ideas are - like
distributed dataflow itself - not necessarily novel, but the ecosystem
of post-MapReduce dataflow and storage systems have broadly increased
their impact:

\begin{enumerate}[label=\arabic*.)]
\item
\emph{Schema flexibility.} Perhaps most importantly, traditional
data warehouse systems are walled gardens: ingested data is pristine,
curated, and has structure. In contrast, MapReduce systems process
arbitrarily structured data, whether clean or dirty, curated or not.
There is no loading step. This means users can store data first and
consider what to do with it later. Coupled with the fact that storage
(e.g., in the Hadoop File System) is considerably cheaper than in a
traditional data warehouse, users can afford to retain data for longer
and longer. This is a major shift from traditional data warehouses and
is a key factor behind the rise and gathering of ``Big Data.'' A growing
number of storage formats (e.g., Avro, Parquet, RCFile) marry
semi-structured data and advances in storage such as columnar layouts.
In contrast with XML, this newest wave of semi-structured data is even
more flexible. As a result, extract-transform-load (ETL) tasks are major
workload for post-MapReduce engines. It is difficult to overstate the
impact of schema flexibility on the modern practice of data management
at all levels, from analyst to programmer and analytics vendor, and we
believe it will become even more important in the future. However, this
heterogeneity is not free: curating such ``data lakes'' is expensive
(much more than storage) and is a topic we consider in depth in
\hyperref[ch12-dataintegration]{Chapter 12}.
\item
\emph{Interface flexibility.} Today, most all users interact with
Big Data engines in SQL-like languages. However, these engines also
allow users to program using a combination of paradigms. For example, an
organization might use imperative code to perform file parsing, SQL to
project a column, and machine learning subroutines to cluster the
results - all within a single framework. Tight, idiomatic language
integration as in DryadLINQ is commonplace, further improving
programmability. While traditional database engines historically
supported user-defined functions, these new engines' interfaces make
user-defined computations simpler to express and also make it easier to
integrate the results of user-defined computations with the results of
queries expressed using traditional relational constructs like SQL.
Interface flexibility and integration is a strong selling point for data
analytics offerings; the ability to combine ETL, analytics, and
post-processing in a single system is remarkably convenient to
programmers --- but not necessarily to users of traditional BI tools,
which make use of traditional JDBC interfaces.
\item
\emph{Architectural flexibility.} A common critique of RDBMSs is
that their architecture is too tightly coupled: storage, query
processing, memory management, transaction processing, and so on are
closely intertwined, with a lack of clear interfaces between them in
practice. In contrast, as a result of its bottom-up development, the
Hadoop ecosystem has effectively built a data warehouse as a series of
modules. Today, organizations can write and run programs against the raw
file system (e.g., HDFS), any number of dataflow engines (e.g., Spark),
using advanced analytics packages (e.g.,
GraphLab~{{[}\protect\hyperlink{ref-graphlab}{18}{]}}, Parameter
Server~{{[}\protect\hyperlink{ref-parameterserver}{17}{]}}), or via SQL
(e.g., Impala~{{[}\protect\hyperlink{ref-impala}{16}{]}}). This
flexibility adds performance overhead, but the ability to mix and match
components and analytics packages is unprecedented at this scale. This
architectural flexibility is perhaps most interesting to systems
builders and vendors, who have additional degrees of freedom in
designing their infrastructure offerings.
\end{enumerate}

To summarize, a dominant theme in today's distributed data management
infrastructure is flexibility and heterogeneity: of storage formats, of
computation paradigms, and of systems implementations. Of these, storage
format heterogeneity is probably the highest impact by an order of
magnitude or more, simply because it impacts novices, experts, and
architects alike. In contrast, heterogeneity of computation paradigms
most impacts experts and architects, while heterogeneity of systems
implementations most impacts architects. All three are relevant and
exciting developments for database research, with lingering questions
regarding market impact and longevity.

\Section{looking-ahead}{%
Looking Ahead
}

In a sense, MapReduce was a short-lived, extreme architecture that blew
open a design space. The architecture was simple and highly scalable,
and its success in the open source domain led many to realize that there
was demand for alternative solutions and the principle of flexibility
that it embodied (not to mention a market opportunity for cheaper data
warehousing solutions based on open source). The resulting interest is
still surprising to many and is due to many factors, including community
zeitgeist, clever marketing, economics, and technology shifts. It is
interesting to consider which differences between these new systems and
RDBMSs are fundamental and which are due to engineering improvements.

Today, there is still debate about the appropriate architecture for
large-scale data processing. As an example, Rasmussen et al. provide a
strong argument for why intermediate fault tolerance is not necessary
except in very large (100+ node)
clusters~{{[}\protect\hyperlink{ref-themismr}{23}{]}}. As another
example, McSherry et al. have colorfully illustrated that many workloads
can be efficiently processed using a single server (or thread!),
eliminating the need for distribution at
all~{{[}\protect\hyperlink{ref-cost}{20}{]}}. Recently, systems such as
the GraphLab project~{{[}\protect\hyperlink{ref-graphlab}{18}{]}}
suggested that domain-specific systems are necessary for performance;
later work, including Grail~{{[}\protect\hyperlink{ref-grail}{9}{]}} and
GraphX~{{[}\protect\hyperlink{ref-graphx}{13}{]}}, argued this need not
be the case. A further wave of recent proposals have also suggested new
interfaces and systems for stream processing, graph processing,
asynchronous programming, and general-purpose machine learning. Are
these specialized systems actually required, or can one analytics engine
rule them all? Time will tell, but I perceive a push towards
consolidation.

Finally, we would be remiss not to mention Spark, which is only six
years old but is increasingly popular with developers and is very well
supported both by VC-backed startups (e.g., Databricks) and by
established firms such as Cloudera and IBM. While we have included
DryadLINQ as an example of a post-MapReduce system due to its historical
significance and technical depth, the Spark
paper~{{[}\protect\hyperlink{ref-spark}{27}{]}}, written in the early
days of the project, and recent extensions including
SparkSQL~{{[}\protect\hyperlink{ref-sparksql}{4}{]}}, are worthwhile
additional reads. Like Hadoop, Spark rallied major interest at a
relatively early stage of maturity. Today, Spark still has a ways to go
before its feature set rivals that of a traditional data warehouse.
However, its feature set is rapidly growing and expectations of Spark as
the successor to MapReduce in the Hadoop ecosystem are high; for
example, Cloudera is working to replace MapReduce with Spark in the
Hadoop ecosystem~{{[}\protect\hyperlink{ref-cloudera-spark}{10}{]}}.
Time will tell whether these expectations are accurate; in the meantime,
the gaps between traditional warehouses and post-MapReduce systems are
quickly closing, resulting in systems that are as good at data
warehousing as traditional systems, but also much more.

\section*{References}

\leavevmode\hypertarget{ref-stratosphere}{}%
{[}1{]} Alexandrov, A., Bergmann, R., Ewen, S., Freytag, J.-C., Hueske,
F., Heise, A., Kao, O., Leich, M., Leser, U., Markl, V. and others The
Stratosphere platform for big data analytics. \emph{The VLDB Journal}.
23, 6 (2014), 939-964.

\leavevmode\hypertarget{ref-asterixdb}{}%
{[}2{]} Alsubaiee, S., Altowim, Y., Altwaijry, H., Behm, A., Borkar, V.,
Bu, Y., Carey, M., Cetindil, I., Cheelangi, M., Faraaz, K. and others
Asterixdb: A scalable, open source bDMS. \emph{VLDB}, 2014.

\leavevmode\hypertarget{ref-tez}{}%
{[}3{]} Apache Tez: Apache Tez. \url{https://tez.apache.org/}.

\leavevmode\hypertarget{ref-sparksql}{}%
{[}4{]} Armbrust, M., Xin, R.S., Lian, C., Huai, Y., Liu, D., Bradley,
J.K., Meng, X., Kaftan, T., Franklin, M.J., Ghodsi, A. and others Spark
SQL: Relational data processing in spark. \emph{SIGMOD}, 2015.

\leavevmode\hypertarget{ref-fnt-mr}{}%
{[}5{]} Babu, S. and Herodotou, H. Massively parallel databases and
MapReduce systems. \emph{Foundations and Trends in Databases}. 5, 1
(2013), 1-104.

\leavevmode\hypertarget{ref-chubby}{}%
{[}6{]} Burrows, M. The chubby lock service for loosely-coupled
distributed systems. \emph{OSDI}, 2006.

\leavevmode\hypertarget{ref-bigtable}{}%
{[}7{]} Chang, F., Dean, J., Ghemawat, S., Hsieh, W.C., Wallach, D.A.,
Burrows, M., Chandra, T., Fikes, A. and Gruber, R.E. Bigtable: A
distributed storage system for structured data. \emph{OSDI}, 2006.

\leavevmode\hypertarget{ref-mr-majorstep}{}%
{[}8{]} DeWitt, D. and Stonebraker, M. MapReduce: A major step
backwards. \emph{The Database Column}. (2008).

\leavevmode\hypertarget{ref-grail}{}%
{[}9{]} Fan, J., Gerald, A., Raj, S. and Patel, J.M. The case against
specialized graph analytics engines. \emph{CIDR}, 2015.

\leavevmode\hypertarget{ref-cloudera-spark}{}%
{[}10{]} Forbes: Why Cloudera is saying 'Goodbye, MapReduce' and 'Hello,
Spark': Forbes: Why Cloudera is saying 'Goodbye, MapReduce' and 'Hello,
Spark'. 2015.
\url{http://fortune.com/2015/09/09/cloudera-spark-mapreduce/}.

\leavevmode\hypertarget{ref-gfs}{}%
{[}11{]} Ghemawat, S., Gobioff, H. and Leung, S.-T. The google file
system. \emph{SOSP}, 2003.

\leavevmode\hypertarget{ref-systemml}{}%
{[}12{]} Ghoting, A., Krishnamurthy, R., Pednault, E., Reinwald, B.,
Sindhwani, V., Tatikonda, S., Tian, Y. and Vaithyanathan, S. SystemML:
Declarative machine learning on mapReduce. \emph{ICDE}, 2011.

\leavevmode\hypertarget{ref-graphx}{}%
{[}13{]} Gonzales, J.E., Xin, R.S., Crankshaw, D., Dave, A., Franklin,
M.J. and Stoica, I. GraphX: Unifying data-parallel and graph-parallel
analytics. \emph{OSDI}, 2014.

\leavevmode\hypertarget{ref-drill}{}%
{[}14{]} Hausenblas, M. and Nadeau, J. Apache Drill: Interactive ad-hoc
analysis at scale. \emph{Big Data}. 1, 2 (2013), 100-104.

\leavevmode\hypertarget{ref-dryad}{}%
{[}15{]} Isard, M., Budiu, M., Yu, Y., Birrell, A. and Fetterly, D.
Dryad: Distributed data-parallel programs from sequential building
blocks. \emph{EuroSys}, 2007.

\leavevmode\hypertarget{ref-impala}{}%
{[}16{]} Kornacker, M., Behm, A., Bittorf, V., Bobrovytsky, T., Ching,
C., Choi, A., Erickson, J., Grund, M., Hecht, D., Jacobs, M. and others
Impala: A modern, open-source sQL engine for hadoop. \emph{CIDR}, 2015.

\leavevmode\hypertarget{ref-parameterserver}{}%
{[}17{]} Li, M., Andersen, D.G., Park, J.W., Smola, A.J., Ahmed, A.,
Josifovski, V., Long, J., Shekita, E.J. and Su, B.-Y. Scaling
distributed machine learning with the parameter server. \emph{OSDI},
2014.

\leavevmode\hypertarget{ref-graphlab}{}%
{[}18{]} Low, Y., Bickson, D., Gonzalez, J., Guestrin, C., Kyrola, A.
and Hellerstein, J.M. Distributed graphLab: A framework for machine
learning and data mining in the cloud. \emph{VLDB}, 2012.

\leavevmode\hypertarget{ref-pregel}{}%
{[}19{]} Malewicz, G., Austern, M.H., Bik, A.J., Dehnert, J.C., Horn,
I., Leiser, N. and Czajkowski, G. Pregel: A system for large-scale graph
processing. \emph{SIGMOD}, 2010.

\leavevmode\hypertarget{ref-cost}{}%
{[}20{]} McSherry, F., Isard, M. and Murray, D.G. Scalability! But at
what COST? \emph{HotOS}, 2015.

\leavevmode\hypertarget{ref-naiad}{}%
{[}21{]} Murray, D.G., McSherry, F., Isaacs, R., Isard, M., Barham, P.
and Abadi, M. Naiad: A timely dataflow system. \emph{SOSP}, 2013.

\leavevmode\hypertarget{ref-pig}{}%
{[}22{]} Olston, C., Reed, B., Srivastava, U., Kumar, R. and Tomkins, A.
Pig latin: A not-so-foreign language for data processing. \emph{SIGMOD},
2008.

\leavevmode\hypertarget{ref-themismr}{}%
{[}23{]} Rasmussen, A., Lam, V.T., Conley, M., Porter, G., Kapoor, R.
and Vahdat, A. Themis: An i/O-efficient mapReduce. \emph{SoCC}, 2012.

\leavevmode\hypertarget{ref-f1}{}%
{[}24{]} Shute, J., Vingralek, R., Samwel, B., Handy, B., Whipkey, C.,
Rollins, E., Oancea, M., Littlefield, K., Menestrina, D., Ellner, S. and
others F1: A distributed sQL database that scales. \emph{VLDB}, 2013.

\leavevmode\hypertarget{ref-mllib}{}%
{[}25{]} Sparks, E.R., Talwalkar, A., Smith, V., Kottalam, J., Pan, X.,
Gonzalez, J., Franklin, M.J., Jordan, M., Kraska, T. and others MLI: An
aPI for distributed machine learning. \emph{ICDM}, 2013.

\leavevmode\hypertarget{ref-hive}{}%
{[}26{]} Thusoo, A., Sarma, J.S., Jain, N., Shao, Z., Chakka, P.,
Anthony, S., Liu, H., Wyckoff, P. and Murthy, R. Hive: A warehousing
solution over a map-reduce framework. \emph{VLDB}, 2009.

\leavevmode\hypertarget{ref-spark}{}%
{[}27{]} Zaharia, M., Chowdhury, M., Das, T., Dave, A., Ma, J.,
McCauley, M., Franklin, M.J., Shenker, S. and Stoica, I. Resilient
distributed datasets: A fault-tolerant abstraction for in-memory cluster
computing. \emph{NSDI}, 2012.

\section*[Comments]{Comments\\%
{\normalsize Michael Stonebraker}\\%
{\normalsize 26 October 2015}%
}

Recently, there has been considerable interest in data analytics as part
of the marketing moniker ``big data''. Historically, this meant business
intelligence (BI) analytics and was serviced by BI applications (Cognos,
Business Objects, etc.) talking to a relational data warehouse (such as
Teradata, Vertica, Red Shift, Greenplum, etc.). More recently it has
become associated with ``data science''. In this context, let's start
ten years ago with Map-Reduce, which was purpose-built by Google to
support their web crawl data base. Then, the marketing guys took over
with the basic argument: ``Google is smart; Map-Reduce is Google's next
big thing, so it must be good''. Cloudera, Hortonworks and Facebook were
in the vanguard in hyping Map-Reduce (and its open source look-alike
Hadoop). A few years ago, the market was abuzz drinking the Map-Reduce
koolaid. About the same time, Google stopped using Map-Reduce for the
application that it was purpose-built for, moving instead to Big Table.
With a delay of about 5 years, the rest of the world is seeing what
Google figured out earlier; Map-Reduce is not an architecture with any
broad scale applicability.

\enlargethispage{-3\baselineskip}
In effect Map-Reduce suffers from the following two problems:

\begin{enumerate}
  \item
  It is inappropriate as a platform on which to build data warehouse
  products. There is no interface inside any commercial data warehouse
  product which looks like Map-Reduce, and for good reason. Hence, DBMSs
  do not want this sort of platform.
  \item
  It is inappropriate as a platform on which to build distributed
  applications. Not only is the Map-Reduce interface not flexible enough
  for distributed applications but also a message passing system that
  uses the file system is way too slow to be interesting.
\end{enumerate}

Of course, that has not stopped the Map-Reduce vendors. They have simply
rebranded their platform to be HDFS (a file system) and have built
products based on HDFS that do not include Map-Reduce. For example,
Cloudera has recently introduced Impala, which is a SQL engine, not
built on Map-Reduce. In point of fact, Impala does not really use HDFS
either, choosing to drill through that layer to read and write the
underlying local Linux files directly. HortonWorks and Facebook have
similar projects underway. As such the Map-Reduce crowd has turned into
a SQL crowd and Map-Reduce, as an interface, is history. Of course, HDFS
has serious problems when used by a SQL engine, so it is not clear that
it will have any legs, but that remains to be seen. In any case, the
Map-Reduce-HDFS market will merge with the SQL-data warehouse market;
and may the best systems prevail. In summary, Map-Reduce has failed as a
distributed systems platform, and vendors are using HDFS as a file
system under data warehouse products.

This brings us to Spark. The original argument for Spark is that it is a
faster version of Map-Reduce. It is a main memory platform with a fast
message passing interface. Hence, it should not suffer from the
performance problems of Map-Reduce when used for distributed
applications. However, according to Spark's lead author Matei Zaharia,
more than 70\% of the Spark accesses are through SparkSQL. In effect,
Spark is being used as a SQL engine, not as a distributed applications
platform! In this context Spark has an identity problem. If it is a SQL
platform, then it needs some mechanism for persistence, indexing,
sharing of main memory between users, meta data catalogs, etc. to be
competitive in the SQL/data warehouse space. It seems likely that Spark
will turn into a data warehouse platform, following Hadoop along the
same path.

On the other hand, 30\% of Spark accesses are not to SparkSQL and are
primarily from Scala. Presumably this is a distributed computing load.
In this context, Spark is a reasonable distributed computing platform.
However, there are a few issues to consider. First, the average data
scientist does a mixture of data management and analytics. Higher
performance comes from tightly coupling the two. In Spark there is no
such coupling, since Spark's data formats are not necessarily common
across these two tasks. Second, Spark is main memory-only (at least for
now). Scalability requirements will presumably get this fixed over time.
As such, it will be interesting to see how Spark evolves off into the
future.

In summary, I would like to offer the following takeaways:

\begin{itemize}
  \item
  Just because Google thinks something is a good idea does not mean you
  should adopt it.
  \item
  Disbelieve all marketing spin, and figure out what benefit any given
  product actually has. This should be especially applied to performance
  claims.
  \item
  The community of programmers has a love affair with ``the next shiny
  object''. This is likely to create ``churn'' in your organization, as
  the ``half-life'' of shiny objects may be quite short.
\end{itemize}


\Chapter{ch6-isolation}{%
Weak Isolation and Distribution
}{%
Introduced by Peter Bailis
}

\begin{framed}
Readings:
\begin{itemize}
\item
\href{https://scholar.google.com/scholar?cluster=12975897967422539576}{Atul
  Adya, Barbara Liskov, and Patrick O'Neil. {Generalized Isolation Level
    Definitions}. {ICDE}, 2000.}
\item
\href{https://scholar.google.com/scholar?cluster=5432858092023181552}{Giuseppe
  DeCandia, Deniz Hastorun, Madan Jampani, Gunavardhan Kakulapati, Avinash
  Lakshman, Alex Pilchin, Swaminathan Sivasubramanian, Peter Vosshall, and
  Werner Vogels. {Dynamo: Amazon's Highly Available Key-Value Store}.
  {SOSP}, 2007.}
\item
\href{https://scholar.google.com/scholar?cluster=17642052422667212790}{Eric
  Brewer. {CAP Twelve Years Later: How the "Rules" Have Changed}. {IEEE
    Computer}, 45, 2 (2012).}
\end{itemize}
\end{framed}

Conventional database wisdom dictates that serializable transactions are
the canonical solution to the problem of concurrent programming, but
this is seldom the case in real-world databases. In practice, database
systems instead overwhelmingly implement concurrency control that is
non-serializable, exposing users to the possibility that their
transactions will not appear to have executed in some serial order. In
this chapter, we discuss why the use of this so-called ``weak
isolation'' is so widespread, what these non-serializable isolation
modes actually do, and why reasoning about them is so difficult.

\Section{overview-and-prevalence}{%
Overview and Prevalence
}

Even in the earliest days of database systems (see
\hyperref[ch3-techniques]{Chapter 3}), systems builders realized that
implementing serializability is expensive. The requirement that
transactions appear to execute sequentially has profound consequences
for the degree of concurrency a database can achieve. If transactions
access disjoint sets of data items in the database, serializability is
effectively ``free'': under these disjoint access patterns, a
serializable schedule admits data parallelism. However, if transactions
contend on the same data items, in the worst case, the system cannot
process them with any parallelism whatsoever. This property is
fundamental to serializability and independent of the actual
implementation: because transactions cannot safely make progress
independently under all workloads (i.e., they must \emph{coordinate}),
any implementation of serializability may, in effect, require serial
execution. In practice, this means that transactions may need to wait,
decreasing throughput while increasing latency. Transaction processing
expert Phil Bernstein suggests that serializability typically incurs a
three-fold performance penalty on a single-node database compared to one
of the most common weak isolation levels called Read
Committed~{{[}\protect\hyperlink{ref-bernstein-tutorial}{12}{]}}.
Depending on the implementation, serializability may also lead to more
aborts, restarted transactions, and/or deadlocks. In distributed
databases, these costs increase because networked communication is
expensive, increasing the time required to execute serial critical
sections (e.g., holding locks); we have observed multiple
order-of-magnitude performance penalties under adverse
conditions~{{[}\protect\hyperlink{ref-coord-avoid}{7}{]}}.

As a result, instead of implementing serializability, database system
designers instead often implemented weaker models. Under weak isolation,
transactions are not guaranteed to observe serializable behavior.
Instead, transactions will observe a range of anomalies (or
``phenomena''): behaviors that could not have arisen in a serial
execution. The exact anomalies depend on which model is provided, but
example anomalies include reading intermediate data that another
transaction produced, reading aborted data, reading two or more
different values for the same item during execution of the same
transaction, and ``losing'' some effects of transactions due to
concurrent writes to the same item.

These weak isolation modes are surprisingly prevalent. In a recent
survey of eighteen SQL and ``NewSQL''
databases~{{[}\protect\hyperlink{ref-hat-vldb}{5}{]}}, we found that
only three of eighteen provided serializability by default and eight
(including Oracle and SAP's flagship offerings) did not offer
serializability at all! This state of affairs is further complicated by
often inaccurate use of terminology: for example, Oracle's
``serializable'' isolation guarantee actually provides Snapshot
Isolation, a weak isolation
mode~{{[}\protect\hyperlink{ref-fekete-ssi}{14}{]}}. There is also a
race to to bottom among vendors. Anecdotally, when vendor A, a major
player in the transaction processing market, switched its default
isolation mode from serializability to Read Committed, vendor B, who
still defaulted to serializability, began to lose sales contracts during
bake-offs with vendor A. Vendor B's database was clearly slower, so why
would the customer choose B instead of A? Unsurprisingly, vendor B now
provides Read Committed isolation by default, too.

\hypertarget{the-key-challenge-reasoning-about-anomalies}{%
\section[Reasoning about Anomalies]{The Key Challenge: Reasoning about Anomalies}
\label{the-key-challenge-reasoning-about-anomalies}
}

The primary reason why weak isolation is problematic is that, depending
on the application, weak isolation anomalies can result in
application-level inconsistency: the invariants that each transaction
preserves in a serializable execution may no longer hold under weak
isolation. For example, if two users attempt to withdraw from a bank
account at the same time and their transactions run under a weak
isolation mode allowing concurrent writes to the same data item (e.g.,
the common Read Committed model), the users may successfully withdraw
more money than the account contained (i.e., each reads the current
amount, each calculates the amount following their withdrawal, then each
writes the ``new'' total to the database). This is not a hypothetical
scenario. In a recent, colorful example, an attacker systematically
exploited weak isolation behavior in the Flexcoin Bitcoin exchange; by
repeatedly and programmatically triggering non-transactional
read-modify-write behavior in the Flexcoin application (an vulnerability
under Read Committed isolation and, under a more sophisticated access
pattern, Snapshot Isolation), the attacker was able to withdraw more
Bitcoins than she should have, thus bankrupting the
exchange~{{[}\protect\hyperlink{ref-flexcoin}{15}{]}}.

Perhaps surprisingly, few developers I talk with regarding their use of
transactions are even aware that they are running under non-serializable
isolation. In fact, in our research, we have found that many open-source
ORM-backed applications assume serializable isolation, leading to a
range of possible application integrity violations when deployed on
commodity database engines~{{[}\protect\hyperlink{ref-feral}{6}{]}}. The
developers who are aware of weak isolation tend to employ a range of
alternative techniques at the application level, including explicitly
acquiring locks (e.g., SQL ``SELECT FOR UPDATE'') and introducing false
conflicts (e.g., writing to a dummy key under Snapshot Isolation). This
is highly error-prone and negates many of the benefits of the
transaction concept.

Unfortunately, specifications for weak isolation are often incomplete,
ambiguous, and even inaccurate. These specifications have a long history
dating to the 1970s. While they have improved over time, they remain
problematic.

The earliest weak isolation modes were specified operationally: as we
saw in \hyperref[ch3-techniques]{Chapter 3}, popular models like Read
Committed were originally invented by modifying the duration for which
read locks were held~{{[}\protect\hyperlink{ref-gray-isolation}{17}{]}}.
The definition of Read Committed was: ``hold read locks for a short
duration, and hold write locks for a long duration.''

The ANSI SQL Standard later attempted to provide an
implementation-independent description of several weak isolation modes
that would apply not only to lock-based mechanisms but also to
multi-versioned and optimistic methods as well. However, as Gray et al.
describe in~{{[}\protect\hyperlink{ref-ansi-critique}{10}{]}}, the SQL
Standard is both ambiguous and under-specified: there are multiple
possible interpretations of the English-language description, and the
formalization does not capture all behaviors of the lock-based
implementations. Additionally, the ANSI SQL Standard does not cover all
isolation modes: for example, vendors had already begun shipping
production databases providing Snapshot Isolation (and labeling it as
serializable!) before Gray et al. defined it in their 1995 paper.
(Sadly, as of 2015, the ANSI SQL Standard remains unchanged.)

To complicate matters further, Gray et al.'s 1995 revised formalism is
also problematic: it focuses on lock-related semantics and rules out
behavior that might be considered safe in a multi-versioned concurrency
control system. Accordingly, for his 1999 Ph.D.
thesis~{{[}\protect\hyperlink{ref-adya-phd}{1}{]}}, Atul Adya introduced
the best formalism for weak isolation that we have to date. Adya's
thesis adapts the formalism of multi-version serialization
graphs~{{[}\protect\hyperlink{ref-bernstein-book}{11}{]}} to the domain
of weak isolation and describes anomalies according to restrictions on
those graphs. We include Adya's corresponding ICDE 2000 paper, but
isolation aficionados should consult the full thesis. Unfortunately,
Adya's model is still underspecified in some cases (e.g., what exactly
does G0 mean if no reads are involved?), and implementations of these
guarantees differ across databases.

Even with a perfect specification, weak isolation is still a real
challenge to reason about. To decide whether weak isolation is ``safe,''
programmers must mentally translate their application-level consistency
concerns down to low-level read and write
behavior~{{[}\protect\hyperlink{ref-consistency-borders}{2}{]}}. This is
ridiculously difficult, even for seasoned concurrency control experts.
In fact, one might wonder what benefits of transactions remain if
serializability is compromised? Why is it easier to reason about Read
Committed isolation than no isolation at all? Given how many database
engines like Oracle run under weak isolation, how does modern society
function at all - whether users are booking airline flights,
administering hospitals, or performing stock trades? The literature
lends few clues, casting serious questions about the success of the
transaction concept as deployed in practice today.

The most compelling argument I have encountered for why weak isolation
seems to be ``okay'' in practice is that few applications today
experience high degrees of concurrency. Without concurrency, most
implementations of weak isolation deliver serializable results. This in
turn has led to a fruitful set of research results. Even in a
distributed setting, weakly isolated databases deliver ``consistent''
results: for example, at Facebook, only 0.0004\% of results returned
from their eventually consistent store were
``stale''~{{[}\protect\hyperlink{ref-fb-inconsistency}{19}{]}}, and
others have found similar results~{{[}\protect\hyperlink{ref-pbs}{9},
  \protect\hyperlink{ref-wada}{25}{]}}. However, while for many
applications weak isolation is apparently not problematic, it can be: as
our Flexcoin example illustrates, given the possibility of errors,
application writers must be vigilant in accounting for (or otherwise
explicitly ignoring) concurrency-related anomalies.

\Section{weak-isolation-distribution-and-nosql}{%
Weak Isolation, Distribution, and ``NoSQL''
}

With the rise of Internet-scale services and cloud computing, weak
isolation has become even more prevalent. As I mentioned earlier,
distribution exacerbates overheads of serializability, and, in the event
of partial system failures (e.g., servers crashing), transactions may
stall indefinitely. As more and more programmers began to write
distributed applications and used distributed databases, these concerns
became mainstream.

The past decade saw the introduction of a range of new data stores
optimized for the distributed environment, collectively called
``NoSQL.'' The ``NoSQL'' label is unfortunately overloaded and refers to
many aspects of these stores, from lack of literal SQL support to
simpler data models (e.g., key-value pairs) and little to no
transactional support. Today, as in MapReduce-like systems (Chapter 5),
NoSQL stores are adding many these features. However, a notable,
fundamental difference is that these NoSQL stores frequently focus on
providing better availability of operations via weaker models, with an
explicit focus on fault tolerance. (It is somewhat ironic that, while
NoSQL stores are commonly associated with the use of non-serializable
guarantees, classic RDBMSs do not provide serializability by default
either.)

As an example of these NoSQL stores, we include a paper on the Dynamo
system, from Amazon, presented at SOSP 2007. Dynamo was introduced to
provide highly available and low latency operation for Amazon's shopping
cart. The paper is technically interesting as it combines several
techniques, including quorum replication, Merkle tree anti-entropy,
consistent hashing, and version vectors. The system is entirely
non-transactional, does not provide any kind of atomic operation (e.g.,
compare and swap), and relies on the application writer to reconcile
divergent updates. In the limit, any node can update any item (under
hinted handoff).

By using a merge function, Dynamo adopts an ``optimistic replication''
policy: accept writes first, reconcile divergent versions
later~{{[}\protect\hyperlink{ref-gray-dangers}{16},
  \protect\hyperlink{ref-optimistic}{21}{]}}. On the one hand, presenting
a set of divergent versions to the user is more friendly than simply
discarding some concurrent updates, as in Read Committed isolation. On
the other hand, programmers must reason about merge functions. This
raises many questions: what is a suitable merge for an application? How
do we avoid throwing away committed data? What if an operation should
not have been performed concurrently in the first place? Some open
source Dynamo clones, like Apache Cassandra, do not provide merge
operators and simply choose a ``winning'' write based on a numerical
timestamp. Others, like Basho Riak, have adopted ``libraries'' of
automatically mergeable datatypes like counters, called Commutative
Replicated Data Types~{{[}\protect\hyperlink{ref-crdt}{22}{]}}.

Dynamo also does not make promises about recency of reads. Instead, it
guarantees that, if writes stop, eventually all replicas of a data item
will contain the same set of writes. This eventual consistency is a
remarkably weak guarantee: technically, an eventually consistent
datastore can return stale (or even garbage) data for an indefinite
amount of time~{{[}\protect\hyperlink{ref-ec-queue}{4}{]}}. In practice,
data store deployments frequently return recent
data~{{[}\protect\hyperlink{ref-pbs}{9},
  \protect\hyperlink{ref-wada}{25}{]}}, but, nevertheless, users must
reason about non-serializable behavior. Moreover, in practice, many
stores offer intermediate forms of isolation called ``session
guarantees'' that ensure that users read their own writes (but not the
writes of other users); interestingly, these techniques were developed
in the early 1990s as part of the Bayou project on mobile computing and
have recently come to prominence
again~{{[}\protect\hyperlink{ref-terry-baseball}{23},
  \protect\hyperlink{ref-bayou-session}{24}{]}}.

\Section{trade-offs-and-the-cap-theorem}{
Trade-offs and the CAP Theorem
}

We have also included Brewer's 12 year retrospective on the CAP Theorem.
Originally formulated following Brewer's time building Inktomi, one of
the first scalable search engines, Brewer's CAP Theorem pithily
describes trade-offs between the requirement for coordination (or
``availability'') and strong guarantees like serializability. While
earlier results described this
trade-off~{{[}\protect\hyperlink{ref-davidson-survey}{13},
  \protect\hyperlink{ref-caprfc}{18}{]}}, CAP became a rallying cry for
mid-2000s developers and has considerable impact. Brewer's article
briefly discusses performance implications of CAP, as well as the
possibility of maintaining some consistency criteria without relying on
coordination.

\Section{programmability-and-practice}{%
Programmability and Practice
}

As we have seen, weak isolation is a real challenge: its performance and
availability benefits mean it is extremely popular in deployments
despite the fact that we have little understanding of its behavior. Even
with a perfect specification, existing formulations of weak isolation
would still be a extremely difficult to reason about. To decide whether
weak isolation is ``safe,'' programmers must mentally translate their
application-level consistency concerns down to low-level read and write
behavior~{{[}\protect\hyperlink{ref-consistency-borders}{2}{]}}. This is
ridiculously difficult, even for seasoned concurrency control experts.

As a result, I believe there is a serious opportunity to investigate
semantics that are not subject to the performance and availability
overheads of serializability but are more intuitive, usable, and
programmable than existing guarantees. Weak isolation has historically
been highly challenging to reason about, but this need not be the case.
We and others have found that several high-value use cases, including
index and view maintenance, constraint maintenance, and distributed
aggregation, frequently do not actually require coordination for
``correct'' behavior; thus, for these use cases, serializability is
overkill~{{[}\protect\hyperlink{ref-bailis-phd}{3},
  \protect\hyperlink{ref-ramp}{8},
  \protect\hyperlink{ref-homeostasis}{20},
  \protect\hyperlink{ref-crdt}{22}{]}}. That is, by providing databases
with additional knowledge about their applications, database users can
have their cake and eat it too. Further identifying and exploiting these
use cases is an area ripe for research.

\Section{conclusions}{
Conclusions
}

In summary, weak isolation is prevalent due to its many benefits: less
coordination, higher performance, and greater availability. However, its
semantics, risks, and usage are poorly understood, even in an academic
context. This is particularly baffling given the amount of research
devoted to serializable transaction processing, which is considered by
many to be a ``solved problem.'' Weak isolation is arguably even more
deserving of such a thorough treatment. As I have highlighted, many
challenges remain: how do modern systems even work, and how should users
program weak isolation? For now, I offer the following take-aways:

\begin{itemize}
  \item
  Non-serializable isolation is prevalent in practice (in both classical
  RDBMSs and recent NoSQL upstarts) due to its concurrency-related
  benefits.
  \item
  Despite this prevalence, many existing formulations of
  non-serializable isolation are poorly specified and difficult to use.
  \item
  Research into new forms of weak isolation show how to preserve
  meaningful semantics and improve programmability without the expense
  of serializability.
\end{itemize}

\section*{References}

\leavevmode\hypertarget{ref-adya-phd}{}%
{[}1{]} Adya, A. \emph{Weak consistency: A generalized theory and
  optimistic implementations for distributed transactions}. Ph.D. Thesis,
MIT, 1999

\leavevmode\hypertarget{ref-consistency-borders}{}%
{[}2{]} Alvaro, P., Bailis, P., Conway, N. and Hellerstein, J.M.
Consistency without borders. \emph{SoCC}, 2013.

\leavevmode\hypertarget{ref-bailis-phd}{}%
{[}3{]} Bailis, P. \emph{Coordination avoidance in distributed
  databases}. Ph.D. Thesis, University of California at Berkeley, 2015.

\leavevmode\hypertarget{ref-ec-queue}{}%
{[}4{]} Bailis, P. and Ghodsi, A. Eventual consistency today:
Limitations, extensions, and beyond. \emph{ACM Queue}. 11, 3 (2013).

\leavevmode\hypertarget{ref-hat-vldb}{}%
{[}5{]} Bailis, P., Davidson, A., Fekete, A., Ghodsi, A., Hellerstein,
J.M. and Stoica, I. Highly Available Transactions: Virtues and
limitations. \emph{VLDB}, 2014.

\leavevmode\hypertarget{ref-feral}{}%
{[}6{]} Bailis, P., Fekete, A., Franklin, M.J., Ghodsi, A., Hellerstein,
J.M. and Stoica, I. Feral Concurrency Control: An empirical
investigation of modern application integrity. \emph{SIGMOD}, 2015.

\leavevmode\hypertarget{ref-coord-avoid}{}%
{[}7{]} Bailis, P., Fekete, A., Franklin, M.J., Hellerstein, J.M.,
Ghodsi, A. and Stoica, I. Coordination avoidance in database systems.
\emph{VLDB}, 2015.

\leavevmode\hypertarget{ref-ramp}{}%
{[}8{]} Bailis, P., Fekete, A., Ghodsi, A., Hellerstein, J.M. and
Stoica, I. Scalable atomic visibility with RAMP transactions.
\emph{SIGMOD}, 2014.

\leavevmode\hypertarget{ref-pbs}{}%
{[}9{]} Bailis, P., Venkataraman, S., Franklin, M.J., Hellerstein, J.M.
and Stoica, I. Probabilistically Bounded Staleness for practical partial
quorums. \emph{VLDB}, 2012.

\leavevmode\hypertarget{ref-ansi-critique}{}%
{[}10{]} Berenson, H., Bernstein, P., Gray, J., Melton, J., O'Neil, E.
and O'Neil, P. A critique of ANSI SQL isolation levels. \emph{SIGMOD},
1995.

\leavevmode\hypertarget{ref-bernstein-book}{}%
{[}11{]} Bernstein, P., Hadzilacos, V. and Goodman, N. Concurrency
control and recovery in database systems. Addison-Wesley New York. 1987.

\leavevmode\hypertarget{ref-bernstein-tutorial}{}%
{[}12{]} Bernstein, P.A. and Das, S. Rethinking eventual consistency.
\emph{SIGMOD}, 2013.

\leavevmode\hypertarget{ref-davidson-survey}{}%
{[}13{]} Davidson, S., Garcia-Molina, H. and Skeen, D. Consistency in
partitioned networks. \emph{ACM CSUR}. 17, 3 (1985), 341-370.

\leavevmode\hypertarget{ref-fekete-ssi}{}%
{[}14{]} Fekete, A., Liarokapis, D., O'Neil, E., O'Neil, P. and Shasha,
D. Making snapshot isolation serializable. \emph{ACM TODS}. 30, 2 (Jun.
2005), 492-528.

\leavevmode\hypertarget{ref-flexcoin}{}%
{[}15{]} Flexcoin: The Bitcoin Bank: Flexcoin: The Bitcoin Bank. 2014.
\url{http://www.flexcoin.com/}; originally via Emin G眉n Sirer.

\leavevmode\hypertarget{ref-gray-dangers}{}%
{[}16{]} Gray, J., Helland, P., O'Neil, P. and Shasha, D. The dangers of
replication and a solution. \emph{SIGMOD}, 1996.

\leavevmode\hypertarget{ref-gray-isolation}{}%
{[}17{]} Gray, J., Lorie, R., Putzolu, G. and Traiger, I. Granularity of
locks and degrees of consistency in a shared data base. IBM. 1976.

\leavevmode\hypertarget{ref-caprfc}{}%
{[}18{]} Johnson, P.R. and Thomas, R.H. RFC 667: The maintenance of
duplicate databases. 1975.

\leavevmode\hypertarget{ref-fb-inconsistency}{}%
{[}19{]} Lu, H., Veeraraghavan, K., Ajoux, P., Hunt, J., Song, Y.J.,
Tobagus, W., Kumar, S. and Lloyd, W. Existential consistency: Measuring
and understanding consistency at Facebook. \emph{SOSP}, 2015.

\leavevmode\hypertarget{ref-homeostasis}{}%
{[}20{]} Roy, S., Kot, L., Bender, G., Ding, B., Hojjat, H., Koch, C.,
Foster, N. and Gehrke, J. The homeostasis protocol: Avoiding transaction
coordination through program analysis. \emph{SIGMOD}, 2015.

\leavevmode\hypertarget{ref-optimistic}{}%
{[}21{]} Saito, Y. and Shapiro, M. Optimistic replication. \emph{ACM
  Comput. Surv.} 37, 1 (Mar. 2005).

\leavevmode\hypertarget{ref-crdt}{}%
{[}22{]} Shapiro, M., Preguica, N., Baquero, C. and Zawirski, M. Shapiro
Marc. Preguica Nuno. Baquero Carlos. Zawirski Marek. A comprehensive
study of convergent and commutative replicated data types. INRIA TR
7506. 2011.

\leavevmode\hypertarget{ref-terry-baseball}{}%
{[}23{]} Terry, D. Replicated data consistency explained through
baseball. \emph{Communications of the ACM}. 56, 12 (2013), 82-89.

\leavevmode\hypertarget{ref-bayou-session}{}%
{[}24{]} Terry, D.B., Demers, A.J., Petersen, K., Spreitzer, M.J.,
Theimer, M.M. and others Session guarantees for weakly consistent
replicated data. \emph{PDIS}, 1994.

\leavevmode\hypertarget{ref-wada}{}%
{[}25{]} Wada, H., Fekete, A., Zhao, L., Lee, K. and Liu, A. Data
consistency properties and the trade-offs in commercial cloud storage:
The consumers' perspective. \emph{CIDR}, 2011.


\Chapter{ch7-queryoptimization}{%
Query Optimization
}{%
Introduced by Joe Hellerstein
}

\begin{framed}
Readings:
\begin{itemize}
\item
\href{https://scholar.google.com/scholar?cluster=2304531151126477511}{Goetz
  Graefe and William J. McKenna. {The Volcano Optimizer Generator:
    Extensibility and Efficient Search}. {ICDE}, 1993.}
\item
\href{https://scholar.google.com/scholar?cluster=13049208738754012194}{Ron
  Avnur and Joseph M. Hellerstein. {Eddies: Continuously Adaptive Query
    Processing}. {SIGMOD}, 2000.}
\item
\href{https://scholar.google.com/scholar?cluster=4929312332613352080}{Volker
  Markl, Vijayshankar Raman, David Simmen, Guy Lohman, Hamid Pirahesh,
  Miso Cilimdzic. {Robust Query Processing Through Progressive
    Optimization}. {SIGMOD}, 2004.}
\end{itemize}
\end{framed}

Query optimization is one of the signature components of database
technology---the bridge that connects declarative languages to efficient
execution. Query optimizers have a reputation as one of the hardest
parts of a DBMS to implement well, so it's no surprise they remain a
clear differentiator for mature commercial DBMSs. The best of the
open-source relational database optimizers are limited by comparison,
and some have relatively naive optimizers that only work for the
simplest of queries.

It's important to remember that no query optimizer is truly producing
``optimal'' plans. First, they all use estimation techniques to guess at
real plan costs, and it's well known that errors in these estimation
techniques can balloon---in some circumstances being as bad as random
guesses~{{[}\protect\hyperlink{ref-ic91}{7}{]}}. Second, optimizers use
heuristics to limit the search space of plans they choose, since the
problem is
NP-hard~{{[}\protect\hyperlink{ref-ibaraki1984optimal}{6}{]}}. One
assumption that's gotten significant attention recently is the
traditional use of 2-table join operators; this has been shown to be
theoretically inferior to new multi-way join algorithms in certain
cases~{{[}\protect\hyperlink{ref-ngo2012worst}{12}{]}}.

Despite these caveats, relational query optimization has proven
successful, and has enabled relational database systems to serve a wide
range of bread-and-butter use cases fairly well in practice. Database
vendors have invested many years into getting their optimizers to
perform reliably on a range of use cases. Users have learned to live
with limitations on the number of joins. Optimizers still, for the most
part, make declarative SQL queries a far better choice than imperative
code for most uses.

In addition to being hard to build and tune, serious query optimizers
also have a tendency to grow increasingly complex over time as they
evolve to handle richer workloads and more corner cases. The research
literature on database query optimization is practically a field unto
itself, full of technical details---many of which have been discussed in
the literature by researchers at mature vendors like IBM and Microsoft
who work closely with product groups. For this book, we focus on the big
picture: the main architectures that have been considered for query
optimization and how have they been reevaluated over time.

\Section{volcanocascades}{%
Volcano/Cascades
}

We begin with the state of the art. There are two reference
architectures for query optimization from the early days of database
research that cover most of the serious optimizer implementations today.
The first is Selinger et al.'s System R optimizer described in
\hyperref[ch3-techniques]{Chapter 3}. System R's optimizer is textbook
material, implemented in many commercial systems; every database
researcher is expected to understand it in detail. The second is the
architecture that Goetz Graefe and his collaborators refined across a
series of research projects: Exodus, Volcano, and Cascades. Graefe's
work is not covered as frequently in the research literature or the
textbooks as the System R work, but it is widely used in practice,
notably in Microsoft SQL Server, but purportedly in a number of other
commercial systems as well. Graefe's papers on the topic have something
of an insider's flavor---targeted for people who know and care about
implementing query optimizers. We chose the Volcano paper for this book
as the most approachable representative of the work, but aficionados
should also read the Cascades
paper~{{[}\protect\hyperlink{ref-graefe1995cascades}{5}{]}}---not only
does it raise and address a number of detailed deficiencies of Volcano,
but it's the latest (and hence standard) reference for the approach.
Recently, two open-source Cascades-style optimizers have emerged:
Greenplum's Orca optimizer is now part of the Greenplum open source, and
Apache Calcite is an optimizer that can be used with multiple backend
query executors and languages, including LINQ.

Graefe's optimizer architecture is notable for two main reasons. First,
it was expressly designed to be extensible. Volcano deserves credit for
being quite forward-looking---long predating MapReduce and the big data
stacks---in exploring the idea that dataflow could be useful for a wide
range of data-intensive applications. As a result, the Graefe optimizers
are not just for compiling SQL into a plan of dataflow iterators. They
can be parameterized for other input languages and execution targets;
this is a highly relevant topic in recent years with the rise of
specialized data models and languages on the one hand (see
\hyperref[ch2-importantdbms]{Chapter 2} and 9), and specialized
execution engines on the other (Chapter 5). The second innovation in
these optimizers was the use of a top-down or goal-oriented search
strategy for finding the cheapest plan in the space of possible plans.
This design choice is connected to the extensibility API in Graefe's
designs, but that is not intrinsic: the Starburst system showed how to
do extensibility for Selinger's bottom-up
algorithm~{{[}\protect\hyperlink{ref-lohman1988grammar}{9}{]}}. This
``top-down'' vs ``bottom-up'' debate for query optimization has
advocates on both sides, but no clear winner; a similar
top-down/bottom-up debate came out to be more or less a tie in the
recursive query processing literature as
well~{{[}\protect\hyperlink{ref-ramakrishnan1991top}{13}{]}}.
Aficionados will be interested to note that these two bodies of
literature-recursive query processing and query optimizer search-were
connected directly in the Evita Raced optimizer, which implemented both
top-down and bottom-up optimizer search by using recursive queries as
the language for implementing an
optimizer~{{[}\protect\hyperlink{ref-condie2008evita}{1}{]}}.

\Section{adaptive-query-processing}{%
Adaptive Query Processing
}

By the late 1990's, a handful of trends suggested that the overall
architecture of query optimization deserved a significant rethink. These
trends included:

\begin{itemize}
  \item
  Continuous queries over streaming data.
  \item
  Interactive approaches to data exploration like Online Aggregation.
  \item
  Queries over data sources that are outside the DBMS and do not provide
  reliable statistics or performance.
  \item
  Unpredictable and dynamic execution environments, including elastic
  and multitenant settings and widely distributed systems like sensor
  networks.
  \item
  Opaque data and user-defined functions in queries, where statistics
  can only be estimated by observing behavior.
\end{itemize}

In addition, there was ongoing practical concern about the theoretical
fact that plan cost estimation was often erratic for multi-operator
queries~{{[}\protect\hyperlink{ref-ic91}{7}{]}}. As a result of these
trends, interest emerged in adaptive techniques for processing queries,
where execution plans could change mid-query. We present two
complementary points in the design space for adaptive query processing;
there is a long survey with a more comprehensive
overview~{{[}\protect\hyperlink{ref-deshpande2007adaptive}{4}{]}}.

\hypertarget{eddies}{%
  \subsection{Eddies}\label{eddies}}

The work on eddies, represented by our second paper, pushed hard on the
issue of adaptivity: if query ``re-planning'' has to occur
mid-execution, why not remove the architectural distinction between
planning and execution entirely? In the eddies approach, the optimizer
is encapsulated as a dataflow operator that is itself interposed along
other dataflow edges. It can monitor the rates of dataflow along those
edges, so it has dynamic knowledge of their behavior, with whatever
history it cares to record. With that ongoing flow of information, it
can dynamically control the rest of the aspects of query planning via
dataflow routing: the order of commutative operators is determined by
the order tuples are routed through operators (the focus of the first
eddies paper that we include here) the choice of physical operators
(e.g. join algorithms, index selection) is determined by routing tuples
among multiple alternative, potentially redundant physical operators in
the flow~{{[}\protect\hyperlink{ref-deshpande2004lifting}{3},
  \protect\hyperlink{ref-raman2003using}{15}{]}} the scheduling of
operators is determined by buffering inputs and deciding which output to
deliver to next~{{[}\protect\hyperlink{ref-raman2002partial}{14}{]}}. As
an extension, multiple queries can be scheduled by interposing on their
flows and sharing common
operators~{{[}\protect\hyperlink{ref-madden2002continuously}{10}{]}}.
Eddies intercept the ongoing dataflow of query operators while they are
in flight, pipelining data from their inputs to their output. For this
reason it's important that eddy routing be implemented efficiently;
Deshpande developed implementation enhancements along these
lines~{{[}\protect\hyperlink{ref-deshpande2004initial}{2}{]}}. The
advantage of this pipelined approach is that eddies can adaptively
change strategies in the middle of executing a pipelined operator like a
join, which is useful if a query operator is either very long-lived (as
in a streaming system) or a very poor choice that should be abandoned
long before it runs to completion. Interestingly, the original Ingres
optimizer also had the ability to make certain query optimization
decisions on a per-tuple
basis~{{[}\protect\hyperlink{ref-wong1976decomposition}{18}{]}}.

\hypertarget{progressive-optimization}{%
  \subsection{Progressive Optimization}\label{progressive-optimization}}

The third paper in this section from IBM represents a much more
evolutionary approach, which extends a System R style optimizer with
adaptivity features; this general technique was pioneered by Kabra and
DeWitt~{{[}\protect\hyperlink{ref-kabra1998efficient}{8}{]}} but
receives a more complete treatment here. Where eddies focused on
intra-operator reoptimization (while data is ``in motion''), this work
focuses on inter-operator reoptimization (when data is ``at rest'').
Some of the traditional relational operators including sorting and most
hash-joins are blocking: they consume their entire input before
producing any output. This presents an opportunity after input is
consumed to compare observed statistics to optimizer predictions, and
reoptimize the ``remainder'' of the query plan using traditional query
optimization technique. The downside of this approach is that it does no
reoptimization while operators are consuming their inputs, making it
inappropriate for continuous queries over streams, for pipelining
operators like symmetric hash
join~{{[}\protect\hyperlink{ref-wilschut1991dataflow}{17}{]}} or for
long-running relational queries that have poorly-chosen operators in the
initial parts of the plan - e.g. when data is being accessed from data
sources outside the DBMS that do not provide useful
statistics~{{[}\protect\hyperlink{ref-melton2002sql}{11},
  \protect\hyperlink{ref-urhan1998cost}{16}{]}}.

It's worth noting that these two architectures for adaptivity could in
principle coexist: an eddy is ``just'' a dataflow operator, meaning that
a traditional optimizer can generate a query plan with an eddy
connecting a set of streaming operators, and also do reoptimization at
blocking points in the dataflow in the manner of our third paper.

\Section{discussion}{
Discussion
}

This brings us to a discussion of current trends in dataflow
architectures, especially in the open source big data stack. Google
MapReduce set back by a decade the conversation about adaptivity of data
in motion, by baking blocking operators into the execution model as a
fault-tolerance mechanism. It was nearly impossible to have a reasoned
conversation about optimizing dataflow pipelines in the mid-to-late
2000's because it was inconsistent with the Google/Hadoop fault
tolerance model. In the last few years the discussion about execution
frameworks for big data has suddenly opened up wide, with a
quickly-growing variety of dataflow and query systems being deployed
that have more similarities than differences (Tenzing, F1, Dremel,
DryadLINQ, Naiad, Spark, Impala, Tez, Drill, Flink, etc.) Note that all
of the motivating issues for adaptive optimization listed above are very
topical in today's big data discussion, but not well treated.

More generally, I would say that the ``big data'' community in both
research and open source has been far too slow to focus on query
optimization, to the detriment of both the current systems and the query
optimization field. To begin with, the ``hand-planned'' MapReduce
programming model remained a topic of conversation for far longer than
it should have. It took a long time for the Hadoop and systems research
communities to accept that a declarative language like SQL or LINQ is a
good general-purpose interface, even while maintaining low-level
MapReduce-style dataflow programming as a special-case ``fast path''.
More puzzling is the fact that even when the community started building
SQL interfaces like Hive, query optimization remained a little-discussed
and poorly-implemented topic. Maybe it's because query optimizers are
harder to build well than query executors. Or maybe it was fallout from
the historical quality divide between commercial and open source
databases. MySQL was the open source de facto reference for ``database
technology'' for the preceding decade, with a naive heuristic optimizer.
Perhaps as a result, many (most?) open source big data developers didn't
understand---or trust---query optimizer technology.

In any case, this tide is turning in the big data community. Declarative
queries have returned as the primary interface to big data, and there
are efforts underway in essentially all the projects to start building
at least a 1980's-era optimizer. Given the list of issues I mention
above, I'm confident we'll also see more innovative query optimization
approaches deployed in new systems over the coming years.

\section*{References}

\leavevmode\hypertarget{ref-condie2008evita}{}%
{[}1{]} Condie, T., Chu, D., Hellerstein, J.M. and Maniatis, P. Evita
raced: Metacompilation for declarative networks. \emph{Proceedings of
  the VLDB Endowment}. 1, 1 (2008), 1153-1165.

\leavevmode\hypertarget{ref-deshpande2004initial}{}%
{[}2{]} Deshpande, A. An initial study of overheads of eddies. \emph{ACM
  SIGMOD Record}. 33, 1 (2004), 44-49.

\leavevmode\hypertarget{ref-deshpande2004lifting}{}%
{[}3{]} Deshpande, A. and Hellerstein, J.M. Lifting the burden of
history from adaptive query processing. \emph{VLDB}, 2004.

\leavevmode\hypertarget{ref-deshpande2007adaptive}{}%
{[}4{]} Deshpande, A., Ives, Z. and Raman, V. Adaptive query processing.
\emph{Foundations and Trends in Databases}. 1, 1 (2007), 1-140.

\leavevmode\hypertarget{ref-graefe1995cascades}{}%
{[}5{]} Graefe, G. The cascades framework for query optimization.
\emph{IEEE Data Eng. Bull.} 18, 3 (1995), 19-29.

\leavevmode\hypertarget{ref-ibaraki1984optimal}{}%
{[}6{]} Ibaraki, T. and Kameda, T. On the optimal nesting order for
computing n-relational joins. \emph{ACM Transactions on Database Systems
  (TODS)}. 9, 3 (1984), 482-502.

\leavevmode\hypertarget{ref-ic91}{}%
{[}7{]} Ioannidis, Y.E. and Christodoulakis, S. On the propagation of
errors in the size of join results. \emph{SIGMOD}, 1991.

\leavevmode\hypertarget{ref-kabra1998efficient}{}%
{[}8{]} Kabra, N. and DeWitt, D.J. Efficient mid-query re-optimization
of sub-optimal query execution plans. \emph{SIGMOD}, 1998.

\leavevmode\hypertarget{ref-lohman1988grammar}{}%
{[}9{]} Lohman, G.M. Grammar-like functional rules for representing
query optimization alternatives. \emph{SIGMOD}, 1988.

\leavevmode\hypertarget{ref-madden2002continuously}{}%
{[}10{]} Madden, S., Shah, M., Hellerstein, J.M. and Raman, V.
Continuously adaptive continuous queries over streams. \emph{SIGMOD},
2002.

\leavevmode\hypertarget{ref-melton2002sql}{}%
{[}11{]} Melton, J., Michels, J.E., Josifovski, V., Kulkarni, K. and
Schwarz, P. SQL/MED: A status report. \emph{ACM SIGMOD Record}. 31, 3
(2002), 81-89.

\leavevmode\hypertarget{ref-ngo2012worst}{}%
{[}12{]} Ngo, H.Q., Porat, E., R茅, C. and Rudra, A. Worst-case optimal
join algorithms:{[}extended abstract{]}. \emph{Proceedings of the 31st
  symposium on principles of database systems}, 2012, 37-48.

\leavevmode\hypertarget{ref-ramakrishnan1991top}{}%
{[}13{]} Ramakrishnan, R. and Sudarshan, S. Top-down vs. bottom-up
revisited. \emph{Proceedings of the international logic programming
  symposium}, 1991, 321-336.

\leavevmode\hypertarget{ref-raman2002partial}{}%
{[}14{]} Raman, V. and Hellerstein, J.M. Partial results for online
query processing. \emph{SIGMOD}, 2002, 275-286.

\leavevmode\hypertarget{ref-raman2003using}{}%
{[}15{]} Raman, V., Deshpande, A. and Hellerstein, J.M. Using state
modules for adaptive query processing. \emph{ICDE}, 2003.

\leavevmode\hypertarget{ref-urhan1998cost}{}%
{[}16{]} Urhan, T., Franklin, M.J. and Amsaleg, L. Cost-based query
scrambling for initial delays. \emph{ACM SIGMOD Record}. 27, 2 (1998),
130-141.

\leavevmode\hypertarget{ref-wilschut1991dataflow}{}%
{[}17{]} Wilschut, A.N. and Apers, P.M. Dataflow query execution in a
parallel main-memory environment. \emph{Parallel and distributed
  information systems, 1991., proceedings of the first international
  conference on}, 1991, 68-77.

\leavevmode\hypertarget{ref-wong1976decomposition}{}%
{[}18{]} Wong, E. and Youssefi, K. Decomposition---a strategy for query
processing. \emph{ACM Transactions on Database Systems (TODS)}. 1, 3
(1976), 223-241.


\Chapter{ch8-interactive}{%
Interactive Analytics
}{%
Introduced by Joe Hellerstein
}

\begin{framed}
Readings:
\begin{itemize}
\item
\href{https://scholar.google.com/scholar?cluster=9112921129698038148}{Venky
  Harinarayan, Anand Rajaraman, Jeffrey D. Ullman. {Implementing Data
    Cubes Efficiently}. {SIGMOD}, 1996.}
\item
\href{https://scholar.google.com/scholar?cluster=1761772586638103323}{Yihong
  Zhao, Prasad M. Deshpande, Jeffrey F. Naughton. {An Array-Based
    Algorithm for Simultaneous Multidimensional Aggregates}. {SIGMOD},
  1997.}
\item
\href{https://scholar.google.com/scholar?cluster=1434575511619007556}{Joseph
  M. Hellerstein, Ron Avnur, Vijayshankar Raman. {Informix under CONTROL:
    Online Query Processing}. {Data Mining and Knowledge Discovery}, 4(4),
  2000, 281-314.}
\item
\href{https://scholar.google.com/scholar?cluster=4916926405792203059}{Sameer
  Agarwal, Barzan Mozafari, Aurojit Panda, Henry Milner, Samuel Madden,
  Ion Stoica. {BlinkDB: Queries with Bounded Errors and Bounded Response
    Times on Very Large Data}. {EuroSys}, 2013.}
\end{itemize}
\end{framed}

For decades, most database workloads have been partitioned into two
categories: (1) many small ``transaction processing'' queries that do
lookups and updates on a small number of items in a large database, and
(2) fewer big ``analytic'' queries that summarize large volumes of data
for analysis. This section is concerned with ideas for accelerating the
second category of queries---particularly to answer them at interactive
speeds, and allow for summarization, exploration and visualization of
data.

Over the years there has been a great deal of buzzword bingo in industry
to capture some or all of this latter workload, from ``Decision Support
Systems'' (DSS) to ``Online Analytic Processing'' (OLAP) to ``Business
Intelligence'' (BI) to ``Dashboards'' and more generally just
``Analytics''. Billions of dollars of revenue have been associated with
these labels over time, so marketers and industry analysts worked hard
over the years to define, distinguish and sometimes try to subvert them.
By now it's a bit of a mess of nomenclature. The interested reader can
examine Wikipedia and assess conventional wisdom on what these buzzwords
came to mean and how they might be different; be warned that it will not
be a scientifically satisfying exercise.

Here, I will try to keep things simple and somewhat technically
grounded.

Human cognition cannot process large amounts of raw data. In order for a
human to make sense of data, the data has to be ``distilled'' in some
way to a relatively small set of records or visual marks. Typically this
is done by partitioning the data and running simple arithmetic
aggregation functions on the partitions --- think of SQL's ``GROUP BY''
functionality as a canonical
pattern\footnote{Database-savvy folks take GROUP BY and aggregation for granted. In
  statistical programming packages (e.g., R's plyr library, or Python's
  pandas), this is apparently a relatively new issue, referred to as the
  ``Split-Apply-Combine Strategy''. A wrinkle in that context is the
  need to support both array and table
  notation.}. Subsequently the
data typically needs to be visualized for users to relate it to their
task at hand.

The primary challenge we address in this chapter is to make large-scale
grouping/aggregation queries run at interactive speeds---even in cases
where it is not feasible to iterate through all the data associated with
the query.

How do we make a query run in less time than it takes to look at the
data? There is really only one answer: we answer the query without
looking at (all) the data. Two variants of this idea emerge:

Precomputation: If we know something about the query workload in
advance, we can distill the data in various ways to allow us to support
quick answers (either accurate or approximate) to certain queries. The
simplest version of this idea is to precompute the answers to a set of
queries, and only support those queries. We discuss more sophisticated
answers below. Sampling: If we cannot anticipate the queries well in
advance, our only choice is to look at a subset of the data at query
time. This amounts to sampling from the data, and approximating the true
answer based on the sample.

The papers in this section focus on one or both of these approaches.

Our first two papers address what the database community dubbed ``data
cubes'' {[}DataCubes{]}. Data cubes were originally supported by
standalone query/visualization tools called On Line Analytic Processing
(OLAP) systems. The name is due to relational pioneer Ted Codd, who was
hired as a consultant to promote an early OLAP vendor called Essbase
(subsequently bought by Oracle). This was not one of Codd's more
scholarly endeavors.

Early OLAP tools used a pure ``precomputation'' approach. They ingested
a table, and computed and stored the answers to a family of GROUP BY
queries over that table: each query grouped on a different subset of the
columns, and computed summary aggregates over the non-grouped numerical
columns. For example, in a table of car sales, it might show total sales
by Make, total sales by Model, total sales by Region, and total sales by
combinations of any 2 or 3 of those attributes. A graphical user
interface allowed users to navigate the resulting space of group-by
queries interactively---this space of queries is what became known as a
data cube\footnote{Note that this idea did not originate in databases. In statistics, and
  later in spreadsheets, there is an old, well-known idea of a
  contingency table or cross tabulation
  (crosstab).}. Originally, OLAP
systems were pitched as standalone ``Multidimensional Databases'' that
were fundamentally different than relational databases. However, Jim
Gray and a consortium of authors in the relational industry explained
how the notion of a data cube can fit in the relational
context~{{[}\protect\hyperlink{ref-gray-cube}{4}{]}}, and the concept
subsequently migrated into the SQL standard as a single query construct:
``CUBE BY''. There is also a standard alternative to SQL called MDX that
is less verbose for OLAP purposes. Some of the terminology from data
cubes has become common parlance---in particular, the idea of ``drilling
down'' into details and ``rolling up'' to coarser summaries.

A naive relational precomputation approach for precomputing a full data
cube does not scale well. For a table with $k$ potential grouping columns,
such an approach would have to run and store the results for
$2^k$ GROUP BY queries, one for each subset of
the columns. Each query would require a full pass of the table.

Our first paper by Harinarayan, Rajaraman and Ullman reduces this space:
it chooses a judicious subset of queries in the cube that are worthy of
precomputation; it then uses the results of those queries to compute the
results to any other query in the cube. This paper is one of the
most-cited papers in the area, in part because it was early in observing
that the structure of the data cube problem is a set-containment
lattice. This lattice structure underlies their solution, and recurs in
many other papers on data cubes (including our next paper), as well as
on certain data mining algorithms like Association Rules (a.k.a.
Frequent Itemsets)~{{[}\protect\hyperlink{ref-associationrules}{2}{]}}.
Everyone working in the OLAP space should have read this paper.

Our second paper by Zhao, Deshpande and Naughton focuses on the actual
computation of results in the cube. The paper uses an ``array-based''
approach: that is, it assumes the data is stored in an Essbase-like
sparse array structure, rather than a relational table structure, and
presents a very fast algorithm that exploits that structure. However, it
makes the surprising observation that even for relational tables, it is
worthwhile to convert tables to arrays in order to run this algorithm,
rather than to run a (far less efficient) traditional relational
algorithm. This substantially widens the design space for query engines.
The implication is that you can decouple your data model from the
internal model of your query engine. So a special-purpose ``engine''
(Multidimensional OLAP in this case) may add value by being embedded in
a more general-purpose engine (Relational in this case). Some years
after the OLAP wars, Stonebraker started arguing that ``one size doesn't
fit all'' for database engines, and hence that specialized database
engines (not unlike Essbase) are indeed
important~{{[}\protect\hyperlink{ref-onesize}{6}{]}}. This paper is an
example of how that line of reasoning sometimes plays out: clever
specialized techniques get developed, and if they're good enough they
can pay off in more general contexts as well. Innovating on both sides
of that line---specialization and generalization---has led to good
research results over the years. Meanwhile, anyone building a query
engine should keep in mind the possibility that the internal
representations of data and operations can be a superset of the
representations of the API.

Related to this issue is the fact that analytic databases have become
much more efficient in the last decade due to in-database compression,
and the march of Moore's Law. Stonebraker has asserted to me that column
stores make OLAP accelerators irrelevant. This is an interesting
argument to consider, though hasn't been validated by the market.
Vendors still build cubing engines, and BI tools commonly implement them
as accelerators on top of relational databases and Hadoop. Certainly the
caching techniques of our first paper remain relevant. But the live
query processing tradeoffs between high-performance analytic database
techniques and data cubing techniques may deserve a revisit.

Our third paper on ``online aggregation'' starts exploring from the
opposite side of the territory from OLAP, attempting to handle ad-hoc
queries quickly without precomputation by producing incrementally
refining approximate answers. The paper was inspired by the kind of
triage that people perform every day in gathering evidence to make
decisions; rather than pre-specifying hard deadlines, we often make
qualitative decisions about when to stop evaluating and to act. Specific
data-centric examples include the ``early returns'' provided in election
coverage, or the multiresolution delivery of images over low-bandwidth
connections---in both cases we have a good enough sense of what is
likely to happen long before the process completed.

Online aggregation typically makes use of sampling to achieve
incrementally refining results. This is not the first (or last!) use of
database sampling to provide approximate query answers. (Frank Olken's
thesis~{{[}\protect\hyperlink{ref-olken-phd}{5}{]}} is a good early
required reference for database sampling.) But online aggregation helped
kick off an ongoing sequence of work on approximate query processing
that has persisted over time, and is of particular interest in the
current era of Big Data and structure-on-use.

We include the first paper on online aggregation here. To appreciate the
paper, it's important to remember that databases at the time had long
operated under a mythology of ``correctness'' that is a bit hard to
appreciate in today's research environment. Up until approximately the
21st century, computers were viewed by the general populace---and the
business community---as engines of accurate, deterministic calculation.
Phrases like ``Garbage In, Garbage Out'' were invented to remind users
to put ``correct'' data into the computer, so it could do its job and
produce ``correct'' outputs. In general, computers weren't expected to
produce ``sloppy'' approximate results.

So the first battle being fought in this paper is the idea that the
complete accuracy in large-scale analytics queries is unimportant, and
that users should be able to balance accuracy and running time in a
flexible way. This line of thinking quickly leads to three research
directions that need to work in harmony: fast query processing,
statistical approximation, and user interface design. The
inter-dependencies of these three themes make for an interesting design
space that researchers and products are still exploring today.

The initial paper we include here explores how to embed this
functionality in a traditional DBMS. It also provides statistical
estimators for standard SQL aggregates over samples, and shows how
stratified sampling can be achieved using standard B-trees via ``index
striding'', to enable different groups in a GROUP BY query to be sampled
at different rates. Subsequent papers in the area have explored
integrating online aggregation with many of the other standard issues in
query processing, many of which are surprisingly tricky: joins,
parallelism, subqueries, and more recently the specifics of recent Big
Data systems like MapReduce and Spark.

Both IBM and Informix pursued commercial efforts for online aggregation
in the late 1990s, and Microsoft also had a research agenda in
approximate query processing as well. None of these efforts came to
market. One reason for this at the time was the hidebound idea that
``database customers won't tolerate wrong
answers''\footnote{This was particularly ironic given that the sampling support provided
  by some of the vendors was biased (by sampling blocks instead of
  tuples).}. A more compelling
reason related to the coupling of user interface with query engine and
approximation. At that time, many of the BI vendors were independent of
the database vendors. As a result, the database vendors didn't ``own''
the end-user experience in general, and could not deliver the online
aggregation functionality directly to users via standard APIs. For
example, traditional query cursor APIs do not allow for multiple
approximations of the same query, nor do they support confidence
intervals associated with aggregate columns. The way the market was
structured at the time did not support aggressive new technology
spanning both the back-end and front-end.

Many of these factors have changed today, and online aggregation is
getting a fresh look in research and in industry. The first motivation,
not surprisingly, is the interest in Big Data. Big Data is not only
large in volume, but also has wide ``variety'' of formats and uses which
means that it may not be parsed and understood until users want to
analyze. For exploratory analytics on Big Data, the combination of large
volumes and schema-on-use makes precomputation unattractive or
impossible. But sampling on-the-fly remains cheap and useful.

In addition, the structure of the industry and its interfaces has
changed since the 1990s. From the bottom up, query engine standards
today often emerge and evolve through open source development, and the
winning projects (e.g., Hadoop and Spark) become close enough to
monopolies that their APIs can dictate client design. At the same time
from the top down, hosted data visualization products in the cloud are
often vertically integrated: the front-end experience is the primary
concern, and is driven by a (often special-purpose) back-end
implementation without concern for standardization. In both cases, it's
possible to deliver a unique feature like online aggregation through the
stack from engine to applications.

In that context we present one of the more widely-read recent papers in
the area, on BlinkDB. The system makes use of what Olken calls
``materialized sample views'': precomputed samples over base tables,
stored to speed up approximate query answering. Like the early OLAP
papers, BlinkDB makes the case that only a few GROUP BY clauses need to
be precomputed to get good performance on (stable) workloads. Similar
arguments are made by the authors of the early AQUA project on
approximate queries~{{[}\protect\hyperlink{ref-aqua}{1}{]}}, but they
focused on precomputed synopses (``sketches'') rather than materialized
sample views as their underlying approximation mechanism. The BlinkDB
paper also makes the case for stratification in its views to capture
small groups, reminiscent of the Index Striding in the online
aggregation paper. BlinkDB has received interest in industry, and the
Spark team has recently proposed augmenting its precomputed samples with
sampling on the fly---a sensible mixture of techniques to achieve online
aggregation as efficiently as possible. Recent commercial BI tools like
ZoomData seem to use online aggregation as well (they call it ``query
sharpening'').

With all this discussion of online aggregation, it's worth taking a
snapshot of current market realities. In the 25 years since it was
widely introduced, OLAP-style precomputation has underpinned what is now
a multi-billion dollar BI industry. By contrast, approximation at the
user interface is practically non-existent. So for those of you keeping
score at home based on revenue generation, the simple solution of
precomputation is the current winner by a mile. It's still an open
question when and if approximation will become a bread-and-butter
technique in practice. At a technical level, the fundamental benefits of
sampling seem inevitably useful, and the technology trends around data
growth and exploratory analytics make it compelling in the Big Data
market. But today this is still a technology that is before its time.

A final algorithmic note: approximate queries via sketches are in fact
very widely used by engineers and data scientists in the field today as
building blocks for analysis. Outside of the systems work covered here,
well-read database students should be familiar with techniques like
CountMin sketches, HyperLogLog sketches, Bloom filters, and so on. A
comprehensive survey of the field can be found
in~{{[}\protect\hyperlink{ref-fnt-sketch}{3}{]}}; implementations of
various sketches can be found in a number of languages online, including
as user-defined functions in the MADlib library mentioned in
\hyperref[ch11-complexanalytics]{Chapter 11}.

\section*{References}

\leavevmode\hypertarget{ref-aqua}{}%
{[}1{]} Acharya, S., Gibbons, P.B., Poosala, V. and Ramaswamy, S. The
Aqua approximate query answering system. \emph{SIGMOD}, 1999.

\leavevmode\hypertarget{ref-associationrules}{}%
{[}2{]} Agrawal, R., Imieli艅ski, T. and Swami, A. Mining association
rules between sets of items in large databases. \emph{SIGMOD}, 1993.

\leavevmode\hypertarget{ref-fnt-sketch}{}%
{[}3{]} Cormode, G., Garofalakis, M., Haas, P.J. and Jermaine, C.
Synopses for massive data: Samples, histograms, wavelets, sketches.
\emph{Foundations and Trends in Databases}. 4, 1-3 (2012), 1-294.

\leavevmode\hypertarget{ref-gray-cube}{}%
{[}4{]} Gray, J., Chaudhuri, S., Bosworth, A., Layman, A., Reichart, D.,
Venkatrao, M., Pellow, F. and Pirahesh, H. Data cube: A relational
aggregation operator generalizing group-by, cross-tab, and sub-totals.
\emph{Data Mining and Knowledge Discovery}. 1, 1 (1997), 29-53.

\leavevmode\hypertarget{ref-olken-phd}{}%
{[}5{]} Olken, F. \emph{Random sampling from databases}. Ph.D. Thesis,
University of California at Berkeley, 1993.

\leavevmode\hypertarget{ref-onesize}{}%
{[}6{]} Stonebraker, M. and \c{C}etintemel, U. ``One size fits all'': An
idea whose time has come and gone. \emph{ICDE}, 2005.


\Chapter{ch9-languages}{%
Languages
}{%
Introduced by Joe Hellerstein
}

\begin{framed}
Readings:
\begin{itemize}
\item
\href{https://scholar.google.com/scholar?cluster=5934767222958591409}{Joachim
  W. Schmidt. {Some High Level Language Constructs for Data of Type
    Relation}. {ACM Transactions on Database Systems}, 2(3), 1977, 247-261.}
\item
\href{https://scholar.google.com/scholar?cluster=17215743948117955326}{Arvind
  Arasu, Shivnath Babu, Jennifer Widom. {The CQL Continuous Query
    Language: Semantic Foundations and Query Execution}. {The VLDB Journal},
  15(2), 2006, 121-142.}
\item
\href{https://scholar.google.com/scholar?cluster=9165311711752272482}{Peter
  Alvaro, Neil Conway, Joseph M. Hellerstein, William R. Marczak.
  {Consistency Analysis in Bloom: A CALM and Collected Approach}. {CIDR},
  2011.}
\end{itemize}
\end{framed}

From reading database papers, you might expect that typical database
users are data analysts, business decision-makers, or IT staff. In
practice, the majority of database users are software engineers, who
build database-backed applications that are used further up the stack.
Although SQL was originally designed with non-technical users in mind,
it is rare for people to interact directly with a database via a
language like SQL unless they are coding up a database-backed
application.

So if database systems are mostly just APIs for software development,
what do they offer programmers? Like most good software, database
systems offer powerful abstractions.

\enlargethispage{-2\baselineskip}
Two stand out:

\begin{enumerate}
  \item
  The transaction model gives programmers the abstraction of a
  single-process, sequential machine that never fails mid-task. This
  protects programmers from a gigantic cliff of complexity---namely, the
  innate parallelism of modern computing. A single compute rack today
  has thousands of cores running in parallel across dozens of machines
  that can fail independently. Yet application programmers can still
  blithely write sequential code as if it were 1965 and they were
  loading a deck of punchcards into a mainframe to be executed
  individually from start to finish.
  \item
  Declarative query languages like SQL provide programmers with
  abstractions for manipulating sets of data. Famously, declarative
  languages shield programmers from thinking about \emph{how} to access
  data items, and instead let them focus on \emph{what} data items to
  return. This data independence also shields application programmers
  from changes in the organization of underlying databases, and shields
  database administrators from getting involved in the design and
  maintenance of applications.
\end{enumerate}

Just how useful have these abstractions been over time? How useful are
they today?

\begin{enumerate}
  \item
  As a programming construct, serializable transactions have been very
  influential. It is easy for programmers to bracket their code with
  BEGIN and COMMIT/ROLLBACK. Unfortunately, as we discussed in
  \hyperref[ch6-isolation]{Chapter 6}, transactions are expensive, and
  are often compromised. ``Relaxed'' transactional semantics break the
  serial abstraction for users and expose application logic to the
  potential for races and/or unpredictable exceptions. If application
  developers want to account for this, they have to manage the
  complexity of concurrency, failure, and distributed state. A common
  response to the lack of transactions is to aspire to ``eventual
  consistency''~{{[}\protect\hyperlink{ref-bayou-session}{21}{]}}, as we
  discuss in the weak isolation section. But as we discussed in
  \hyperref[ch6-isolation]{Chapter 6}, this still shifts all the
  correctness burdens to the application developer. In my opinion, this
  situation represents a major crisis in modern software development.
  \item
  Declarative query languages have also been a success---certainly an
  improvement over the navigational languages that preceded them, which
  led to spaghetti code that needed to be rewritten every time you
  reorganized the database. Unfortunately, query languages are quite
  different from the imperative languages that programmers usually use.
  Query languages consume and produce simple unordered ``collection
  types'' (sets, relations, streams); programming languages typically
  carry out ordered execution of instructions, often over complex
  structured data types (trees, hashtables, etc.). Programmers of
  database applications are forced to bridge this so-called ``impedance
  mismatch'' between programs and database queries. This has been a
  hassle for database programmers since the earliest days of relational
  databases.
\end{enumerate}

\Section{database-language-embeddings-pascalr}{%
Database Language Embeddings: Pascal/R
}

The first paper in this section presents a classical example of tackling
the second problem: helping imperative programmers with the impedance
mismatch. The paper begins by defining operations for what we might now
recognize (40-odd years later!) as familiar collection types: the
``dictionary'' type in Python, the ``map'' type in Java or Ruby, etc.
The paper then patiently takes us through the possibilities and pitfalls
of various language constructs that seem to recur in applications across
the decades. A key theme is a desire for differentiating between
enumeration (for generating output) and quantification (for checking
properties)---the latter can often be optimized if you are explicit. In
the end, the paper proposes a declarative, SQL-like sublanguage for
Relation types that is embedded into Pascal. The result is relatively
natural and not unlike some of the better interfaces today.

Although this approach seems natural now, the topic took decades to gain
popular attention. Along the way, database ``connectivity'' APIs like
ODBC and JDBC arose as a crutch for C/C++ and Java---they allowed users
to push queries to the DBMS and iterate through results, but the type
systems remained separate, and bridging from SQL types to host language
types was unpleasant. Perhaps the best modern evolution of ideas like
Pascal/R is Microsoft's LINQ library, which provides language-embedded
collection types and functions so application developers can write
query-like code over a variety of backend databases and other
collections (XML documents, spreadsheets, etc.) We included a taste of
LINQ syntax in the DryadLINQ paper in \hyperref[ch5-dataflow]{Chapter 5}.

In the 2000's, web applications like social media, online forums,
interactive chat, photo sharing and product catalogs were implemented
and reimplemented over relational database backends. Modern scripting
languages for web programming were a bit more convenient than Pascal,
and typically included decent collection types. In that environment,
application developers eventually saw recognized patterns in their code
and codified them into what are now called Object-Relational Mappings
(ORMs). Ruby on Rails was one of the most influential ORMs to begin
with, though by now there are many others. Every popular application
programming language has at least one, and there are variations in
features and philosophy. The interested reader is referred to
Wikipedia's ``List of object-relational mapping software'' wiki.

ORMs do a few handy things for the web programmer. First they provide
language-native primitives for working with collections much like
Pascal/R. Second they can enable updates to in-memory language objects
to be transparently reflected in the database-backed state. They often
offer some language-native syntax for familiar database design concepts
like entities, relationships, keys and foreign keys. Finally, some ORMs
including Rails offer nice tools for tracking the way that database
schemas evolve over time to reflect changes in the application code
(``migrations'' in Rails terminology).

This is an area where the database research community and industry
should pay more attention: these are our users! There are some
surprising---and occasionally disconcerting---disconnects between ORMs
and databases~{{[}\protect\hyperlink{ref-feral}{6}{]}}. The author of
Rails, for example, is a colorful character named David Heinemeier
Hansson (``DHH'') who believes in ``opinionated software'' (that
reflects his opinions, of course). He was quoted saying the following:

\begin{quote}
  I don't want my database to be clever! ...I consider stored procedures
  and constraints vile and reckless destroyers of coherence. No, Mr.
  Database, you can not have my business logic. Your procedural ambitions
  will bear no fruit and you'll have to pry that logic from my dead, cold
  object-oriented hands \mbox{. . .} I want a single layer of cleverness: My
  domain model.
\end{quote}

This unwillingness to trust the DBMS leads to many problems in Rails
applications. Applications written against ORMs are often very
slow---the ORMs themselves don't do much to optimize the way that
queries are generated. Instead, Rails programmers often need to learn to
program ``differently'' to encourage Rails to generate efficient
SQL---similar to the discussion in the Pascal/R paper, they need to
learn to avoid looping and table-at-a-time iteration. A typical
evolution in a Rails app is to write it naively, observe slow
performance, study the SQL logs it generates, and rewrite the app to
convince the ORM to generate ``better'' SQL. Recent work by Cheung and
colleagues explores the idea that program synthesis techniques can
automatically generate these
optimizations~{{[}\protect\hyperlink{ref-statusquo}{9}{]}}; it's an
interesting direction, and time will tell how much complexity it can
automate away. The separation between database and applications can also
have negative effects for correctness. For example, Bailis recently
showed~{{[}\protect\hyperlink{ref-feral}{6}{]}} how a host of existing
open-source Rails applications are susceptible to integrity violations
due to improper enforcement within the application (instead of the
database).

Despite some blind spots, ORMs have generally been an important
practical leap forward in programmability of database-backed
applications, and a validation of ideas that go back as far as Pascal/R.
Some good ideas take time to catch on.

\Section{stream-queries-cql}{%
Stream Queries: CQL
}

Our second paper on CQL is a different flavor of language work---it's a
query language design paper. It presents the design of a new declarative
query language for a data model of streams. The paper is interesting for
a few reasons. First, it is a clean, readable, and relatively modern
example of query language design. Every few years a group of people
emerges with yet another data model and query language: examples include
Objects and OQL, XML and XQuery, or RDF and SPARQL. Most of these
exercises begin with an assertion that ``X changes everything'' for some
data model X, leading to the presentation of a new query language that
often seems familiar and yet strangely different than SQL. CQL is a
refreshing example of language design because it does the opposite: it
highlights that fact that streaming data, viewed through the right lens,
actually changes very little. CQL evolves SQL just enough to isolate the
key distinctions between querying ``resting'' tables and ``moving''
streams. This leaves us with a crisp understanding of what's really
different, semantically, when you have to talk about streams; many other
current streaming languages are quite a bit more ad hoc and messy than
CQL.

In addition to this paper being a nice exemplar of thoughtful query
language design, it also represents a research area that received a lot
of attention in the database literature, and remains intriguing in
practice. The first generation of streaming data research systems from
the early 2000's~{{[}\protect\hyperlink{ref-abadi2003aurora}{1},
  \protect\hyperlink{ref-chandrasekarantelegraphcq}{8},
  \protect\hyperlink{ref-motwani2003query}{16},
  \protect\hyperlink{ref-naughton2001niagara}{17}{]}} did not have major
uptake either as open source or in the variety of startups that grew out
of those systems. However the topic of stream queries has been gaining
interest again in industry in recent years, with open source systems
like SparkStreaming, Storm and Heron seeing uptake, and companies like
Google espousing the importance of continuous dataflow as a new reality
of modern services~{{[}\protect\hyperlink{ref-googledataflow}{2}{]}}. We
may yet see stream query systems occupy more than their current small
niche in financial services.

Another reason CQL is interesting is that streams are something of a
middle ground between databases and ``events''. Databases store and
retrieve collection types; event systems transmit and handle discrete
events. Once you view your events as data, then event programming and
stream programming look quite similar. Given that event programming is a
widely-used programming model in certain domains (e.g. Javascript for
user interfaces, Erlang for distributed systems), there should be a
relatively small impedance mismatch between an event programming
language like Javascript and a data stream system. An interesting
example of this approach is the Rx (Reactive extensions) language, which
is a streaming addition to LINQ that makes programming event streams
feel like writing functional query plans; or as its author Erik Meijer
puts it, ``your mouse is a
database''~{{[}\protect\hyperlink{ref-meijer2012your}{15}{]}}.

\Section{programming-correct-applications-without-transactions-bloom}{%
Programming Correct Applications without Transactions: Bloom
}

The third paper on Bloom connects to a number of the points above; it
has a relational model of state at the application level, and a notion
of network channels that relates to CQL streams. But the primary goal is
to help programmers manage the loss of the first abstraction at the
beginning of this chapter introduction; the one I described as a major
crisis. A big question for modern developers is this: can you find a
correct distributed implementation for your program without using
transactions or other other expensive schemes to control orders of
operation?

Bloom's answer to this question is to give programmers a ``disorderly''
programming language: one that discourages them from accidentally using
ordering. Bloom's default data structures are relations; its basic
programming constructs are logical rules that can run in any order. In
short, it's a general-purpose language that is similar to a relational
query language. For the same reason that SQL queries can be optimized
and parallelized without changing output, simple Bloom programs have a
well-defined (consistent) result independent of the order of execution.
The exception to this intuition comes with lines of Bloom code that are
``non-monotonic'', testing for a property that can oscillate between
true and false as time passes (e.g. ``NOT EXISTS x'' or ``HAVING COUNT()
= x''.) These rules are sensitive to execution and message ordering, and
need to be ``protected'' by coordination mechanisms.

The CALM theorem formalizes this notion, answering the question above
definitively: you can find a consistent, distributed, coordination-free
implementation for your program if and only if its specification is
monotonic~{{[}\protect\hyperlink{ref-ameloot2013relational}{5},
  \protect\hyperlink{ref-hellerstein2010declarative}{13}{]}}. The Bloom
paper also illustrates how a compiler can use CALM in practice to
pinpoint the need for coordination in Bloom programs. CALM analysis can
also be applied to data stream languages in systems like Storm with the
help of programmer
annotations~{{[}\protect\hyperlink{ref-alvaro2014blazes}{3}{]}}. A
survey of the theoretical results in this area is given
in~{{[}\protect\hyperlink{ref-ameloot2014declarative}{4}{]}}.

There has been a flurry of related language work on avoiding
coordination: a number of papers propose using associative, commutative,
idempotent
operations~{{[}\protect\hyperlink{ref-clements2015scalable}{10},
  \protect\hyperlink{ref-quicksand}{12},
  \protect\hyperlink{ref-crdt}{19}{]}}; these are inherently monotonic.
Another set of work examines alternative correctness criteria, e.g.,
ensuring only specific invariants over database
state~{{[}\protect\hyperlink{ref-coord-avoid}{7}{]}} or using
alternative program analysis to deliver serializable outcomes without
impelementing traditional read-write
concurrency~{{[}\protect\hyperlink{ref-roy2015homeostasis}{18}{]}}. The
area is still new; papers have different models (e.g. some with
transactional boundaries and some not) and often don't agree on
definitions of ``consistency'' or ``coordination''. (CALM defines
consistency in terms of globally deterministic outcomes, coordination as
messaging that is required regardless of data partitioning or
replication~{{[}\protect\hyperlink{ref-ameloot2013relational}{5}{]}}.)
It's important to get more clarity and ideas here---if programmers can't
have transactions then they need help at the app-development layer.

Bloom also serves as an example of a recurring theme in database
research: general-purpose declarative languages (a.k.a. ``logic
programming''). Datalog is the standard example, and has a long and
controversial history in database research. A favorite topic of database
theoreticians in the 1980's, Datalog met ferocious backlash from systems
researchers of the day as being irrelevant in
practice~{{[}\protect\hyperlink{ref-lagunabeach}{20}{]}}. More recently
it has gotten some attention from (younger) researchers in databases and
other applied
fields~{{[}\protect\hyperlink{ref-green2013datalog}{11}{]}}---for
example, Nicira's software-defined networking stack (acquired by VMWare
for a cool billion dollars) uses a Datalog language for network
forwarding state~{{[}\protect\hyperlink{ref-koponen2014network}{14}{]}}.
There is a spectrum between using declarative sublanguages for accessing
database state, and very aggressive uses of declarative programming like
Bloom for specifying application logic. Time will tell how this
declarative-imperative boundary shifts for programmers in various
contexts, including infrastructure, applications, web clients and mobile
devices.

\section*{References}

\leavevmode\hypertarget{ref-abadi2003aurora}{}%
{[}1{]} Abadi, D.J., Carney, D., 脟etintemel, U., Cherniack, M., Convey,
C., Lee, S., Stonebraker, M., Tatbul, N. and Zdonik, S. Aurora: A new
model and architecture for data stream management. \emph{The VLDB
  Journal---The International Journal on Very Large Data Bases}. 12, 2
(2003), 120-139.

\leavevmode\hypertarget{ref-googledataflow}{}%
{[}2{]} Akidau, T. and others The dataflow model: A practical approach
to balancing correctness, latency, and cost in massive-scale, unbounded,
out-of-order data processing. \emph{VLDB}, 2015.

\leavevmode\hypertarget{ref-alvaro2014blazes}{}%
{[}3{]} Alvaro, P., Conway, N., Hellerstein, J.M. and Maier, D. Blazes:
Coordination analysis for distributed programs. \emph{Data engineering
  (iCDE), 2014 iEEE 30th international conference on}, 2014, 52-63.

\leavevmode\hypertarget{ref-ameloot2014declarative}{}%
{[}4{]} Ameloot, T.J. Declarative networking: Recent theoretical work on
coordination, correctness, and declarative semantics. \emph{ACM SIGMOD
  Record}. 43, 2 (2014), 5-16.

\leavevmode\hypertarget{ref-ameloot2013relational}{}%
{[}5{]} Ameloot, T.J., Neven, F. and Van den Bussche, J. Relational
transducers for declarative networking. \emph{Journal of the ACM
  (JACM)}. 60, 2 (2013), 15.

\leavevmode\hypertarget{ref-feral}{}%
{[}6{]} Bailis, P., Fekete, A., Franklin, M.J., Ghodsi, A., Hellerstein,
J.M. and Stoica, I. Feral Concurrency Control: An empirical
investigation of modern application integrity. \emph{SIGMOD}, 2015.

\leavevmode\hypertarget{ref-coord-avoid}{}%
{[}7{]} Bailis, P., Fekete, A., Franklin, M.J., Hellerstein, J.M.,
Ghodsi, A. and Stoica, I. Coordination avoidance in database systems.
\emph{VLDB}, 2015.

\leavevmode\hypertarget{ref-chandrasekarantelegraphcq}{}%
{[}8{]} Chandrasekaran, S., Cooper, O., Deshpande, A., Franklin, M.J.,
Hellerstein, J.M., Hong, W., Krishnamurthy, S., Madden, S., Raman, V.,
Reiss, F. and others TelegraphCQ: Continuous dataflow processing for an
uncertain world. \emph{CIDR}, 2003.

\leavevmode\hypertarget{ref-statusquo}{}%
{[}9{]} Cheung, A., Arden, O., Madden, S., Solar-Lezama, A. and Myers,
A.C. StatusQuo: Making familiar abstractions perform using program
analysis. \emph{CIDR}, 2013.

\leavevmode\hypertarget{ref-clements2015scalable}{}%
{[}10{]} Clements, A.T., Kaashoek, M.F., Zeldovich, N., Morris, R.T. and
Kohler, E. The scalable commutativity rule: Designing scalable software
for multicore processors. \emph{ACM Transactions on Computer Systems
  (TOCS)}. 32, 4 (2015), 10.

\leavevmode\hypertarget{ref-green2013datalog}{}%
{[}11{]} Green, T.J., Huang, S.S., Loo, B.T. and Zhou, W. Datalog and
recursive query processing. \emph{Foundations and Trends in Databases}.
5, 2 (2013), 105-195.

\leavevmode\hypertarget{ref-quicksand}{}%
{[}12{]} Helland, P. and Campbell, D. Building on quicksand.
\emph{CIDR}, 2009.

\leavevmode\hypertarget{ref-hellerstein2010declarative}{}%
{[}13{]} Hellerstein, J.M. The declarative imperative: Experiences and
conjectures in distributed logic. \emph{ACM SIGMOD Record}. 39, 1
(2010), 5-19.

\leavevmode\hypertarget{ref-koponen2014network}{}%
{[}14{]} Koponen, T., Amidon, K., Balland, P., Casado, M., Chanda, A.,
Fulton, B., Ganichev, I., Gross, J., Gude, N., Ingram, P. and others
Network virtualization in multi-tenant datacenters. \emph{USENIX nSDI},
2014.

\leavevmode\hypertarget{ref-meijer2012your}{}%
{[}15{]} Meijer, E. Your mouse is a database. \emph{Queue}. 10, 3
(2012), 20.

\leavevmode\hypertarget{ref-motwani2003query}{}%
{[}16{]} Motwani, R., Widom, J., Arasu, A., Babcock, B., Babu, S.,
Datar, M., Manku, G., Olston, C., Rosenstein, J. and Varma, R. Query
processing, resource management, and approximation in a data stream
management system. \emph{CIDR}, 2003.

\leavevmode\hypertarget{ref-naughton2001niagara}{}%
{[}17{]} Naughton, J.F., DeWitt, D.J., Maier, D., Aboulnaga, A., Chen,
J., Galanis, L., Kang, J., Krishnamurthy, R., Luo, Q., Prakash, N. and
others The niagara internet query system. \emph{IEEE Data Eng. Bull.}
24, 2 (2001), 27-33.

\leavevmode\hypertarget{ref-roy2015homeostasis}{}%
{[}18{]} Roy, S., Kot, L., Bender, G., Ding, B., Hojjat, H., Koch, C.,
Foster, N. and Gehrke, J. The homeostasis protocol: Avoiding transaction
coordination through program analysis. \emph{SIGMOD}, 2015.

\leavevmode\hypertarget{ref-crdt}{}%
{[}19{]} Shapiro, M., Preguica, N., Baquero, C. and Zawirski, M. Shapiro
Marc. Preguica Nuno. Baquero Carlos. Zawirski Marek. A comprehensive
study of convergent and commutative replicated data types. INRIA TR
7506. 2011.

\leavevmode\hypertarget{ref-lagunabeach}{}%
{[}20{]} Stonebraker, M. and Neuhold, E. The laguna beach report.
International Institute of Computer Science. 1989.

\leavevmode\hypertarget{ref-bayou-session}{}%
{[}21{]} Terry, D.B., Demers, A.J., Petersen, K., Spreitzer, M.J.,
Theimer, M.M. and others Session guarantees for weakly consistent
replicated data. \emph{PDIS}, 1994.



\Chapter{ch10-webdata}{%
Web Data
}{%
Introduced by Peter Bailis
}

\begin{framed}
Readings:
\begin{itemize}
\item
\href{https://scholar.google.com/scholar?cluster=9820961755208603037}{Sergey
  Brin and Larry Page. {The Anatomy of a Large-scale Hypertextual Web
    Search Engine}. {WWW}, 1998.}
\item
\href{https://scholar.google.com/scholar?cluster=15869287167041695406}{Eric
  A. Brewer. {Combining Systems and Databases: A Search Engine
    Retrospective}. {Readings in Database Systems, Fourth Edition}, 2005.}
\item
\href{https://scholar.google.com/scholar?cluster=11659194181134800008}{Michael
  J. Cafarella, Alon Halevy, Daisy Zhe Wang, Eugene Wu, Yang Zhang.
  {WebTables: Exploring the Power of Tables on the Web}. {VLDB}, 2008.}
\end{itemize}
\end{framed}

Since the previous edition of this collection, the World Wide Web has
unequivocally laid any lingering questions regarding its longevity and
global impact to rest. Several multi-Billion-user services including
Google and Facebook have become central to modern life in the first
world, while Internet- and Web-related technology has permeated both
business and personal interactions. The Web is undoubtedly here to
stay---at least for the foreseeable future.

Web data systems bring a new set of challenges, including high scale,
data heterogeneity, and a complex and evolving set of user interaction
modes. Classical relational database system designs did not have the Web
workload in mind, and are not the technology of choice in this context.
Rather, Web data management requires a melange of techniques spanning
information retrieval, database internals, data integration, and
distributed systems. In this section, we include three papers that
highlight technical solutions to problems inherent in Web data
management.

Our first two papers describe the internals of search engine and
indexing technology. Our first paper, from Larry Page and Sergey Brin,
Google co-founders, describes the internals of an early prototype of
Google. The paper is interesting both from a historical perspective as
well as a technical one. The first Web indices, such as Yahoo!,
consisted of human-curated ``directories''. While directory curation
proved useful, directories were difficult to scale and required
considerable human power to maintain. As a result, a number of search
engines, including Google but also Inktomi, co-created by Eric Brewer,
author of the second paper, sought automated approaches. The design of
these engines is conceptually straightforward: a set of crawlers
downloads copies of web data and builds (and maintains) read-only
indices that are used to compute a relevance scoring function. Queries,
in turn, are serviced by a front-end web service that reads from the
indices and presents an ordered set of results, ranked by scoring
function.

The implementation and realization of these engines is complex. For
example, scoring algorithms are highly tuned, and their implementation
is considered a trade secret even within search engines today: Web
authors have a large incentive to manipulate the scoring function to
their advantage. The PageRank algorithm described in the Google paper
(and detailed in~{{[}\protect\hyperlink{ref-pagerank}{5}{]}}) is an
famous example of a scoring function, and measures the ``influence'' of
each page measured according to the hyperlink graph. Both papers
describe how a combination of mostly unspecified attributes is used for
scoring in practice, including ``anchor text'' (providing context on the
source of a link) and other forms of metadata. The algorithmic
foundations of these techniques, such as keyword indexing date, date to
the 1950s~{{[}\protect\hyperlink{ref-luhn}{4}{]}}, while others, such as
TFxIDF ranking and inverted indices, date to the
1960s~{{[}\protect\hyperlink{ref-salton-indexing}{6}{]}}. Many of the
key systems innovations in building Internet search engines came in
scaling them and in handling dirty, heterogenous data sources.

While the high-level details of these papers are helpful in
understanding how modern search engines operate, these papers are also
interesting for their commentary on the process of building a production
Web search engine. A central message in each is that Web services must
account for variety; the Google authors describe how assumptions made in
typical information retrieval techniques may no longer hold in a Web
context (e.g., the ``Bill Clinton sucks'' web page). Web sources change
at varying rates, necessitating prioritized crawling for maintaining
fresh indices. Brewer also highlights the importance of fault tolerance
and availability of operation, echoing his experience in the field
building Inktomi (which also led to the development of concepts
including \emph{harvest} and
\emph{yield}~{{[}\protect\hyperlink{ref-harvest-yield}{2}{]}} and the
CAP Theorem; see \hyperref[ch7-queryoptimization]{Chapter 7}). Brewer
outlines the difficulty in building a search engine using commodity
database engines (e.g., Informix was 10x slower than Inktomi's custom
solution). However, he notes that the principles of database system
design, including ``top-down'' design, data independence, and a
declarative query engine, are valuable in this context---if
appropriately adapted.

Today, Web search engines are considered mature technology. However,
competing services continually improve search experience by adding
additional functionality. Today's search engines are much more than
information retrieval engines for textual data web pages; the content of
the first two papers is a small subset of the internals of a service
like Google or Baidu. These services provide a range of functionality,
including targeted advertisement, image search, navigation, shopping,
and mobile search. There is undoubtedly bleed-over in retrieval, entity
resolution, and indexing techniques between these domains, but each
requires domain-specific adaptation.

As an example of a new type of search enabled by massive Web data, we
include a paper from the WebTables project led by Alon Halevy at Google.
WebTables allows users to query and understand relationships between
data stored in HTML tables. HTML tables are inherently varied in
structure due to a lack of fixed schema. However, aggregating enough of
them at Web scale and performing some lightweight automated data
integration enables some interesting queries (e.g., a table of influenza
outbreak locations can be combined with a table containing data about
city populations). Mining the schema of these tables, determining their
structure and veracity (e.g., only 1\% of the tables in the paper corpus
were, in fact, relations), and efficiently inferring their relationships
is difficult. The paper we have included describes techniques for
building an attribute correlation statistics database (AcsDB) to answer
queries about the table metadata, enabling novel functionality including
schema auto-complete. The WebTables project continues today in various
forms, including Google Table Search and integration with Google's core
search technology; an update on the project can be found
in~{{[}\protect\hyperlink{ref-webtables-update}{1}{]}}. The ability to
produce structured search results is desirable in several
non-traditional domains, including mobile, contextual, and audio-based
search.

The WebTables paper in particular highlights the power of working with
Web data at scale. In a 2009 article, Halevy and colleagues describe the
``Unreasonable Effectiveness of Data,'' effectively arguing that, with
sufficient amount of data, enough latent structure is captured to make
modeling simpler: relatively simple data mining techniques often beat
more mathematically sophisticated statistical
models~{{[}\protect\hyperlink{ref-unreasonable-data}{3}{]}}. This
argument stresses the potential for unlocking hidden structure by sheer
volume of data and computation, whether mining schema correlations or
performing machine translation between languages. With a big enough
haystack, needles become large. Even examining 1\% of the tables in the
web corpus, the VLDB 2009 paper studies 154M distinct relations, a
corpus that was ``five orders of magnitude larger than the largest one
{[}previously{]} considered.''

The barrier for performing analysis of massive datasets and system
architectures outside of these companies is decreasing, due to cheap
commodity storage and cloud computing resources. However, it is
difficult to replicate the feedback loop between users (e.g., spammers)
and algorithms (e.g., search ranking algorithms). Internet companies are
uniquely positioned to pioneer systems designs that account for this
feedback loop. As database technologies power additional interactive
domains, we believe this paradigm will become even more important. That
is, the database market and interesting database workloads may benefit
from similar analyses. For example, it would be interesting to perform a
similar analysis on hosted database platforms such as Amazon Redshift
and Microsoft SQL Azure, enabling a variety of functionality including
index auto-tuning, adaptive query optimization, schema discovery from
unstructured data, query autocomplete, and visualization
recommendations.

\section*{References}

\leavevmode\hypertarget{ref-webtables-update}{}%
{[}1{]} Balakrishnan, S., Halevy, A., Harb, B., Lee, H., Madhavan, J.,
Rostamizadeh, A., Shen, W., Wilder, K., Wu, F. and Yu, C. Applying
webTables in practice. \emph{CIDR}, 2015.

\leavevmode\hypertarget{ref-harvest-yield}{}%
{[}2{]} Brewer, E. and others Lessons from giant-scale services.
\emph{Internet Computing, IEEE}. 5, 4 (2001), 46-55.

\leavevmode\hypertarget{ref-unreasonable-data}{}%
{[}3{]} Halevy, A., Norvig, P. and Pereira, F. The unreasonable
effectiveness of data. \emph{IEEE Intelligent Systems}. 24, 2 (Mar.
2009), 8-12.

\leavevmode\hypertarget{ref-luhn}{}%
{[}4{]} Luhn, H.P. Auto-encoding of documents for information retrieval
systems. \emph{Modern Trends in Documentation}. (1959), 45-58.

\leavevmode\hypertarget{ref-pagerank}{}%
{[}5{]} Page, L., Brin, S., Motwani, R. and Winograd, T. The PageRank
citation ranking: Bringing order to the web. Stanford InfoLab. 1999.
SIDL-WP-1999-0120.

\leavevmode\hypertarget{ref-salton-indexing}{}%
{[}6{]} Salton, G. and Lesk, M.E. Computer evaluation of indexing and
text processing. \emph{Journal of the ACM (JACM)}. 15, 1 (1968), 8-36.



\hypertarget{ch11-complexanalytics}{
  \chapter[Complex Analytics]{A Biased Take on a Moving Target: Complex Analytics\\{\Large by Michael Stonebraker}}\label{ch11-complexanalytics}
}

In the past 5-10 years, new analytic workloads have emerged that are
more complex than the typical business intelligence (BI) use case. For
example, internet advertisers might want to know ``How do women who
bought an Apple computer in the last four days differ statistically from
women who purchased a Ford pickup truck in the same time period?''
\&nbspThe next question might be: ``Among all our ads, which one is the
most profitable to show to the female Ford buyers based on their
click-through likelihood?'' These are the questions asked by today's
data scientists, and represent a very different use case from the
traditional SQL analytics run by business intelligence specialists. It
is widely assumed that data science will completely replace business
intelligence over the next decade or two, since it represents a more
sophisticated approach to mining data warehouses for new insights. As
such, this document focuses on the needs of data scientists.

I will start this section with a description of what I see as the job
description of a data scientist. After cleaning and wrangling his data,
which currently consumes the vast majority of his time and which is
discussed in the section on data integration, he generally performs the
following iteration:

\begin{verbatim}
Until(tired) {
  Data management operation(s);
  Analytic operation(s);
}
\end{verbatim}

In other words, he has an iterative discovery process, whereby he
isolates a data set of interest and then performs some analytic
operation on it. This often suggests either a different data set to try
the same analytic on or a different analytic on the same data set. By
and large what distinguishes data science from business intelligence is
that the analytics are predictive modeling, machine learning,
regressions, ... and not SQL analytics.

In general, there is a pipeline of computations that constitutes the
analytics. For example, Tamr has a module which performs entity
consolidation (deduplication) on a collection of records, say N of them,
at scale. To avoid the N ** 2 complexity of brute force algorithms, Tamr
identifies a collection of ``features'', divides them into ranges that
are unlikely to co-occur, computes (perhaps multiple) ``bins'' for each
record based on these ranges, reshuffles the data in parallel so it is
partitioned by bin number, deduplicates each bin, merges the results,
and finally constructs composite records out of the various clusters of
duplicates. This pipeline is partly SQL-oriented (partitioning) and
partly array-oriented analytics. Tamr seems to be typical of data
science workloads in that it is a pipeline with half a dozen steps.

Some analytic pipelines are ``one-shots'' which are run once on a batch
of records. However, most production applications are incremental in
nature. For example, Tamr is run on an initial batch of input records
and then periodically a new ``delta'' must be processed as new or
changed input records arrive. There are two approaches to incremental
operation. If deltas are processed as ``mini batches'' at periodic
intervals of (say) one day, one can add the next delta to the previously
processed batch and rerun the entire pipeline on all the data each time
the input changes. Such a strategy will be very wasteful of computing
resources. Instead, we will focus on the case where incremental
algorithms must be run after an initial batch processing operation. Such
incremental algorithms require intermediate states of the analysis to be
saved to persistent storage at each interation. Although the Tamr
pipeline is of length 6 or so, each step must be saved to persistent
storage to support incremental operation. Since saving state is a data
management operation, this make the analytics pipeline of length one.

The ultimate ``real time'' solution is to run incremental analytics
continuously services by a streaming platform such as discussed in the
section on new DBMS technology. Depending on the arrival rate of new
records, either solution may be preferred.

Most complex analytics are array-oriented, i.e. they are a collection of
linear algebra operations defined on arrays. Some analytics are graph
oriented, such as social network analysis. It is clear that arrays can
be simulated on table-based systems and that graphs can be simulated on
either table systems or array systems. As such, later in this document,
we discuss how many different architectures are needed for this used
case.

Some problems deal with dense arrays, which are ones where almost all
cells have a value. For example, an array of closing stock prices over
time for all securities on the NYSE will be dense, since every stock has
a closing price for each trading day. On the other hand, some problems
are sparse. For example, a social networking use case represented as a
matrix would have a cell value for every pair of persons that were
associated in some way. Clearly, this matrix will be very sparse.
Analytics on sparse arrays are quite different from analytics on dense
arrays.

In this section we will discuss such workloads at scale. If one wants to
perform such pipelines on ``small data'' then any solution will work
fine.

The goal of a data science platform is to support this iterative
discovery process. We begin with a sad truth. Most data science
platforms are file-based and have nothing to do with DBMSs. The
preponderance of analytic codes are run in R, MatLab, SPSS, SAS and
operate on file data. In addition, many Spark users are reading data
from files. An exemplar of this state of affairs is the NERSC high
performance computing (HPC) system at Lawrence Berkeley Labs. This
machine is used essentially exclusively for complex analytics; however,
we were unable to get the Vertica DBMS to run at all, because of
configuration restrictions. In addition, most ``big science'' projects
build an entire software stack from the bare metal on up. It is
plausible that this state of affairs will continue, and DBMSs will not
become a player in this market. However, there are some hopeful signs
such as the fact that genetic data is starting to be deployed on DBMSs,
for example the 1000 Genomes
Project~{{[}\protect\hyperlink{ref-1000-genomes}{13}{]}} is based on
SciDB.

In my opinion, file system technology suffers from several big
disadvantages. First metadata (calibration, time, etc.) is often not
captured or is encoded in the name of the file, and is therefore not
searchable. Second, sophisticated data processing to do the data
management piece of the data science workload is not available and must
be written (somehow). Third, file data is difficult to share data among
colleagues. I know of several projects which export their data along
with their parsing program. The recipient may be unable to recompile
this accessor program or it generates an error. In the rest of this
discussion, I will assume that data scientists over time wish to use
DBMS technology. Hence, there will be no further discussion of
file-based solutions.

\begin{table}
\footnotesize
\centering
\begin{tabu}{lll}
  \toprule
  & \textbf{Loosely coupled} & \textbf{Tightly coupled}\\\midrule
  \textbf{Array representation} & & SciDB, TileDB, Rasdaman\\
  \textbf{Table representation} & Spark + HBase & MADLib, Vertica +
  R\\
  \bottomrule
\end{tabu}

\caption{A Classification of Data Science Platforms\label{table:data-science-platforms}}
\end{table}

With this backdrop, we show in \autoref{table:data-science-platforms} a classification of data science
platforms. To perform the data management portion, one needs a DBMS,
according to our assumption above. This DBMS can have one of two
flavors. First, it can be record-oriented as in a relational row store
or a NoSQL engine or column-oriented as in most data warehouse systems.
In these cases, the DBMS data structure is not focused on the needs of
analytics, which are essentially all array-oriented, so a more natural
choice would be an array DBMS. The latter case has the advantage that no
conversion from a record or column structure is required to perform
analytics. Hence, an array structure will have an innate advantage in
performance. In addition, an array-oriented storage structure is
multi-dimensional in nature, as opposed to table structures which are
usually one-dimensional. Again, this is likely to result in higher
performance.

The second dimension concerns the coupling between the analytics and the
DBMS. On the one hand, they can be independent, and one can run a query,
copying the result to a different address space where the analytics are
run. At the end of the analytics pipeline (often of length one), the
result can be saved back to persistent storage. This will result in lots
of data churn between the DBMS and the analytics. On the other hand, one
can run analytics as user-defined functions in the same address space as
the DBMS. Obviously the tight coupling alternative will lower data churn
and should result in superior performance.

In this light, there are four cells, as noted in \autoref{table:data-science-platforms}. In the lower
left corner, Map-Reduce used to be the exemplar; more recently Spark has
eclipsed Map-Reduce as the platform with the most interest. There is no
persistence mechanism in Spark, which depends on RedShift or H-Base, or
... for this purpose. Hence, in Spark a user runs a query in some DBMS
to generate a data set, which is loaded into Spark, where analytics are
performed. The DBMSs supported by Spark are all record or
column-oriented, so a conversion to array representation is required for
the analytics.

A notable example in the lower right hand corner is
MADLIB~{{[}\protect\hyperlink{ref-madlib}{8}{]}}, which is a
user-defined function library supported by the RDBMS Greenplum. Other
vendors have more recently started supporting other alternatives; for
example Vertica supports user-defined functions in R. In the upper right
hand corner are array systems with built-in analytics such as
SciDB~{{[}\protect\hyperlink{ref-scidb}{15}{]}},
TileDB~{{[}\protect\hyperlink{ref-bigdawg}{6}{]}} or
Rasdaman~{{[}\protect\hyperlink{ref-rasdaman}{1}{]}}.

In the rest of this document, we discuss performance implications.
First, one would expect performance to improve as one moves from lower
left to upper right in \autoref{table:data-science-platforms}. Second, most complex analytics reduce to
a small collection of ``inner loop'' operations, such as matrix
multiply, singular-value decomposition and QR decomposition. All are
computationally intensive, typically floating point codes. It is
accepted by most that hardware-specific pipelining can make nearly an
order of magnitude difference in performance on these sorts of codes. As
such, libraries such as BLAS, LAPACK, and ScaLAPACK, which call the
hardware-optimized Intel MKL library, will be wildly faster than codes
which don't use hardware optimization. Of course, hardware optimization
will make a big difference on dense array calculations, where the
majority of the effort is in floating point computation. It will be less
significance on sparse arrays, where indexing issues may dominate the
computation time.

Third, codes that provide approximate answers are way faster than ones
that produce exact answers. If you can deal with an approximate answer,
then you will save mountains of time.

Fourth, High Performance Computing (HPC) hardware are generally
configured to support large batch jobs. As such, they are often
structured as a computation server connected to a storage server by
networking, whereby a program must pre-allocation disk space in a
computation server cache for its storage needs. This is obviously at
odds with a DBMS, which expects to be continuously running as a service.
Hence, be aware that you may have trouble with DBMS systems on HPC
environments. An interesting area of exploration is whether HPC machines
can deal with both interactive and batch workloads simultaneously
without sacrificing performance.

Fifth, scalable data science codes invariably run on multiple nodes in a
computer network and are often
network-bound~{{[}\protect\hyperlink{ref-duggan-array}{5}{]}}. In this
case, you must pay careful attention to networking costs and TCP-IP may
not be a good choice. In general MPI is a higher performance
alternative.

Sixth, most analytics codes that we have tested fail to scale to large
data set sizes, either because they run out of main memory or because
they generate temporaries that are too large. Make sure you test any
platform you would consider running on the data set sizes you expect in
production!

Seventh, the structure of your analytics pipeline is crucial. If your
pipeline is on length one, then tight coupling is almost certainly a
good idea. On the other hand, if the pipeline is on length 10, loose
coupling will perform almost as well. In incremental operation, expect
pipelines of length one.

In general, all solutions we know of have scalability and performance
problems. Moreover, most of the exemplars noted above are rapidly moving
targets, so performance and scalability will undoubtedly improve. In
summary, it will be interesting to see which cells in \autoref{table:data-science-platforms} have legs
and which ones don't. The commercial marketplace will be the ultimate
arbitrer!

In my opinion, complex analytics is current in its ``wild west'' phase,
and we hope that the next edition of the red book can identify a
collection of core seminal papers. In the meantime, there is substantial
research to be performed. Specifically, we would encourage more
benchmarking in this space in order to identify flaws in existing
platforms and to spur further research and development, especially
benchmarks that look at end-to-end tasks involving both data management
tasks and analytics. This space is moving fast, so the benchmark results
will likely be transient. That's probably a good thing: we're in a phase
where the various projects should be learning from each other.

There is currently a lot of interest in custom parallel algorithms for
core analytics tasks like convex optimization; some of it from the
database community. It will be interesting to see if these algorithms
can be incorporated into analytic DBMSs, since they don't typically
follow a traditional dataflow execution style. An exemplar here is
Hogwild! ~{{[}\protect\hyperlink{ref-hogwild}{12}{]}}, which achieves
very fast performance by allowing lock-free parallelism in shared
memory. Google Downpour~{{[}\protect\hyperlink{ref-dean2012large}{4}{]}}
and Microsoft's Project
Adam~{{[}\protect\hyperlink{ref-projectadam}{2}{]}} both adapt this
basic idea to a distributed context for deep learning.

Another area where exploration is warranted is out-of-memory algorithms.
For example, Spark requires your data structures to fit into the
combined amount of main memory present on the machines in your cluster.
Such solutions will be brittle, and will almost certainly have
scalability problems.

Furthermore, an interesting topic is the desirable approach to graph
analytics. One can either build special purpose graph analytics, such as
GraphX~{{[}\protect\hyperlink{ref-graphx}{7}{]}} or
GraphLab~{{[}\protect\hyperlink{ref-graphlab}{11}{]}} and connect them
to some sort of DBMS. Alternately, one can simulate such codes with
either array analytics, as espoused in
D4M~{{[}\protect\hyperlink{ref-d4m}{10}{]}} or table analytics, as
suggested in~{{[}\protect\hyperlink{ref-vertexica}{9}{]}}. Again, may
the solution space bloom, and the commercial market place be the
arbiter!

Lastly, many analytics codes use MPI for communication, whereas DBMSs
invariably use TCP-IP. Also, parallel dense analytic packages, such as
ScaLAPACK, organize data into a block-cyclic organization across nodes
in a computing cluster~{{[}\protect\hyperlink{ref-scalapack}{3}{]}}. I
know of no DBMS that supports block-cyclic partitioning. Removing this
impedance mismatch between analytic packages and DBMSs is an interesting
research area, one that is targeted by the Intel-sponsored ISTC on Big
Data~{{[}\protect\hyperlink{ref-istc-bigdata}{14}{]}}.

\section*{References}

\leavevmode\hypertarget{ref-rasdaman}{}%
{[}1{]} Baumann, P., Dehmel, A., Furtado, P., Ritsch, R. and Widmann, N.
The multidimensional database system rasDaMan. \emph{SIGMOD}, 1998.

\leavevmode\hypertarget{ref-projectadam}{}%
{[}2{]} Chilimbi, T., Suzue, Y., Apacible, J. and Kalyanaraman, K.
Project adam: Building an efficient and scalable deep learning training
system. \emph{OSDI}, 2014.

\leavevmode\hypertarget{ref-scalapack}{}%
{[}3{]} Choi, J. and others ScaLAPACK: A portable linear algebra library
for distributed memory computers---Design issues and performance.
\emph{Applied parallel computing computations in physics, chemistry and
  engineering science}. Springer. 95-106.

\leavevmode\hypertarget{ref-dean2012large}{}%
{[}4{]} Dean, J., Corrado, G., Monga, R., Chen, K., Devin, M., Mao, M.,
Senior, A., Tucker, P., Yang, K., Le, Q.V. and others Large scale
distributed deep networks. \emph{Advances in neural information
  processing systems}, 2012, 1223-1231.

\leavevmode\hypertarget{ref-duggan-array}{}%
{[}5{]} Duggan, J. and Stonebraker, M. Incremental elasticity for array
databases. \emph{Proceedings of the 2014 aCM sIGMOD international
  conference on management of data}, 2014, 409-420.

\leavevmode\hypertarget{ref-bigdawg}{}%
{[}6{]} Elmore, A., Duggan, J., Stonebraker, M., Balazinska, M.,
Cetintemel, U., Gadepally, V., Heer, J., Howe, B., Kepner, J., Kraska,
T. and others A demonstration of the BigDAWG polystore system.
\emph{VLDB}, 2015.

\leavevmode\hypertarget{ref-graphx}{}%
{[}7{]} Gonzales, J.E., Xin, R.S., Crankshaw, D., Dave, A., Franklin,
M.J. and Stoica, I. GraphX: Unifying data-parallel and graph-parallel
analytics. \emph{OSDI}, 2014.

\leavevmode\hypertarget{ref-madlib}{}%
{[}8{]} Hellerstein, J.M., R茅, C., Schoppmann, F., Wang, D.Z., Fratkin,
E., Gorajek, A., Ng, K.S., Welton, C., Feng, X., Li, K. and others The
MADlib analytics library: or MAD skills, the SQL. \emph{VLDB}, 2012.

\leavevmode\hypertarget{ref-vertexica}{}%
{[}9{]} Jindal, A., Rawlani, P., Wu, E., Madden, S., Deshpande, A. and
Stonebraker, M. Vertexica: Your relational friend for graph analytics!
\emph{VLDB}, 2014.

\leavevmode\hypertarget{ref-d4m}{}%
{[}10{]} Kepner, J. and others Dynamic distributed dimensional data
model (D4M) database and computation system. \emph{Acoustics, speech and
  signal processing (iCASSP), 2012 iEEE international conference on},
2012, 5349-5352.

\leavevmode\hypertarget{ref-graphlab}{}%
{[}11{]} Low, Y., Bickson, D., Gonzalez, J., Guestrin, C., Kyrola, A.
and Hellerstein, J.M. Distributed graphLab: A framework for machine
learning and data mining in the cloud. \emph{VLDB}, 2012.

\leavevmode\hypertarget{ref-hogwild}{}%
{[}12{]} Recht, B., Re, C., Wright, S. and Niu, F. Hogwild: A lock-free
approach to parallelizing stochastic gradient descent. \emph{Advances in
  neural information processing systems}, 2011, 693-701.

\leavevmode\hypertarget{ref-1000-genomes}{}%
{[}13{]} Siva, N. 1000 genomes project. \emph{Nature biotechnology}. 26,
3 (2008), 256-256.

\leavevmode\hypertarget{ref-istc-bigdata}{}%
{[}14{]} Stonebraker, M., Madden, S. and Dubey, P. Intel big data
science and technology center vision and execution plan. \emph{ACM
  SIGMOD Record}. 42, 1 (2013), 44-49.

\leavevmode\hypertarget{ref-scidb}{}%
{[}15{]} The SciDB Development Team Overview of SciDB: Large scale array
storage, processing and analysis. \emph{SIGMOD}, 2010.

\section*[Comments]{Comments\\%
{\normalsize Joe Hellerstein}\\%
{\normalsize 6 December 2015}%
}

I have a rather different take on this area than Mike, both from a
business perspective and in terms of research opportunities. At base, I
recommend a ``big tent'' approach to this area. DB folk have much to
contribute, but we'll do far better if we play well with others.

Let's look at the industry. First off, advanced analytics of the sort
we're discussing here will not replace BI as Mike suggests. The BI
industry is healthy and growing. More fundamentally, as noted
statistician John Tukey pointed out in his foundational work on
Exploratory Data Analysis,\footnote{Tukey, John. Exploratory Data Analysis. Pearson,
  1977.} a
chart is often much more valuable than a complex statistical model.
Respect the chart!

That said, the advanced analytics and data science market is indeed
growing and poised for change. But unlike the BI market, this is not a
category where database technology currently plays a significant role.
The incumbent in this space is SAS, a company that makes multiple
billions of dollars in revenue each year, and is decidedly not a
database company. When VCs look at companies in this space, they're
looking for ``the next SAS''. SAS users are not database users. And the
users of open-source alternatives like R are also not database users. If
you assume as Mike does that ``data scientists will want to use DBMS
technology'' --- particularly a monolithic ``analytic DBMS'' --- you're
swimming upstream in a strong current.

For a more natural approach to the advanced analytics market, ask
yourself this: what is a serious threat to SAS? Who could take a
significant bite out of the cash that enterprises currently spend there?
Here are some starting points to consider:

\begin{enumerate}
  \item
  {\textbf{Open source stats programming}}: This includes R and the
  Python data science ecosystem (NumPy, SciKit-Learn, iPython Notebook).
  These solutions don't currently don't scale well, but efforts are
  underway aggressively to address those limitations. This ecosystem
  could evolve more quickly than SAS.
  \item
  {\textbf{Tight couplings to big data platforms}}. When the data is big
  enough, performance requirements may ``drag'' users to a new platform
  --- namely a platform that already hosts the big data in their
  organization. Hence the interest in ``DataFrame'' interfaces to
  platforms like Spark/MLLib, PivotalR/MADlib, and Vertica dplyr. Note
  that the advanced analytics community is highly biased toward open
  source. The cloud is also an interesting platform here, and not one
  where SAS has an advantage.
  \item
  {\textbf{Analytic Services}}. By this I mean interactive online
  services that use analytic methods at their core: recommender systems,
  real-time fraud detection, predictive equipment maintenance and so on.
  This space has aggressive system requirements for response times,
  request scaling, fault tolerance and ongoing evolution that products
  like SAS don't address. Today, these services are largely built with
  custom code. This doesn't scale across industries --- most companies
  can't recruit developers that can work at this level. So there is
  ostensibly an opportunity here in commoditizing this technology for
  the majority of use cases. But it's early days for this market --- it
  remains to be seen whether analytics service platforms can be made
  simple enough for commodity deployment. If the tech evolves, then
  cloud-based services may have significant opportunities for disruption
  here as well.
\end{enumerate}

On the research front, I think it's critical to think outside the
database box, and collaborate aggressively. To me this almost goes
without saying. Nearly every subfield in computing is working on big
data analytics in some fashion, and smart people from a variety of areas
are quickly learning their own lessons about data and scale. We can have
fun playing with these folks, or we can ignore them to our detriment.

So where can database research have a big impact in this space? Some
possiblities that look good to me include these:

\begin{enumerate}
  \item
  {\textbf{New approaches to Scalability}}. We have successfully shown
  that parallel dataflow --- think MADlib, MLlib or the work of Ordonez
  at Teradata\footnote{e.g., Ordonez, C. Integrating K-means clustering with a relational
    DBMS using SQL. TKDE 18(2) 2006. Also Ordonez, C. Statistical Model
    Computation with UDFs. TKDE 22(12), 2010.} --- can take
  you a long way toward implementing scalable analytics \emph{without}
  doing violence at the system architecture level. That's useful to
  know. Moving forward, can we do something that is usefully faster and
  more scalable than parallel dataflow over partitioned data? Is that
  necessary? Hogwild! has generated some of the biggest excitement here;
  note that it's work that spans the DB and ML communities.
  \item
  {\textbf{Distributed infrastructure for analytic services}}. As I
  mentioned above, analytic services are an interesting opportunity for
  innovation. The system infrastructure issues on this front are fairly
  wide open. What are the main components of architectures for analytics
  services? How are they stitched together? What kind of data
  consistency is required across the components? So-called Parameter
  Servers are a topic of interest right now, but only address a piece of
  the puzzle.\footnote{Ho, Q., et al. More effective distributed ML via a stale synchronous
    parallel parameter server. NIPS 2013.} There has been
  some initial work on online serving, evolution and deployment of
  models.\footnote{Crankshaw, D, et al. The missing piece in complex analytics: Low
    latency, scalable model management and serving with Velox. CIDR 2015.
    See also Schleier-Smith, J. An Architecture for Agile Machine Learning
    in Real-Time Applications. KDD 2015.} I hope there will
  be more.
  \item
  {\textbf{Analytic lifecycle and metadata management}}. This is an area
  where I agree with Mike. Analytics is often a people-intensive
  exercise, involving data exploration and transformation in addition to
  core statistical modeling. Along the way, a good deal of context needs
  to be managed to understand how models and data products get developed
  across a range of tools and systems. The database commmunity has
  perspectives on this area that are highly relevant, including workflow
  management, data lineage and materialized view maintenance. VisTrails
  is an example of research in this space that is being used in
  practice.\footnote{See http://www.vistrails.org.} This is an area
  of pressing need in industry as well --- especially work that takes
  into account the real-world diversity of analytics tools and systems
  in the field.
\end{enumerate}



\hypertarget{ch12-dataintegration}{
  \chapter[Data Integration]{A Biased Take on a Moving Target: Data Integration\\{\Large by Michael Stonebraker}}\label{ch12-dataintegration}
}

I will start this treatise with a history of two major themes in data
integration. In my opinion, the topic began with the major retailers in
the 1990s consolidating their sales data into a data warehouse. To do
this they needed to extract data from in-store sales systems, transform
it into a predefined common representation (think of this as a global
schema), and then load it into a data warehouse. This data warehouse
kept historical sales data for a couple of years and was used by the
buyers in the organization to rotate stock. In other words, a buyer
would figure out that pet rocks are ``out'' and barbie dolls are ``in.''
Hence, he would tie up the barbie doll factory with a big order and move
the pet rocks up front and discount them to get rid of them. A typical
retail data warehouse paid for itself within a year through better
buying and stock rotation decisions. In the late 1990s and early 2000's
there was a giant ``pile on'' as essentially all enterprises followed
the lead of retailers and organized their customer-facing data into a
data warehouse.

A new industry was spawned to support the loading of data warehouses,
called extract, transform, and load (ETL) systems. The basic methodology
was:

\begin{enumerate}[label=\alph*)]
\item
Construct a global schema in advance.
\item
Send a programmer out to the owner of each data source and have him
figure out how to do the extraction. Historically, writing such
``connectors'' was a lot of work because of arcane formats. Hopefully,
this will become less problematic in the future with more open source
and standardized formats.
\item
Have him write transformations, often in a scripting language, and
any necessary cleaning routines
\item
Have him write a script to load the data into a data warehouse
\end{enumerate}

It became apparent that this methodology scales to perhaps a dozen data
sources, because of three big issues:

\begin{enumerate}
  \item
  A global schema is needed up front. About this time, there was a push
  in many enterprises to write an enterprise-wide schema for all company
  objects. A team was charged with doing this and would work on it for a
  couple of years. At the end of this time, their result was two years
  out of date, and was declared a failure. Hence, an upfront global
  schema is incredibly difficult to construct for a broad domain. This
  limits the plausible scope of data warehouses.
  \item
  Too much manual labor. A skilled programmer is required to perform
  most of the steps in the methodology for each data source.
  \item
  Data integration and cleaning is fundamentally difficult. The typical
  data warehouse project in the 1990's was a factor of two over budget
  and a factor of two late. The problem was that planners underestimated
  the difficulty of the data integration challenge. There are two big
  issues. First, data is dirty. A rule of thumb is that 10\% of your
  data is incorrect. This results from using nicknames for people or
  products, having stale addresses for suppliers, having incorrect ages
  for people, etc. The second is that deduplication is hard. One has to
  decide if Mike Stonebraker and M.R. Stonebraker are the same entity or
  different ones. Equally challenging is two restaurants at the same
  address. They might be in a food court, one might have replaced the
  other in a stand-alone location or this might be a data error. It is
  expensive to figure out ground truth in such situations.
\end{enumerate}

In spite of these issues, data warehouses have been a huge success for
customer facing data, and are in use by most major enterprises. In this
use case, the pain of assembling composite data is justified by the
better decision making that results. I hardly ever hear enterprises
complaining about the operational cost of their data warehouse. What I
hear instead is an incessant desire by business analysts for more data
sources, whether these be public data off the web or other enterprise
data. For example, the average large enterprise has about 5000
operational data stores, and only a few are in the data warehouse.

As a quick example, I visited a major beer manufacturer a while ago. He
had a typical data warehouse of sales of his products by brand, by
distributor, by time period, etc. I told the analysts that El Nino was
widely predicted to occur that winter and it would be wetter than normal
on the west coast and warmer than normal in the Northeast. I then asked
if beer sales are correlated to temperature or precipitation. They
replied ``I wish we could answer that question, but weather data is not
in our warehouse''. Supporting data source scalability is very difficult
using ETL technology.

Fast forward to the 2000's, and the new buzzword is \emph{master data
  management (MDM)}. The idea behind MDM is to standardize the enterprise
representation of important entities such as customers, employees,
sales, purchases, suppliers, etc. Then carefully curate a master data
set for each entity type and get everyone in the enterprise to use this
master. This sometimes goes by the mantra ``golden records''. In effect,
the former ETL vendors are now selling MDM, as a broader scope offering.
In my opinion, MDM is way over-hyped.

Let me start with ``Who can be against standards?'' Certainly not me.
However, MDM has the following problems, which I will illustrate by
vignette.

When I worked for Informix 15 years ago, the new CEO asked the Human
Resources VP at an early staff meeting ``How many employees do we
have?'' She returned the next week with the answer ``I don't know and
there is no way to find out?'' Informix operated in 58 countries, each
with its own labor laws, definition of an employee, etc. There was no
golden record for employees. Hence, the only way to answer the CEOs
question would be to perform data integration on these 58 data sources
to resolve the semantic issues. Getting 58 country managers to agree to
do this would be challenging, made more difficult by the fact that
Informix did not even own all the organizations involved. The new CEO
quickly realized the futility of this exercise.

So why would a company allow this situation to occur? The answer is
simple: business agility. Informix set up country operations on a
regular basis, and wanted the sales team up and running quickly.
Inevitably they would hire a country manager and tell him to ``make it
happen''. Sometimes it was a distributor or other independent entity. If
they had said ``here are the MDM golden records you need to conform
to'', then the country manager or distributor would spend months trying
to reconcile his needs with the MDM system in place. In other words, MDM
is the opposite of business agility. Obviously every enterprise needs to
strike a balance between standards and agility.

A second vignette concerns a large manufacturing enterprise. They are
decentralized into business units for business agility reasons. Each
business unit has its own purchasing system - to specify the terms and
conditions under which the business unit interacts with its suppliers.
There are some 300 of these systems. There is an obvious return on
investment to consolidate these systems. After all it is less code to
maintain and the enterprise can presumably get better-consolidated terms
than each business unit can individually. 鈥⊿o why are there so many
purchasing systems? Acquisitions. This enterprise grew largely by
acquisition. Each acquisition became a new business unit, and came with
its own data systems, contracts in place, etc. It is often simply not
feasible to merge all these data systems into the parent's IT
infrastructure. In summary, acquisitions screw up MDM.

So what is entailed in data integration (or data curation)? It is the
following steps:

\begin{enumerate}
  \item
  \emph{Ingest.} A data source must be located and captured. This
  requires parsing whatever data structure is used for storage.
  \item
  \emph{Transform.} For example, Euros to dollars or airport code to
  city name.
  \item
  \emph{Clean.} Data errors must be found and rectified.
  \item
  \emph{Schema integration.} Your wages is my salary.
  \item
  \emph{Consolidate (deduplication).} Mike Stonebraker and M.R.
  Stonebraker must be consolidated into a single record.
\end{enumerate}

The ETL vendors do this at high cost and with low scalability. The MDM
vendors have a similar profile. So there is a big unmet need. Data
curation at scale is the ``800 pound gorilla in the corner.'' So what
are the research challenges in this area?

Let's go through the steps one by one.

Ingest is simply a matter of parsing data sources. Others have written
such ``connectors'', and they are generally expensive to construct. An
interesting challenge would be to semi-automatically generate
connectors.

Data transformations have also been extensively researched, mostly in
the last decade or so. Scripting/visualization facilities to specify
transforms have been studied in~{{[}\protect\hyperlink{ref-morpheus}{3},
  \protect\hyperlink{ref-potterswheel}{8}{]}}. Data
Wrangler~{{[}\protect\hyperlink{ref-wrangler}{6}{]}} appears to be the
state of the art in this area, and the interested reader is encouraged
to take a look. In addition, there are a bunch of commercial offerings
that offer transformation services for a fee (e.g. address to (lat,long)
or company to canonical company representation). In addition, work on
finding transformations of interest from the public web is reported
in~{{[}\protect\hyperlink{ref-dataxformer}{1}{]}}.

Data cleaning has been studied using a variety of techniques.
~{{[}\protect\hyperlink{ref-fd-cleaning}{2}{]}}\&nbspapplied functional
dependencies to discover erroneous data and suggest automatic repairs.
Outlier detection (which may correspond to errors) has been studied in
many contexts~{{[}\protect\hyperlink{ref-fnt-cleaning}{5}{]}}.
~{{[}\protect\hyperlink{ref-seedb}{11},
  \protect\hyperlink{ref-scorpion}{12}{]}} are query systems to discover
interesting patterns in the data. Such patterns may correspond to
errors. ~{{[}\protect\hyperlink{ref-tamer}{10}{]}} have studied the
utilization of crowd sourcing and expert sourcing to correct errors,
once they have been identified. Lastly, there are a variety of
commercial services that will clean common domains, such as addresses of
persons and dates. In my opinion, data cleaning research MUST utilize
real-world data. Finding faults that have been injected into other-wise
clean data just is not believable. Unfortunately, real world ``dirty''
data is quite hard to come by.

Schema matching has been extensively worked on for at least 20 years.
The interested reader should
consult~{{[}\protect\hyperlink{ref-garlic1}{4},
  \protect\hyperlink{ref-clio}{7}, \protect\hyperlink{ref-garlic2}{9}{]}}
for the state of the art in this area.

Entity consolidation is a problem of finding records in a high
dimensional space (all of the attributes about an entity - typically 25
or more) that are close together. Effectively this is a clustering
problem in 25 space. This is an $N ^ 2$ problem that will have a very
long running time at scale. Hence, approximate algorithms are clearly
the way to proceed here. A survey of techniques appears
in~{{[}\protect\hyperlink{ref-fnt-cleaning}{5}{]}}.

In my opinion, the real problem is an end-to-end system. Data curation
entails all of these steps, which must be seamlessly integrated, and
enterprise-wide systems must perform curation at scale. An interesting
end-to-end approach that appears to scale well is the Data Tamer
system~{{[}\protect\hyperlink{ref-tamer}{10}{]}}. On the other hand,
data curation problems also occur at the department level, where an
individual contributor wants to integrate a handful of data sources, and
the Data Wrangler system noted above appears to be an interesting
approach. There are commercial companies supporting both of these
systems, so regular enhancements should be forthcoming.

Hopefully, the next edition of the Red Book will have a collection of
seminal papers in this area to replace this (self-serving) call to
action. In my opinion, this is one of the most important topics that
enterprises are dealing with. My one caution is ``the rubber has to meet
the road''. If you want to work in this area, you have got to try your
ideas on real world enterprise data. Constructing artificial data,
injecting faults into it, and then finding these faults is simply not
believable. If you have ideas in this area, I would recommend building
an end-to-end system. In this way, you make sure that you are solving an
important problem, rather than just a ``point problem'' which real world
users may or may not be interested in.

\section*{References}

\leavevmode\hypertarget{ref-dataxformer}{}%
{[}1{]} Abedjan, Z., Morcos, J., Gubanov, M., Ilyas, I.F., Stonebraker,
M., Papotti, P. and Ouzzani, M. DataXFormer: Leveraging the web for
semantic transformations. \emph{CIDR}, 2015.

\leavevmode\hypertarget{ref-fd-cleaning}{}%
{[}2{]} Chu, X., Ilyas, I.F. and Papotti, P. Holistic data cleaning:
Putting violations into context. \emph{ICDE}, 2013.

\leavevmode\hypertarget{ref-morpheus}{}%
{[}3{]} Dohzen, T., Pamuk, M., Seong, S.-W., Hammer, J. and Stonebraker,
M. Data integration through transform reuse in the morpheus project.
\emph{SIGMOD}, 2006.

\leavevmode\hypertarget{ref-garlic1}{}%
{[}4{]} Haas, L., Kossmann, D., Wimmers, E. and Yang, J. Optimizing
queries across diverse data sources. \emph{VLDB}, 1997.

\leavevmode\hypertarget{ref-fnt-cleaning}{}%
{[}5{]} Ilyas, I.F. and Chu, X. Trends in cleaning relational data:
Consistency and deduplication. \emph{Foundations and Trends in
  Databases}. 5, 4 (2012), 281-393.

\leavevmode\hypertarget{ref-wrangler}{}%
{[}6{]} Kandel, S., Paepcke, A., Hellerstein, J. and Heer, J. Wrangler:
Interactive visual specification of data transformation scripts.
\emph{CHI}, 2011.

\leavevmode\hypertarget{ref-clio}{}%
{[}7{]} Miller, R.J., Hernč°Šndez, M.A., Haas, L.M., Yan, L.-L., Ho, C.H.,
Fagin, R. and Popa, L. The clio project: Managing heterogeneity.
\emph{SIGMOD Record}. 30, 1 (2001), 78-83.

\leavevmode\hypertarget{ref-potterswheel}{}%
{[}8{]} Raman, V. and Hellerstein, J.M. Potter's wheel: An interactive
data cleaning system. \emph{VLDB}, 2001.

\leavevmode\hypertarget{ref-garlic2}{}%
{[}9{]} Roth, M.T. and Schwarz, P.M. Don't scrap it, wrap it! A wrapper
architecture for legacy data sources. \emph{VLDB}, 1997.

\leavevmode\hypertarget{ref-tamer}{}%
{[}10{]} Stonebraker, M., Bruckner, D., Ilyas, I.F., Beskales, G.,
Cherniack, M., Zdonik, S.B., Pagan, A. and Xu, S. Data curation at
scale: The data tamer system. \emph{CIDR}, 2013.

\leavevmode\hypertarget{ref-seedb}{}%
{[}11{]} Vartak, M., Madden, S., Parameswaran, A. and Polyzotis, N.
SeeDB: Automatically generating query visualizations. \emph{VLDB}, 2014.

\leavevmode\hypertarget{ref-scorpion}{}%
{[}12{]} Wu, E. and Madden, S. Scorpion: Explaining away outliers in
aggregate queries. \emph{VLDB}, 2013.

\section*[Comments]{Comments\\%
{\normalsize Joe Hellerstein}\\%
{\normalsize 6 December 2015}%
}

I agree with Mike's assessment here in general, but wanted to add my
perspective on the space, relating to the ``department level'' problem
he mentions in passing.

Based on experience with users across a wide range of organizations,
we've seen that data transformation is increasingly a user-centric task,
and depends critically upon the user experience: the interfaces and
languages for interactively assessing and manipulating data.

In many of today's settings, the right outcome from data transformation
depends heavily on context. To understand if data is dirty, you have to
know what it is ``supposed'' to look like. To transform data for use,
you need to understand what it is being used for. Even in a single
organization, the context of how data is going to be used and what it
needs to be like varies across people and across time. Add this to
Mike's critique of the idea of a ``golden master''-it simply doesn't
make sense for many modern use cases, especially in analytical contexts.

Instead, you need to design tools for the people who best understand the
data and the use case, and enable them to do their own data profiling
and transformation in an agile, exploratory manner. Computer scientists
tend to design for technical users-IT professionals and data scientists.
But in most organizations, the users who understand the data and the
context are closer to the ``business'' than the IT department. They are
often less technically skilled as well. Rather than reach for
traditional ETL tools, they tend to fall back on manipulating data in
spreadsheets and desktop database packages, neither of which are well
suited for assessing data quality or doing bulk transformation. For
large datasets they ``code in Microsoft Word'': they describe their
desired workflow in a natural language spec, wait for IT to get around
to writing the code for them, and then when they get the results they
typically realize they don't quite work. At which point their need for
the data has often changed anyhow. No surprise that people often assert
that 50-80\% of their time is spent in ``preparing the data.'' (As a
footnote, in my experience modern ``data scientists'' tend to wrangle
data via ad-hoc scripts in Python, R or SAS DataStep, and are shockingly
lax about code quality and revision control for these scripts. As a
result they're often worse off over time than the old-school ETL users!)

Business users reach for graphical tools for good reason: they want to
understand the data as it is being transformed, and assess whether it is
getting closer to a form that's useful for their business problem. As a
result, the unattended algorithms from the database research literature
have done too little to address the key issues in this space. I believe
the most relevant solutions will be based on interfaces that enable
people to understand the state of their data intuitively, and
collaborate with algorithms to get the data better purposed for use.

This presents a significant opportunity for innovation. Data
transformation is a perfect Petri Dish for exploring the general topic
of visualizing and interacting with data. This is an area where ideas
from Databases, HCI and Machine Learning can be brought together, to
create interactive collaborations between algorithms and people that
solve practical, context-specific problems with data. Backing this up we
need interactive data systems that can do things like provide
instantaneous data profiles (various aggregates) over the results of
ad-hoc transformation steps, and speculate ahead of users in real time
to suggest multiple alternative transformations that could be
useful.\footnote{Heer, J., Hellerstein, J.M. and Kandel, S. ``Predictive Interaction
  for Data Transformation.'' CIDR 2015.} Some of the topics
from the Interactive Analytics chapter are relevant here, particularly
for big data sets. I've been happy to see more work on visualization and
interaction in the database community in recent years; this is a great
application area for that work.

\end{document}
