\PassOptionsToPackage{dvipsnames}{xcolor}
\PassOptionsToPackage{unicode=true,colorlinks=true,urlcolor=blue}{hyperref} % options for packages loaded elsewhere
\PassOptionsToPackage{hyphens}{url}
\documentclass[a4paper,12pt,notitlepage,twoside,openright]{article}

\usepackage{ifxetex}
\ifxetex{}
\else
\errmessage{Must be built with XeLaTeX}
\fi

\usepackage{amsmath}
\usepackage{fontspec}
\usepackage{fourier-otf} % erewhon-math
\setmonofont{iosevka-type-slab-regular}[
  Path=../common/iosevka-type-slab/,
  Extension=.ttf,
  BoldFont=iosevka-type-slab-bold,
  ItalicFont=iosevka-type-slab-italic,
  BoldItalicFont=iosevka-type-slab-bolditalic,
  Scale=MatchLowercase,
]

% Math
\usepackage[binary-units]{siunitx}

\usepackage{caption}
\usepackage{authblk}
\usepackage{enumitem}
\usepackage{footnote}

% Table
\usepackage{tabu}
\usepackage{longtable}
\usepackage{booktabs}
\usepackage{multirow}

% Verbatim & Source code
\usepackage{fancyvrb}
\usepackage{minted}

% Beauty
\usepackage[protrusion]{microtype}
\usepackage[defaultlines=3]{nowidow}
\usepackage{upquote}
\usepackage{parskip}
\usepackage[strict]{changepage}

\usepackage{hyperref}

% Graph
\usepackage{graphicx}
\usepackage{grffile}
\usepackage{tikz}

\usepackage{endnotes}
\hypersetup{
  bookmarksnumbered,
  pdfborder={0 0 0},
  pdfpagemode=UseNone,
  pdfstartview=FitH,
  breaklinks=true}
\urlstyle{same}  % don't use monospace font for urls

\usetikzlibrary{arrows.meta,calc,shapes.geometric,shapes.misc}

\setminted{
  autogobble,
  breakbytokenanywhere,
  breaklines,
  fontsize=\footnotesize,
}
\setmintedinline{
  autogobble,
  breakbytokenanywhere,
  breaklines,
  fontsize=\footnotesize,
}

\makeatletter
\def\maxwidth{\ifdim\Gin@nat@width>\linewidth\linewidth\else\Gin@nat@width\fi}
\def\maxheight{\ifdim\Gin@nat@height>\textheight\textheight\else\Gin@nat@height\fi}
\makeatother

% Scale images if necessary, so that they will not overflow the page
% margins by default, and it is still possible to overwrite the defaults
% using explicit options in \includegraphics[width, height, ...]{}
\setkeys{Gin}{width=\maxwidth,height=\maxheight,keepaspectratio}
\setlength{\emergencystretch}{3em}  % prevent overfull lines
\setcounter{secnumdepth}{3}

% Redefines (sub)paragraphs to behave more like sections
\ifx\paragraph\undefined\else
\let\oldparagraph\paragraph{}
\renewcommand{\paragraph}[1]{\oldparagraph{#1}\mbox{}}
\fi
\ifx\subparagraph\undefined\else
\let\oldsubparagraph\subparagraph{}
\renewcommand{\subparagraph}[1]{\oldsubparagraph{#1}\mbox{}}
\fi

% set default figure placement to htbp
\makeatletter
\def\fps@figure{htbp}
\makeatother


\title{Semantic Versioning 2.0.0}
\author{Tom Preston-Werner}
\date{18 Jun 2013}

\begin{document}

\maketitle

\hypertarget{summary}{%
\section{Summary}\label{summary}}

Given a version number MAJOR.MINOR.PATCH, increment the:

\begin{enumerate}
\def\labelenumi{\arabic{enumi}.}
\item
  MAJOR version when you make incompatible API changes,
\item
  MINOR version when you add functionality in a backwards-compatible
  manner, and
\item
  PATCH version when you make backwards-compatible bug fixes.
\end{enumerate}

Additional labels for pre-release and build metadata are available as
extensions to the MAJOR.MINOR.PATCH format.

\hypertarget{introduction}{%
\section{Introduction}\label{introduction}}

In the world of software management there exists a dread place called
``dependency hell.'' The bigger your system grows and the more packages
you integrate into your software, the more likely you are to find
yourself, one day, in this pit of despair.

In systems with many dependencies, releasing new package versions can
quickly become a nightmare. If the dependency specifications are too
tight, you are in danger of version lock (the inability to upgrade a
package without having to release new versions of every dependent
package). If dependencies are specified too loosely, you will inevitably
be bitten by version promiscuity (assuming compatibility with more
future versions than is reasonable). Dependency hell is where you are
when version lock and/or version promiscuity prevent you from easily and
safely moving your project forward.

As a solution to this problem, I propose a simple set of rules and
requirements that dictate how version numbers are assigned and
incremented. These rules are based on but not necessarily limited to
pre-existing widespread common practices in use in both closed and
open-source software. For this system to work, you first need to declare
a public API. This may consist of documentation or be enforced by the
code itself. Regardless, it is important that this API be clear and
precise. Once you identify your public API, you communicate changes to
it with specific increments to your version number. Consider a version
format of X.Y.Z (Major.Minor.Patch). Bug fixes not affecting the API
increment the patch version, backwards compatible API additions/changes
increment the minor version, and backwards incompatible API changes
increment the major version.

I call this system ``Semantic Versioning.'' Under this scheme, version
numbers and the way they change convey meaning about the underlying code
and what has been modified from one version to the next.

\hypertarget{semantic-versioning-specification-semver}{%
\section{Semantic Versioning Specification
(SemVer)}\label{semantic-versioning-specification-semver}}

The key words ``MUST'', ``MUST NOT'', ``REQUIRED'', ``SHALL'', ``SHALL
NOT'', ``SHOULD'', ``SHOULD NOT'', ``RECOMMENDED'', ``MAY'', and
``OPTIONAL'' in this document are to be interpreted as described in
\href{http://tools.ietf.org/html/rfc2119}{RFC 2119}.

\begin{enumerate}
\def\labelenumi{\arabic{enumi}.}
\item
  Software using Semantic Versioning MUST declare a public API. This API
  could be declared in the code itself or exist strictly in
  documentation. However it is done, it should be precise and
  comprehensive.
\item
  A normal version number MUST take the form X.Y.Z where X, Y, and Z are
  non-negative integers, and MUST NOT contain leading zeroes. X is the
  major version, Y is the minor version, and Z is the patch version.
  Each element MUST increase numerically. For instance: 1.9.0
  -\textgreater{} 1.10.0 -\textgreater{} 1.11.0.
\item
  Once a versioned package has been released, the contents of that
  version MUST NOT be modified. Any modifications MUST be released as a
  new version.
\item
  Major version zero (0.y.z) is for initial development. Anything may
  change at any time. The public API should not be considered stable.
\item
  Version 1.0.0 defines the public API. The way in which the version
  number is incremented after this release is dependent on this public
  API and how it changes.
\item
  Patch version Z (x.y.Z \textbar{} x \textgreater{} 0) MUST be
  incremented if only backwards compatible bug fixes are introduced. A
  bug fix is defined as an internal change that fixes incorrect
  behavior.
\item
  Minor version Y (x.Y.z \textbar{} x \textgreater{} 0) MUST be
  incremented if new, backwards compatible functionality is introduced
  to the public API. It MUST be incremented if any public API
  functionality is marked as deprecated. It MAY be incremented if
  substantial new functionality or improvements are introduced within
  the private code. It MAY include patch level changes. Patch version
  MUST be reset to 0 when minor version is incremented.
\item
  Major version X (X.y.z \textbar{} X \textgreater{} 0) MUST be
  incremented if any backwards incompatible changes are introduced to
  the public API. It MAY include minor and patch level changes. Patch
  and minor version MUST be reset to 0 when major version is
  incremented.
\item
  A pre-release version MAY be denoted by appending a hyphen and a
  series of dot separated identifiers immediately following the patch
  version. Identifiers MUST comprise only ASCII alphanumerics and hyphen
  {[}0-9A-Za-z-{]}. Identifiers MUST NOT be empty. Numeric identifiers
  MUST NOT include leading zeroes. Pre-release versions have a lower
  precedence than the associated normal version. A pre-release version
  indicates that the version is unstable and might not satisfy the
  intended compatibility requirements as denoted by its associated
  normal version. Examples: 1.0.0-alpha, 1.0.0-alpha.1, 1.0.0-0.3.7,
  1.0.0-x.7.z.92.
\item
  Build metadata MAY be denoted by appending a plus sign and a series of
  dot separated identifiers immediately following the patch or
  pre-release version. Identifiers MUST comprise only ASCII
  alphanumerics and hyphen {[}0-9A-Za-z-{]}. Identifiers MUST NOT be
  empty. Build metadata SHOULD be ignored when determining version
  precedence. Thus two versions that differ only in the build metadata,
  have the same precedence. Examples: 1.0.0-alpha+001,
  1.0.0+20130313144700, 1.0.0-beta+exp.sha.5114f85.
\item
  Precedence refers to how versions are compared to each other when
  ordered. Precedence MUST be calculated by separating the version into
  major, minor, patch and pre-release identifiers in that order (Build
  metadata does not figure into precedence). Precedence is determined by
  the first difference when comparing each of these identifiers from
  left to right as follows: Major, minor, and patch versions are always
  compared numerically. Example: 1.0.0 \textless{} 2.0.0 \textless{}
  2.1.0 \textless{} 2.1.1. When major, minor, and patch are equal, a
  pre-release version has lower precedence than a normal version.
  Example: 1.0.0-alpha \textless{} 1.0.0. Precedence for two pre-release
  versions with the same major, minor, and patch version MUST be
  determined by comparing each dot separated identifier from left to
  right until a difference is found as follows: identifiers consisting
  of only digits are compared numerically and identifiers with letters
  or hyphens are compared lexically in ASCII sort order. Numeric
  identifiers always have lower precedence than non-numeric identifiers.
  A larger set of pre-release fields has a higher precedence than a
  smaller set, if all of the preceding identifiers are equal. Example:
  1.0.0-alpha \textless{} 1.0.0-alpha.1 \textless{} 1.0.0-alpha.beta
  \textless{} 1.0.0-beta \textless{} 1.0.0-beta.2 \textless{}
  1.0.0-beta.11 \textless{} 1.0.0-rc.1 \textless{} 1.0.0.
\end{enumerate}

\hypertarget{why-use-semantic-versioning}{%
\section{Why Use Semantic
Versioning?}\label{why-use-semantic-versioning}}

This is not a new or revolutionary idea. In fact, you probably do
something close to this already. The problem is that ``close'' isn't
good enough. Without compliance to some sort of formal specification,
version numbers are essentially useless for dependency management. By
giving a name and clear definition to the above ideas, it becomes easy
to communicate your intentions to the users of your software. Once these
intentions are clear, flexible (but not too flexible) dependency
specifications can finally be made.

A simple example will demonstrate how Semantic Versioning can make
dependency hell a thing of the past. Consider a library called
``Firetruck.'' It requires a Semantically Versioned package named
``Ladder.'' At the time that Firetruck is created, Ladder is at version
3.1.0. Since Firetruck uses some functionality that was first introduced
in 3.1.0, you can safely specify the Ladder dependency as greater than
or equal to 3.1.0 but less than 4.0.0. Now, when Ladder version 3.1.1
and 3.2.0 become available, you can release them to your package
management system and know that they will be compatible with existing
dependent software.

As a responsible developer you will, of course, want to verify that any
package upgrades function as advertised. The real world is a messy
place; there's nothing we can do about that but be vigilant. What you
can do is let Semantic Versioning provide you with a sane way to release
and upgrade packages without having to roll new versions of dependent
packages, saving you time and hassle.

If all of this sounds desirable, all you need to do to start using
Semantic Versioning is to declare that you are doing so and then follow
the rules. Link to this website from your README so others know the
rules and can benefit from them.

\hypertarget{faq}{%
\section{FAQ}\label{faq}}

\hypertarget{how-should-i-deal-with-revisions-in-the-0.y.z-initial-development-phase}{%
\subsection{How should I deal with revisions in the 0.y.z initial
development
phase?}\label{how-should-i-deal-with-revisions-in-the-0.y.z-initial-development-phase}}

The simplest thing to do is start your initial development release at
0.1.0 and then increment the minor version for each subsequent release.

\hypertarget{how-do-i-know-when-to-release-1.0.0}{%
\subsection{How do I know when to release
1.0.0?}\label{how-do-i-know-when-to-release-1.0.0}}

If your software is being used in production, it should probably already
be 1.0.0. If you have a stable API on which users have come to depend,
you should be 1.0.0. If you're worrying a lot about backwards
compatibility, you should probably already be 1.0.0.

\hypertarget{doesnt-this-discourage-rapid-development-and-fast-iteration}{%
\subsection{Doesn't this discourage rapid development and fast
iteration?}\label{doesnt-this-discourage-rapid-development-and-fast-iteration}}

Major version zero is all about rapid development. If you're changing
the API every day you should either still be in version 0.y.z or on a
separate development branch working on the next major version.

\hypertarget{if-even-the-tiniest-backwards-incompatible-changes-to-the-public-api-require-a-major-version-bump-wont-i-end-up-at-version-42.0.0-very-rapidly}{%
\subsection{If even the tiniest backwards incompatible changes to the
public API require a major version bump, won't I end up at version
42.0.0 very
rapidly?}\label{if-even-the-tiniest-backwards-incompatible-changes-to-the-public-api-require-a-major-version-bump-wont-i-end-up-at-version-42.0.0-very-rapidly}}

This is a question of responsible development and foresight.
Incompatible changes should not be introduced lightly to software that
has a lot of dependent code. The cost that must be incurred to upgrade
can be significant. Having to bump major versions to release
incompatible changes means you'll think through the impact of your
changes, and evaluate the cost/benefit ratio involved.

\hypertarget{documenting-the-entire-public-api-is-too-much-work}{%
\subsection{Documenting the entire public API is too much
work!}\label{documenting-the-entire-public-api-is-too-much-work}}

It is your responsibility as a professional developer to properly
document software that is intended for use by others. Managing software
complexity is a hugely important part of keeping a project efficient,
and that's hard to do if nobody knows how to use your software, or what
methods are safe to call. In the long run, Semantic Versioning, and the
insistence on a well defined public API can keep everyone and everything
running smoothly.

\hypertarget{what-do-i-do-if-i-accidentally-release-a-backwards-incompatible-change-as-a-minor-version}{%
\subsection{What do I do if I accidentally release a backwards
incompatible change as a minor
version?}\label{what-do-i-do-if-i-accidentally-release-a-backwards-incompatible-change-as-a-minor-version}}

As soon as you realize that you've broken the Semantic Versioning spec,
fix the problem and release a new minor version that corrects the
problem and restores backwards compatibility. Even under this
circumstance, it is unacceptable to modify versioned releases. If it's
appropriate, document the offending version and inform your users of the
problem so that they are aware of the offending version.

\hypertarget{what-should-i-do-if-i-update-my-own-dependencies-without-changing-the-public-api}{%
\subsection{What should I do if I update my own dependencies without
changing the public
API?}\label{what-should-i-do-if-i-update-my-own-dependencies-without-changing-the-public-api}}

That would be considered compatible since it does not affect the public
API. Software that explicitly depends on the same dependencies as your
package should have their own dependency specifications and the author
will notice any conflicts. Determining whether the change is a patch
level or minor level modification depends on whether you updated your
dependencies in order to fix a bug or introduce new functionality. I
would usually expect additional code for the latter instance, in which
case it's obviously a minor level increment.

\hypertarget{what-if-i-inadvertently-alter-the-public-api-in-a-way-that-is-not-compliant-with-the-version-number-change-i.e.-the-code-incorrectly-introduces-a-major-breaking-change-in-a-patch-release}{%
\subsection{What if I inadvertently alter the public API in a way
that is not compliant with the version number change (i.e.~the code
incorrectly introduces a major breaking change in a patch
release)}\label{what-if-i-inadvertently-alter-the-public-api-in-a-way-that-is-not-compliant-with-the-version-number-change-i.e.-the-code-incorrectly-introduces-a-major-breaking-change-in-a-patch-release}}

Use your best judgment. If you have a huge audience that will be
drastically impacted by changing the behavior back to what the public
API intended, then it may be best to perform a major version release,
even though the fix could strictly be considered a patch release.
Remember, Semantic Versioning is all about conveying meaning by how the
version number changes. If these changes are important to your users,
use the version number to inform them.

\hypertarget{how-should-i-handle-deprecating-functionality}{%
\subsection{How should I handle deprecating
functionality?}\label{how-should-i-handle-deprecating-functionality}}

Deprecating existing functionality is a normal part of software
development and is often required to make forward progress. When you
deprecate part of your public API, you should do two things: (1) update
your documentation to let users know about the change, (2) issue a new
minor release with the deprecation in place. Before you completely
remove the functionality in a new major release there should be at least
one minor release that contains the deprecation so that users can
smoothly transition to the new API.

\hypertarget{does-semver-have-a-size-limit-on-the-version-string}{%
\subsection{Does semver have a size limit on the version
string?}\label{does-semver-have-a-size-limit-on-the-version-string}}

No, but use good judgment. A 255 character version string is probably
overkill, for example. Also, specific systems may impose their own
limits on the size of the string.

\hypertarget{about}{%
\section{About}\label{about}}

The Semantic Versioning specification is authored by
\href{http://tom.preston-werner.com}{Tom Preston-Werner}, inventor of
Gravatars and cofounder of GitHub.

If you'd like to leave feedback, please
\href{https://github.com/mojombo/semver/issues}{open an issue on
GitHub}.

\hypertarget{license}{%
\section{License}\label{license}}

Creative Commons --- CC BY 3.0 http://creativecommons.org/licenses/by/3.0/

\end{document}
