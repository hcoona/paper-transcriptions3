\PassOptionsToPackage{dvipsnames}{xcolor}
\PassOptionsToPackage{unicode=true,colorlinks=true,urlcolor=blue}{hyperref} % options for packages loaded elsewhere
\PassOptionsToPackage{hyphens}{url}
\documentclass[a4paper,12pt,notitlepage,twoside,openright]{article}

\usepackage{ifxetex}
\ifxetex{}
\else
\errmessage{Must be built with XeLaTeX}
\fi

\usepackage{amsmath}
\usepackage{fontspec}
\usepackage{fourier-otf} % erewhon-math
\setmonofont{iosevka-type-slab-regular}[
  Path=../common/iosevka-type-slab/,
  Extension=.ttf,
  BoldFont=iosevka-type-slab-bold,
  ItalicFont=iosevka-type-slab-italic,
  BoldItalicFont=iosevka-type-slab-bolditalic,
  Scale=MatchLowercase,
]

% Math
\usepackage[binary-units]{siunitx}

\usepackage{caption}
\usepackage{authblk}
\usepackage{enumitem}
\usepackage{footnote}

% Table
\usepackage{tabu}
\usepackage{longtable}
\usepackage{booktabs}
\usepackage{multirow}

% Verbatim & Source code
\usepackage{fancyvrb}
\usepackage{minted}

% Beauty
\usepackage[protrusion]{microtype}
\usepackage[defaultlines=3]{nowidow}
\usepackage{upquote}
\usepackage{parskip}
\usepackage[strict]{changepage}

\usepackage{hyperref}

% Graph
\usepackage{graphicx}
\usepackage{grffile}
\usepackage{tikz}

\usepackage{endnotes}
\hypersetup{
  bookmarksnumbered,
  pdfborder={0 0 0},
  pdfpagemode=UseNone,
  pdfstartview=FitH,
  breaklinks=true}
\urlstyle{same}  % don't use monospace font for urls

\usetikzlibrary{arrows.meta,calc,shapes.geometric,shapes.misc}

\setminted{
  autogobble,
  breakbytokenanywhere,
  breaklines,
  fontsize=\footnotesize,
}
\setmintedinline{
  autogobble,
  breakbytokenanywhere,
  breaklines,
  fontsize=\footnotesize,
}

\makeatletter
\def\maxwidth{\ifdim\Gin@nat@width>\linewidth\linewidth\else\Gin@nat@width\fi}
\def\maxheight{\ifdim\Gin@nat@height>\textheight\textheight\else\Gin@nat@height\fi}
\makeatother

% Scale images if necessary, so that they will not overflow the page
% margins by default, and it is still possible to overwrite the defaults
% using explicit options in \includegraphics[width, height, ...]{}
\setkeys{Gin}{width=\maxwidth,height=\maxheight,keepaspectratio}
\setlength{\emergencystretch}{3em}  % prevent overfull lines
\setcounter{secnumdepth}{3}

% Redefines (sub)paragraphs to behave more like sections
\ifx\paragraph\undefined\else
\let\oldparagraph\paragraph{}
\renewcommand{\paragraph}[1]{\oldparagraph{#1}\mbox{}}
\fi
\ifx\subparagraph\undefined\else
\let\oldsubparagraph\subparagraph{}
\renewcommand{\subparagraph}[1]{\oldsubparagraph{#1}\mbox{}}
\fi

% set default figure placement to htbp
\makeatletter
\def\fps@figure{htbp}
\makeatother


\title{The OpenTracing Semantic Specification}
\author{}
\date{}

\begin{document}

\maketitle

\hypertarget{document-overview}{%
\section{Document Overview}\label{document-overview}}

This is the ``formal'' OpenTracing semantic specification. Since
OpenTracing must work across many languages, this document takes care to
avoid language-specific concepts. That said, there is an understanding
throughout that all languages have some concept of an ``interface''
which encapsulates a set of related capabilities.

\hypertarget{versioning-policy}{%
\subsection{Versioning policy}\label{versioning-policy}}

The OpenTracing specification uses a \texttt{Major.Minor} version number
but has no \texttt{.Patch} component. The major version increments when
backwards-incompatible changes are made to the specification. The minor
version increments for non-breaking changes like the introduction of new
standard tags, log fields, or SpanContext reference types. (You can read
more about the motivation for this versioning scheme at Issue
\href{https://github.com/opentracing/specification/issues/2\#issuecomment-261740811}{specification\#2})

\hypertarget{the-big-picture-opentracings-scope}{%
\section{The Big Picture: OpenTracing's
Scope}\label{the-big-picture-opentracings-scope}}

OpenTracing's core specification (i.e., this document) is intentionally
agnostic about the specifics of particular downstream tracing or
monitoring systems. This is because \textbf{OpenTracing exists to
describe the semantics of transactions in distributed systems.}
Describing those transactions should not be influenced by how --- or how
not --- any particular backend likes to process or represent data. For
instance, detailed OpenTracing instrumentation can be used to simply
measure latencies and apply tags in a timeseries monitoring system
(e.g., Prometheus); or Span start+finish times along with Span logs may
be redirected to a central logging service (e.g., Kibana).

As such, the OpenTracing specification and
\href{./data_conventions.md}{data modelling conventions} have a wider
scope than some tracing systems, ``and that's okay.'' If certain
semantic behavior is out-of-scope for a particular tracing or monitoring
system, said system can summarize or simply ignore the respective data
flowing from OpenTracing instrumentation.

\hypertarget{the-opentracing-data-model}{%
\section{The OpenTracing Data
Model}\label{the-opentracing-data-model}}

\textbf{Traces} in OpenTracing are defined implicitly by their
\textbf{Spans}. In particular, a \textbf{Trace} can be thought of as a
directed acyclic graph (DAG) of \textbf{Spans}, where the edges between
\textbf{Spans} are called \textbf{References}.

For example, the following is an example \textbf{Trace} made up of 8
\textbf{Spans}:

\begin{verbatim}
Causal relationships between Spans in a single Trace


        [Span A]  ←←←(the root span)
            |
     +------+------+
     |             |
 [Span B]      [Span C] ←←←(Span C is a `ChildOf` Span A)
     |             |
 [Span D]      +---+-------+
               |           |
           [Span E]    [Span F] >>> [Span G] >>> [Span H]
                                       ↑
                                       ↑
                                       ↑
                         (Span G `FollowsFrom` Span F)
\end{verbatim}

Sometimes it's easier to visualize \textbf{Traces} with a time axis as
in the diagram below:

\begin{verbatim}
Temporal relationships between Spans in a single Trace


––|–––––––|–––––––|–––––––|–––––––|–––––––|–––––––|–––––––|–> time

 [Span A···················································]
   [Span B··············································]
      [Span D··········································]
    [Span C········································]
         [Span E·······]        [Span F··] [Span G··] [Span H··]
\end{verbatim}

Each \textbf{Span} encapsulates the following state:

\begin{itemize}

\item
  An operation name
\item
  A start timestamp
\item
  A finish timestamp
\item
  A set of zero or more key:value \textbf{Span Tags}. The keys must be
  strings. The values may be strings, bools, or numeric types.
\item
  A set of zero or more \textbf{Span Logs}, each of which is itself a
  key:value map paired with a timestamp. The keys must be strings,
  though the values may be of any type. Not all OpenTracing
  implementations must support every value type.
\item
  A \textbf{SpanContext} (see below)
\item
  \protect\hyperlink{references-between-spans}{\textbf{References}} to
  zero or more causally-related \textbf{Spans} (via the
  \textbf{SpanContext} of those related \textbf{Spans})
\end{itemize}

Each \textbf{SpanContext} encapsulates the following state:

\begin{itemize}

\item
  Any OpenTracing-implementation-dependent state (for example, trace and
  span ids) needed to refer to a distinct \textbf{Span} across a process
  boundary
\item
  \textbf{Baggage Items}, which are just key:value pairs that cross
  process boundaries
\end{itemize}

\hypertarget{references-between-spans}{%
\subsection{References between
Spans}\label{references-between-spans}}

A Span may reference zero or more other \textbf{SpanContexts} that are
causally related. OpenTracing presently defines two types of references:
\texttt{ChildOf} and \texttt{FollowsFrom}. \textbf{Both reference types
specifically model direct causal relationships between a child Span and
a parent Span.} In the future, OpenTracing may also support reference
types for Spans with non-causal relationships (e.g., Spans that are
batched together, Spans that are stuck in the same queue, etc).

\textbf{\texttt{ChildOf} references:} A Span may be the \texttt{ChildOf}
a parent Span. In a \texttt{ChildOf} reference, the parent Span depends
on the child Span in some capacity. All of the following would
constitute \texttt{ChildOf} relationships:

\begin{itemize}

\item
  A Span representing the server side of an RPC may be the
  \texttt{ChildOf} a Span representing the client side of that RPC
\item
  A Span representing a SQL insert may be the \texttt{ChildOf} a Span
  representing an ORM save method
\item
  Many Spans doing concurrent (perhaps distributed) work may all
  individually be the \texttt{ChildOf} a single parent Span that merges
  the results for all children that return within a deadline
\end{itemize}

These could all be valid timing diagrams for children that are the
\texttt{ChildOf} a parent.

\begin{verbatim}
    [-Parent Span---------]
         [-Child Span----]

    [-Parent Span--------------]
         [-Child Span A----]
          [-Child Span B----]
        [-Child Span C----]
         [-Child Span D---------------]
         [-Child Span E----]
\end{verbatim}

\textbf{\texttt{FollowsFrom} references:} Some parent Spans do not
depend in any way on the result of their child Spans. In these cases, we
say merely that the child Span \texttt{FollowsFrom} the parent Span in a
causal sense. There are many distinct \texttt{FollowsFrom} reference
sub-categories, and in future versions of OpenTracing they may be
distinguished more formally.

These can all be valid timing diagrams for children that ``FollowsFrom''
a parent.

\begin{verbatim}
    [-Parent Span-]  [-Child Span-]


    [-Parent Span--]
     [-Child Span-]


    [-Parent Span-]
                [-Child Span-]
\end{verbatim}

\hypertarget{the-opentracing-api}{%
\section{The OpenTracing API}\label{the-opentracing-api}}

There are three critical and inter-related types in the OpenTracing
specification: \texttt{Tracer}, \texttt{Span}, and \texttt{SpanContext}.
Below, we go through the behaviors of each type; roughly speaking, each
behavior becomes a ``method'' in a typical programming language, though
it may actually be a set of related sibling methods due to type
overloading and so on.

When we discuss ``optional'' parameters, it is understood that different
languages have different ways to construe such concepts. For example, in
Go we might use the ``functional Options'' idiom, whereas in Java we
might use a builder pattern.

\hypertarget{tracer}{%
\subsection{\texorpdfstring{\texttt{Tracer}}{Tracer}}\label{tracer}}

The \texttt{Tracer} interface creates \texttt{Span}s and understands how
to \texttt{Inject} (serialize) and \texttt{Extract} (deserialize) them
across process boundaries. Formally, it has the following capabilities:

\hypertarget{start-a-new-span}{%
\paragraph{\texorpdfstring{Start a new
\texttt{Span}}{Start a new Span}}\label{start-a-new-span}}

Required parameters

\begin{itemize}

\item
  An \textbf{operation name}, a human-readable string which concisely
  represents the work done by the Span (for example, an RPC method name,
  a function name, or the name of a subtask or stage within a larger
  computation). The operation name should be \textbf{the most general
  string that identifies a (statistically) interesting class of
  \texttt{Span} instances}. That is, \texttt{"get\_user"} is better than
  \texttt{"get\_user/314159"}.
\end{itemize}

For example, here are potential \textbf{operation names} for a
\texttt{Span} that gets hypothetical account information:

\begin{longtable}[]{@{}ll@{}}
\toprule
Operation Name & Guidance\tabularnewline
\midrule
\endhead
\texttt{get} & Too general\tabularnewline
\texttt{get\_account/792} & Too specific\tabularnewline
\texttt{get\_account} & Good, and \texttt{account\_id=792} would make a
nice \textbf{\texttt{Span} tag}\tabularnewline
\bottomrule
\end{longtable}

Optional parameters

\begin{itemize}

\item
  Zero or more \textbf{references} to related \texttt{SpanContext}s,
  including a shorthand for \texttt{ChildOf} and \texttt{FollowsFrom}
  reference types if possible.
\item
  An optional explicit \textbf{start timestamp}; if omitted, the current
  walltime is used by default
\item
  Zero or more \textbf{tags}
\end{itemize}

\textbf{Returns} a \texttt{Span} instance that's already started (but
not \texttt{Finish}ed)

\hypertarget{inject-a-spancontext-into-a-carrier}{%
\paragraph{\texorpdfstring{Inject a \texttt{SpanContext} into a
carrier}{Inject a SpanContext into a carrier}}\label{inject-a-spancontext-into-a-carrier}}

Required parameters

\begin{itemize}

\item
  A \textbf{\texttt{SpanContext}} instance
\item
  A \textbf{format} descriptor (typically but not necessarily a string
  constant) which tells the \texttt{Tracer} implementation how to encode
  the \texttt{SpanContext} in the carrier parameter
\item
  A \textbf{carrier}, whose type is dictated by the \textbf{format}. The
  \texttt{Tracer} implementation will encode the \texttt{SpanContext} in
  this carrier object according to the \textbf{format}.
\end{itemize}

\hypertarget{extract-a-spancontext-from-a-carrier}{%
\paragraph{\texorpdfstring{Extract a \texttt{SpanContext} from a
carrier}{Extract a SpanContext from a carrier}}\label{extract-a-spancontext-from-a-carrier}}

Required parameters

\begin{itemize}

\item
  A \textbf{format} descriptor (typically but not necessarily a string
  constant) which tells the \texttt{Tracer} implementation how to decode
  \texttt{SpanContext} from the carrier parameter
\item
  A \textbf{carrier}, whose type is dictated by the \textbf{format}. The
  \texttt{Tracer} implementation will decode the \texttt{SpanContext}
  from this carrier object according to \textbf{format}.
\end{itemize}

\textbf{Returns} a \texttt{SpanContext} instance suitable for use as a
\textbf{reference} when starting a new \texttt{Span} via the
\texttt{Tracer}.

\hypertarget{note-required-formats-for-injection-and-extraction}{%
\paragraph{\texorpdfstring{Note: required \textbf{format}s for injection
and
extraction}{Note: required formats for injection and extraction}}\label{note-required-formats-for-injection-and-extraction}}

Both injection and extraction rely on an extensible \textbf{format}
parameter that dictates the type of the associated ``carrier'' as well
as how a \texttt{SpanContext} is encoded in that carrier. All of the
following \textbf{format}s must be supported by all Tracer
implementations.

\begin{itemize}

\item
  \textbf{Text Map}: an arbitrary string-to-string map with an
  unrestricted character set for both keys and values
\item
  \textbf{HTTP Headers}: a string-to-string map with keys and values
  that are suitable for use in HTTP headers (a la
  \href{https://tools.ietf.org/html/rfc7230\#section-3.2.4}{RFC 7230}).
  In practice, since there is such ``diversity'' in the way that HTTP
  headers are treated in the wild, it is strongly recommended that
  Tracer implementations use a limited HTTP header key space and escape
  values conservatively.
\item
  \textbf{Binary}: a (single) arbitrary binary blob representing a
  \texttt{SpanContext}
\end{itemize}

\hypertarget{span}{%
\subsection{\texorpdfstring{\texttt{Span}}{Span}}\label{span}}

With the exception of the method to retrieve the \texttt{Span}'s
\texttt{SpanContext}, none of the below may be called after the
\texttt{Span} is finished.

\hypertarget{retrieve-the-spans-spancontext}{%
\paragraph{\texorpdfstring{Retrieve the \texttt{Span}s
\texttt{SpanContext}}{Retrieve the Spans SpanContext}}\label{retrieve-the-spans-spancontext}}

There should be no parameters.

\textbf{Returns} the \texttt{SpanContext} for the given \texttt{Span}.
The returned value may be used even after the \texttt{Span} is finished.

\hypertarget{overwrite-the-operation-name}{%
\paragraph{Overwrite the operation
name}\label{overwrite-the-operation-name}}

Required parameters

\begin{itemize}

\item
  The new \textbf{operation name}, which supersedes whatever was passed
  in when the \texttt{Span} was started
\end{itemize}

\hypertarget{finish-the-span}{%
\paragraph{\texorpdfstring{Finish the
\texttt{Span}}{Finish the Span}}\label{finish-the-span}}

Optional parameters

\begin{itemize}

\item
  An explicit \textbf{finish timestamp} for the \texttt{Span}; if
  omitted, the current walltime is used implicitly.
\end{itemize}

With the exception of the method to retrieve a \texttt{Span}'s
\texttt{SpanContext}, no method may be called on a \texttt{Span}
instance after it's finished.

\hypertarget{set-a-span-tag}{%
\paragraph{\texorpdfstring{Set a \texttt{Span}
tag}{Set a Span tag}}\label{set-a-span-tag}}

Required parameters

\begin{itemize}

\item
  The tag key, which must be a string
\item
  The tag value, which must be either a string, a boolean value, or a
  numeric type
\end{itemize}

Note that the OpenTracing project documents certain
\textbf{\href{./semantic_conventions.md\#span-tags-table}{``standard
tags''}} that have prescribed semantic meanings.

\hypertarget{log-structured-data}{%
\paragraph{Log structured data}\label{log-structured-data}}

Required parameters

\begin{itemize}

\item
  One or more key:value pairs, where the keys must be strings and the
  values may have any type at all. Some OpenTracing implementations may
  handle more (or more of) certain log values than others.
\end{itemize}

Optional parameters

\begin{itemize}

\item
  An explicit timestamp. If specified, it must fall between the local
  start and finish time for the span.
\end{itemize}

Note that the OpenTracing project documents certain
\textbf{\href{./semantic_conventions.md\#log-fields-table}{``standard
log keys''}} which have prescribed semantic meanings.

\hypertarget{an-aside-logging-in-general-and-what-it-means-in-opentracing}{%
\paragraph{An aside: ``Logging'' in general, and what it means in
OpenTracing}\label{an-aside-logging-in-general-and-what-it-means-in-opentracing}}

``Logging'' is an overloaded term in our industry; one could reasonably
argue that all tracing is just a particularly organized form of logging.
OpenTracing ``logs'' are really just key:value maps that describe a
particular moment within the context of a Span.

While it's possible to redirect general-purpose process-level logging
into OpenTracing, doing so requires care. For instance, logging
statements that aren't anchored in specific transactions or traces may
not make sense within a tracing system. That said, in enviroments where
an overwhelming fraction of conventional logging statements already
refer to distributed transactions, \texttt{tee}ing that logging data
into OpenTracing is reasonable and often beneficial.

The granularity of Span logs is intended to be finer than typical
``info''-style logging in process-level logging frameworks. Since
tracing systems usually have a smart, all-or-nothing per-trace sampling
mechanism, the verbosity within a single trace can be higher than what
would be appropriate for a process as a whole ---~especially when that
process contends with high concurrency.

\hypertarget{set-a-baggage-item}{%
\paragraph{\texorpdfstring{Set a \textbf{baggage}
item}{Set a baggage item}}\label{set-a-baggage-item}}

Baggage items are key:value string pairs that apply to the given
\texttt{Span}, its \texttt{SpanContext}, and \textbf{all \texttt{Spans}
which directly or transitively \emph{reference} the local
\texttt{Span}.} That is, baggage items propagate in-band along with the
trace itself.

Baggage items enable powerful functionality given a full-stack
OpenTracing integration (for example, arbitrary application data from a
mobile app can make it, transparently, all the way into the depths of a
storage system), and with it some powerful costs: use this feature with
care.

Use this feature thoughtfully and with care. Every key and value is
copied into every local \emph{and remote} child of the associated Span,
and that can add up to a lot of network and cpu overhead.

Required parameters

\begin{itemize}

\item
  The \textbf{baggage key}, a string
\item
  The \textbf{baggage value}, a string
\end{itemize}

\hypertarget{get-a-baggage-item}{%
\paragraph{\texorpdfstring{Get a \textbf{baggage}
item}{Get a baggage item}}\label{get-a-baggage-item}}

Required parameters

\begin{itemize}

\item
  The \textbf{baggage key}, a string
\end{itemize}

\textbf{Returns} either the corresponding \textbf{baggage value}, or
some indication that such a value was missing.

\hypertarget{spancontext}{%
\subsection{\texorpdfstring{\texttt{SpanContext}}{SpanContext}}\label{spancontext}}

The \texttt{SpanContext} is more of a ``concept'' than a useful piece of
functionality at the generic OpenTracing layer. That said, it is of
critical importance to OpenTracing \emph{implementations} and does
present a thin API of its own. Most OpenTracing users only interact with
\texttt{SpanContext} via
\protect\hyperlink{references-between-spans}{\textbf{references}} when
starting new \texttt{Span}s, or when injecting/extracting a trace
to/from some transport protocol.

In OpenTracing we force \texttt{SpanContext} instances to be
\textbf{immutable} in order to avoid complicated lifetime issues around
\texttt{Span} finish and references.

\hypertarget{iterate-through-all-baggage-items}{%
\paragraph{Iterate through all baggage
items}\label{iterate-through-all-baggage-items}}

This is modeled in different ways depending on the language, but
semantically the caller should be able to efficiently iterate through
all baggage items in one pass given a \texttt{SpanContext} instance.

\hypertarget{nooptracer}{%
\subsection{\texorpdfstring{\texttt{NoopTracer}}{NoopTracer}}\label{nooptracer}}

All OpenTracing language APIs must also provide some sort of
\texttt{NoopTracer} implementation which can be used to flag-control
OpenTracing or inject something harmless for tests (et cetera). In some
cases (for example, Java) the \texttt{NoopTracer} may be in its own
packaging artifact.

\hypertarget{optional-api-elements}{%
\subsection{Optional API Elements}\label{optional-api-elements}}

Some languages also provide utilities to pass an active \texttt{Span}
and/or \texttt{SpanContext} around a single process. For instance,
\texttt{opentracing-go} provides helpers to set and get the active
\texttt{Span} in Go's \texttt{context.Context} mechanism.

\end{document}
