\PassOptionsToPackage{unicode=true}{hyperref} % options for packages loaded elsewhere
\PassOptionsToPackage{hyphens}{url}
\documentclass[a4paper,12pt,notitlepage,twoside,openright]{article}

\usepackage{ifxetex}
\ifxetex{}
\else
\errmessage{Must be built with XeLaTeX}
\fi

\usepackage{amsmath}
\usepackage{fontspec}
\usepackage{fourier-otf} % erewhon-math
\setmonofont{iosevka-type-slab-regular}[
  Path=../common/iosevka-type-slab/,
  Extension=.ttf,
  BoldFont=iosevka-type-slab-bold,
  ItalicFont=iosevka-type-slab-italic,
  BoldItalicFont=iosevka-type-slab-bolditalic,
  Scale=MatchLowercase,
]

% Math
\usepackage[binary-units]{siunitx}

\usepackage{caption}
\usepackage{authblk}
\usepackage{enumitem}
\usepackage{footnote}
\usepackage{bookmark}

% Table
\usepackage{tabu}
\usepackage{longtable}
\usepackage{booktabs}
\usepackage{multirow}

% Verbatim & Source code
\usepackage{fancyvrb}
\usepackage{minted}

% Beauty
\usepackage[protrusion]{microtype}
\usepackage[defaultlines=3]{nowidow}
\usepackage{upquote}
\usepackage{parskip}
\usepackage[strict]{changepage}

\usepackage{hyperref}

% Graph
\usepackage{graphicx}
\usepackage{grffile}
\usepackage{tikz}

\setlist{noitemsep}

\hypersetup{
	bookmarksnumbered,
	colorlinks, % hidelinks,
	pdfpagemode=UseNone,
	pdfstartview=FitH
}

\usetikzlibrary{calc,shapes.multipart,chains,arrows,matrix}
\tikzset{>=latex'}
\tikzstyle{list}=[
	 rectangle split,
	 rectangle split parts=2, draw,
	 rectangle split horizontal,
	 rectangle split part align=base,
	 text width=3ex, text centered
    ]
    
\newcommand{\pFF}[2]{\draw[->] let \p1 = (#1.one), \p2 = (#1.center), \p3 = (#2.one), \p4 = (#2.north) in (\x1,\y2) -- (\x3,\y4)}
\newcommand{\pSF}[2]{\draw[->] let \p1 = (#1.two), \p2 = (#1.center), \p3 = (#2.west), \p4 = (#2.center) in (\x1,\y2) -- (\x3,\y4)}

\newcolumntype{L}{>{\RaggedRight}X}

\newcommand{\SpecialChapter}[1]{%
  \chapter*{#1}%
  \addcontentsline{toc}{chapter}{#1}%
  \markboth{\MakeUppercase{#1}}{\MakeUppercase{#1}}
}

\title{Data Management: Past, Present, and Future}
\author{Jim Gray}
\date{ June 1996}

\begin{document}
\maketitle

\begin{abstract}

Soon most information will be available at your
fingertips, anytime, anywhere. Rapid advances in storage,
communications, and processing allow us move all information into
Cyberspace. Software to define, search, and visualize online information
is also a key to creating and accessing online information. This article
traces the evolution of data management systems and outlines current
trends. Data management systems began by automating traditional tasks:
recording transactions in business, science, and commerce. This data
consisted primarily of numbers and character strings. Today these
systems provide the infrastructure for much of our society, allowing
fast, reliable, secure, and automatic access to data distributed
throughout the world. Increasingly these systems automatically design
and manage access to the data. The next steps are to automate access to
richer forms of data: images, sound, video, maps, and other media. A
second major challenge is automatically summarizing and abstracting data
in anticipation of user requests. These multi-media databases and tools
to access them will be a cornerstone of our move to Cyberspace.

\end{abstract}

\hypertarget{introduction-and-overview}{%
\section{Introduction And Overview}\label{introduction-and-overview}}

\begin{figure}
  \centering
  \includegraphics[width=0.8\columnwidth]{fig-1.png}
  \caption{The six generations of data management, evolving from manual
methods, through several stages of automated data management.}
\end{figure}

Computers can now store all forms of information: records, documents,
images, sound recordings, videos, scientific data, and many new data
formats. We have made great strides in capturing, storing, managing,
analyzing, and visualizing this data. These tasks are generically called
data management. This paper sketches the evolution of data management
systems describing six generations of data managers shown in Figure 1.
The article then outlines current trends,

Data management systems typically store huge quantities of data
representing the historical records of an organization. These databases
grow by accretion. It is important that the old data and applications
continue to work as new data and applications are added. The systems are
in constant change. Indeed, most of the larger database systems in
operation today were designed several decades ago and have evolved with
technology. A historical perspective helps to understand current
systems.

There have been six distinct phases in data management. Initially, data
was manually processed. The next step used punched-card equipment and
electro-mechanical machines to sort and tabulate millions of records.
The third phase stored data on magnetic tape and used stored program
computers to perform batch processing on sequential files. The fourth
phase introduced the concept of a database schema and online
navigational access to the data. The fifth step automated access to
relational databases and added distributed and client-server processing.
We are now in the early stages of sixth generation systems that store
richer data types, notably documents, images, voice, and video data.
These sixth generation systems are the storage engines for the emerging
Internet and Intranets.

\hypertarget{historical-perspective-the-six-generations-of-data-management}{%
\section{Historical perspective: The Six Generations of Data
Management}\label{historical-perspective-the-six-generations-of-data-management}}

\hypertarget{zeroth-generation-record-managers-4000bc--1900}{%
\subsection{Zeroth generation: Record Managers 4000BC--1900}\label{zeroth-generation-record-managers-4000bc--1900}}

The first known writing describes the royal assets and taxes in Sumeria.
Record keeping has a long history. The next six thousand years saw a
technological evolution from clay tablets to papyrus to parchment and
then to paper. There were many innovations in data representation:
phonetic alphabets, novels, ledgers, libraries, paper and the printing
press. These were great advances, but the information processing in this
era was manual. \emph{(Note to editor: it would be nice to have a photo
of a Sumarian tablet or a Hollerith machine here. U. Penn has a good
collection of photos of Sumerian tablets.)}

\hypertarget{first-generation-record-managers-1900--1955}{%
\subsection{First Generation: Record Managers 1900--1955}\label{first-generation-record-managers-1900--1955}}

The first practical automated information processing began circa 1800
with the Jacquard Loom that produced fabric from patterns represented by
punched cards. Player pianos later used similar technology. In 1890,
Hollerith used punched card technology to perform the US census. His
system had a record for each household. Each data record was represented
as binary patterns on a punched card. Machines tabulated counts for
blocks, census tracts, Congressional Districts, and States. Hollerith
formed a company to produce equipment that recorded data on cards,
sorted, and tabulate the cards {[}1{]}. Hollerith's business eventually
became International Business Machines. This small company, IBM,
prospered as it supplied unit-record equipment for business and
government between 1915 and 1960.

By 1955, many companies had entire floors dedicated to storing punched
cards, much as the Sumerian archives had stored clay tablets. Other
floors contained banks of card punches, sorters, and tabulators. These
machines were programmed by rewiring control panels (patch-boards) that
managed some accumulator registers, and that selectively reproduced
cards onto other cards or onto paper. Large companies were processing
and generating millions of records each night. This would have been
impossible with manual techniques. Still, it was clearly time for a new
technology to replace punched cards and electro-mechanical computers.

\hypertarget{second-generation-programmed-unit-record-equipment-1955--1970}{%
\subsection{Second Generation: Programmed Unit Record Equipment 1955--1970}\label{second-generation-programmed-unit-record-equipment-1955-1970}}

Stored program electronic computers had been developed in the 1940's and
early 1950's for scientific and numerical calculations. At about the
same time, Univac had developed a magnetic tape that could store as much
information as ten thousand cards: giving huge improvements in space,
time, convenience, and reliability. The 1951 delivery of the UNIVAC1 to
the Census Bureau echoed the development of punched card equipment.
These new computers could process hundreds of records per second, and
they could fit in a fraction of the space occupied by the unit-record
equipment.

Software was a key component of this new technology. It made them
relatively easy to program and use. It was much easier to sort, analyze,
and process the data with languages like COBOL and RPG. Indeed, standard
packages began to emerge for common business applications like
general-ledger, payroll, inventory control, subscription management,
banking, and document libraries.

The response to these new technologies was predictable. Large businesses
recorded even more information, and demanded faster and faster
equipment. As prices declined, even medium-sized businesses began to
capture transactions on cards and use a computer to process the cards
against a tape-based master file.

The software of the day provided a \textbf{file-oriented record
processing} model. Typical programs sequentially read several input
files and produced new files as output. COBOL and several other
programming languages were designed to make it easy to define these
record-oriented sequential tasks. Operating systems provided the file
abstraction to store these records, a job control language to run the
jobs, and a job scheduler to manage the workflow.

\textbf{Batch transaction processing} systems captured transactions on
cards or tape and collected them in a batch for later processing. Once a
day these transaction batches were sorted. The sorted transactions were
merged with the much larger database (master file) stored on tape to
produce a new master file. This master file also produced a report that
was used as the ledger for the next day's business. Batch processing
used computers very efficiently, but it had two serious shortcomings. If
there was an error in a transaction, it was not detected until that
evening's run against the master file, and the transaction might take
several days to correct. More significantly, the business did not know
the current state of the database --- so transactions were not really
processed until the next morning. Solving these two problems required
the next evolutionary step, online systems. This step also made it much
easier to write applications.

\hypertarget{third-generation-online-network-databases-1965--1980}{%
\subsection{Third Generation: Online Network Databases
1965--1980}\label{third-generation-online-network-databases-1965-1980}}

Applications like stock-market trading and travel reservation need to
know the current information. They could not use the day-old information
provided by off-line batch transaction processing --- rather they need
immediate access to current data. Starting in the late 1950's, leaders
in several industries began innovating with online transaction databases
which interactively processed transactions against online databases.
Several technologies were key to enabling online data access. The
hardware to connect interactive computer terminals to a computer evolved
from teletypes, to simple CRT displays, and to today's intelligent
terminals based on PC technology. \textbf{Teleprocessing monitors}
provided the specialized software to multiplex thousands of terminals
onto the modest server computers of the day. These TP monitors collected
request messages from a terminal, quickly dispatched server programs to
process each message, and then dispatched the response back to the
requesting terminal. \textbf{Online transaction processing} augmented
the batch transaction processing that performed background reporting
tasks.

Online databases stored on magnetic disks or drums provided sub-second
access to any data item. These devices and data management software
allowed programs to read a few records, update them, and then return the
new values to the online user. Initially, the systems provided simple
record lookup: either by direct lookup by record number or associative
lookup by a record key.

Simple indexed-sequential record organizations soon evolved to a more
powerful \textbf{set-oriented record model.} Applications often want to
relate two or more records. Figure 2.a shows some record types of a
simple airline reservation system and their relationships. Each city has
a set of outgoing flights. Each customer has a set of trips, and each
trip consists of a set of flights. In addition, each flight has a set of
passengers. This information can be represented as three
set-hierarchies, as shown in figure 2.b. Each of the three hierarchies
answers a different question: the first is the flight schedule by city.
The second hierarchy gives the customer's view of his flights. The third
hierarchy tells which customers are on each flight. The travel
reservation application needs all three of these data views.

The hierarchical representation of figure 2.b has a major shortcoming.
Storing data redundantly is expensive, but also creates update problems:
when a flight is created or is altered the flight information must be
updated in all three places (all three hierarchies.) To solve these
problems, the information could be represented with a \textbf{network
data model} shown in figure 2.c. Figure 2.c depicts a single database
where each record is stored once and is related to a set of other
records via a relationship. For example, all the flights involved in a
specific customer's trip are related to that trip. A program can ask the
database system to enumerate those flights. New relationships among
records can be created as needed. Figure 2.c is variously called a
Bachman diagram or an Entity-Relationship diagram {[}2{]}, {[}5{]}. The
relational diagram of figure 2 (figure 2.d) is described in the next
section.

\begin{figure}
  \centering
  \includegraphics[width=0.8\columnwidth]{fig-2.png}
  \caption[The evolution of data models.]{The evolution of data models. (a) A pure hierarchical model
with records grouped under other records. (b) As the application grows,
different users want different views of the data expressed as different
hierarchies. (c) A Bachman diagram showing the record sets and
relationships among record types. (d) The same information represented
as in the relational model where all data and all relationships are
explicitly represented as records. The relations are shown at the top of
the figure. Some details of the Segment relation are shown at the lower
right; it has a record for each flight (segment) in any passenger's
itinerary.}
\end{figure}

Managing associative access and set-oriented processing was so common
that the COBOL community chartered a Data Base Task Group (DBTG) to
define a standard way to define and access such data. Charles Bachman
had built a prototype data navigation system at GE. Bachman received the
Turing award for leading the DBTG effort which defined a standard data
definition and data manipulation language. In his Turing lecture he
described the evolution from flat-file models to the new world where
programs could navigate among records by following the relationships
among the records {[}2{]}. Bachman's model is reminiscent of Vannevar
Bush's Memex system {[}2{]} or the pages-and-links navigational model of
today's Internet.

The COBOL database community crystallized the concept of
\textbf{schemas} and data independence. They understood the need to hide
the physical details of record layouts. Programs should see only the
logical organization of records and relationships, so that the programs
continued to work as the data layout was reorganized and evolved over
time. Records, fields, and relationships not used by the program should
be hidden --- both for security reasons, and to insulate the program from
the inevitable changes to the database design over time. These early
databases supported three kinds of data schemas: (1) a \textbf{logical
schema} that defines the global logical design of the database records
and relationships among records, (2) a \textbf{physical schema} that
describes the physical layout of the database records on storage devices
and files, and the indices needed to support the logical relationships,
and (3) each application was given a \textbf{sub-schema} exposing just
the subset of the logical schema used by the program. The
logical-physical-sub-schema mechanism provided \textbf{data
independence}. Indeed, may programs written in that era are still
running today using the same sub-schema the programs started with, even
though the logical and physical schemas have evolved to completely new
designs.

These online systems had to solve the problem of running many concurrent
transactions against a database shared among many terminal users. Prior
to this, the single-program-at-a-time old-master new-master approach
eliminated concurrency and recovery problems. The early online systems
pioneered the concept of \textbf{transactions} that lock just the
records that they access. Transaction locking allows concurrent
transactions to access different records. The systems also kept a log of
the records that each transaction changed. If the transaction failed,
the log was used to undo the effects of the transaction. The transaction
log was also used for media recovery. If the system failed, the log was
re-applied to an archive copy of the database to reconstruct the current
database.

By 1980 the set-oriented network (and hierarchical) data models were
very popular. Cullinet, a company founded by Bachman, was the largest
and fastest-growing software company in the world.

\hypertarget{fourth-generation-relational-databases-and-client-server-computing-1980--1995}{%
\subsection{Fourth Generation: Relational Databases and client-server computing 1980--1995}\label{fourth-generation-relational-databases-and-client-server-computing-1980-1995}}

Despite the success of the network data model, many software designers
felt that a navigational programming interface was too low-level. It was
difficult to design and program these databases. E.F. Codd's 1970 paper
outlined the relational model {[}4{]} that seemed to provide an
alternative to the low-level navigational interfaces. The idea of the
\textbf{relational model} is to represent both entities and
relationships in a uniform way. The relational data model has a unified
language for data definition, data navigation, and data manipulation,
rather than separate languages for each task. More importantly, the
relational algebra deals with record sets (relations) as a group,
applying operators to whole record sets and producing record sets as a
result. The relational data model and operators gives much shorter and
simpler programs to perform record management tasks. To give a concrete
example, the airline database of the previous section would be
represented by five tables as shown in Figure 2.d. Rather than
implicitly storing the relationship between flights and trips, a
relational system explicitly stores each flight-trip pair as a record in
the database. This is the ``Segment'' table in Figure 2.d.

To find all segments reserved for customer Jones going to San Francisco,
one would write the SQL query:

\begin{verbatim}
Select Flight#
From City, Flight, Segment, Trip, Customer
Where Flight.to = "SF" AND
      Flight.flight# = Segment.flight# AND
      Segment.trip# = trip.trip# AND
      trip.customer# = customer.customer# AND
      customer.name = "Jones"
\end{verbatim}

The English equivalent of this SQL query is: ``Find the flight numbers
for flights to San Francisco which are a segment of a trip booked by any
customer named ``Jones.'' Combine the City, Flight, Segment, Tip, and
Customer tables to find this flight.'' This program may seem complex,
but it is vastly simpler than the corresponding navigational program.

Given this non-procedural query, the relational database system
automatically finds the best way to match up records in the City,
Flight, Segment, Trip, and Customer tables. The query does not depend on
which relationships are defined. It will continue to work even after the
database is logically reorganized. Consequently, it has much better data
independence than a navigational query based on the network data model.
In addition to improving data independence, relational programs are
often five or ten times simpler than the corresponding navigational
program.

Inspired by Codd's ideas, researchers in academe and industry
experimented throughout the 1970's with this new approach to structuring
and accessing databases promising dramatically easier data modeling and
application programming. The many relational prototypes developed during
this period converged on a common model and language. Work at IBM
Research led by Ted Codd, Raymond Boyce, and Don Chamberlin and work at
UC Berkeley led by Michael Stonebraker gave rise to a language called
SQL. This language was first standardized in 1985. There have been two
major additions to the standard since then {[}5{]}, {[}6{]}. Virtually
all database systems provide an SQL interface today. In addition, all
systems provide unique extensions that go beyond the standard.

The relational model had some unexpected benefits beyond programmer
productivity and ease-of-use. The relational model was well suited to
client-server computing, to parallel processing, and to graphical user
interfaces. \textbf{Client-server} application designs divide
applications in two parts. The \textbf{client} part is responsible for
capturing inputs and presenting data outputs to the user or client
device. The \textbf{server} is responsible for storing the database,
processing client requests against a database, and responding with a
summary answer. The relational interface is especially convenient for
client-server computing because it exchanges high-level requests and
responses. SQL's high-level language minimizes communication between
client and server. Today, many client-server tools are built around the
Open Database Connectivity (ODBC) protocol that provides a standard way
for clients to make high-level requests to servers. The client-server
paradigm continues to evolve. As explained in the next section, there is
an increasing trend to integrate procedures into database servers. In
particular, procedural languages like BASIC and Java have been added to
servers so that clients can invoke application procedures running at the
server.

\textbf{Parallel database processing} was the second unanticipated
benefit of the relational model. Relations are uniform sets of records.
The relational model consists of operators closed under composition:
each operator takes relations as inputs and produces a relation as a
result. Consequently, relational operators naturally give pipeline
parallelism by piping the output of one operator to the input of the
next. It is rare to find long pipelines, but relational operators can
often be partitioned so that each operator can be cloned \emph{N} ways
and each clone can work on \emph{1/Nth} of the input relation. These
ideas were pioneered by academe and by Teradata Corporation (now NCR).
Today, it is routine for relational systems to provide hundred-fold
speedups by using parallelism. Data mining jobs that might takes weeks
or months to search multi-terabyte databases are done within hours by
using parallelism. This parallelism is completely automatic. Designers
just present the data to the database system, and the system partitions
and indexes the data. Users present queries to the system (as ODBC
requests) and the system automatically picks a parallel plan for the
query and executes it.

Relational data is also well suited for \textbf{graphical user
interfaces} (GUIs). It is very easy to render a relation as a set of
records --- relations fit a spreadsheet metaphor. Users can easily create
spreadsheet-like relations and can visually manipulate them. Indeed,
there are many tools that move relational data between documents,
spreadsheets, and databases. Explicitly representing data,
relationships, and meta-data in a uniform way makes this possible.

Relational systems combined with GUIs allow hundreds of thousands of
people to pose complex database queries each day. The combinations of
GUIs and relational systems has come closest to the goal of
automatic-programming. GUIs allow very complex queries to be easily
constructed. Given a non-procedural query, relational systems find the
most efficient way to execute that query.

Continuing the historical perspective, by 1980 Oracle, Informix, and
Ingress had brought relational database management systems to market.
Within a few more years, IBM and Sybase had brought their products to
market. By 1990, the relational systems had become more popular than the
earlier set-oriented navigational systems. Meanwhile file systems, and
set-oriented systems were still the workhorses of many corporations.
These corporations had built huge applications over the years and could
not easily change to relational systems. Rather, relational systems
became the key tool for new client-server applications.

\hypertarget{fifth-generation-multimedia-databases-1995--}{%
\subsection{Fifth Generation: Multimedia Databases 1995--}\label{fifth-generation-multimedia-databases-1995-}}

Relational systems offered huge improvements in ease-of-use, graphical
interfaces, client-server applications, distributed databases, parallel
data search, and data mining. Nonetheless, in about 1985, the research
community began to look beyond the relational model. Traditionally,
there had been a clear separation between programs and data. This worked
well when the data was just numbers, characters, arrays, lists, or sets
of records. As new applications appeared, the separation between
programs and data became problematic. The applications needed to give
the data behavior. For example, if the data was a complex object, then
the methods to search, compare, and manipulate the data were peculiar to
the, document, image, sound, or map datatype (see figure 3).

\begin{figure}
  \centering
  \includegraphics[width=0.8\columnwidth]{fig-3.png}
  \caption{The transition of traditional databases storing numbers and
characters into an object-relational database where each record can
contain data with complex behavior. These behaviors are encapsulated in
the class libraries that support the new types. In this model, the
database system stores and retrieves the data and provides relationships
among data items, but the class libraries provide the item behavior.}
\end{figure}

The traditional approach was to build the datatypes right into the
database system. SQL added new datatypes for time, time intervals, and
two-byte character strings. Each of these extensions was a significant
effort. When they were done, the results were not appropriate for
everyone. For example, SQL time cannot represent dates before the
Christian Era and the multi-character design does not include Unicode (a
universal character set for almost all languages). Users wanting Unicode
or pre-Christian dates must define their own datatypes. These simple
examples, and many others convinced the database community that the
database system must allow domain specialists to implement the datatypes
for their domains. Geographers should implement maps, text specialists
should implement text indexing and retrieval, and image specialists
should implement the type libraries for images. To give a specific
example, a data time series is a common object type. Rather than build
this object into the database system, it is recommended that the type be
implemented as a class library with methods to create, update and delete
a time series. Additional methods summarize trends and interpolate
points in a series, and compare, combine and difference two series. Once
this class library is built, it can be ``plugged into'' any database
system. The database system will store objects of this type and will
manage the data (security, concurrency, recovery, and indexing) but the
datatype will manage the contents and behavior of time-series objects.

People coming from the object-oriented programming community saw the
problem clearly: datatype design requires a good data model and a
unification of procedures and data. Indeed, programs encapsulate the
data and provide all the methods to manipulate the data. Researchers,
startups, and established relational database vendors have labored long
and hard since 1985 to either replace the relational model or unify the
object-oriented and relational systems. Over a dozen Object-Oriented
database products came to market in the late 1980's, but customers were
slow to accept these systems. Meanwhile, the traditional vendors tried
to extend the SQL language to embrace object oriented concepts, while
preserving the benefits of the relational model.

There is still heated debate on the outcome of this evolution vs.
revolution in data models. There is no debate that database systems must
store and retrieve objects that are managed by class libraries. The
debate revolves around the role of SQL, around the details of the object
model, and around the core class libraries that the database system
should support.

The rapid evolution of the Internet amplifies these debates. Internet
clients and servers are being built around ``applets'' and ``helpers''
that capture, process, and render one data type or another. Users plug
these applets into a browser or server. The common applets manage sound,
image, text, video, spreadsheets, graphs. These applets are each class
libraries for their associated types. Desktops and web browsers are
ubiquitous sources and destinations for much of the data. Hence, the
types and object models used on the desktop will drive the server class
libraries that database systems must support.

To summarize, databases are being called upon to store more than just
numbers and text strings. They are being used to store the many kinds of
objects we see on the World Wide Web, and to store relationships among
them. The distinction between the database and the rest of the web is
being blurred. Indeed, each database vendor is promising a ``universal
server'' that will store and analyze all forms of data (all class
libraries and their objects).

Unifying procedures and data extends the traditional client-server
computing model in two interesting ways: (1) active databases and (2)
workflow. Active databases autonomously perform tasks when the database
changes. The idea is that a user-defined trigger procedure fires when a
database condition becomes true. Using the database procedure language,
database designers can define pre-conditions and triggers procedures.
For example, if a re-order trigger has been defined on an inventory
database, then the database will invoke a reorder procedure on an item
anytime the item's inventory falls below the reorder threshold. Triggers
simplify applications by moving logic from the applications to the data.
The trigger mechanism is a powerful way to build active databases that
are self-managing.

Workflow generalizes the typical request-response model of computing. A
workflow is a script of tasks that must be executed. For example, a
simple purchase agreement consists of a seven step workflow for: (1)
buyer request, (2) bid, (3) agree, (4) ship, (5) invoice, (6) pay.
Systems to script, execute and mange workflows are becoming common.

To close on the current status of data management technology, it makes
sense to describe two large data management projects that stretch the
limits of our technology today. The Earth Observation System Data /
Information System (EOS/DIS) is being built by NASA and its contractors
to store all the satellite data that will start arriving from the
Mission to Planet Earth satellites in 1997. The database, consisting of
remote sensor data, will grow by 5 terabytes a day (a terabyte is a
million megabytes). By 2007, the database will have grown to 15
petabytes. This is a thousand times larger than the largest online
databases today. NASA wants this database to be available to everyone,
everywhere, all the time. Anyone should be able to search, analyze, and
visualize the data in this database. Building EOS/DIS will require
advances in data storage, data management, data search, and data
visualization. Most of the data has both spatial and temporal
characteristics, so the system requires substantial advances storing
those data types, as well as class libraries for the various scientific
data sets. For example, this application will need a library to
recognize snow cover, vegetation index, clouds, and other physical
features in LandSat images. This class library must easily plug into the
EOS/DIS data manager.

The emerging world-wide library gives another challenging database
example. Many institutional libraries are putting their holdings online.
New scientific literature is being published online. Online publishing
poses difficult societal issues about copyrights and intellectual
property, but it also poses deep technical challenges. The size and
diversity of this information are daunting. The information appears in
many languages, in many data formats, and in huge volumes. Traditional
or approaches to organizing this information (author, subject, title) do
not exploit the power of computers to search documents by content, to
link documents, and to cluster similar documents together. Information
discovery, finding relevant information in the sea of text documents,
maps, photographs, sounds, and videos, poses an exciting and challenging
problem.

\hypertarget{reflections-and-predictions}{%
\section{Reflections And
Predictions}\label{reflections-and-predictions}}

Advances in computer hardware have enabled the evolution of data
management from paper-based manual processing to modern information
search engines. This progress in hardware is expected to continue for
many more years.

Data management software has advanced in parallel to these hardware
advances. The record and set-oriented systems gave way to relational
systems that are now evolving to object-relational systems. These
innovations give one of the best examples of research prototypes turning
into products. The relational model, parallel database systems, active
databases, and object-relational databases all came from the academic
and industrial research labs. The development of database technology has
been a textbook case of successful collaboration between academe and
industry.

Inexpensive hardware and easy software have made computers accessible to
almost everyone. It is now easy and inexpensive to create a web server
or a database. Millions of people have done it. These users expect
computers to automatically design and manage themselves. These users do
not want to be computer operators. They expect to add new applications
with almost no effort: a plug-and-play mentality. This view extends from
simple desktop systems to very high-end servers. Users expect automated
management with intuitive graphical interfaces for all administration,
operations, and design tasks. Once the database is built and
operational, users expect simple and powerful tools to browse, search,
analyze and visualize the data. These requirements stretch the limits of
what we know how to do today.

Many data management challenges remain, both technical and societal.
Large online databases raise serious societal issues. Electronic data
interchange and data mining software makes it relatively easy for a
large organization to track all your financial transactions. By doing
that, someone can build a very detailed profile of your interests,
travel, and finances. Is this an invasion of your privacy? Indeed, it is
possible to do this for almost everyone in the developed world. What are
the implications of that? What are the privacy and security rules
surrounding online medical records? Who should be allowed to see your
records? How will copyrights work when anyone anywhere can access an
electronic copy of a document? Cyberspace crosses national boundaries.
What are the rights and responsibility of people operating in
Cyberspace?

Our grandchildren will probably still be wrestling with these societal
issues 50 years hence. The technical challenges are more tractable.
There is broad consensus within the database community on the main
challenges and a research agenda to attach those problems. Every five
years, the database community does a self-assessment that outlines this
agenda. The most recent self-assessment, called the Lagunita II report
{[}8{]}, emphasizes the following challenges:

\begin{itemize}
\item
  Defining the data models for new types (e.g., spatial, temporal,
  image, \ldots) and integrating them with the traditional database
  systems.
\item
  Scaling databases in size (to petabytes), space (distributed), and
  diversity (heterogeneous).
\item
  Automatically discovering data trends, patterns, and anomalies (data
  mining, data analysis).
\item
  Integrating (combining) data from multiple sources.
\item
  Scripting and managing the flow of work (process) and data in human
  organizations.
\item
  Automating database design and administration.
\end{itemize}

These are challenging problems. Solving them will open up new
applications for computers both for organizations and for individuals.
These systems will allow us to access and analyze all information from
anywhere at any time. This easy access to information will transform the
way we do science, the way we manage our businesses, the way we learn,
and the way we play. It will both enrich and empower us and future
generations.

Perhaps the most challenging problem is understanding the data. There is
little question that most data will be online -- both because it is
inexpensive to store the data in computers and because it is convenient
to store it in computers. Organizing these huge data archives so that
people can easily find the information they need is the real challenge
we face. Finding patterns, trends, anomalies, and relevant information
from a large database is one of the most exciting new areas of data
management {[}7{]}. Indeed, my hope is that computers will be able to
condense and summarize information for us so that we will be spared the
drudgery searching through irrelevant data for the nuggets we seek. The
solution to this will require contributions from many disciplines.

\hypertarget{references}{%
\section{References}\label{references}}

{[}1{]} \emph{Engines of the Mind: A History of the Computer,} J.
Shurkin, W.W. Norton \& Co. 1984.

{[}2{]} ``The Programmer as Navigator,'' C.W. Bachman, CACM
\textbf{16.11}, Nov. 1973.

{[}3{]} ``As We May Think,'' \emph{The Atlantic Monthly}, V. Bush, July
1945.

{[}4{]} ``A Relational Model of Data for Large Shared Databanks,'' E. F.
Codd, CACM \textbf{13.6}, June 1970.

{[}5{]} \emph{An Introduction to Database Systems},
6\textsuperscript{th} edition,C. J. Date, Addison Wesley, 1995

{[}6{]} \emph{Understanding the New SQL: A Complete Guide}, J. Melton,
A. R. Simon, Morgan Kaufmann, 1993.

{[}7{]} \emph{Advances in Data Mining and Knowledge Discovery,} U.M.
Fayyad, G. Piatetsky-Shapiro, P. Smyth, R. Uthurusamy, MIT Press,
Cambridge, MA., 1995

{[}8{]} ``Database Research: Achievements and Opportunities Into the
21\textsuperscript{st} Century'', A. Silbershatz, M. J. Stonebraker,
J.D. Ullman, editors. ACM SIGMOD Record 25:1 (March, 1996), pp.
52-\/-63.

\end{document}
