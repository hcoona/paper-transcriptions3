\PassOptionsToPackage{unicode=true}{hyperref} % options for packages loaded elsewhere
\PassOptionsToPackage{hyphens}{url}
\documentclass[a4paper,12pt,notitlepage,twoside,openright]{article}

\usepackage{ifxetex}
\ifxetex{}
\else
\errmessage{Must be built with XeLaTeX}
\fi

\usepackage{amsmath}
\usepackage{fontspec}
\usepackage{fourier-otf} % erewhon-math
\setmonofont{iosevka-type-slab-regular}[
  Path=../common/iosevka-type-slab/,
  Extension=.ttf,
  BoldFont=iosevka-type-slab-bold,
  ItalicFont=iosevka-type-slab-italic,
  BoldItalicFont=iosevka-type-slab-bolditalic,
  Scale=MatchLowercase,
]

% Math
\usepackage[binary-units]{siunitx}

\usepackage{caption}
\usepackage{authblk}
\usepackage{enumitem}
\usepackage{footnote}

% Table
\usepackage{tabu}
\usepackage{longtable}
\usepackage{booktabs}
\usepackage{multirow}

% Verbatim & Source code
\usepackage{fancyvrb}
\usepackage{minted}

% Beauty
\usepackage[protrusion]{microtype}
\usepackage[defaultlines=3]{nowidow}
\usepackage{upquote}
\usepackage{parskip}
\usepackage[strict]{changepage}

\usepackage{hyperref}

% Graph
\usepackage{graphicx}
\usepackage{grffile}
\usepackage{tikz}

\usepackage{wasysym}
\hypersetup{
  bookmarksnumbered,
  pdfborder={0 0 0},
  pdfpagemode=UseNone,
  pdfstartview=FitH,
  breaklinks=true}
\urlstyle{same}  % don't use monospace font for urls

\usetikzlibrary{arrows.meta,calc,shapes.geometric,shapes.misc}

\setminted{
  autogobble,
  breakbytokenanywhere,
  breaklines,
  fontsize=\footnotesize,
}
\setmintedinline{
  autogobble,
  breakbytokenanywhere,
  breaklines,
  fontsize=\footnotesize,
}

\makeatletter
\def\maxwidth{\ifdim\Gin@nat@width>\linewidth\linewidth\else\Gin@nat@width\fi}
\def\maxheight{\ifdim\Gin@nat@height>\textheight\textheight\else\Gin@nat@height\fi}
\makeatother

% Scale images if necessary, so that they will not overflow the page
% margins by default, and it is still possible to overwrite the defaults
% using explicit options in \includegraphics[width, height, ...]{}
\setkeys{Gin}{width=\maxwidth,height=\maxheight,keepaspectratio}
\setlength{\emergencystretch}{3em}  % prevent overfull lines
\setcounter{secnumdepth}{3}

% Redefines (sub)paragraphs to behave more like sections
\ifx\paragraph\undefined\else
\let\oldparagraph\paragraph{}
\renewcommand{\paragraph}[1]{\oldparagraph{#1}\mbox{}}
\fi
\ifx\subparagraph\undefined\else
\let\oldsubparagraph\subparagraph{}
\renewcommand{\subparagraph}[1]{\oldsubparagraph{#1}\mbox{}}
\fi

% set default figure placement to htbp
\makeatletter
\def\fps@figure{htbp}
\makeatother


\title{The Volcano Optimizer Generator: Extensibility and Efficient
Search}
\author{Goetz Graefe}
\affil{Portland State University}
\author{William J. McKenna}
\affil{University of Colorado at Boulder}
\date{1993}

\begin{document}
\maketitle

\begin{abstract}

Emerging database application domains demand not only new functionality
but also high performance. To satisfy these two requirements, the
Volcano project provides efficient, extensible tools for query and
request processing, particularly for object-oriented and scientific
database systems. One of these tools is a new optimizer generator. Data
model, logical algebra, physical algebra, and optimization rules are
translated by the optimizer generator into optimizer source code.
Compared with our earlier EXODUS optimizer generator prototype, the
search engine is more extensible and powerful; it provides effective
support for non-trivial cost models and for physical properties such as
sort order. At the same time, it is much more efficient as it combines
dynamic programming, which until now had been used only for relational
select-project-join optimization, with goal-directed search and
branch-and-bound pruning. Compared with other rule-based optimization
systems, it provides complete data model independence and more natural
extensibility.

\end{abstract}

\hypertarget{introduction}{%
\section{Introduction}\label{introduction}}

While extensibility is an important goal and requirement for many
current database research projects and system prototypes, performance
must not be sacrificed for two reasons. First, data volumes stored in
database systems continue to grow, in many application domains far
beyond the capabilities of most existing database systems. Second, in
order to overcome acceptance problems in emerging database application
areas such as scientific computation, database systems must achieve at
least the same performance as the file systems currently in use.
Additional software layers for database management must be
counterbalanced by database performance advantages normally not used in
these application areas. Optimization and parallelization are prime
candidates to provide these performance advantages, and tools and
techniques for optimization and parallelization are crucial for the
wider use of extensible database technology.

For a number of research projects, namely the Volcano extensible,
parallel query processor {[}4{]}, the REVELATION OODBMS project {[}11{]}
and optimization and parallelization in scientific databases {[}20{]} as
well as to assist research efforts by other researchers, we have built a
new extensible query optimization system. Our earlier experience with
the EXODUS optimizer generator had been inconclusive; while it had
proven the feasibility and validity of the optimizer generator paradigm,
it was difficult to construct efficient, production-quality optimizers.
Therefore, we designed a new optimizer generator, requiring several
important improvements over the EXODUS prototype.

First, this new optimizer generator had to be usable both in the Volcano
project with the existing query execution software as well as in other
projects as a stand-alone tool. Second, the new system had to be more
efficient, both in optimization time and in memory consumption for the
search. Third, it had to provide effective, efficient, and extensible
support for physical properties such as sort order and compression
status. Fourth, it had to permit use of heuristics and data model
semantics to guide the search and to prune futile parts of the search
space. Finally, it had to support flexible cost models that permit
generating dynamic plans for incompletely specified queries.

In this paper, we describe the Volcano Optimizer Generator, which will
soon fulfill all the requirements above. Section 2 introduces the main
concepts of the Volcano optimizer generator and enumerates facilities
for tailoring a new optimizer. Section 3 discusses the optimizer search
strategy in detail. Functionality, extensibility, and search efficiency
of the EXODUS and Volcano optimizer generators are compared in Section
4. In Section 5, we describe and compare other research into extensible
query optimization. We offer our conclusions from this research in
Section 6.

\hypertarget{the-outside-view-of-the-volcano-optimizer-generator}{%
\section{The Outside View of the Volcano Optimizer
Generator}\label{the-outside-view-of-the-volcano-optimizer-generator}}

In this section, we describe the Volcano optimizer generator as seen by
the person who is implementing a database system and its query
optimizer. The focus is the wide array of facilities given to the
optimizer implementor, i.e., modularity and extensibility of the Volcano
optimizer generator design. After considering the design principles of
the Volcano optimizer generator, we discuss generator input and
operation. Section 3 discusses the search strategy used by optimizers
generated with the Volcano optimizer generator.

Figure 1 shows the optimizer generator paradigm. When the DBMS software
is being built, a model specification is translated into optimizer
source code, which is then compiled and linked with the other DBMS
software such as the query execution engine. Some of this software is
written by the optimizer implementor, e.g., cost functions. After a data
model description has been translated into source code for the
optimizer, the generated code is compiled and linked with the search
engine that is part of the Volcano optimization software. When the DBMS
is operational and a query is entered, the query is passed to the
optimizer, which generates an optimized plan for it. We call the person
who specifies the data model and implements the DBMS software the
``optimizer implementor''. The person who poses queries to be optimized
and executed by the database system is called the DBMS user.

\begin{figure}
  \centering
  \includegraphics[width=0.8\columnwidth]{fig-1.png}
  \caption{The Generator Paradigm}
\end{figure}

\hypertarget{design-principles}{%
\subsection{Design Principles}\label{design-principles}}

There are five fundamental design decisions embodied in the system,
which contribute to the extensibility and search efficiency of
optimizers designed and implemented with the Volcano optimizer
generator. We explain and justify these decisions in turn.

First, while query processing in relational systems has always been
based on the relational algebra, it is becoming increasingly clear that
query processing in extensible and object-oriented systems will also be
based on algebraic techniques, i.e., by defining algebra operators,
algebraic equivalence laws, and suitable implementation algorithms.
Several object-oriented algebras have recently been proposed, e.g.
{[}16-18{]} among many others. Their common thread is that algebra
operators consume one or more bulk types (e.g., a set, bag, array, time
series, or list) and produce another one suitable as input into the next
operator. The execution engines for these systems are also based on
algebra operators, i.e., algorithms consuming and producing bulk types.
However, the set of operators and the set of algorithms are different,
and selecting the most efficient algorithms is one of the central tasks
of query optimization. Therefore, the Volcano optimizer generator uses
two algebras, called the logical and the physical algebras, and
generates optimizers that map an expression of the logical algebra (a
query) into an expression of the physical algebra (a query evaluation
plan consisting of algorithms). To do so, it uses transformations within
the logical algebra and cost-based mapping of logical operators to
algorithms.

Second, rules have been identified as a general concept to specify
knowledge about patterns in a concise and modular fashion, and knowledge
of algebraic laws as required for equivalence transformations in query
optimization can easily be expressed using patterns and rules. Thus,
most extensible query optimization systems use rules, including the
Volcano optimizer generator. Furthermore, the focus on independent rules
ensures modularity. In our design, rules are translated independently
from one another and are combined only by the search engine when
optimizing a query. Considering that query optimization is one of the
conceptually most complex components of any database system,
modularization is an advantage in itself both for initial construction
of an optimizer and for its maintenance.

Third, the choices that the query optimizer can make to map a query into
an optimal equivalent query evaluation plan are represented as algebraic
equivalences in the Volcano optimizer generator's input. Other systems
use multiple intermediate levels when transforming a query into a plan.
For example, the cost-based optimizer component of the extensible
relational Starburst database system uses an ``expansion grammar'' with
multiple levels of ``non-terminals'' such as commutative binary join,
noncommutative binary join, etc. {[}10{]}. We felt that multiple
intermediate levels and the need to re-design them for a new or extended
algebra confuse issues of equivalence, i.e., defining the choices open
to the optimizer, and of search method, i.e., the order in which the
optimizer considers possible query evaluation plans. Just as
navigational query languages are less user-friendly than nonnavigational
ones, an extensible query optimization system that requires control
information from the database implementor is less convenient than one
that does not. Therefore, optimizer choices are represented in the
Volcano optimizer generator's input file as algebraic equivalences, and
the optimizer generator's search engine applies them in a suitable
manner. However, for database implementors who wish to exert control
over the search, e.g., who wish to specify search and pruning
heuristics, there will be optional facilities to do so.

The fourth fundamental design decision concerns rule interpretation vs.
compilation. In general, interpretation can be made more flexible (in
particular the rule set can be augmented at run-time), while compiled
rule sets typically execute faster. Since query optimization is very
CPU-intensive, we decided on rule compilation similar to the EXODUS
optimizer generator. Moreover, we believe that extending a query
processing system and its optimizer is so complex and time-consuming
that it can never be done quickly, making the strongest argument for an
interpreter pointless. In order to gain additional flexibility with
compiled rule sets, it may be useful to parameterize the rules and their
conditions, e.g., to control the thoroughness of the search, and to
observe and exploit repeated sequences of rule applications. In general,
the issue of flexibility in the search engine and the choice between
interpretation vs. compilation are orthogonal.

Finally, the search engine used by optimizers generated with the Volcano
optimizer generator is based on dynamic programming. We will discuss the
use of dynamic programming in Section 3.


\hypertarget{optimizer-generator-input-and-optimizer-operation}{%
\subsection{Optimizer Generator Input and Optimizer
Operation}\label{optimizer-generator-input-and-optimizer-operation}}

Since one major design goal of the Volcano optimizer generator was to
minimize the assumptions about the data model to be implemented, the
optimizer generator only provides a framework into which an optimizer
implementor can integrate data model specific operations and functions.
In this section, we discuss the components that the optimizer
implementor defines when implementing a new database query optimizer.
The actual user queries and execution plans are input and output of the
generated optimizer, as shown in Figure 1. All other components
discussed in this section are specified by the optimizer implementor
before optimizer generation in the form of equivalence rules and support
functions, compiled and linked during optimizer generation, and then
used by the generated optimizer when optimizing queries. We discuss
parts of the operation of generated optimizers here, but leave it to the
section on search to draw all the pieces together.

The user queries to be optimized by a generated optimizer are specified
as an algebra expression (tree) of \emph{logical operators}. The translation
from a user interface into a logical algebra expression must be
performed by the parser and is not discussed here. The set of logical
operators is declared in the model specification and compiled into the
optimizer during generation. Operators can have zero or more inputs; the
number of inputs is not restricted. The output of the optimizer is a
plan, which is an expression over the algebra of algorithms. The \emph{set of
algorithms}, their capabilities and their costs represents the data
formats and physical storage structures used by the database system for
permanent and temporary data.

Optimization consists of mapping a
logical algebra expression into the optimal equivalent physical algebra
expression. In other words, a generated optimizer reorders operators and
selects implementation algorithms. The algebraic rules of expression
equivalence, e.g., commutativity or associativity, are specified using
\emph{transformation rules}. The possible mappings of operators to algorithms
are specified using \emph{implementation rules}. It is important that the rule
language allow for complex mappings. For example, a join followed by a
projection (without duplicate removal) should be implemented in a single
procedure; therefore, it is possible to map multiple logical operators
to a single physical operator. Beyond simple pattern matching of
operators and algorithms, additional conditions may be specified with
both kinds of rules. This is done by attaching condition code to a rule,
which will be invoke after a pattern match has succeeded.

The results of expressions are described using properties, similar to
the concepts of properties in the EXODUS optimizer generator and the
Starburst optimizer. \emph{Logical properties} can be derived from the logical
algebra expression and include schema, expected size, etc., while
\emph{physical properties} depend on algorithms, e.g., sort order,
partitioning, etc. When optimizing a many-sorted algebra, the logical
properties also include the type (or sort) of an intermediate result,
which can be inspected by a rule's condition code to ensure that rules
are only applied to expressions of the correct type. Logical properties
are attached to equivalence classes --- sets of equivalent logical
expressions and plans --- whereas physical properties are attached to
specific plans and algorithm choices.

The set of physical properties is summarized for each intermediate
result in a \emph{physical property vector}, which is defined by the
optimizer implementor and treated as an abstract data type by the
Volcano optimizer generator and its search engine. In other words, the
types and semantics of physical properties can be designed by the
optimizer implementor.

There are some operators in the physical algebra that do not correspond
to any operator in the logical algebra, for example sorting and
decompression. The purpose of these operators is not to perform any
logical data manipulation but to enforce physical properties in their
outputs that are required for subsequent query processing algorithms. We
call these operators \emph{enforcers}; they are comparable to the ``glue''
operators in Starburst. It is possible for an enforcer to ensure two
properties, or to enforce one but destroy another.

Each optimization goal (and subgoal) is a pair of a logical expression
and a physical property vector. In order to decide whether or not an
algorithm or enforcer can be used to execute the root node of a logical
expression, a generated optimizer matches the implementation rule,
executes the condition code associated with the rule, and then invokes
an \emph{applicability function} that determines whether or not the algorithm
or enforcer can deliver the logical expression with physical properties
that satisfy the physical property vector. The applicability functions
also determine the physical property vectors that the algorithm's inputs
must satisfy. For example, when optimizing a join expression whose
result should be sorted on the join attribute, hybrid hash join does not
qualify while merge-join qualifies with the requirement that its inputs
be sorted. The sort enforcer also passes the test, and the requirements
for its input do not include sort order. When the input to the sort is
optimized, hybrid hash join qualifies. There is also a provision to
ensure that algorithms do not qualify redundantly, e.g., merge-join must
not be considered as input to the sort in this example.


After the optimizer decides to explore using an algorithm or enforcer,
it invokes the algorithm's \emph{cost function} to estimate its cost.
Cost is an \emph{abstract data type} for the optimizer generator;
therefore, the optimizer implementor can choose cost to be a number
(e.g., estimated elapsed time), a record (e.g., estimated CPU time and
I/O count), or any other type. Cost arithmetic and comparisons are
performed by invoking functions associated with the abstract data type
``cost. ''

For each logical and physical algebra expression, logical and physical
properties are derived using \emph{property functions}. There must be one
property function for each logical operator, algorithm, and enforcer.
The \emph{logical properties} are determined based on the logical expression,
before any optimization is performed, by the property functions
associated with the logical operators. For example, the schema of an
intermediate result can be determined independently of which one of many
equivalent algebra expressions creates it. The logical property
functions also encapsulate selectivity estimation. On the other hand,
\emph{physical properties} such as sort order can only be determined after an
execution plan has been chosen. As one of many consistency checks,
generated optimizers verify that the physical properties of a chosen
plan really do satisfy the physical property vector given as part of the
optimization goal.

To summarize this section, the optimizer implementor provides (1) a set
of logical operators, (2) algebraic transformation rules, possibly with
condition code, (3) a set of algorithms and enforcers, (4)
implementation rules, possibly with condition code, (5) an ADT ``cost''
with functions for basic arithmetic and comparison, (6) an ADT ``logical
properties'', (7) an ADT ``physical property vector'' including comparisons
functions (equality and cover), (8) an applicability function for each
algorithm and enforcer, (9) a cost function for each algorithm and
enforcer, (10) a property function for each operator, algorithm, and
enforcer. This might seem to be a lot of code; however, all this
functionality is required to construct a database query optimizer with
or without an optimizer generator. Considering that query optimizers are
typically one of the most intricate modules of a database management
systems and that the optimizer generator prescribes a clean
modularization for these necessary optimizer components, the effort of
building a new database query optimizer using the Volcano optimizer
generator should be significantly less than designing and implementing a
new optimizer from scratch. This is particularly true since the
optimizer implementor using the Volcano optimizer generator does not
need to design and implement a new search algorithm.

\hypertarget{the-search-engine}{%
\section{The Search Engine}\label{the-search-engine}}

Since the general paradigm of database query optimization is to create
alternative (equivalent) query evaluation plans and then to choose among
the many possible plans, the search engine and its algorithm are central
components of any query optimizer. Instead of forcing each database and
optimizer implementor to implement an entirely new search engine and
algorithm, the Volcano optimizer generator provides a search engine to
be used in all created optimizers. This search engine is linked
automatically with the pattern matching and rule application code
generated from the data model description.

Since our experience with the EXODUS optimizer generator indicated that
it is easy to waste a lot of search effort in extensible query
optimization, we designed the search algorithm for the Volcano optimizer
generator to use dynamic programming and to be very goal-oriented, i.e.,
driven by needs rather than by possibilities.

Dynamic programming has been used
before in database query optimization, in particular in the System R
optimizer {[}15{]} and in Starburst's cost-based optimizer {[}8, 10{]},
but only for relational select-project-join queries. The search strategy
designed with the Volcano optimizer generator extends dynamic
programming from relational join optimization to general algebraic query
and request optimization and combines it with a top-down, goal-oriented
control strategy for algebras in which the number of possible plans
exceeds practical limits of pre-computation. Our dynamic programming
approach derives equivalent expressions and plans only for those partial
queries that are considered as parts of larger subqueries (and the
entire query), not all equivalent expressions and plans that are
feasible or seem interesting by their sort order {[}15{]}. Thus, the
exploration and optimization of subqueries and their alternative plans
is tightly directed and very goal-oriented. In a way, while the search
engines of the EXODUS optimizer generator as well as of the System R and
Starburst relational systems use forward chaining (in the sense in which
this term is used in AI), the Volcano search algorithm uses backward
chaining, because it explores only those subqueries and plans that truly
participate in a larger expression. We call our search algorithms
\emph{directed dynamic programming}.

Dynamic programming is used in optimizers created with the Volcano
optimizer generator by retaining a large set of partial optimization
results and using these earlier results in later optimization decisions.
Currently, this set of partial optimization results is reinitialized for
each query being optimized. In other words, earlier partial optimization
results are used during the optimization of only a single query. We are
considering research into longer. lived partial results in the future.

Algebraic transformation systems always include the possibility of
deriving the same expression in several different ways. In order to
prevent redundant optimization effort by detecting redundant (i.e.,
multiple equivalent) derivations of the same logical expressions and
plans during optimization, expression and plans are captured in a hash
table of expressions and equivalence classes. An equivalence class
represents two collections, one of equivalent logical and one of
physical expressions (plans). The logical algebra expressions are used
for efficient and complete exploration of the search space, and plans
are used for a fast choice of a suitable input plan that satisfies
physical property requirements. For each combination of physical
properties for which an equivalence class has already been optimized,
e.g., unsorted, sorted on A, and sorted on B, the best plan found is
kept.

Figure 2 shows an outline of the search algorithm used by the Volcano
optimizer generator. The original invocation of the FindBestPlan
procedure indicates the logical expression passed to the optimizer as
the query to be optimized, physical properties as requested by the user
(for example, sort order as in the ORDER BY clause of SQL), and a cost
limit. This limit is typically infinity for a user query, but the user
interface may permit users to set their own limits to ``catch''
unreasonable queries, e.g., ones using a Cartesian product due to a
missing join predicate.

\begin{figure}
  \centering
  \begin{verbatim}
FindBestPlan (LogExpr, PhysProp, Limit)
if the pair LogExpr and PhysProp is in the look-up table
  if the cost in the look-up table < Limit
    return Plan and Cost
  else
    return failure
/* else: optimization required */
create the set of possible "moves" from
  applicable transformations
  algorithms that give the required PhysProp
  enforcers for required PhysProp
order the set of moves by promise
for the most promising moves
  if the move uses a transformation
    apply the transformation creating NewLogExpr
    call FindBestPlan (NewLogExpr, PhysProp, Limit)
  else if the move uses an algorithm
    TotalCost := cost of the algorithm
    for each input I while TotalCost < Limit
      determine required physical properties PP for I
      Cost = FindBestPlan (I, PP, Limit - TotalCost)
      add Cost to TotalCost
  else/* move uses an enforcer */
    TotalCost := cost of the enforcer
    modify PhysProp for enforced property
    call FindBestPlan for LogExpr with new PhysProp
/* maintain the look-up table of explored facts */
if LogExpr is not in the look-up table
  insert LogExpr into the look-up table
insert PhysProp and best plan found into look-up table
return best Plan and Cost
  \end{verbatim}
  \caption{Outline of the Search Algorithm}
\end{figure}

The FindBestPlan procedure is broken into two parts. First, if a plan
for the expression satisfying the physical property vector can be found
in the hash table, either the plan and its cost or a failure indication
are returned depending on whether or not the found plan satisfies the
given cost limit. If the expression cannot be found in the hash table,
or if the expression has been optimized before but not for the presently
required physical properties, actual optimization is begun.

There are three sets of possible ``moves'' the optimizer can explore at
any point. First, the expression can be transformed using a
transformation rule. Second, there might be some algorithms that can
deliver the logical expression with the desired physical properties,
e.g., hybrid hash join for unsorted output and merge-join for join
output sorted on the join attribute. Third, an enforcer might be useful
to permit additional algorithm choices, e.g., a sort operator to permit
using hybrid hash join even if the final output is to be sorted.

After all possible moves have been generated and assessed, the most
promising moves are pursued. Currently, with only exhaustive search
implemented, all moves are pursued. In the future, a subset of the moves
will be selected, determined and ordered by another function provided by
the optimizer implementor. Pursuing all moves or only a selected few is
a major heuristic placed into the hands of the optimizer implementor. In
the extreme case, an optimizer implementor can choose to transform a
logical expression without any algorithm selection and cost analysis,
which covers the optimizations that in Starburst are separated into the
query rewrite level. The difference between Starburst's two-level and
Volcano's approach is that this separation is mandatory in Starburst
while Volcano will leave it as a choice to be made by the optimizer
implementor.

The cost limit is used to improve the search algorithm using
brunch-and-bound pruning. Once a complete plan is known for a logical
expression (the user query or some part of it) and a physical property
vector, no other plan or partial plan with higher cost can be part of
the optimal query evaluation plan. Therefore, it is important (for
optimization speed, not for correctness) that a relatively good plan be
found fast, even if the optimizer uses exhaustive search. Furthermore,
cost limits are passed down in the optimization of subexpressions, and
tight upper bounds also speed their optimization.

If a move to be pursued is a transformation, the new expression is
formed and optimized using FindBestPlan. In order to detect the case
that two (or more) rules are inverses of each other, the current
expression and physical property vector is marked as ``in progress''. If a
newly formed expression already exists in the hash table and is marked
as ``in progress'', it is ignored because its optimal plan will be
considered when it is finished.

Often a new equivalence class is created during a transformation.
Consider the associativity rule in Figure 3. The expressions rooted at A
and B are equivalent and therefore belong into the same class. However,
expression C is not equivalent to any expression in the left expression
and requires a new equivalence class. In this case, a new equivalence
class is created and optimized as required for cost analysis and
optimization of expression B.

\begin{figure}
  \centering
  \includegraphics[width=0.8\columnwidth]{fig-3.png}
  \caption{Associativity Rule}
\end{figure}

If a move to be pursued is the exploration of a normal query processing
algorithm such as merge-join, its cost is calculated by the algorithm's
cost function. The algorithm's applicability function determines the
physical property vectors for the algorithms inputs, and their costs and
optimal plans are found by invoking FindBestPlan for the inputs.

For some binary operators, the actual physical properties of the inputs
are not as important as the consistency of physical properties among the
inputs. For example, for a sort-based implementation of intersection,
i.e., an algorithm very similar to merge-join, any sort order of the two
inputs will suffice as long as the two inputs are sorted in the same
way. Similarly, for a parallel join, any partitioning of join inputs
across multiple processing nodes is acceptable if both inputs are
partitioned using compatible partitioning rules. For these cases, the
search engine permits the optimizer implementor to specify a number of
physical property vectors to be tried. For example, for the intersection
of two inputs R and S with attributes A, B, and C where R is sorted on
(A,B,C) and S is sorted on (B,A,C), both these sort orders can be
specified by the optimizer implementor and will be optimized by the
generated optimizer, while other possible sort orders, e.g., (C,B,A),
will be ignored.

If the move to be pursued is the use of an enforcer such as sort, its
cost is estimated by a cost function provided by the optimizer
implementor and the original logical expression is optimized using
FindBestPlan with a suitably modified (i.e., relaxed) physical property
vector. In many respects, enforcers are dealt with exactly like
algorithms, which is not surprising considering that both are operators
of the physical algebra During optimization with the modified physical
property vector, algorithms that already applied before relaxing the
physical properties must not be explored again. For example, if a join
result is required sorted on the join column, merge-join (an algorithm)
and sort (an enforcer) will apply. When optimizing the sort input, i.e.,
the join expression without the sort requirement, hybrid hash join
should apply but merge-join should not. To ensure this, FindBestPlan
uses an additional parameter, not shown in Figure 2, called the
excluding physical property vector that is used only when inputs to
enforcers are optimized. In the example, the excluding physical property
vector would contain the sort condition, and since merge-join is able to
satisfy the excluding properties, it would not be considered a suitable
algorithm for the sort input.

At the end of (or actually already during) the optimization procedure
FindBestPlan, newly derived interesting facts are captured in the hash
table. ``Interesting'' is defined with respect to possible future use,
which includes both plans optimal for given physical properties as well
as failures that can save future optimization effort for a logical
expression and a physical property vector with the same or even lower
cost limits.

In summary, the search algorithm employed by optimizers created with the
Volcano optimizer generator uses dynamic programming by storing all
optimal subplans as well as optimization failures until a query is
completely optimized. Without any a-priori assumptions about the
algebras themselves, it is designed to map an expressions over the
logical algebra into the optimal equivalent expressions over the
physical algebra. Since it is very goal-oriented through the use of
physical properties and derives only those expressions and plans that
truly participate in promising larger plans, the algorithm is more
efficient than previous approaches to using dynamic programming in
database query optimization.

\hypertarget{comparison-with-the-exodus-optimizer-generator}{%
\section{Comparison with the EXODUS Optimizer
Generator}\label{comparison-with-the-exodus-optimizer-generator}}

Since
the EXODUS optimizer generator was our first attempt to design and
implement an extensible query optimization system or tool, this section
compares the EXODUS and Volcano optimizer generators in some detail. The
EXODUS optimizer generator was successful to the extent that it defined
a general approach to the problem based on query algebras, the generator
paradigm (data model specification as input data), separation of logical
and physical algebras, separation of logical and physical properties,
extensive use of algebraic rules (transformation rules and
implementation rules), and its focus on software modularization {[}2,
3{]}. Considering the complexity of typical query optimization software
and the importance of well-defined modules to conquer the complexities
of software design and maintenance, the latter two points might well be
the most important contributions of the EXODUS optimizer generator
research.

The generator concept was very successful because the input data (data
model specification) could be turned into machine code; in particular,
all strings were translated into integers, which ensured very fast
pattern matching. However, the EXODUS optimizer generator's search
engine was far from optimal. First, the modifications required for
unforeseen algebras and their peculiarities made it a bad patchwork of
code. Second, the organization of the ``MESH'' data structure (which
held all logical and physical algebra expressions explored so far) was
extremely cumbersome, both in its time and space complexities. Third,
the almost random transformations of expressions in MESH resulted in
significant overhead in ``reanalyzing'' existing plans. In fact, for
larger queries, most of the time was spent reanalyzing existing plans.

The Volcano optimizer generator has solved these three problems, and
includes new functionality not found in the EXODUS optimizer generator.
We first summarize their differences in functionality and then present a
performance comparison for relational queries.

\hypertarget{functionality-and-extensibility}{%
\subsection{Functionality and
Extensibility}\label{functionality-and-extensibility}}

There are several important differences in the functionality and
extensibility of the EXODUS and Volcano optimizer generators. First,
Volcano makes a distinction between logical expressions and physical
expressions. In EXODUS, only one type of node existed in the hash table
called MESH, which contained both a logical operator such as join and a
physical algorithm such as hybrid hash join. To retain equivalent plans
using merge-join and hybrid hash join, the logical expression (or at
least one node) had to be kept twice, resulting in a large number of
nodes in MESH.

Second, physical properties were handled rather haphazardly in EXODUS.
If the algorithm with the lowest cost happened to deliver results with
useful physical properties, this was recorded in VIESH and used in
subsequent optimization decisions. Otherwise, the cost of enforcers (to
use a Volcano term) had to be included in the cost function of other
algorithms such as merge-join. In other words, the ability to specify
required physical properties and let these properties, together with the
logical expression, drive the optimization process was entirely absent
in EXODUS and has contributed significantly to the efficiency of the
Volcano optimizer generator search engine.

The concept of physical property is very powerful and extensible. The
most obvious and well-known candidate for a physical property in
database query processing is the sort order of intermediate results.
Other properties can be defined by the optimizer implementor at will.
Depending on the semantics of the data model, uniqueness might be a
physical property with two enforcers, sort- and hash-based. Location and
partitioning in parallel and distributed systems can be enforced with a
network and parallelism operator such as Volcano's \emph{exchange}
operator {[}4{]}. For query optimization in object-oriented systems, we
plan on defining ``assembledness'' of complex objects in memory as a
physical property and using the assembly operator described in {[}5{]}
as the enforcer for this property.

Third, the Volcano algorithm is driven top-down; subexpressions are
optimized only if warranted. In the extreme case, if the only move
pursued is a transformation, a logical expression is transformed on the
logical algebra level without optimizing its subexpressions and without
performing algorithm selection and cost analysis for the subexpressions.
In EXODUS, a transformation is always followed immediately by algorithm
selection and cost analysis. Moreover, transformations were explored
whether or not they were part of the currently most promising logical
expression and physical plan for the overall query. Worst of all for
optimizer performance, however, was the decision to perform
transformations with the highest expected cost improvement first. Since
the expected cost improvement was calculated as product of a factor
associated with the transformation rule and the current cost before
transformation, nodes at the top of the expression (with high total
cost) were preferred over lower expressions. When the lower expression
were finally transformed, all consumer nodes above (of which there were
many at this time) had to be reanalyzed creating an large
number of MESH nodes.

Fourth, cost is defined in much more general terms in Volcano than in
the EXODUS optimizer generator. In Volcano, cost is an abstract data
type for which all calculations and comparisons are performed by
invoking functions provided by the optimizer implementor. It can be a
simple number, e.g., estimated elapsed seconds, a structure, e.g., a
record consisting of CPU time and I/O count for a cost model similar to
the one in System R {[}15{]}, or even a function, e.g., of the amount of
available main memory.

Finally, we believe that the Volcano optimizer generator is more
extensible than the EXODUS prototype, in particular with respect to the
search strategy. The hash table that holds logical expressions and
physical plans and operations on this hash table are quite general, and
would support a variety of search strategies, not only the procedure
outlined in the previous section. We are still modifying (extending and
refining) the search strategy, and plan on modifying it further in
subsequent years and on using the Volcano optimizer generator for
further research.

\hypertarget{search-efficiency-and-effectiveness}{%
\subsection{Search Efficiency and
Effectiveness}\label{search-efficiency-and-effectiveness}}

In this section, we experimentally compare efficiency and effectiveness
of the mechanisms built into the EXODUS and Volcano search engines. The
example used for this comparison is a rather small ``data model''
consisting of relational select and join operators only; as we will see,
however, even this small data model and query language suffices to
demonstrate that the search strategy of the Volcano optimizer generator
is superior to the one designed for the earlier EXODUS prototype. The
effects exposed here would be even stronger for richer and more complex
data models, (logical) query algebras, and (physical) execution
algebras.

For the experiments, we specified the data model descriptions as
similarly as possible for the EXODUS and Volcano optimizer generators.
In particular, we specified the same operators (get, select, join) and
algorithms (file scan, filter for selections, sort, merge-join, hybrid
hash join), the same transformation and implementation rules, and the
same property and cost functions. Sorting was modeled as an enforcer in
Volcano while it was implicit in the cost function for merge-join in
EXODUS. The transformation rules permitted generating all plans
including bushy ones (composite inner inputs). The test relations
contained 1,200 to 7,200 records of 100 bytes. The cost functions
included both I/O and CPU costs. Hash join was presumed to proceed
without partition files, while sorting costs were calculated based on a
single-level merge.

As a first comparison between the two search engines, we performed
exhaustive optimizations of relational select-join queries. Figure 4
shows the average optimization effort and, to show the quality of the
optimizer output, the estimated execution time of produced plans for
queries with 1 to 7 binary joins, i.e., 2 to 8 input relations, and as
many selections as input relations. Solid lines indicate optimization
times on a Sun SparcStation-1 delivering about 12 MIPS. Dashed lines
indicate estimated plan execution times. Note that the y-axis are
logarithmic. Measurements from the EXODUS optimizer generator are marked
with \( \Box \)'s, Volcano measurements are marked with \( \ocircle \)'s.

For each complexity level, we generated and optimized 50 queries. For
some of the more complex queries, the EXODUS optimizer generator aborted
due to lack of memory or was aborted because it ran much longer than the
Volcano optimizer generator. Furthermore, we observed in repeated
experiments that the EXODUS optimizer generator measurements were quite
volatile. Similar problems were observed in EXODUS experiments reported
in {[}3{]}. The Volcano-generated optimizer performed exhaustive search
for all queries with less than 1 MB of work space. The data points in
Figure 4 represent only those queries for which the EXODUS optimizer
generator completed the optimization.

The search times reflect Volcano's more efficient search strategy,
visible in the large distance between the two solid lines. For the
EXODUS-generated optimizer, the search effort increases dramatically
from 3 to 4 input relations because reanalyzing becomes a substantial
part of the query optimization effort in EXODUS at this point. The
increase of Volcano's optimization costs is about exponential, shown in
an almost straight line, which mirrors exactly the increase in the
number of equivalent logical algebra expressions {[}13{]}. For more
complex queries, the EXODUS' and Volcano's optimization times differ by
about an order of magnitude.

The plan quality (shown by the estimated execution cost; dashed lines in
Figure 4) is equal for moderately complex queries (up to 4 input
relations). For more complex queries, however, the cost is significantly
higher for EXODUS-optimized plans, because the EXODUS generated
optimizer and its search engine do not systematically explore and
exploit physical properties and interesting orderings.

In summary, the Volcano optimizer generator is not only more extensible,
it is also much more efficient and effective than the earlier EXODUS
prototype. In the next section, we compare our work with other related
work.

\hypertarget{other-related-work}{%
\section{Other Related Work}\label{other-related-work}}

The query optimizer of the Starburst extensible-relational database
management system consists of two rule-based subsystems with nested
scopes. The two subsystems are connected by a common data structure,
which represents an entire query and is called query graph model (QGM).
The first subsystem, called query rewrite, merges nested subqueries and
bundles selection and join predicates for optimization in a second,
cost-based optimizer. Optimization during the query rewrite phase, i.e.,
nested SQL queries, union, outer join, grouping, and aggregation, is
based entirely on heuristics and is not cost-sensitive.
Select-project-join query components are covered by the second
optimizer\footnote{Actually, the cost-based optimizer covers all
  operators. However, its optimization and algorithm choices are very
  limited for all but the select-project-join blocks in a query}, also
called the cost-based optimizer, which performs rule-based expansion of
select-project-join queries from relational calculus into access plans
and compares the resulting plans by estimated execution costs {[}8,
10{]}. The cost-based optimizer performs exhaustive search within
certain structural boundaries. For example, it is possible to restrict
the search space to left-deep trees (no composite inner), to include all
bushy trees, or to set a parameter for exploration of some but not all
bushy trees. For moderately complex join queries, the exhaustive search
of Starburst's cost-based optimizer is very fast because of its use of
dynamic programming. Moreover, the cost-based optimizer considers
physical properties such as sort order and creates efficient access
plans that include ``glue'' operators to enforce physical properties.

As we see it, there are two fundamental problems in Starburst's approach
to extensible query optimization. First, the design of the cost-based
optimizer is focused on step-wise expansion of join expressions based on
grammar-like rules. me ``grammar'' depends on a hierarchy of intermediate
levels (similar to non-terminals in a parsing grammar), e.g.,
commutative join and noncommutative join, and the sets of rules and
intermediate levels are tailored specifically to relational join
optimization. The problem is that it is not obvious how the existing
rule set would interact with additional operators and expansion rules.
For example, which level of the hierarchy is the right place for a
multi-way join algorithm? What new intermediate levels (non-terminals)
must be defined for the expansion grammar? In order to integrate a new
operator into Starburst's cost-based optimizer, the database implementor
must design a number of new intermediate levels and their new grammar
rules. These rules may interact with existing ones, making any extension
of Starburst's cost-based optimizer a complex and tedious task.
Volcano's algebraic approach seems much more natural and easier to
understand. Most recent work in object-oriented query optimization and
some work on database programming languages has focused on algebras and
algebraic transformations, e.g. {[}9, 16-18{]} among many others.

Second, in order to avoid the problems associated with adding new
operators to the cost-based optimizer, new operators are integrated at
the query rewrite level. However, query optimization on the query
rewrite level is heuristic; in other words, it does not include cost
estimation. While heuristics are sufficient for some transformations,
e.g., rewriting nested SQL queries into join expressions, they are not
sufficient for the relational operators already in Starburst's query
rewrite level and certainly not for an extensible query optimization
system in which future algebra operators and their properties are yet
unknown. As an example for insufficient optimization capabilities for
existing Starburst operators, consider that optimizing the union or
intersection of N sets is very similar to optimizing a join of N
relations; however, while join optimization uses exhaustive search of
tree shapes and join orderings as well as selectivity and cost
estimation, union and intersection are optimized using query rewrite
heuristics and commutativity only. We believe that a single-level
approach, in which all algebraic equivalences and transformations are
specified in a single language and performed by a single optimizer
component, is much more conducive for future research and exploration of
database query algebras and their optimization. Note that the Volcano
optimizer generator will permit heuristic transformations by suitable
ranking and selection of ``moves''; however, it leaves the choice to the
database implementor when and how to use heuristics vs. cost-sensitive
optimization rather than making this choice a priori as in the Starburst
design.

Sciore and Sieg criticized earlier rule-based query optimizers and
concluded that modularity is a major requirement for extensible query
optimization systems, e.g., in the rule set and in the control
structures for rule application {[}14{]}. The different tasks of query
optimization, such as rule application and selectivity estimation,
should be encapsulated in separate and cooperating ``experts''. Mitchell
et al. recently proposed a very similar approach for query optimization
in object-oriented database systems {[}12{]}. While promising as a
conceptual approach, we feel that this separation can be sustained for
some aspects of query optimization (and have tried to do so in the
abstract data types for cost etc. in the Volcano optimizer generator),
but we have found it extremely hard to maintain encapsulation of all
desirably separate concerns in an actual implementation.

Kemper and Moerkotte designed a rule-based query optimizer for the
Generic Object Model {[}6{]}. The rules operate almost entirely on path
expressions (e.g., employee.department.floor) by extending and cutting
them to permit effective use of access support relations [7]. While
the use of rules makes the optimizer extensible, it is not clear to what
extent these techniques can be used for different data models and for
different execution engines.

\hypertarget{summary-and-conclusions}{%
\section{Summary and Conclusions}\label{summary-and-conclusions}}

Emerging database application domains demand not only high functionality
but also high performance. To satisfy these two requirements, the
Volcano project provides efficient, extensible tools for query and
request processing, particularly for object-oriented and scientific
database systems. We do not propose to reintroduce relational query
processing into next-generation database systems; instead, we work on a
new kind of query processing engine that is independent of any data
model. The basic assumption is that high-level query and request
languages are and will continue to be based on sets, other bulk types,
predicates, and operators. Therefore, operators consuming and producing
sets or sequences of items are the fundamental building blocks of
next-generation query and request processing systems. In other words, we
assume that some algebra of sets is the basis of query processing, and
our research tries to support any algebra of collections, including
heterogeneous collections and many-sorted algebras. Fortunately,
algebras and algebraic equivalence rules are a very suitable basis for
database query optimization. Moreover, sets (permitting definition and
exploitation of subsets) and operators with data passed (or pipelined)
between them are also the foundations of parallel algorithms for
database query processing. Thus, our fundamental assumption for query
processing in extensible database systems are compatible with
high-performance parallel processing.

One of the tools provided by the Volcano research is a new optimizer
generator, designed and implemented to further explore extensibility,
search algorithms, effectiveness (i.e., the quality of produced plans),
heuristics, and time and space efficiency in the search engine.
Extensibility was achieved by generating optimizer source code from data
model specifications and by encapsulating costs as well as logical and
physical properties into abstract data types. Effectiveness was achieved
by permitting exhaustive search, which will be pruned only at the
discretion of the optimizer implementor. Efficiency was achieved by
combining dynamic programming with directed search based on physical
properties, branch-and-bound pruning, and heuristic guidance into a new
search algorithm that we have called directed dynamic programming. A
preliminary performance comparison with the EXODUS optimizer generator
demonstrated that optimizers built with the Volcano optimizer generator
are much more efficient than those built with the EXODUS prototype. We
hope that the new Volcano optimizer generator will permit our own
research group as well as others to develop more rapidly new database
query optimizers for novel data models, query algebras, and database
management systems. The Volcano optimizer generator has been used to
develop optimizers for computations over scientific databases {[}20{]}
and for Texas Instruments' Open OODB project {[}l, 19{]}, which
introduces a new ``materialize'' or scope operator that captures the
semantics of path expressions in a logical algebra expression. Both of
these optimizers have recently become operational. Moreover, the Volcano
optimizer generator is currently being evaluated by several academic and
industrial researchers in three continents.

In addition to combining an efficient implementation of exhaustive
search based on dynamic programming (as also found in the cost-based
component of the Starburst's relational optimizer) with the generality
of the EXODUS optimizer generator and the more natural single-level
algebraic transformation approach, the Volcano optimizer generator has a
number of new features that enhance its value as a software development
and research tool beyond all earlier extensible query optimization
efforts.

First, the choice when and how to use heuristic transformations vs.
cost-sensitive optimization is not prescribed or ``wired in''. In EXODUS,
cost analysis was always performed after a transformation; in Starburst,
one level can only perform heuristic optimization while the other level
performs cost-sensitive exhaustive search. Thus, the Volcano optimizer
generator has removed the restrictions on the search strategy imposed by
the earlier extensible query optimizer designs.

Second, optimizers generated with the Volcano optimizer generator use
physical properties very efficiently to direct the search. Rather than
optimizing an expression first and then adding ``glue'' operators and
their cost to a plan (the Starburst approach), the Volcano optimizer
generator's search algorithm immediately considers which physical
properties are to be enforced and can be enforced by which enforcer
algorithms, and subtracts the cost of the enforcer algorithms from the
bound that is used for branch-and-bound pruning. Thus, the Volcano
optimizer generator promises to be even more efficient in its search and
pruning than the relational Starburst optimizer.

Third, for binary (ternary, etc.) operations that can benefit from
multiple, alternative combinations of physical properties, the
subexpressions can be optimized multiple times. For example, any sort
order can be exploited by an intersection algorithm based on merge-join
as long as the two inputs are sorted in the same way. Although the same
consideration applies to location and partitioning in parallel and
distributed relational query processing, no earlier query optimizer has
provided this feature. Fourth, the internal structure for equivalence
classes is sufficiently modular and extensible to support alternative
search strategies, far beyond the parameterization of rule condition
codes, which can be found to a roughly similar extent in Starburst and
EXODUS. We are exploring several directions with respect to the search
strategy, namely preoptimized subplans, learning of transformation
sequences, an alternative, even more parameterized search algorithm that
can be ``switched'' to different existing algorithms, and parallel search
(on shared-memory machines). Finally, the consistent separation of
logical and physical algebras makes specification as well as
modifications at either level particularly easy for the database and
optimizer implementor and makes the search engine very efficient. For
example, the introduction of a new, non-trivial algorithm such as a
multi-way join (rather than binary joins) requires one or two
implementation rules in Volcano, whereas the design of Starburst's
cost-based optimizer requires reconsideration of almost the entire rule
set. While the separation of logical and physical algebras was already
present in the EXODUS rule language, the Volcano design also exploits
this separation in the search engine, which makes extending the code
supplied by the optimizer implementor (which sometimes must inspect the
internal data structures, e.g., in rule condition code) significantly
easier to write, understand, and modify. In summary, the Volcano
optimizer generator is a much more extensible and useful research tool
than both the Starburst optimizer and the EXODUS optimizer generator.

\hypertarget{acknowledgements}{%
\section{Acknowledgements}\label{acknowledgements}}

Jim Martin, David Maier, and Guy Lohman have made valuable contributions
to this research. Jim Martin, Guy Lohman, Barb Peters, Rick Cole, Diane
Davison, and Richard Wolniewicz suggested numerous improvements to
drafts of this paper. We thank José A. Blakeley and •his colleagues at
Texas Instruments for using the Volcano Optimizer Generator in the Open
OODB project. --- This research was performed at the University of
Colorado at Boulder with partial support by NSF with awards IRI-8805200,
IRI-8912618, and IRI-9116547, DARPA with contract DAAB07-91-C-Q518, and
Texas Instruments.

\hypertarget{references}{%
\section{References}\label{references}}

\end{document}
