\SpecialChapter{References}

\label{bibliography_h0}





\noindent[\label{bibliography_g219}1] 
Michael Adams and R. Kent Dybvig.
Efficient nondestructive equality checking for trees and graphs.
In \textit{Proceedings of the 13th ACM SIGPLAN International
  Conference on Functional Programming},  179-188, September 2008.


\noindent[\label{bibliography_g220}2] 
J. Michael Ashley and R. Kent Dybvig.
An efficient implementation of multiple return values in Scheme.
In \textit{Proceedings of the 1994 ACM Conference on Lisp and
  Functional Programming},  140-149, June 1994.


\noindent[\label{bibliography_g221}3] 
Alan Bawden.
Quasiquotation in lisp.
In \textit{Partial Evaluation and Semantic-Based Program Manipulation},
   88-99, 1999.


\noindent[\label{bibliography_g222}4] 
William Briggs and Van Emden Henson.
\textit{The DFT: An Owner's Manual for the Discrete Fourier
  Transform}.
Society for Industrial and Applied Mathematics, Philadelphia, PA,
  1995.


\noindent[\label{bibliography_g223}5] 
Robert G. Burger and R. Kent Dybvig.
Printing floating-point numbers quickly and accurately.
In \textit{Proceedings of the ACM SIGPLAN '96 Conference on
  Programming Language Design and Implementation},  108-116, May 1996.


\noindent[\label{bibliography_g224}6] 
William F. Clocksin and Christopher S. Mellish.
\textit{Programming in Prolog}, second edition.
Springer-Verlag, Berlin, 1984.


\noindent[\label{bibliography_g225}7] 
Sam M. Daniel.
Efficient recursive FFT implementation in Prolog.
In \textit{Proceedings of the Second International Conference on the
  Practical Application of Prolog},  175-185, 1994.


\noindent[\label{bibliography_g226}8] 
Mark Davis.
Unicode Standard Annex \#{}29: Text boundaries, 2006.
 \textit{http://www.unicode.org/reports/tr29/}.
 
\noindent[\label{bibliography_g227}9] 
R. Kent Dybvig.
\textit{Chez Scheme User's Guide: Version 8}.
Cadence Research Systems, 2009.
 \textit{http://www.scheme.com/csug8/}.
 
\noindent[\label{bibliography_g228}10] 
R. Kent Dybvig and Robert Hieb.
Engines from continuations.
\textit{Computer Languages}, 14(2):109-123, 1989.


\noindent[\label{bibliography_g229}11] 
R. Kent Dybvig and Robert Hieb.
A new approach to procedures with variable arity.
\textit{Lisp and Symbolic Computation}, 3(3):229-244, September
  1990.


\noindent[\label{bibliography_g230}12] 
R. Kent Dybvig, Robert Hieb, and Carl Bruggeman.
Syntactic abstraction in Scheme.
\textit{Lisp and Symbolic Computation}, 5(4):295-326, 1993.


\noindent[\label{bibliography_g231}13] 
Daniel P. Friedman and Matthias Felleisen.
\textit{The Little Schemer}, fourth edition.
MIT Press, Cambridge, MA, 1996.


\noindent[\label{bibliography_g232}14] 
Daniel P. Friedman, Christopher T. Haynes, and Eugene E. Kohlbecker.
Programming with continuations.
In P. Pepper, editor, \textit{Program Transformation and Programming
  Environments},  263-274. Springer-Verlag, New York, 1984.


\noindent[\label{bibliography_g233}15] 
Christopher T. Haynes and Daniel P. Friedman.
Abstracting timed preemption with engines.
\textit{Computer Languages}, 12(2):109-121, 1987.


\noindent[\label{bibliography_g234}16] 
Christopher T. Haynes, Daniel P. Friedman, and Mitchell Wand.
Obtaining coroutines with continuations.
\textit{Computer Languages}, 11(3/4):143-153, 1986.


\noindent[\label{bibliography_g235}17] 
Robert Hieb, R. Kent Dybvig, and Carl Bruggeman.
Representing control in the presence of first-class continuations.
In \textit{Proceedings of the SIGPLAN '90 Conference on Programming
  Language Design and Implementation},  66-77, June 1990.


\noindent[\label{bibliography_g236}18] 
IEEE Computer Society.
\textit{IEEE Standard for the Scheme Programming Language}, May
  1991.
IEEE Std 1178-1990.


\noindent[\label{bibliography_g237}19] 
Brian W. Kernighan and Dennis M. Ritchie.
\textit{The C Programming Language}, second edition.
Prentice Hall, Englewood Cliffs, NJ, 1988.


\noindent[\label{bibliography_g238}20] 
P. Leach, M. Mealling, and R. Salz.
A Universally Unique IDentifier (UUID) URN namespace, July
  2005.
RFC 4122.
 \textit{http://www.ietf.org/rfc/rfc4122.txt}.

\noindent[\label{bibliography_g239}21] 
Peter Naur et al.
Revised report on the algorithmic language ALGOL 60.
\textit{Communications of the ACM}, 6(1):1-17, January 1963.


\noindent[\label{bibliography_g240}22] 
David A. Plaisted.
Constructs for sets, quantifiers, and rewrite rules in Lisp.
Technical Report UIUCDCS-R-84-1176, University of Illinois at
  Urbana-Champaign Department of Computer Science, June 1984.


\noindent[\label{bibliography_g241}23] 
J. A. Robinson.
A machine-oriented logic based on the resolution principle.
\textit{Journal of the ACM}, 12(1):23-41, 1965.


\noindent[\label{bibliography_g242}24] 
Michael Sperber, R. Kent Dybvig, Matthew Flatt, and Anton van Straaten (eds.).
Revised\textsuperscript{6} report on the algorithmic language Scheme, September
  2007.
 \textit{http://www.r6rs.org/}.
 
\noindent[\label{bibliography_g243}25] 
Michael Sperber, R. Kent Dybvig, Matthew Flatt, and Anton van Straaten (eds.).
Revised\textsuperscript{6} report on the algorithmic language
  Scheme---non-normative appendices, September 2007.
 \textit{http://www.r6rs.org/}.
 
\noindent[\label{bibliography_g244}26] 
Michael Sperber, R. Kent Dybvig, Matthew Flatt, and Anton van Straaten (eds.).
Revised\textsuperscript{6} report on the algorithmic language Scheme---standard
  libraries, September 2007.
 \textit{http://www.r6rs.org/}.
 
\noindent[\label{bibliography_g245}27] 
Guy L. Steele Jr.
\textit{Common Lisp, the Language}, second edition.
Digital Press, Bedford, Massachusetts, 1990.


\noindent[\label{bibliography_g246}28] 
Guy L. Steele Jr. and Gerald J. Sussman.
The revised report on Scheme, a dialect of Lisp.
MIT AI Memo 452, Massachusetts Institute of Technology, January 1978.


\noindent[\label{bibliography_g247}29] 
Gerald J. Sussman and Guy L. Steele Jr.
Scheme: An interpreter for extended lambda calculus.
\textit{Higher-Order and Symbolic Computation}, 11(4):405-439, 1998.
Reprinted from the AI Memo 349, MIT (1975), with a foreword.


\noindent[\label{bibliography_g248}30] 
The Unicode Consortium.
\textit{The Unicode Standard, Version 5.0}, fifth edition.
Addison-Wesley Professional, Boston, MA, 2006.


\noindent[\label{bibliography_g249}31] 
Oscar Waddell, Dipanwita Sarkar, and R. Kent Dybvig.
Fixing letrec: A faithful yet efficient implementation of Scheme's
  recursive binding construct.
\textit{Higher-Order and Symbolic Computation}, 18(3/4):299-326, 2005.


\noindent[\label{bibliography_g250}32] 
Mitchell Wand.
Continuation-based multiprocessing.
\textit{Higher-Order and Symbolic Computation}, 12(3):285-299, 1999.
Reprinted from the proceedings of the 1980 Lisp Conference, with a
  foreword.




