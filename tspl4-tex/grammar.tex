\SpecialChapter{Formal Syntax\label{grammar_APPENDIXFORMALSYNTAX}}

\label{grammar_h0}




\label{grammar_s0}The formal grammars and accompanying text appearing here describe the
written syntax of Scheme data values, or \textit{datums}.
The grammars also effectively cover the written syntax of Scheme syntactic
forms, since every Scheme syntactic form has a representation as a Scheme
datum.
In particular, parenthesized syntactic forms are written
as lists, and identifiers (e.g., keywords and variables) are written
as symbols.
The high-level structure of each syntactic form is described in detail by
the entries marked ``syntax'' in Chapters \ref{binding_g88}
through \hyperref[exceptions_g147]{11}, and the syntactic forms are summarized in
the Summary of Forms.


\label{grammar_s1}The written representation of a datum involves tokens, whitespace, and comments.
\textit{Tokens} are sequences of one or more characters representing atomic
datums or serving as punctuation marks.
The tokens that represent atomic datums are symbols, numbers, strings,
booleans, and characters, while the tokens
serving as punctuation marks are
open and close parentheses, open and close brackets, the open vector
parenthesis \texttt{\#{}(}, the open bytevector parenthesis \texttt{\#{}vu8(},
the dotted pair marker \texttt{.} (dot), the quotation marks
\texttt{'} and \texttt{`}, the
unquotation marks \texttt{,} and \texttt{,\@{}}, the
syntax quotation marks 
\texttt{\#{}'} and \texttt{\#{}`}, and the syntax
unquotation marks \texttt{\#{},} and \texttt{\#{},\@{}}.


\label{grammar_s2}\label{grammar_s3}\label{grammar_s4}\textit{Whitespace} consists of space, tab, newline, form-feed,
carriage-return, and next-line characters along with any additional
characters categorized as Zs, Zl, or Zp by the Unicode
standard [\hyperref[bibliography_g248]{30}].
A newline character is also called a linefeed character.
Some whitespace characters or character sequences serve as \textit{line endings},
which are recognized as part of the syntax of line comments and strings.
A line ending is a newline character, a next-line character,
a line-separator character, a carriage-return character followed by a
newline character, a carriage return followed by a next-line
character, or a carriage return not followed by a newline
or next-line character.
A different set of whitspace characters serve as \textit{intraline whitespace},
which are recognized as part of the syntax of strings.
Intraline whitespace includes spaces, tabs, and any additional Unicode
characters whose general category is Zs.
The sets of intraline whitespace characters and line endings are disjoint,
and there are other whitespace characters, such as form feed, that are
not in either set.


\label{grammar_s5}\label{grammar_s6}\label{grammar_s7}\label{grammar_s8}\label{grammar_s9}\label{grammar_s10}\label{grammar_s11}\textit{Comments} come in three flavors:  line comments, block comments, and datum
comments.
A line comment consists of a semicolon ( \texttt{;} )
followed by any number of characters up to the next line ending or
end of input.
A block comment consists of a \texttt{\#{}\textbar{}} prefix, any number of
characters and nested block comments, and a \texttt{\textbar{}\#{}}
suffix.
A datum comment consists of a \texttt{\#{};} prefix followed by
any datum.


Symbols, numbers, characters, booleans, and the dotted pair
marker ( \texttt{.} ) must be delimited by the end the
input, whitespace, the start of a comment, an open or close
parenthesis, an open or close bracket, a string quote ( \texttt{"} ), or
a hash mark ( \texttt{\#{}} ).
Any token may be preceded or followed by any number of whitespace
characters and comments.


Case is significant in the syntax of characters, strings, and symbols
except within a hex scalar value, where the
hexadecimal digits ``a'' through ``f'' may be written in either
upper or lower case.
(Hex scalar values are hexadecimal numbers denoting Unicode scalar values.)
Case is insignificant in the syntax of booleans and numbers.
For example, \texttt{Hello} is distinct from
\texttt{hello}, \texttt{\#{}\textbackslash{}A} is distinct from \texttt{\#{}\textbackslash{}a}, and
\texttt{"String"} is distinct from \texttt{"string"}, while
\texttt{\#{}T} is equivalent to \texttt{\#{}t}, \texttt{\#{}E1E3} is equivalent to
\texttt{\#{}e1e3}, \texttt{\#{}X2aBc} is equivalent to \texttt{\#{}x2abc},
and \texttt{\#{}\textbackslash{}x3BA} is equivalent to \texttt{\#{}\textbackslash{}x3ba}.


A conforming implementation of the Revised\textsuperscript{6} Report is not permitted to
extend the syntax of datums, with one exception:  it is permitted to
recognize any token starting with the prefix \texttt{\#{}!} as a flag
indicating certain extensions are valid in the text following the flag.
So, for example, an implementation might recognize the flag
\texttt{\#{}!braces} and switch to a mode in which lists may be enclosed in
braces as well as in parentheses and brackets.


\texttt{\#{}!braces '\{a b c\} \(\Rightarrow\) (a b c)}

\label{grammar_s12}The flag \texttt{\#{}!r6rs} may be used to declare that the subsequent text is
written in R6RS syntax.
It is good practice to include \texttt{\#{}!r6rs} at the start of any file
containing a portable library or top-level program to specify that R6RS
syntax is being used, in the event that future reports extend the syntax
in ways that are incompatible with the text of the library or program.
\texttt{\#{}!r6rs} is otherwise treated as a comment.


In the grammars appearing below, \textless{}empty\textgreater{} stands for an
empty sequence of characters.
An item followed by an asterisk ( * ) represents zero or more
occurrences of the item, and an item followed by a raised plus sign
( \textsuperscript{+} ) represents one or more occurrences.
Spacing between items within a production appears for readability only and
should be treated as if it were not present.


\textbf{Datums\label{grammar_grammar_datums}.}  \label{grammar_s13}A datum is a boolean, character, symbol, string, number, list, vector,
or bytevector.


  
  
  {\footnotesize
\begin{tabular}[H]{lcl}

\textless{}datum\textgreater{} & \(\longrightarrow\) & \textless{}boolean\textgreater{} \\

     & \textbar{} & \textless{}character\textgreater{} \\

     & \textbar{} & \textless{}symbol\textgreater{} \\

     & \textbar{} & \textless{}string\textgreater{} \\

     & \textbar{} & \textless{}number\textgreater{} \\

     & \textbar{} & \textless{}list\textgreater{} \\

     & \textbar{} & \textless{}vector\textgreater{} \\

     & \textbar{} & \textless{}bytevector\textgreater{}
 \\
\end{tabular}
}


Lists, vectors, and bytevectors are compound datums formed from groups of
tokens possibly separated by whitespace and comments.
The others are single tokens.


\textbf{Booleans\label{grammar_grammar_booleans}.}  \label{grammar_s14}Boolean false is written \texttt{\#{}f}.
While all other values count as true, the canonical true value (and only
other value to be considered a boolean value by the \texttt{boolean?}
predicate) is written \texttt{\#{}t}.


  
  
  {\footnotesize
\begin{tabular}[H]{lcl}

\textless{}boolean\textgreater{} & \(\longrightarrow\) & \texttt{\#{}t} \textbar{} \texttt{\#{}f}
 \\
\end{tabular}
}


Case is not significant in the syntax of booleans, so these may also be written
as \texttt{\#{}T} and \texttt{\#{}F}.


\textbf{Characters\label{grammar_grammar_characters}.}  \label{grammar_s15}A character object is written as the prefix \texttt{\#{}\textbackslash{}} followed by a single
character, a character name, or a sequence of characters specifying a
Unicode scalar value.


  
  
  {\footnotesize
\begin{tabular}[H]{lcl}

\textless{}character\textgreater{} & \(\longrightarrow\) & \texttt{\#{}\textbackslash{}} \textless{}any character\textgreater{} \textbar{} \texttt{\#{}\textbackslash{}} \textless{}character name\textgreater{}
                    \textbar{} \texttt{\#{}\textbackslash{}x} \textless{}hex scalar value\textgreater{} \\

\textless{}character name\textgreater{} & \(\longrightarrow\) & \texttt{alarm} \textbar{} \texttt{backspace} \textbar{} \texttt{delete} \textbar{} \texttt{esc} \textbar{}\texttt{linefeed} \\

   & \textbar{} & \texttt{newline} \textbar{} \texttt{page} \textbar{} \texttt{return} \textbar{} \texttt{space} \textbar{} \texttt{tab} \textbar{} \texttt{vtab} \\

\textless{}hex scalar value\textgreater{} & \(\longrightarrow\) & \textless{}digit 16\textgreater{}\textsuperscript{+} \\
\end{tabular}
}


The named characters correspond to the Unicode characters
alarm (Unicode scalar value 7, i.e., U+0007),
backspace (U+0008),
delete (U+007F),
esc (U+001B),
linefeed (U+000A; same as newline),
newline (U+000A),
page (U+000C),
return (U+000D),
space (U+0020),
tab (U+0009) and 
vertical tab (U+000B).


A hex scalar value represents a Unicode scalar value \textit{n},
\(0 \leq n \leq \mathrm{D800_16}\) or
\(\mathrm{E000_16} \leq n \leq \mathrm{10FFF_16}\).
The \textless{}digit 16\textgreater{} nonterminal is defined under \textbf{Numbers} below.


A \texttt{\#{}\textbackslash{}} prefix followed by a character name is always interpreted as
a named character, e.g., \texttt{\#{}\textbackslash{}page} is treated as \texttt{\#{}\textbackslash{}page}
rather than \texttt{\#{}\textbackslash{}p} followed by the symbol \texttt{age}.
Characters must also be delimited, as described above, so that \texttt{\#{}\textbackslash{}pager}
is treated as a syntax error rather than as the character \texttt{\#{}\textbackslash{}p} followed
by the symbol \texttt{ager} or the character \texttt{\#{}\textbackslash{}page} followed by the
symbol \texttt{r}.


Case is significant in the syntax of character objects, except within a hex scalar value.


\textbf{Strings\label{grammar_grammar_strings}.}  \label{grammar_s16}A string is written as a sequence of string elements enclosed in string quotes
(double quotes).
Any character other than a string quote or backslash can appear as a
string element.
A string element can also consist of a backslash followed by a single
character, a backslash followed by sequence of characters specifying a
Unicode scalar value, or a backslash followed by sequence of intraline
whitespace characters that includes a single line ending.


  
  
  {\footnotesize
\begin{tabular}[H]{lcl}

\textless{}string\textgreater{} & \(\longrightarrow\) & \texttt{"} \textless{}string character\textgreater{}* \texttt{"} \\

\textless{}string element\textgreater{} & \(\longrightarrow\) & \textless{}any character except \texttt{"} or \texttt{\textbackslash{}}\textgreater{} \\

   & \textbar{} & \texttt{\textbackslash{}"} \textbar{} \texttt{\textbackslash{}\textbackslash{}} \textbar{} \texttt{\textbackslash{}a} \textbar{} \texttt{\textbackslash{}b}
        \textbar{} \texttt{\textbackslash{}f} \textbar{} \texttt{\textbackslash{}n} \textbar{} \texttt{\textbackslash{}r} \textbar{} \texttt{\textbackslash{}t} \textbar{} \texttt{\textbackslash{}v} \\

   & \textbar{} & \texttt{\textbackslash{}x} \textless{}hex scalar value\textgreater{} \texttt{;} \\

   & \textbar{} & \texttt{\textbackslash{}} \textless{}intraline whitespace\textgreater{}* \textless{}line ending\textgreater{} \textless{}intraline whitespace\textgreater{}* \\
\end{tabular}
}


A string element consisting of a single character represents that
character, except that any single character or pair of characters
representing a line ending represents a single newline character.
A backslash followed by a double quote represents a double quote, while
a backslash followed by a backslash represents a backslash.
A backslash followed by
\texttt{a} represents the alarm character (U+0007);
by \texttt{b}, backspace (U+0008);
by \texttt{f}, form feed (U+000C);
by \texttt{n}, newline (U+000A);
by \texttt{r}, carriage return (U+000D);
by \texttt{t}, tab (U+0009); and
by \texttt{v}, vertical tab (U+000B).
A backslash followed by \texttt{x}, a hex scalar value, and a semi-colon
( \texttt{;} ) represents the Unicode character specified by the scalar value.
The \textless{}hex scalar value\textgreater{} nonterminal is defined under \textbf{Characters} above.
Finally, a sequence of characters consisting of a backslash followed by
intraline whitespace that includes a single line ending represents no characters.


Case is significant in the syntax of strings, except within a hex scalar value.


\textbf{Symbols\label{grammar_grammar_symbols}.}  \label{grammar_s17}A symbol is written either as an ``initial'' character followed by a sequence of
``subsequent'' characters or as a ``peculiar symbol.''
Initial characters are letters, certain special characters, an additional
set of Unicode characters, or arbitrary characters specified by Unicode
scalar values.
Subsequent characters are initial characters, digits, certain additional
special characters, and a set of additional Unicode characters.
The peculiar symbols are \texttt{+}, \texttt{-}, \texttt{...}, and any
sequence of subsequent characters prefixed by \texttt{-\textgreater{}}.


  
  
  {\footnotesize
\begin{tabular}[H]{lcl}

\textless{}symbol\textgreater{} & \(\longrightarrow\) & \textless{}initial\textgreater{} \textless{}subsequent\textgreater{}* \\

\textless{}initial\textgreater{} & \(\longrightarrow\) & \textless{}letter\textgreater{} \textbar{} \texttt{!} \textbar{} \texttt{\${}} \textbar{} \texttt{\%{}}
    \textbar{} \texttt{\&{}} \textbar{} \texttt{*} \textbar{} \texttt{/} \textbar{} \texttt{:}
    \textbar{} \texttt{\textless{}} \textbar{} \texttt{=} \textbar{} \texttt{\textgreater{}} \textbar{} \texttt{?}
    \textbar{} \texttt{\~{}} \textbar{} \texttt{\_{}} \textbar{} \texttt{\^{}} \\

   & \textbar{} & \textless{}Unicode Lu, Ll, Lt, Lm, Lo, Mn, Nl, No, Pd, Pc, Po, Sc, Sm, Sk, So, or Co\textgreater{} \\

   & \textbar{} & \texttt{\textbackslash{}x} \textless{}hex scalar value\textgreater{} \texttt{;} \\

\textless{}subsequent\textgreater{} & \(\longrightarrow\) & \textless{}initial\textgreater{} \textbar{} \textless{}digit 10\textgreater{} \textbar{} \texttt{.}
    \textbar{} \texttt{+} \textbar{} \texttt{-} \textbar{} \texttt{\@{}} \textbar{} \textless{}Unicode Nd, Mc, or Me\textgreater{} \\

\textless{}letter\textgreater{} & \(\longrightarrow\) & \texttt{a} \textbar{} \texttt{b} \textbar{} ... \textbar{} \texttt{z}
              \textbar{} \texttt{A} \textbar{} \texttt{B} \textbar{} ... \textbar{} \texttt{Z}
 \\
\end{tabular}
}


\textless{}Unicode Lu, Ll, Lt, Lm, Lo, Mn, Nl, No, Pd, Pc, Po, Sc, Sm, Sk, So,
or Co\textgreater{} represents any character whose Unicode scalar value is greater than
127 and whose Unicode category is one of the listed categories.
\textless{}Unicode Nd, Mc, or Me\textgreater{} represents any character whose Unicode
category is one of the listed categories.
The \textless{}hex scalar value\textgreater{} nonterminal is defined under \textbf{Characters} above,
and \textless{}digit 10\textgreater{} is defined under \textbf{Numbers} below.


Case is significant in symbols.


\textbf{Numbers\label{grammar_grammar_numbers}.}  \label{grammar_s18}Numbers can appear in one of four radices:
2, 8, 10, and 16, with 10 the default.
Several of the productions below are parameterized by the radix,
\texttt{\textit{r}}, and each represents four productions, one for each of the
four possible radices.
Numbers that contain radix points or exponents are constrained to
appear in radix
10, so \textless{}decimal \texttt{\textit{r}}\textgreater{} is valid only when \texttt{\textit{r}} is 10.


  
  
  {\footnotesize
\begin{tabular}[H]{lcl}

\textless{}number\textgreater{} & \(\longrightarrow\) & \textless{}num 2\textgreater{} \textbar{} \textless{}num 8\textgreater{} \textbar{} \textless{}num 10\textgreater{} \textbar{} \textless{}num 16\textgreater{} \\

\textless{}num \texttt{\textit{r}}\textgreater{} & \(\longrightarrow\) & \textless{}prefix \texttt{\textit{r}}\textgreater{} \textless{}complex \texttt{\textit{r}}\textgreater{} \\

\textless{}prefix \texttt{\textit{r}}\textgreater{} & \(\longrightarrow\) & \textless{}radix \texttt{\textit{r}}\textgreater{} \textless{}exactness\textgreater{} \textbar{} \textless{}exactness\textgreater{} \textless{}radix \texttt{\textit{r}}\textgreater{} \\

\textless{}radix 2\textgreater{} & \(\longrightarrow\) & \texttt{\#{}b} \\

\textless{}radix 8\textgreater{} & \(\longrightarrow\) & \texttt{\#{}o} \\

\textless{}radix 10\textgreater{} & \(\longrightarrow\) & \textless{}empty\textgreater{} \textbar{} \texttt{\#{}d} \\

\textless{}radix 16\textgreater{} & \(\longrightarrow\) & \texttt{\#{}x} \\

\textless{}exactness\textgreater{} & \(\longrightarrow\) & \textless{}empty\textgreater{} \textbar{} \texttt{\#{}i} \textbar{} \texttt{\#{}e} \\

\textless{}complex \texttt{\textit{r}}\textgreater{} & \(\longrightarrow\) & \textless{}real \texttt{\textit{r}}\textgreater{} \textbar{} \textless{}real \texttt{\textit{r}}\textgreater{} \@{} \textless{}real \texttt{\textit{r}}\textgreater{} \\

  & \textbar{} & \textless{}real \texttt{\textit{r}}\textgreater{} \texttt{+} \textless{}imag \texttt{\textit{r}}\textgreater{} \textbar{} \textless{}real \texttt{\textit{r}}\textgreater{} \texttt{-} \textless{}imag \texttt{\textit{r}}\textgreater{} \\

  & \textbar{} & \texttt{+} \textless{}imag \texttt{\textit{r}}\textgreater{} \textbar{} \texttt{-} \textless{}imag \texttt{\textit{r}}\textgreater{} \\

\textless{}real \texttt{\textit{r}}\textgreater{} & \(\longrightarrow\) & \textless{}sign\textgreater{} \textless{}ureal \texttt{\textit{r}}\textgreater{} \textbar{} \texttt{+nan.0} \textbar{} \texttt{-nan.0} \textbar{} \texttt{+inf.0} \textbar{} \texttt{-inf.0} \\

\textless{}imag \texttt{\textit{r}}\textgreater{} & \(\longrightarrow\) & \texttt{i} \textbar{} \textless{}ureal \texttt{\textit{r}}\textgreater{} \texttt{i} \textbar{} \texttt{inf.0} \texttt{i} \textbar{} \texttt{nan.0} \texttt{i} \\

\textless{}ureal \texttt{\textit{r}}\textgreater{} & \(\longrightarrow\) & \textless{}uinteger \texttt{\textit{r}}\textgreater{} \textbar{} \textless{}uinteger \texttt{\textit{r}}\textgreater{} \texttt{/} \textless{}uinteger \texttt{\textit{r}}\textgreater{} \textbar{} \textless{}decimal \texttt{\textit{r}}\textgreater{} \textless{}suffix\textgreater{} \\

\textless{}uinteger \texttt{\textit{r}}\textgreater{} & \(\longrightarrow\) & \textless{}digit \texttt{\textit{r}}\textgreater{}\textsuperscript{+} \\

\textless{}decimal 10\textgreater{} & \(\longrightarrow\) & \textless{}uinteger 10\textgreater{} \textless{}suffix\textgreater{} \\

  & \textbar{} & \texttt{.} \textless{}digit 10\textgreater{}\textsuperscript{+} \textless{}suffix\textgreater{} \\

  & \textbar{} & \textless{}digit 10\textgreater{}\textsuperscript{+} \texttt{.} \textless{}digit 10\textgreater{}* \textless{}suffix\textgreater{} \\

\textless{}suffix\textgreater{} & \(\longrightarrow\) & \textless{}exponent\textgreater{} \textless{}mantissa width\textgreater{} \\

\textless{}exponent\textgreater{} & \(\longrightarrow\) & \textless{}empty\textgreater{} \textbar{} \textless{}exponent marker\textgreater{} \textless{}sign\textgreater{} \textless{}digit 10\textgreater{}\textsuperscript{+} \\

\textless{}exponent marker\textgreater{} & \(\longrightarrow\) & \texttt{e} \textbar{} \texttt{s} \textbar{} \texttt{f}
                 \textbar{} \texttt{d} \textbar{} \texttt{l} \\

\textless{}mantissa width\textgreater{} & \(\longrightarrow\) & \textless{}empty\textgreater{} \textbar{} \texttt{\textbar{}} \textless{}digit 10\textgreater{}\textsuperscript{+} \\

\textless{}sign\textgreater{} & \(\longrightarrow\) & \textless{}empty\textgreater{} \textbar{} \texttt{+} \textbar{} \texttt{-} \\

\textless{}digit 2\textgreater{} & \(\longrightarrow\) & \texttt{0} \textbar{} \texttt{1} \\

\textless{}digit 8\textgreater{} & \(\longrightarrow\) & \texttt{0} \textbar{} \texttt{1} \textbar{} \texttt{2} \textbar{} \texttt{3} \textbar{} \texttt{4}
                     \textbar{} \texttt{5} \textbar{} \texttt{6} \textbar{} \texttt{7} \\

\textless{}digit 10\textgreater{} & \(\longrightarrow\) & \texttt{0} \textbar{} \texttt{1} \textbar{} \texttt{2} \textbar{} \texttt{3} \textbar{} \texttt{4}
                     \textbar{} \texttt{5} \textbar{} \texttt{6} \textbar{} \texttt{7} \textbar{} \texttt{8} \textbar{} \texttt{9} \\

\textless{}digit 16\textgreater{} & \(\longrightarrow\) & \textless{}digit 10\textgreater{} \textbar{} \texttt{a} \textbar{} \texttt{b}
               \textbar{} \texttt{c} \textbar{} \texttt{d} \textbar{} \texttt{e}
               \textbar{} \texttt{f}
 \\
\end{tabular}
}


A number written as above is inexact if it is prefixed by \texttt{\#{}i}
or if it is not prefixed by \texttt{\#{}e} and contains a decimal point,
nonempty exponent, or nonempty mantissa width.
Otherwise, it is exact.


Case is not significant in the syntax of numbers.


\textbf{Lists\label{grammar_grammar_lists}.}  \label{grammar_s19}\label{grammar_s20}\label{grammar_s21}Lists are compound datums formed from groups of tokens and possibly
involving other datums, including other lists.
Lists are written as a sequence of datums within parentheses
or brackets; as a nonempty sequence of datums,
dotted-pair marker ( . ), and single datum enclosed within
parentheses or brackets; or as an abbreviation.


  
  
  {\footnotesize
\begin{tabular}[H]{lcl}

\textless{}list\textgreater{} & \(\longrightarrow\) & \texttt{(}\textless{}datum\textgreater{}*\texttt{)} \textbar{} \texttt{[}\textless{}datum\textgreater{}*\texttt{]} \\

   & \textbar{} & \texttt{(}\textless{}datum\textgreater{}\textsuperscript{+} \texttt{.} \textless{}datum\textgreater{}\texttt{)} \textbar{} \texttt{[}\textless{}datum\textgreater{}\textsuperscript{+} \texttt{.} \textless{}datum\textgreater{}\texttt{]} \\

   & \textbar{} & \textless{}abbreviation\textgreater{} \\

\textless{}abbreviation\textgreater{} & \(\longrightarrow\) & \texttt{'} \textless{}datum\textgreater{} \textbar{} \texttt{`} \textless{}datum\textgreater{} 
                       \textbar{} \texttt{,} \textless{}datum\textgreater{} \textbar{} \texttt{,\@{}} \textless{}datum\textgreater{} \\

 & \textbar{} & \texttt{\#{}'} \textless{}datum\textgreater{} \textbar{} \texttt{\#{}`} \textless{}datum\textgreater{} 
                       \textbar{} \texttt{\#{},} \textless{}datum\textgreater{} \textbar{} \texttt{\#{},\@{}} \textless{}datum\textgreater{}
 \\
\end{tabular}
}


If no dotted-pair marker appears in a list enclosed in parentheses or brackets,
it is a proper list, and the datums are the elements of the list, in the order
given.
If a dotted-pair marker appears, the initial elements of the list are those
before the marker, and the datum that follows the marker is the tail of the
list.
The dotted-pair marker is typically used only when the datum that follows
the marker is not itself a list.
While any proper list may be written without a dotted-pair marker, a
proper list can be written in dotted-pair notation by placing a list after
the dotted-pair marker.


The abbreviations are equivalent to the corresponding two-element lists
shown below. 
Once an abbreviation has been read, the result is indistinguishable from
its nonabbreviated form.


\begin{alltt}
'\textless{}datum\textgreater{} \(\rightarrow\) (quote \textless{}datum\textgreater{})
`\textless{}datum\textgreater{} \(\rightarrow\) (quasiquote \textless{}datum\textgreater{})
,\textless{}datum\textgreater{} \(\rightarrow\) (unquote \textless{}datum\textgreater{})
,\@{}\textless{}datum\textgreater{} \(\rightarrow\) (unquote-splicing \textless{}datum\textgreater{})
\#{}'\textless{}datum\textgreater{} \(\rightarrow\) (syntax \textless{}datum\textgreater{})
\#{}`\textless{}datum\textgreater{} \(\rightarrow\) (quasisyntax \textless{}datum\textgreater{})
\#{},\textless{}datum\textgreater{} \(\rightarrow\) (unsyntax \textless{}datum\textgreater{})
\#{},\@{}\textless{}datum\textgreater{} \(\rightarrow\) (unsyntax-splicing \textless{}datum\textgreater{})
\end{alltt}


\textbf{Vectors\label{grammar_grammar_vectors}.}  \label{grammar_s22}Vectors are compound datums formed from groups of tokens and possibly
involving other datums, including other vectors.
A vector is written as an open vector parenthesis followed by a
sequence of datums and a close parenthesis.


  
  
  {\footnotesize
\begin{tabular}[H]{lcl}

\textless{}vector\textgreater{} & \(\longrightarrow\) & \texttt{\#{}(}\textless{}datum\textgreater{}*\texttt{)}
 \\
\end{tabular}
}


\textbf{Bytevectors\label{grammar_grammar_bytevectors}.}  \label{grammar_s23}Bytevectors are compound datums formed from groups of tokens, but the
syntax does not permit them to contain arbitrary nested datums.
A bytevector is written as an open bytevector parenthesis followed by a
sequence of octets (unsigned 8-bit exact integers) and a close
parenthesis.


  
  
  {\footnotesize
\begin{tabular}[H]{lcl}

\textless{}bytevector\textgreater{} & \(\longrightarrow\) & \texttt{\#{}vu8(}\textless{}octet\textgreater{}*\texttt{)} \\

\textless{}octet\textgreater{} & \(\longrightarrow\) & \textless{}any \textless{}number\textgreater{} representing an exact integer \texttt{\textit{n}}, 0 ≤ \textit{n} ≤ 255\textgreater{}
 \\
\end{tabular}
}



