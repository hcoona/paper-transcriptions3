\SpecialChapter{Preface}

\label{preface_h0}




Scheme was introduced in 1975 by Gerald J. Sussman
and
Guy L. Steele Jr. [\hyperref[bibliography_g246]{28},\hyperref[bibliography_g247]{29}],
and was the
first dialect of \label{preface_s0}Lisp to fully support lexical scoping, first-class
procedures, and continuations.
In its earliest form it was a small language intended primarily
for research and teaching, supporting only a handful of predefined
syntactic forms and procedures.
Scheme is now a complete general-purpose programming language,
though it still derives its power from a small set of key concepts.
Early implementations of the language were interpreter-based
and slow, but some current Scheme implementations boast sophisticated
compilers that generate code on par with code generated by the best
optimizing compilers for lower-level languages such as C and Fortran.


This book is intended to provide an introduction to the Scheme programming language
but not an introduction to programming in general.
The reader is expected to have had some experience programming and
to be familiar with terms commonly associated with computers and
programming languages.
Readers unfamiliar with Scheme or Lisp should
also consider reading \textit{The Little Schemer} [\hyperref[bibliography_g231]{13}] to become familiar with
the concepts of list processing and recursion.
Readers new to programming should begin with an introductory text on
programming.


Scheme has been standardized both formally and informally.
The \label{preface_s1}\textit{IEEE Standard for the Scheme Programming
Language} [\hyperref[bibliography_g236]{18}], describes a formal
ANSI/IEEE Standard for Scheme but dates back to 1991.
A related series of reports, the "\label{preface_s2}Revised Reports on the
Algorithmic Language Scheme," document an evolving informal standard that
most implementations support.
The current report in this series is the "Revised\textsuperscript{6} Report on the
Algorithmic Language Scheme" [\hyperref[bibliography_g242]{24}], which was completed in 2007.


This book covers the language of the Revised\textsuperscript{6} Report.
It is not intended to supplant the Revised\textsuperscript{6} Report but rather to
provide a more comprehensive introduction and reference
manual for the language, with more explanatory text and examples,
suitable more for users than for implementors.
Features specific to particular implementations of Scheme are not included.
In particular, features specific to the author's
\label{preface_s3}Chez Scheme and
\label{preface_s4}Petite Chez Scheme implementations are described separately
in the \textit{Chez Scheme User's Guide} [\hyperref[bibliography_g227]{9}].
On the other hand, no book on Scheme would be complete without some coverage
of the interactive top level, since nearly every Scheme system supports interactive
use in one form or another, even though the behavior is not standardized by the
Revised\textsuperscript{6} Report.
Chapters 2 and 3 are thus written assuming that the reader has available a Scheme
implementation that supports an interactive top level, with behavior consistent
with the description of the top-level environment in earlier reports and the
IEEE/ANSI standard.


A large number of small- to medium-sized examples are spread throughout
the text, and one entire chapter is dedicated to the presentation of a
set of longer examples.
Many of the examples show how a standard Scheme syntactic form or
procedure might be implemented;
others implement useful extensions.
All of the examples can
be entered directly from the keyboard into an interactive Scheme
session.


This book is organized into twelve chapters, plus back matter.
Chapter \ref{intro_g0} describes the properties and features of
Scheme that make it a useful and enjoyable language to use.
Chapter \ref{intro_g0} also describes Scheme's notational conventions and the
typographical conventions employed in this book.


Chapter \ref{start_g4} is an introduction to Scheme programming
for the novice Scheme
programmer that leads the reader through a series of examples, beginning with
simple Scheme expressions and working toward progressively more
difficult ones.
Each section of Chapter \ref{start_g4} introduces
a small set of related features, and
the end of each section contains a set of exercises for further practice.
The reader will learn the most from Chapter \ref{start_g4} by
sitting at the keyboard
and typing in the examples and trying the exercises.


Chapter \ref{further_g49} continues the introduction but covers
more advanced features and concepts.
Even readers with prior Scheme experience may wish to work
through the examples and exercises found there.


Chapters \ref{binding_g88} through \hyperref[exceptions_g147]{11} make up the reference
portion of the text.
They present each of Scheme's primitive procedures and syntactic
forms in turn, grouping them into short sections of related procedures
and forms.
Chapter \ref{binding_g88} describes operations for creating procedures
and variable bindings;
Chapter \ref{control_g96}, program control operations;
Chapter \ref{objects_g106}, operations on the various object types
(including lists, numbers, and strings);
Chapter \ref{io_g121}, input and output operations;
Chapter \ref{syntax_g133}, syntactic extension;
Chapter \ref{records_g138}, record-type definitions;
Chapter \ref{libraries_g142}, libraries and top-level programs; and
Chapter \ref{exceptions_g147}, exceptions and conditions.


Chapter \ref{examples_g151} contains a collection of example procedures,
libraries, and programs, each with a short overview, some examples of
its use, the implementation with brief explanation, and a set of
exercises for further work.
Each of these programs demonstrates a particular set of features,
and together they illustrate an appropriate style for
programming in Scheme.


Following Chapter \ref{examples_g151} are
bibliographical references,
answers to selected exercises,
a detailed description of the formal syntax of Scheme programs and data,
a concise summary of Scheme syntactic forms and procedures,
and
the index.
The summary of forms and procedures is a useful first stop for
programmers unsure of the structure of a syntactic form or the
arguments expected by a primitive procedure.
The page numbers appearing in the summary of forms and procedures and
the italicized page numbers appearing in the index indicate the
locations in the text where forms and procedures are defined.


Because the reference portion describes a number of aspects of the
language not covered by the introductory chapters along with a number
of interesting short examples,
most readers will find it profitable to read through most of the
material to become familiar with each feature and how it relates to
other features.
Chapter \ref{objects_g106} is lengthy, however, and may be skimmed
and later referenced as needed.


An online version of this book is available at \href{http://www.scheme.com/tspl/}{\textit{http://www.scheme.com/tspl/}}.
The summary of forms and index in the online edition include page
numbers for the printed version and are thus useful as searchable
indexes.


\textit{About the illustrations:}
The cover illustration and the illustration at the front of each chapter
are algorithmic line fields created by artist Jean-Pierre Hébert, based
on an idea inspired by the writings of John Cage.
Each line field is created by the composition of any number of grids of
parallel lines.
The grids are regular, but they are not.
For instance, the lines are of irregular length, which creates ragged
edges.
Their tone and thickness vary slightly.
They are not exactly equidistant.
They intersect with each other at a certain angle. 
When this angle is small, patterns of interference develop.
The lines are first steeped into various scalar fields that perturb their
original straight shape, then projected on the plane of the paper.
Masks introduce holes in some layers. 
For the cover illustration, the grids are colored in different hues.


All the images are created by a single Scheme program that makes most of
the decisions, based heavily on chance.
The artist controls only canvas size, aspect ratio, the overall palette of
colors, and levels of chance and fuzziness. 
The task of the artist is to introduce just enough chance at the right
place so that the results are at the same time surprising, interesting,
and in line with the artist's sense of aesthetics. 
This is a game of uncertainty, chaos, and harmony.


\textit{Acknowledgments:}
Many individuals contributed in one way
or another to the preparation of one or more editions
of this book, including
Bruce Smith,
Eugene Kohlbecker,
Matthias Felleisen,
Dan Friedman,
Bruce Duba,
Phil Dybvig,
Guy Steele,
Bob Hieb,
Chris Haynes,
Dave Plaisted,
Joan Curry,
Frank Silbermann,
Pavel Curtis,
John Wait,
Carl Bruggeman,
Sam Daniel,
Oscar Waddell,
Mike Ashley,
John LaLonde,
John Zuckerman,
John Simmons,
Bob Prior,
Bob Burger,
and
Aziz Ghuloum.
Many others have offered minor corrections and suggestions.
Oscar Waddell helped create the typesetting system used to
format the printed and online versions of this book.
A small amount of text and a few examples have been adapted from
the Revised\textsuperscript{6} Report for this book, for which credit goes to the
editors of that report and many others who contributed to it.
Finally and most importantly, my wife, Susan Dybvig, suggested that
I write this book in the first place and lent her expertise and
assistance to the production and publication of this and the
previous editions.


