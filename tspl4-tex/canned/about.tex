\chapter*{The Scheme Programming Language, 4th Edition}

\textbf{\huge Written by R. Kent Dybvig}

\textbf{\Large Illustrations by Jean-Pierre Hébert}

 Scheme is a general-purpose programming language, descended from Algol and Lisp, widely used in computing education and research and a broad range of industrial applications. This thoroughly updated edition of \textit{The Scheme Programming Language} provides an introduction to Scheme and a definitive reference for standard Scheme, presented in a clear and concise manner. Written for professionals and students with some prior programming experience, it begins by leading the programmer gently through the basics of Scheme and continues with an introduction to some of the more advanced features of the language.

The fourth edition has been substantially revised and expanded to bring the content up to date with the current Scheme standard, the Revised\textsuperscript{6} Report on Scheme. All parts of the book were updated and three new chapters were added, covering the language's new library, exception handling, and record-definition features.

The book offers three chapters of introductory material with numerous examples, eight chapters of reference material, and one chapter of extended examples and additional exercises. All of the examples can be entered directly from the keyboard into an interactive Scheme session. Answers to many of the exercises, a complete formal syntax of Scheme, and a summary of forms and procedures are provided in appendixes.

\textit{The Scheme Programming Language} is the only book available that serves both as an introductory text in a variety of courses and as an essential reference for Scheme programmers. 

\section*{About the Author}

R. Kent Dybvig is Professor of Computer Science at Indiana University and principal developer of Chez Scheme. 

\section*{Endorsements}

``Kent Dybvig's \textit{The Scheme Programming Language} is to Scheme what Kernighan and Ritchie's \textit{The C Programming Language} is to C. Kent's book is the book for either the novice or serious Scheme programmer. Its style, wit, and organization has reached a new high with the publication of the fourth edition.'' \\
---\textbf{Daniel P. Friedman}, Department of Computer Science, Indiana University

``Students in my Programming Language Concepts class need to learn the basics of Scheme in a few days, and to pick up harder concepts throughout the course. For 19 years, \textit{The Scheme Programming Language} has been an excellent guide for them. Dybvig's rapid-fire prose and examples serve both the Scheme beginner and the experienced programmer in need of a reference. Seldom do my students make a point of praising a computer science textbook; that happens over and over with this one.'' \\
---\textbf{Claude W. Anderson}, Rose-Hulman Institute of Technology

``Eric Raymond once wrote that learning Lisp makes one a better programmer for the rest of one's days. Scheme is the best dialect of Lisp to learn for this purpose, and Kent Dybvig's book provides a comprehensive and beautiful introduction to learning Scheme and becoming a better programmer.'' \\
---\textbf{Olivier Danvy}, Aarhus University, Denmark, co-Editor-in-Chief of \textit{Higher-Order and Symbolic Computation}