\PassOptionsToPackage{unicode=true,colorlinks=true,urlcolor=blue}{hyperref}
\PassOptionsToPackage{hyphens}{url}
%
\documentclass[a4paper,12pt,notitlepage,twoside,openright]{article}

\usepackage{ifxetex}
\ifxetex{}
\else
  \errmessage{Must be built with xelatex}
\fi

\usepackage{amsmath}
\usepackage{fontspec}
\usepackage{fourier-otf} % erewhon-math
\setmonofont{iosevka-type-slab-regular}[
  Path=../common/iosevka-type-slab/,
  Extension=.ttf,
  BoldFont=iosevka-type-slab-bold,
  ItalicFont=iosevka-type-slab-italic,
  BoldItalicFont=iosevka-type-slab-bolditalic,
  Scale=MatchLowercase,
]

% Math
\usepackage[binary-units]{siunitx}

\usepackage{caption}
\usepackage{authblk}
\usepackage{enumitem}
\usepackage{footnote}

% Table
\usepackage{tabu}
\usepackage{longtable}
\usepackage{booktabs}
\usepackage{multirow}

% Verbatim & Source code
\usepackage{fancyvrb}
\usepackage{minted}

% Beauty
\usepackage[protrusion]{microtype}
\usepackage[defaultlines=3]{nowidow}
\usepackage{upquote}
\usepackage{parskip}
\usepackage[strict]{changepage}

\usepackage{hyperref}

% Graph
\usepackage{graphicx}
\usepackage{grffile}
\usepackage{tikz}

\hypersetup{
  bookmarksnumbered,
  pdfborder={0 0 0},
  pdfpagemode=UseNone,
  pdfstartview=FitH,
  breaklinks=true}
\urlstyle{same}  % don't use monospace font for urls

\usetikzlibrary{arrows.meta,calc,shapes.geometric,shapes.misc}

\setminted{
  autogobble,
  breakbytokenanywhere,
  breaklines,
  fontsize=\footnotesize,
}
\setmintedinline{
  autogobble,
  breakbytokenanywhere,
  breaklines,
  fontsize=\footnotesize,
}

\makeatletter
\def\maxwidth{\ifdim\Gin@nat@width>\linewidth\linewidth\else\Gin@nat@width\fi}
\def\maxheight{\ifdim\Gin@nat@height>\textheight\textheight\else\Gin@nat@height\fi}
\makeatother

% Scale images if necessary, so that they will not overflow the page
% margins by default, and it is still possible to overwrite the defaults
% using explicit options in \includegraphics[width, height, ...]{}
\setkeys{Gin}{width=\maxwidth,height=\maxheight,keepaspectratio}
\setlength{\emergencystretch}{3em}  % prevent overfull lines
\setcounter{secnumdepth}{3}

% Redefines (sub)paragraphs to behave more like sections
\ifx\paragraph\undefined\else
\let\oldparagraph\paragraph{}
\renewcommand{\paragraph}[1]{\oldparagraph{#1}\mbox{}}
\fi
\ifx\subparagraph\undefined\else
\let\oldsubparagraph\subparagraph{}
\renewcommand{\subparagraph}[1]{\oldsubparagraph{#1}\mbox{}}
\fi

% set default figure placement to htbp
\makeatletter
\def\fps@figure{htbp}
\makeatother


\title{Data Lakes: Trends and Perspectives}
\author{Franck Ravat and Yan Zhao}
\date{DEXA 2019}

\begin{document}
\maketitle

\begin{abstract}
As a relatively new concept, data lake has neither a
standard definition nor an acknowledged architecture. Thus, we study the
existing work and propose a complete definition and a generic and
extensible architecture of data lake. What's more, we introduce three
future research axes in connection with our health-care Information
Technology (IT) activities. They are related to (i) metadata management
that consists of intra- and inter-metadata, (ii) a unified ecosystem for
companies' data warehouses and data lakes and (iii) data lake
governance.
\end{abstract}

\hypertarget{introduction}{%
\section{Introduction}\label{introduction}}

In the big data era, a great volume of structured, semi-structured and
unstructured data are created much faster than before by smart-phones,
social media, connected objects, and other data creators. These data
have a great value for companies' Decision Support System (DSS) whose
cornerstone is built upon data. Nevertheless, handling heterogeneous and
voluminous data is especially challenging for DSS. Nowadays, Data
Warehouse (DW) is a commonly used solution in DSS. Data have been
extracted, transformed and loaded (ETL processes) according to
predefined schemas. DW is popular thanks to its fast response,
consistent performance and cross functional analysis. However, according
to {[}4,5{]}, DWs are not adapted for the big data analytics for the
following reasons: (i) only predefined requirements can be answered.
(ii) some information is lost through ETL processes. And (iii) the cost
of a DW can grow exponentially because of the requirements of better
performance, the growth of data volume and the complexity of database.

To face the challenges of big data and the deficiencies of DW, Dixon
{[}4{]} put forward the concept data lake (DL): ``If a data warehouse
may be a store of bottled water - cleansed and packaged and structured
for easy consumption the data lake is a large body of water in a more
natural state.'' This explication sketches the outline of DL, but it can
not be considered as a formal definition. DL is a relatively new
concept. Even though there are some so-called DL solutions in the
market, there is not a standard definition nor an acknowledged
architecture.

The goal of this prospective/survey paper is twofold. Firstly, we
summarize the state of the art work and present a more complete vision
of DL concept and a generic architecture. Secondly, we present future
research axes by identifying major issues that appeared in our
health-care IT activities. The remainder of the paper is organized as
follows: Sect.2 introduces the DL concept, we analyze different
definitions and propose our own definition; Sect.3 discusses DL
architectures and introduces a generic and extensible architecture;
Sect.4 describes future research axes that includes metadata management,
position of a DL in an information system and data lake governance.

\hypertarget{data-lake-concept}{%
\section{Data Lake Concept}\label{data-lake-concept}}

\hypertarget{state-of-the-art}{%
\subsection{State of the Art}\label{state-of-the-art}}

Data lake, as a relatively new concept, is defined in both scientific
community and industrial world {[}3,5,7,15,18,21,25,31{]}. All the
existing definitions respect the idea that a DL is a repository storing
raw data in their native format. Yet, different definitions have
different emphases. Regarding input, {[}5{]} introduces that the input
of a DL is the data within an enterprise. Regarding process, {[}21{]}
emphasizes that there is no process during the ingestion phase and
{[}3,7,21,25{]} introduce that data will be processed upon usage.
Regarding architecture, {[}5{]} presents that DLs are based on an
architecture with low cost technologies. Regarding governance, metadata
management is emphasized in {[}7,31{]}. And regarding users, {[}18{]}
presents that data scientists and statisticians are DL users.

\hypertarget{dara-lake-definition}{%
\subsection{Dara Lake Definition}\label{dara-lake-definition}}

Existing definitions have evolved over time from experience feedback.
Nevertheless, as mentioned in the previous paragraph, these different
definitions are vague, they are not integrated with each other or even
contradictory. To be as complete as possible, we propose a definition
that includes input, process, output and governance of data lakes.

In the context of big data analytics, user requirements are not clearly
defined at the time of the initial design and the implementation of a
DL. A data lake is a big data analytics solution that ingests
heterogeneously structured raw data from various sources (local or
external to the organization) and stores these raw data in their native
format, allows to process data according to different requirements and
provides accesses of available data to different users (data scientists,
data analysts, BI professionals etc.) for statistical analysis, Business
Intelligence (BI), Machine Learning (ML) etc., and governs data to
insure the data quality, data security and data life-cycle.

\hypertarget{data-lake-architecture}{%
\section{Data Lake Architecture}\label{data-lake-architecture}}

To the best of our knowledge, there does not exist an acknowledged DL
architecture in literature. Firstly, we present different existing
architectures and then propose a generic and extensible architecture.

\hypertarget{state-of-the-art-1}{%
\subsection{State of the Art}\label{state-of-the-art-1}}

Data lake functional architecture has evolved from mono-zone to
multi-zone, and it is always presented with technical solutions.

The first vision of DL architecture is a flat architecture with a
mono-zone that stores all the raw data in their native format. This
architecture, closely tied to the HADOOP environment, enables load
heterogeneous and voluminous data with low cost. Nevertheless, it does
not allow users to process data and does not record any user operations.

A second vision of DL architecture contains five data ponds {[}10{]}. A
\emph{Raw data pond} that stores the just ingested data and the data
that do not fit in other ponds. \emph{Analog, application and textual
data ponds} stores classified data from raw data pond by their
characteristics. And \emph{achival data pond} stores the data that are
no longer used. This architecture classifies different types of data and
achieves useless data, which make data finding faster and data analytics
easier. However, the division of different ponds, especially the
archival pond can not ensure the availability of all the raw data,
contradicts the general recognition of DL which is to ingest all the raw
data and process them upon usage.

To overcome these drawbacks, a third vision of architecture with
multi-zones is proposed with a more diverse technological environment in
the academic and industrial world. The author of {[}22{]} presents
Amazon Web Services (AWS) DL architecture with four zones: ingestion,
storage, processing and govern \& secure. Raw data are loaded in the
ingestion zone. The ingested raw data are stored in the storage zone.
When data are needed, they are processed in the processing zone. The
objective of Govern \& secure zone is to control data security, data
quality, metadata management and data life-cycle. The author of {[}19{]}
separates the data processing zone into batch-processing and real time
processing zones. He also adds a processed data zone to store all the
cleansed data. Zaloni's DL architecture {[}14{]} separates the
processing and storage zones into refined data zone, trusted data zone
and discovery sandbox zone. The refined zone allows to integrate and
structure data. Trusted data zone stores all the cleansed data. Data for
exploratory analysis moves to the discovery sandbox.

As mentioned, a lot of DL architectures are supported with technical
solutions. They are not independent of the inherent technical
environment. Consequently, none of the existing architectures draws a
clear distinction between functionality-related and technology-related
components. What's more, the concept of multi-zone architecture is
interesting and deserves further investigations. We believe that some
zones are essential, while others are optional or can be regrouped.
Concerning the essential zones, based on our DL definition, a data lake
should be able to ingest raw data, process data upon usage, store
processed data, provide access for different uses and govern data.

\hypertarget{data-lake-functional-architecture}{%
\subsection{Data Lake Functional
Architecture}\label{data-lake-functional-architecture}}

Unlike several proposals, we want to distinguish functional architecture
from technical architecture. Because a functional architecture concerns
the usage perspective and it can be implemented by different technical
solutions. By adopting to the existing DL architectures and avoiding
their shortcomings, we propose a functional DL architecture (see Fig.1),
which contains four essential zones, and each zone, except the govern
zone, has a treatment area (dotted rectangle) and a data storage area
that stores the result of processes (gray rectangle):

\begin{figure}
\centering
\includegraphics[width=4.72883in,height=2.05383in]{./media/image1.jpg}
\caption{Data lake functional architecture}
\end{figure}

\begin{itemize}
\item
  \emph{Raw data zone}: all types of data are ingested without
  processing and stored in their native format. The ingestion can be
  batch, real-time or hybrid. This zone allows users to find the
  original version of data for their analytics to facilitate subsequent
  treatments. The stored raw data format can be different from the
  source format.
\item
  \emph{Process zone}: in this zone, users can transform data according
  to their requirements and store all the intermediate data. The data
  processing includes batch and/or real-time processing. This zone
  allows users to process data (selection, projection, join,
  aggregation, etc.) for their data analytics.
\item
  \emph{Access zone}: the access zone stores all the available data for
  data analytics and provides the access of data. This zone allows
  self-service data consumption for different analytics (reporting,
  statistical analysis, business intelligence analysis, machine learning
  algorithms).
\item
  \emph{Governance zone}: data governance is applied on all the other
  zones. It is in charge of insuring data security, data quality, data
  life-cycle, data access and metadata management.
\end{itemize}

To exemplify our architecture, we propose an example of implementation
(Fig.2). Raw datasets (RD1, RD2) are ingested in data lake and stored in
the raw data zone in their native format. Data are processed in the
process zone and all the intermediate datasets (PD1, PD2, PD3, PD4) are
stored in this area too. All the available data (AD1, AD2, AD3) are
stored in the access zone for data consumption.

\begin{figure}
\centering
\includegraphics[width=3.6855in,height=1.62383in]{./media/image2.jpg}
\caption{An implementation of the data lake functional
architecture}
\end{figure}

\hypertarget{future-research-axes}{%
\section{Future Research Axes}\label{future-research-axes}}

The University Hospital Center (UHC) of Toulouse owns a great amount of
data produced by different applications, it can also access to many
external data. In order to facilitate data analytics to improve medical
treatments, UHC of Toulouse lunched a project of DL to combine data from
different individual sources. In this context, we encounter some
problems: How to integrate a DL in the existing DSS? How to ensure the
quality of data analytics by tracing back to the various transformations
of data since the ingestion? Based on the questions that we are facing,
we propose some research axes.

\hypertarget{integration-of-a-data-lake-in-an-information-system}{%
\subsection{Integration of a Data Lake in an Information
System}\label{integration-of-a-data-lake-in-an-information-system}}

In a data lake, different users can access and process data for data
exploration or statistical analysis for the purposes of decision making.
Thus, DLs should be considered as one part of the DSS in enterprises'
Information Systems (IS). Nowadays, the commonly used DSS solution is
DW. According to the authors of {[}5,15,18{]}, DLs and DWs are both
created for extracting value of data to support decision makings but
they also have differences. DWs are data repositories which store
cleansed data based on predetermined schema. DLs ingest all types of raw
data in their native format with low cost technologies to provide more
flexibility and scalability. Regarding the similarities and differences,
some questions like how do a DL and DWs work together, will a DL replace
DWs need to be answered.

Many papers compared DLs and DWs but only a few papers introduced the
impact of a DL for a data management ecosystem. Some authors present DL
as a advanced version of DW {[}14,31{]}. The author of {[}5{]}
introduces data lake cloud which is an elastic data storing and
computing platform, DWs are constructed based on the data in the data
lake. The author of {[}15,18{]} introduced that a DL can be fed by DWs
and a DL can also be the source of DWs.

We think DLs should coexist with DWs because they have different
objectives and users, a DL cannot simply replace a DW. To the best of
our knowledge, a coexisting ecosystem has not been studied and
implemented. To propose a such ecosystem, different research problems
are induced. The first problem relates to the functional architecture
definition which must determine precisely the information flow between
DWs and a DL. If DWs feed a DL, the questions to solve are: where are
the extracted data (from a particular DW or DM (Data Mart))? Do the data
get into the ingestion zone or the process zone of a DL (because they
have been processed in the DW)? Is it the same type of ingestion without
data transformation as the ingestion from other data sources? If a DL is
the source of a DW, the questions are similar on the sources (data in
process zone or access zone), the target (a particular DW or DM) and the
transfer process (ETL, ELT or only EL). Once these problems are solved,
we must answer the problem on refreshing and updating data. We need to
therefore answer the following questions: when is the data
transformation done (real time, near real time, batch)? What is the type
of refreshment (never, complete, incremental)? Finally, the third issue
concerns the technical architecture which ensures the power and
reliability of data flows between a DW and a DL.

\hypertarget{metadata}{%
\subsection{Metadata}\label{metadata}}

The main idea of data lake is to ingest raw data without process and
process data upon usage. Therefore, data lakes keep all the information
and have a good flexibility. Nevertheless, data lakes, which contain a
lot of datasets without explicit models or descriptions can easily
become invisible, incomprehensible and inaccessible. So that it is
mandatory to set up a metadata management system for DL. In fact, the
importance of metadata has been emphasized in many papers {[}1,7,31{]}.
The first research problem that needs to be solved is the content of the
metadata.

Data lake metadata, inspirited by DW metadata classification
{[}16,26{]}, are mainly classified in two ways. The first classification
has three categories: \emph{technical metadata} for data type, format
and data structure, \emph{operational metadata} for process history and
\emph{business metadata} for the descriptions of business objective.
This classification focus on each single dataset, but not on the
relationships between different datasets. Nevertheless, for a DL,
datasets relationships are important to help users to find relevant
datasets and verify data lineage. The second classification has two
categories: \emph{inter-metadata} for the relationships between data and
\emph{intra-metadata} for specifying each single dataset {[}17{]}.
Inter-metadata is classified into dataset containment, provenance,
logical cluster and content similarity by the author of {[}9{]}.
Intra-metadata is classified into data characteristics, definitional,
navigational, activity, lineage, rating and assessment {[}2,6,30{]}. The
second classification is evolved, but it can still be improved. Some
sub-categories are not adapted, for instance, the rating sub-category is
not adaptive, because a DL can be accessed by different users who have
different objective {[}30{]}. Moreover, the classification can be
extended by some sub-categories, for instance, data sensitivity needs to
be verified. Due to the specificity of DL with different zones including
both storage and transformation processes, it is important to include
intra- and inter-metadata. Therefore, we propose an metadata
classification which contains inter- and intra-metadata and adapted
subcategories:

\begin{itemize}
\item
  For \emph{inter-metadata} {[}28{]}, we propose to integrate
  \emph{Dataset containment} which means a dataset is contained in
  another dataset. \emph{Partial overlap} which means that some
  attributes with corresponding data in different datasets overlap.
  \emph{Provenance} which means that one dataset is the source of
  another dataset. \emph{Logical clusters} which means that some
  datasets are from the same domain (different versions, duplication
  etc.). And \emph{Content similarity} which means that different
  datasets share the same attributes.
\item
  For \emph{intra-metadata} {[}28{]}, we retain data characteristics,
  definitional, navigational and lineage metadata proposed in {[}2{]}
  and add the access, quality and security metadata.

  \begin{itemize}
  \item
    \emph{Data characteristics}: attributes describing a dataset, such
    as identification name, size and creation date.
  \item
    \emph{Definitional metadata}: datasets' meanings. Structured and
    unstructured datasets can be described semantically with a textual
    description or a set of keywords (vocabularies). Definitional
    metadata help users to understand datasets and make their data
    exploitation easier.
  \item
    \emph{Navigational metadata}: location information like file paths
    and database connection URLs.
  \item
    \emph{Lineage metadata}: information concerns data life-cycle. For
    example, data source, data process history.
  \item
    \emph{Quality metadata}: data consistency and completeness {[}26{]}
    to ensure dataset's reliability.
  \item
    \emph{Security metadata}: data sensitivity and access level. Some
    datasets may contain sensitive information that can only be access
    by certain users. Security metadata can support the verification of
    access.
  \end{itemize}
\end{itemize}

The second research problem is to define an appropriate solution for
metadata management. To the best our knowledge, there isn't a general
metadata management system that works on heterogeneous data for the
whole data life-cycle in DLs. We only have partial solutions in the
literature. Some works concentrate on the detection of relationships
between different datasets {[}1,9,27{]}. Some other work focus on the
extraction of metadata for unstructured data (mostly textual data)
{[}27,29{]}.

To propose a appropriate solution, different research problems are
induced. The first one relates to the way that metadata are stored: what
is the conceptual schema of metadata? Which attributes should be
recorded? How should we store the metadata (distributed RDBMS, NOSQL
DBMS, RDF Stores etc.)? The second problem relates to the data feeding
process: how to extract metadata from structured data as well as semi or
unstructured data. What is the text analysis engine to detect
automatically the keywords from unstructured data?

\hypertarget{data-lake-governance}{%
\subsection{Data Lake Governance}\label{data-lake-governance}}

A DL ingests and stores various types of data and can be accessed by
different users. Without best practices in place to manage it, many
issues that concern accessing, querying, and analyzing data can appear
{[}23{]}. Given this context, data lake governance is required. In the
state of the art work we find some partials solutions. A ``Just-enough
Governance'' for DLs is proposed by {[}8{]} with data quality policy,
data on-boarding policy, metadata management and compliance \& audit
policy. {[}11{]} indicated that the data governance needs to ensure data
quality and availability throughout the full data life-cycle. {[}24{]}
presented data governance with data lineage, data quality, security and
data life-cycle management. However, these papers don't integrate the
data lake governance through a complete vision.

We firstly propose a definition: the data governance of DL enables
policies, standards and practices to be applied to manage data from
heterogeneous sources and associated process (transformation and
analysis) to ensure an efficient, secure usage, and a reliable quality
of the analysis results. We propose to classify the DL data governance
into data assets and IT assets {[}12,32{]}. Data assets refers to the
value or potential value of data, while IT assets refers to
technologies. Secondly, we identify some research axes based on the
classification:

\begin{itemize}
\item
  Concerning the IT assets, the future researches must integrate the
  four fol-lowing points: \emph{data lake principles} concerns the
  position of the data lake in an information system. \emph{Data lake
  functional architecture} concerns the evolution of different zones
  adapted to the enterprises' needs. \emph{Data lake technical
  infrastructure} concerns the decisions about the technological
  solutions. \emph{Data lake investment and prioritization} concerns the
  decisions about how much to invest for a data lake and the
  distribution for different zones.
\item
  Concerning the data assets, the future research must be related to:

  \begin{itemize}
  \item
    \emph{Metadata management} (cf. previous section).
  \item
    \emph{Data quality management}: it influences the reliability of
    analytics result. In a data lake, data quality should at least be
    evaluated upon usage. New models must be proposed to make sure that
    data are valid in different data lake zones to ensure the data
    quality. Automatic validation {[}13,20{]} in the raw data zone,
    users' comments of analytics in the access zone can be one of the
    solutions.
  \item
    \emph{Data life-cycle management (DLM)}: it's necessary to define
    specific workflows to model the life-cycle of all the data that
    stored in different zones of a data lake and the relationships
    between different data.
  \item
    \emph{Security/Privacy}: a data lake can be accessed by different
    users with different privileges. Data sensitivity has to be
    authenticated, users access has to be managed {[}24{]}. New models
    and systems must be defined to be adapted to the data lake
    governance.
  \end{itemize}
\end{itemize}

\hypertarget{conclusions}{%
\section{Conclusions}\label{conclusions}}

In this prospective/survey paper, we propose a complete definition of
data lake and an extensible functional architecture based on 4 zones.
Our definition has the advantages of being more complete than the
literature and includes both input and output, different functions as
well as users of data lakes. In our data lake architecture, each zone is
defined more formally than the literature and is composed of a process
layer and a data storage layer.

We also introduce future research axes. In the metadata context, we
identify two issues: (i) identifying and modelling intra- and
inter-metadata, (ii) implementing these metadata in an adequate data
management system. In the information system context, the future work
concentrates on the definition of a unified ecosystem in which the
enterprise data warehouses and data lakes coexist. Finally, in the data
lake governance context, we propose to work in the fields of data assets
and IT assets.

\hypertarget{references}{%
\section*{References}\label{references}}

\begin{enumerate}
\def\labelenumi{\arabic{enumi}.}
\item
  Alserafi, A., Abell´o, A., Romero, O., Calders, T.: Towards
  information profiling: data lake content metadata management. In: 2016
  IEEE 16th International Conference on Data Mining Workshops (ICDMW),
  pp. 178--185. IEEE (2016)
\item
  Bilalli, B., Abell´o, A., Aluja-Banet, T., Wrembel, R.: Towards
  intelligent dataanalysis: the metadata challenge. In: Proceedings of
  the International Conference on Internet of Things and Big Data, Rome,
  Italy, pp. 331--338 (2016)
\item
  Campbell, C.: Top five differences between data lakes and data
  warehouse, January 2015.
  \href{https://www.blue-granite.com/blog/bid/402596/top-five-differences-between-data-lakes-and-data-warehouses}{https://www.blue-granite.com/blog/bid/402596/top-five-differencesbetween-data-lakes-and-data-warehouses}
\item
  Dixon, J.: Pentaho, Hadoop, and data lakes, October 2010.
  \href{https://jamesdixon.wordpress.com/2010/10/14/pentaho-hadoop-and-data-lakes/}{https://jamesdixon.
  wordpress.com/2010/10/14/pentaho-hadoop-and-data-lakes/}
\item
  Fang, H.: Managing data lakes in big data era: what's a data lake and
  why hasit became popular in data management ecosystem. In: 2015 IEEE
  International Conference on Cyber Technology in Automation, Control,
  and Intelligent Systems (CYBER), pp. 820--824. IEEE (2015)
\item
  Foshay, N., Mukherjee, A., Taylor, A.: Does data warehouse end-user
  metadataadd value? Commun. ACM \textbf{50}(11), 70--77 (2007)
\item
  Hai, R., Geisler, S., Quix, C.: Constance: an intelligent data lake
  system. In: Proceedings of the 2016 International Conference on
  Management of Data, pp. 2097-- 2100. ACM (2016)
\item
  Haines, R.: What is just enough governance for the data lake?,
  February 2015.
  \href{https://infocus.dellemc.com/rachel---haines/just--enough--governance--data--lake/}{https://infocus.dellemc.com/rachel-\/-\/-haines/just-\/-enough-\/-governance-\/-data-lake/}
\item
  Halevy, A.Y., et al.: Managing google's data lake: an overview of the
  goods system.IEEE Data Eng. Bull. \textbf{39}(3), 5--14 (2016)
\item
  Inmon, B.: Data Lake Architecture: Designing the Data Lake and
  avoiding thegarbage dump. Technics publications (2016)
\item
  Kaluba, K.: Data lake governance - do you need it?, March 2018.
  \href{https://blogs.sas.com/content/datamanagement/2018/03/27/data-lake-governance/}{https://blogs.
  sas.com/content/datamanagement/2018/03/27/data-lake-governance/} 12.
  Khatri, V., Brown, C.V.: Designing data governance. Commun. ACM
  \textbf{53}(1), 148 (2010).
  \href{https://doi.org/10.1145/1629175.1629210}{https://doi.org/10.1145/1629175.1629210.}
  \href{http://portal.acm.org/citation.cfm?doid=1629175.1629210}{http://portal.acm.org/citation.
  cfm?doid=1629175.1629210}
\end{enumerate}

\begin{enumerate}
\def\labelenumi{\arabic{enumi}.}
\setcounter{enumi}{12}
\item
  Kwon, O., Lee, N., Shin, B.: Data quality management, data usage
  experienceand acquisition intention of big data analytics. Int. J.
  Inf. Manag. \textbf{34}(3), 387--394
\end{enumerate}

\begin{quote}
(2014)
\end{quote}

\begin{enumerate}
\def\labelenumi{\arabic{enumi}.}
\setcounter{enumi}{13}
\item
  LaPlante, A., Sharma, B.: Architecting Data Lakes. O'Reilly Media,
  Sebastopol(2014)
\item
  Llave, M.R.: Data lakes in business intelligence: reporting from the
  trenches. Procedia Comput. Sci. \textbf{138}, 516--524 (2018)
\item
  Lopez Pino, J.L.: Metadata in business intelligence, January 2014.
  \href{https://www.slideshare.net/jlpino/metadata-in-business-intelligence}{https://www.
  slideshare.net/jlpino/metadata-in-business-intelligence}
\item
  Maccioni, A., Torlone, R.: Crossing the finish line faster when
  paddling the datalake with kayak. Proc. VLDB Endow. \textbf{10}(12),
  1853--1856 (2017)
\item
  Madera, C., Laurent, A.: The next information architecture evolution:
  the datalake wave. In: Proceedings of the 8th International Conference
  on Management of Digital EcoSystems, pp. 174--180. ACM (2016)
\item
  Menon, P.: Demystifying data lake architecture, July 2017.
  \href{https://medium.com/@rpradeepmenon/demystifying-data-lake-architecture-30cf4ac8aa07}{https://medium.com/
  @rpradeepmenon/demystifying-data-lake-architecture-30cf4ac8aa07}
\item
  Merino, J., Caballero, I., Rivas, B., Serrano, M., Piattini, M.: A
  data quality inuse model for big data. Future Gener. Comput. Syst.
  \textbf{63}, 123--130 (2016)
\item
  Miloslavskaya, N., Tolstoy, A.: Big data, fast data and data lake
  concepts. ProcediaComput. Sci. \textbf{88}, 300--305 (2016)
\item
  Nadipalli, R.: Effective Business Intelligence with QuickSight. Packt
  PublishingLtd., Birmingham (2017)
\item
  O'Leary, D.E.: Embedding AI and crowdsourcing in the big data lake.
  IEEE Intell.Syst. \textbf{29}(5), 70--73 (2014)
\item
  Patel, P., Greg, W., Diaz, A.: Data lake governance best practices,
  April 2017.
  \url{https://dzone.com/articles/data-lake-governance-best-practices}
\item
  Piatetsky-Shapiro, G.: Data lake vs data warehouse: key differences,
  September2015.
  \href{https://www.kdnuggets.com/2015/09/data-lake-vs-data-warehouse-key-differences.html}{https://www.kdnuggets.com/2015/09/data-lake-vs-data-warehouse-keydifferences.html}
\item
  Ponniah, P.: Data Warehousing Fundamentals: a Comprehensive Guide for
  IT Professionals. Wiley, Hoboken (2004)
\item
  Quix, C., Hai, R., Vatov, I.: Metadata extraction and management in
  data lakeswith gemms. Complex Syst. Inf. Model. Q. \textbf{9}, 67--83
  (2016)
\item
  Ravat, F., Zhao, Y.: Metadata management for data lakes. In: East
  EuropeanConference on Advances in Databases and Information Systems.
  Springer (2019)
\item
  Sawadogo, P., Kibata, T., Darmont, J.: Metadata management for textual
  documents in data lakes. In: 21st International Conference on
  Enterprise Information Systems (ICEIS 2019) (2019)
\item
  Varga, J., Romero, O., Pedersen, T.B., Thomsen, C.: Towards next
  generationBI systems: the analytical metadata challenge. In:
  Bellatreche, L., Mohania, M.K. (eds.) DaWaK 2014. LNCS, vol. 8646, pp.
  89--101. Springer, Cham (2014).
  \href{https://doi.org/10.1007/978-3-319-10160-6_9}{https://
  doi.org/10.1007/978-3-319-10160-6}
  \href{https://doi.org/10.1007/978-3-319-10160-6_9}{9}
\item
  Walker, C., Alrehamy, H.: Personal data lake with data gravity pull.
  In: 2015 IEEEFifth International Conference on Big Data and Cloud
  Computing, pp. 160--167. IEEE (2015)
\item
  Weill, P., Ross, J.W.: IT Governance: How Top Performers Manage IT
  Decision
\end{enumerate}

\end{document}
