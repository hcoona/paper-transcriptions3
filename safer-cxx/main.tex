\PassOptionsToPackage{dvipsnames}{xcolor}
\PassOptionsToPackage{unicode=true,colorlinks=true,urlcolor=blue}{hyperref} % options for packages loaded elsewhere
\PassOptionsToPackage{hyphens}{url}
\documentclass[a4paper,12pt,notitlepage,twoside,openright]{article}

\usepackage{ifxetex}
\ifxetex{}
\else
\errmessage{Must be built with XeLaTeX}
\fi

\usepackage{amsmath}
\usepackage{fontspec}
\usepackage{fourier-otf} % erewhon-math
\setmonofont{iosevka-type-slab-regular}[
  Path=../common/iosevka-type-slab/,
  Extension=.ttf,
  BoldFont=iosevka-type-slab-bold,
  ItalicFont=iosevka-type-slab-italic,
  BoldItalicFont=iosevka-type-slab-bolditalic,
  Scale=MatchLowercase,
]

% Math
\usepackage[binary-units]{siunitx}

\usepackage{caption}
\usepackage{authblk}
\usepackage{enumitem}
\usepackage{footnote}

% Table
\usepackage{tabu}
\usepackage{longtable}
\usepackage{booktabs}
\usepackage{multirow}

% Verbatim & Source code
\usepackage{fancyvrb}
\usepackage{minted}

% Beauty
\usepackage[protrusion]{microtype}
\usepackage[defaultlines=3]{nowidow}
\usepackage{upquote}
\usepackage{parskip}
\usepackage[strict]{changepage}

\usepackage{hyperref}

% Graph
\usepackage{graphicx}
\usepackage{grffile}
\usepackage{tikz}

\usepackage{endnotes}
\hypersetup{
  bookmarksnumbered,
  pdfborder={0 0 0},
  pdfpagemode=UseNone,
  pdfstartview=FitH,
  breaklinks=true}
\urlstyle{same}  % don't use monospace font for urls

\usetikzlibrary{arrows.meta,calc,shapes.geometric,shapes.misc}

\setminted{
  autogobble,
  breakbytokenanywhere,
  breaklines,
  fontsize=\footnotesize,
}
\setmintedinline{
  autogobble,
  breakbytokenanywhere,
  breaklines,
  fontsize=\footnotesize,
}

\makeatletter
\def\maxwidth{\ifdim\Gin@nat@width>\linewidth\linewidth\else\Gin@nat@width\fi}
\def\maxheight{\ifdim\Gin@nat@height>\textheight\textheight\else\Gin@nat@height\fi}
\makeatother

% Scale images if necessary, so that they will not overflow the page
% margins by default, and it is still possible to overwrite the defaults
% using explicit options in \includegraphics[width, height, ...]{}
\setkeys{Gin}{width=\maxwidth,height=\maxheight,keepaspectratio}
\setlength{\emergencystretch}{3em}  % prevent overfull lines
\setcounter{secnumdepth}{3}

% Redefines (sub)paragraphs to behave more like sections
\ifx\paragraph\undefined\else
\let\oldparagraph\paragraph{}
\renewcommand{\paragraph}[1]{\oldparagraph{#1}\mbox{}}
\fi
\ifx\subparagraph\undefined\else
\let\oldsubparagraph\subparagraph{}
\renewcommand{\subparagraph}[1]{\oldsubparagraph{#1}\mbox{}}
\fi

% set default figure placement to htbp
\makeatletter
\def\fps@figure{htbp}
\makeatother


\title{Safer Usage Of C++}
\author{adetaylor, palmer}
\date{2021-08-20}

\newcommand{\uline}[1]{#1}

\begin{document}

\maketitle

\textbf{This document is PUBLIC. Chromium committers can comment
on \href{https://www.google.com/url?q=https://docs.google.com/document/d/1eVKfIsPsEVDsBS24ErmuyjNy-9zSBzHgp4Sk6fs6OOI/edit\%23\&sa=D\&source=editors\&ust=1631944129012000\&usg=AOvVaw0F0RgfJgUo0h2Qa_7lRJLS}{the
original doc}. If you want to comment but can't, ping palmer@. Thanks
for reading!}

\section{Introduction}

Chrome
Security \href{https://www.google.com/url?q=https://docs.google.com/document/d/1JWGI-fkEsdRPpwPtS6x3siNtHX8cWoZeiWD6_y781ag/edit\&sa=D\&source=editors\&ust=1631944129014000\&usg=AOvVaw2-sNWSU4CBVsAbGowWToa-}{has
been asked to consider} {[}Google-internal only, sorry{]} what it would
take to make C++ less dangerous. This document outlines various
mechanisms we could use to make it significantly easier to use C++
safely. Some are radical, and adopting them (especially adopting many of
them) may result in code that looks quite different from what C++
programmers expect.

Most of the proposed mechanisms are new usage patterns, libraries, and
classes, but some call for the use of compiler-specific flags that
change the language somewhat. (For example, Chromium already
uses \texttt{-fno-exceptions}, and here we propose \texttt{-ftrapv}, \texttt{-fwrapv},
or \texttt{fsanitize=signed-integer-overflow}.)

Some of these mechanisms are already being built in Chromium, with
varying degrees of success. (Examples: The UAF-resistant smart
pointer \href{https://www.google.com/url?q=http://go/miracleptr\&sa=D\&source=editors\&ust=1631944129016000\&usg=AOvVaw2vE6RkE_iDu216o08aJNOu}{MiraclePtr} is
in performance trials, we have expanded the use of Oilpan to PDFium, and
the hardening of //base, WTF, and Abseil is substantial and has proven
effective.)

Other mechanisms we propose represent significant new directions for C++
and Chromium, and may even require new research and development into
open problem areas (e.g. new forms of static analysis).

The C++ language and culture tend to trade off safety in favor of
efficiency, and therefore many of these proposed changes are complex,
controversial, and not as robust as similar changes might be in another
language. Additionally, they might sometimes have micro- or even
macro-efficiency effects (time, space, or object code size).

\subsection{Prior Work}

Safer C++ is a dream that many people share.

There
is \href{https://www.google.com/url?q=https://isocpp.github.io/CppCoreGuidelines/CppCoreGuidelines\&sa=D\&source=editors\&ust=1631944129018000\&usg=AOvVaw3esgyUEKe_P7ORRHHsQKuw}{the
C++ Core Guidelines project},
and \href{https://www.google.com/url?q=https://github.com/duneroadrunner/scpptool\&sa=D\&source=editors\&ust=1631944129019000\&usg=AOvVaw2Mqr8zeo2INs5TCMiGAXxz}{the
Safe C++
Tool} (and \href{https://www.google.com/url?q=https://github.com/duneroadrunner/SaferCPlusPlus\&sa=D\&source=editors\&ust=1631944129019000\&usg=AOvVaw0WEMnaP_cIAFfrNm_5BkEQ}{its
SaferCPlusPlus library},
and \href{https://www.google.com/url?q=https://github.com/duneroadrunner/SaferCPlusPlus-AutoTranslation2\&sa=D\&source=editors\&ust=1631944129020000\&usg=AOvVaw3DSaV0ss5Z4newR7PeGb_h}{an
auto-translation tool}).

We don't propose a new language, but
e.g. \href{https://www.google.com/url?q=https://web.eecs.umich.edu/ weimerw/p/p232-condit.pdf\&sa=D\&source=editors\&ust=1631944129020000\&usg=AOvVaw3VtjxgjGOY2s7YlopmPcna}{CCured} and \href{https://www.google.com/url?q=https://cyclone.thelanguage.org/\&sa=D\&source=editors\&ust=1631944129020000\&usg=AOvVaw3mmIspB3EW_s2LHVnpv5K1}{Cyclone} have
been interesting previous efforts to make new flavors of C mostly
compatible with existing C.

Also see an analysis
of \href{https://www.google.com/url?q=https://docs.google.com/document/d/1LhcUy_Eoqo8H2yNPJYAPDfaQWmOIY0gS8S8gfPISOIc/edit\%23heading\%3Dh.t1e8jytl71ol\&sa=D\&source=editors\&ust=1631944129021000\&usg=AOvVaw2rd0sTq73D3ofpS8zTWq7U}{how
memory tagging might change the safety
situation} {[}Google-internal{]}.

\subsection{Background}

There are 2 basic types of memory safety: spatial safety and temporal
safety. \textbf{Spatial safety} is the guarantee that the program will
behave in a defined and safe way if it accesses memory outside valid
bounds. Examples include array bounds, struct and union field access,
and iterator access.

\textbf{Temporal safety} is the guarantee that the program will behave
in a defined and safe way if it accesses memory when that memory is not
valid at the time of the access. Examples include use after free (UAF),
double-free, use before initialization, and use after move (UAM).
Temporal safety violations often look like type confusion. For example,
the program mistakenly uses a recently-freed Dog object as if it were
a Cat object. (Attackers often build entirely fake objects with vtables
that give them control of program execution.)

Of these 2 types of safety, spatial safety is relatively easier to
achieve (with changes in Chromium code and/or by boxing a build target
in WASM), albeit at some micro-cost to efficiency. (For example, you
have to perform the array bounds check, which might cost more than not
doing it. This is an entirely empirical question that can only be
resolved in the context of a real program, and the results can be
surprising.)

Temporal safety is more difficult to achieve and more expensive.
Solutions include ubiquitous reference counting (e.g. Objective-C with
ARC, Swift), banning shared mutable state and building a borrow-checker
into the compiler (e.g. Rust), or fully generic garbage collection (e.g.
Go, JavaScript, etc.).

We believe that, \emph{given sufficient tolerance for micro-efficiency
regressions}, we could essentially eliminate spatial unsafety in
C++ \emph{in first-party code}. We could do this (and have started doing
so) with a combination of library changes and additions, compiler
options, and policies/style rules and presubmit checks (including banned
and encouraged classes and constructs). Keep in mind that
while \emph{possible} and usually relatively easily technically, this
work is controversial in C++ communities (including Chromium).

We cannot marginalize or eliminate temporal unsafety in C++ without
adopting one of the known solutions (such as GC). Micro-efficient and
ergonomic temporal safety remains an open problem in software
engineering. However, at some (potentially significant) cost to
efficiency, we can reduce the prevalence and exploitability of temporal
unsafety. \href{https://www.google.com/url?q=https://source.chromium.org/chromium/chromium/src/\%2B/master:base/allocator/partition_allocator/starscan/README.md\&sa=D\&source=editors\&ust=1631944129023000\&usg=AOvVaw0WV-iMbo0WDC3QbBfHcu9Z}{*Scan
is a promising possibility} that we are experimenting with.

\subsection{Undefined Behavior}

Much of the problems of C/C++ come from undefined behavior (UB) built
into the language and library specifications. (Even very recent language
additions continue the tradition.) Spatial and temporal safety are
sub-types of UB; other examples include signed integer overflow.
Merely \href{https://www.google.com/url?q=http://www.open-std.org/jtc1/sc22/wg21/docs/papers/2019/p1705r1.html\&sa=D\&source=editors\&ust=1631944129024000\&usg=AOvVaw105cmRp9-saBo3luPsWpiN}{enumerating
all the UB in C++ is an open project}.

\href{https://www.google.com/url?q=https://blog.regehr.org/archives/213\&sa=D\&source=editors\&ust=1631944129024000\&usg=AOvVaw1I5eRSTln2KNNjipgz2Opq}{Compiler
authors see UB as an opportunity for micro-optimization},
while \href{https://www.google.com/url?q=http://www.dullien.net/thomas/weird-machines-exploitability.pdf\&sa=D\&source=editors\&ust=1631944129025000\&usg=AOvVaw0BM_xMKnu3TtvqeuuCOiBX}{attackers
see UB as an opportunity for exploitation}.

For software that runs in an unpredictable and even hostile environment
like the Internet, there is increasingly broad recognition
that \href{https://www.google.com/url?q=https://alexgaynor.net/2020/may/27/science-on-memory-unsafety-and-security/\&sa=D\&source=editors\&ust=1631944129025000\&usg=AOvVaw2BcG3fkxkIPmIk7JDpUpR9}{writing
reliable and safe software in C/C++ is an extreme uphill battle} due to
the many safety- and ergonomics-relevant forms of UB.

\subsection{Purpose Of This Document}

Given this background, our goal in this document is to enumerate some
likely projects Chromium contributors could, should, and might undertake
to reduce the overall exploitable and un-ergonomic UB in Chromium's
usage of C++.

It is not possible to entirely `fix' C++ without fundamentally
redefining it. That is not our goal here. Instead, we want to identify
and reduce some of the most persistent and impactful types of unsafe UB
in Chromium's usage.

\subsubsection{Prioritization And Motivation}

Here's a guide to the relative importance of each of the solutions
discussed in this document. This is pure numbers of bugs, but note that
attackers favor different classes of bugs differently:

\includegraphics[width=5.76806in,height=3.23542in]{./media/image1.png}

(\href{https://www.google.com/url?q=https://docs.google.com/presentation/d/1UVSc_0CHCdj0iv0c23uJZ6OQ_fr9TrISrYmcahJAbIo/edit?ts\%3D5f31d4f7\%23slide\%3Did.g8f91eb8516_0_2\&sa=D\&source=editors\&ust=1631944129027000\&usg=AOvVaw1rTE-lCvOGTOz6siXuLmUe}{source};
Google-internal but the results are reproducible
from \href{https://www.google.com/url?q=https://bugs.chromium.org/p/chromium/issues/list?colspec\%3DID\%2520Pri\%2520M\%2520Stars\%2520ReleaseBlock\%2520Component\%2520Status\%2520Owner\%2520Summary\%2520OS\%2520Modified\%26x\%3Dm\%26y\%3Dreleaseblock\%26cells\%3Dids\%26q\%3DType\%253DBug-Security\%26can\%3D1\%26sort\%3Dpri\&sa=D\&source=editors\&ust=1631944129027000\&usg=AOvVaw2ewSLXG4Wr_Cfq5rqyOSds}{the
public bug tracker})

The proposals and in-progress work described in the rest of this
document are roughly ordered by how much impact they might have on our
most significant bug classes.

\subsubsection{Managing Expectations}

Note that many of these proposed and in-progress projects are quite
large and complex, even changing the semantics of C++. While we think
most of them are necessary, we also know that they are not sufficient.
That is the nature of the C++ problem.

With that, here they are.

\section{Remove/Reduce Raw Pointers}

\textbf{Problem:} Manual lifetime and ownership management has proven
too difficult for even very experienced engineers to reliably do
correctly. This results in UAF bugs and also memory leaks.

\textbf{Solution:} Ban the direct use of raw pointers, new, and delete.
Instead, require developers to use one of the MiraclePtr
implementations, for example. \textbf{Note:} Wrapping pointers in some
syntax is beneficial for most approaches (Oilpan, MiraclePtr, *Scan, et
c.), and is valuable on its own.

\textbf{Current
status:} \href{https://www.google.com/url?q=https://docs.google.com/document/d/1pnnOAIz_DMWDI4oIOFoMAqLnf_MZ2GsrJNb_dbQ3ZBg/edit\%23\&sa=D\&source=editors\&ust=1631944129029000\&usg=AOvVaw0-NEg9SdvTBD7GYlBrO6gZ}{Work
in progress} (as of August 2021) for T* fields in the browser process.
The MiraclePtr project aims to make a smart pointer type that makes UAF
unexploitable while not regressing run-time performance too badly.

\textbf{Costs:} Performance (TBD; as of August 2021 we are actively
quantifying the impact of several implementations). Deviation from C++
language community norms. Non-Chrome C++ is is much less likely to adopt
this approach and hence will not get the temporal safety that MiraclePtr
provides (e.g. Google code pulled into Chrome, open source
dependencies). Difficulty of diagnosing crashes (only one call stack
from MiraclePtr as opposed to three from ASAN: allocate, free, use). If
we wish to apply these protections to 3rd-party code, we may need to
fork repositories such that their other consumers also use non-standard
C++.
(\href{https://www.google.com/url?q=https://docs.google.com/document/d/1VTJAqsQ004ydSPxPcFirM1ukS8uvmR49ppA4OM_9AQI/edit\%23heading\%3Dh.bcg0gbbh7czw\&sa=D\&source=editors\&ust=1631944129030000\&usg=AOvVaw2DZvpSd5eI5011eUSDEuWz}{Most
UAFs happen in 1st-party code}.) Instantaneous cost of rewriting lots
of code (merge conflicts, etc.).

\textbf{Benefits:} UAF accounts for
around \href{https://docs.google.com/document/d/e/2PACX-1vRZr-HJcYmf2Y76DhewaiJOhRNpjGHCxliAQTBhFxzv1QTae9o8mhBmDl32CRIuaWZLt5kVeH9e9jXv/pub\#h.eoikp3r0cwlf}{48\%
of high-severity security bugs} (and
climbing). \href{https://www.google.com/url?q=https://docs.google.com/document/d/1VTJAqsQ004ydSPxPcFirM1ukS8uvmR49ppA4OM_9AQI/edit\%23heading\%3Dh.x9snb54sjlu9\&sa=D\&source=editors\&ust=1631944129031000\&usg=AOvVaw3aIt6uoCrntz84P6hQy0BL}{MiraclePtr
plans to address \textasciitilde50\% of those now}, and might
theoretically cover 20\% more in the future by banning raw pointers in
containers and in local variables and function parameters. (In practice
using MiraclePtr in local variables and tight loops might have a
prohibitive performance cost.) The proposed
implementation, BackupRefPtr, involves the potential runtime cost of
atomic reference counting. (It can still be defeated by pointer
arithmetic and aliasing.)

See \href{https://www.google.com/url?q=https://docs.google.com/document/d/1qsPh8Bcrma7S-5fobbCkBkXWaAijXOnorEqvIIGKzc0/edit\%23heading\%3Dh.j7d3wg2h6goh\&sa=D\&source=editors\&ust=1631944129032000\&usg=AOvVaw2lgrxYIr80YnxorY2-jq0J}{Pointer
Safety Ideas} for more information.

However, Project Zero believes that \textbf{inside renderer processes},
the majority of UAFs are not of the kind MiraclePtr fixes. Instead, they
are iterator invalidation and other lifetime mishaps.

\section{Annotate Lifetimes}

\textbf{Problem:} C++ lifetimes are unknown to the compiler, and
impossible to follow with static analysis.

\textbf{Solutions:} In some cases, we can annotate lifetimes
with {[}{[}clang::lifetimebound{]}{]} in order to tell the compiler the
lifetime is bound to an object.

The attribute has many limitations, meaning it is not a solution for
memory safety, but it can help with some important scenarios. The
limitations are:

\begin{itemize}
\item
  \begin{quote}
  There is no way to distinguish between different lifetimes.
  \end{quote}
\item
  \begin{quote}
  There is no way to annotate a static lifetime.
  \end{quote}
\item
  \begin{quote}
  The attribute attaches to function parameters and always implicitly
  refers to the outermost reference-like type; it is not possible to
  attach it to part of a type (e.g. to the T* in a const
  std::vector\textless T*\textgreater\&).
  \end{quote}
\item
  \begin{quote}
  The single lifetime is implicitly applied to the outermost
  reference-like type in the function's return type (or the value of the
  constructed object, in the case of a constructor). Again, it is not
  possible to associate the lifetime with inner reference types in the
  return value (e.g. the T* in const
  std::vector\textless T*\textgreater\&).
  \end{quote}
\item
  \begin{quote}
  There is no way to add a lifetime parameter to a struct. This means
  that parameters can only be attached to the object's lifetime if they
  are given to a constructor, not in other setters.
  \end{quote}
\end{itemize}

In theory, we should apply this to:

\begin{itemize}
\item
  \begin{quote}
  Any constructor reference parameter that is stored in a field.
  \end{quote}
\end{itemize}

\begin{itemize}
\item
  \begin{quote}
  However it misses even trivial examples:
  \end{quote}
\end{itemize}

\begin{itemize}
\item
  \begin{quote}
  \href{https://www.google.com/url?q=https://godbolt.org/z/Ysq41G6vb\&sa=D\&source=editors\&ust=1631944129035000\&usg=AOvVaw0Rnch-j0apUhdw0LM5stFJ}{https://godbolt.org/z/Ysq41G6vb}
  \end{quote}
\end{itemize}

\begin{itemize}
\item
  \begin{quote}
  Any constructor pointer parameter that can be held in a const member.
  (In other words, the pointer is never reassigned.)
  \end{quote}
\end{itemize}

\begin{itemize}
\item
  \begin{quote}
  However it misses even trivial examples:
  \end{quote}
\end{itemize}

\begin{itemize}
\item
  \begin{quote}
  \href{https://www.google.com/url?q=https://godbolt.org/z/Ma7P8q8WG\&sa=D\&source=editors\&ust=1631944129036000\&usg=AOvVaw31wGuAGhkcA9WvYkZ-0CG7}{https://godbolt.org/z/Ma7P8q8WG}
  \end{quote}
\end{itemize}

\begin{itemize}
\item
  \begin{quote}
  Any class method that returns a reference or pointer to a class member
  (but not to pointers inside class members, unfortunately).
  \end{quote}
\end{itemize}

\begin{itemize}
\item
  \begin{quote}
  However it misses even trivial examples:
  \end{quote}
\end{itemize}

\begin{itemize}
\item
  \begin{quote}
  \href{https://www.google.com/url?q=https://godbolt.org/z/9er4WE6zK\&sa=D\&source=editors\&ust=1631944129036000\&usg=AOvVaw2Kh9RhXzwFffwRp6Fm6Y0s}{https://godbolt.org/z/9er4WE6zK}
  \end{quote}
\item
  \begin{quote}
  \href{https://www.google.com/url?q=https://godbolt.org/z/GW9j4zrdT\&sa=D\&source=editors\&ust=1631944129037000\&usg=AOvVaw1nrVEhav1-RhjoWdbBOu8e}{https://godbolt.org/z/GW9j4zrdT}
  \end{quote}
\end{itemize}

\begin{itemize}
\item
  \begin{quote}
  Any function that receives references or pointers and returns a
  reference or pointer back to its input. This includes templated
  functions that return one of their inputs (such as min/max/clamp).
  \end{quote}
\end{itemize}

The only cases this attribute only actually catches at the moment
are \href{https://www.google.com/url?q=https://reviews.llvm.org/D49922\%23inline-440637\&sa=D\&source=editors\&ust=1631944129038000\&usg=AOvVaw0ZgDB7wthr0uf0eXbnCh7s}{invalid
use of temporaries}. While this is a valid/important memory safety bug
when it happens, it is not representative of the type of bug we see in
our UAF security bugs.

\begin{itemize}
\item
  \begin{quote}
  And it does not even catch all temporaries:
  \end{quote}
\end{itemize}

\begin{itemize}
\item
  \begin{quote}
  \href{https://www.google.com/url?q=https://godbolt.org/z/PbnM1TY8n\&sa=D\&source=editors\&ust=1631944129038000\&usg=AOvVaw2rfmJ9A2vbf5NxysSnFJQD}{https://godbolt.org/z/PbnM1TY8n}
  \end{quote}
\item
  \begin{quote}
  \href{https://www.google.com/url?q=https://godbolt.org/z/aMbaq4W55\&sa=D\&source=editors\&ust=1631944129039000\&usg=AOvVaw2QHvbnCGBCtLIR1zR43GFJ}{https://godbolt.org/z/aMbaq4W55}
  \end{quote}
\item
  \begin{quote}
  \href{https://www.google.com/url?q=https://godbolt.org/z/9G77fcE18\&sa=D\&source=editors\&ust=1631944129039000\&usg=AOvVaw298-Iy0OCDeOVx7H82I5J0}{https://godbolt.org/z/9G77fcE18}
  \end{quote}
\end{itemize}

\begin{itemize}
\item
  \begin{quote}
  In fact it's hard to construct an example that it does catch.
  \end{quote}
\end{itemize}

Some key potential places to do this are:

\begin{itemize}
\item
  \begin{quote}
  base::span constructor
  \end{quote}
\item
  \begin{quote}
  base::StringPiece constructor
  \end{quote}
\item
  \begin{quote}
  base::clamp
  \end{quote}
\item
  \begin{quote}
  ????
  \end{quote}
\item
  \begin{quote}
  Reference/pointer-returning methods everywhere --- but only if we can
  show that the attribute actually helps (see above godbolts for
  counter-examples).
  \end{quote}
\end{itemize}

Note that base::span is designed to be able to hold an invalid pointer
past the end of the container it's ``pointing'' to, and the lifetime
analysis can not help with this problem. This is a spatial memory safety
problem built into C++. But it could potentially help with using
the base::span beyond the lifetime of the object it points to, if the
attribute caught more misuses.

\textbf{Current status:} Not started.

\textbf{Costs:} Visually noisy annotations (the Abseil macro is a
mouthful) present in the code. We will learn to become familiar with
them. But there are many places where the annotation can not be used.
The annotation can be written incorrectly, where it is more strict than
the object requires, or it may become incorrect over time if an object
is changed. For instance, if base::span grew a method to re-assign the
pointers.

Since the attribute is defined to generate a warning \emph{if} a
violation can be spotted, it does not actually guarantee violations are
caught. And it seems that most are not (see godbolt examples above).
This can give a false impression of safety, which may lead to developers
trying to rely on the attribute catching their mistakes and actually
writing \emph{more }UAFs.

Furthermore, there is no way to enforce that the annotations are present
where they are possible, which would allow new code to be written
without them. Code authors that rely on annotations to check their
correctness would be left without any checks.

\textbf{Benefits:} Compiler errors when some set of simple lifetime
errors are written.

\section{Implement Automatic Memory Management}

\textbf{Problem:} Temporal safety and correctness (UAF, leaks).

\textbf{Solutions:} Reference counting
(e.g. \href{https://www.google.com/url?q=https://en.wikipedia.org/wiki/Automatic_Reference_Counting\&sa=D\&source=editors\&ust=1631944129042000\&usg=AOvVaw0oMAe91IrprtUI2zFzkmi7}{ARC}-like
semantics), and/or full GC.

\textbf{Current status:} Oilpan is now a generic reusable library (no
longer special to Blink),
and \href{https://www.google.com/url?q=https://bugs.chromium.org/p/pdfium/issues/detail?id\%3D1563\&sa=D\&source=editors\&ust=1631944129042000\&usg=AOvVaw1uctKa94n_u0n9AU7jui4R}{we }\href{https://www.google.com/url?q=https://bugs.chromium.org/p/pdfium/issues/detail?id\%3D1563\&sa=D\&source=editors\&ust=1631944129043000\&usg=AOvVaw0MGd1-LSOfDe38KK48Tt0p}{have
adopted it in PDFium} to resolve many or most of our temporal safety
problems in that project. This enabled us to ship XFA Forms support in
production, for example! (Currently off by default, due to functionality
gaps.)

\textbf{Costs:} Reference counting needs to be atomic, which costs
micro-time. Fully generic GC can be expensive.

\textbf{Benefits:} UAFs account for
around \href{https://docs.google.com/document/d/e/2PACX-1vRZr-HJcYmf2Y76DhewaiJOhRNpjGHCxliAQTBhFxzv1QTae9o8mhBmDl32CRIuaWZLt5kVeH9e9jXv/pub\#h.eoikp3r0cwlf}{48\%
of high-severity security bugs} (and climbing). This approach is an
alternative to the universal application of checked pointer types
(\href{https://docs.google.com/document/d/e/2PACX-1vRZr-HJcYmf2Y76DhewaiJOhRNpjGHCxliAQTBhFxzv1QTae9o8mhBmDl32CRIuaWZLt5kVeH9e9jXv/pub\#h.c3notccb295u}{see
above}). Additionally, GC has excellent developer ergonomics.

\section{Implement Ownership Analysis}

\textbf{Problem:} Temporal safety and correctness (UAF, leaks).

\textbf{Solutions:} Enforce at run-time that there is a single `owner'
of any object, which can only be changed via std::move. Allow `borrows'
with Rust-like rules which prevent multiple mutable references existing
at the same time, and ensures objects aren't accessed beyond their
lifetime bounds.

This solution seems a poor fit to C++, but it keeps being proposed so it
seems important to discuss it here.

Rust achieves this model through fairly
complex \href{https://www.google.com/url?q=https://doc.rust-lang.org/edition-guide/rust-2018/ownership-and-lifetimes/non-lexical-lifetimes.html\&sa=D\&source=editors\&ust=1631944129045000\&usg=AOvVaw0Sc7SumvQIHrlzjTV7oqUB}{compiler
support}. The majority of objects therefore incur no runtime cost for
this sort of ownership checking; it's all static. Developers can
optionally instead use a runtime version
(\href{https://www.google.com/url?q=https://doc.rust-lang.org/std/cell/struct.RefCell.html\&sa=D\&source=editors\&ust=1631944129045000\&usg=AOvVaw2-AksYW18PTZd07aOxYU5o}{RefCell\textless\textgreater{}})
which does the same checks at runtime. We presume this model would be
far too expensive if \emph{every} object were tracked at runtime,
and \href{https://www.google.com/url?q=https://docs.google.com/document/d/1oVTxJ-4VItkcA7rAMylIW74SOmKsnc4aS6bylr1B8ZY/edit?resourcekey\%3D0-RNrtKRt8CQ_BgGdGzrDwUA\&sa=D\&source=editors\&ust=1631944129046000\&usg=AOvVaw1vKF2_PTEYFrWd4ikG4Sf1}{we
don't see a way to do static build-time enforcement in C++ without
radical compiler and language changes}. (Clang has
added \href{https://www.google.com/url?q=https://reviews.llvm.org/rL338464\&sa=D\&source=editors\&ust=1631944129046000\&usg=AOvVaw0A8ddFZa3y3m3AXUyQIMD0}{lifetime
bounds} for simple
cases, \href{https://docs.google.com/document/d/e/2PACX-1vRZr-HJcYmf2Y76DhewaiJOhRNpjGHCxliAQTBhFxzv1QTae9o8mhBmDl32CRIuaWZLt5kVeH9e9jXv/pub\#h.j8svxi9w9c1i}{but
see above}).

\textbf{Current status:} We
have \href{https://www.google.com/url?q=https://docs.google.com/document/d/1YnJBXshxxpUkgzgEsuGTvxsFqVq_YFzsueQHxpbDZJU/edit\&sa=D\&source=editors\&ust=1631944129047000\&usg=AOvVaw2B9xqIF8Z6_YCzRSAf9YY1}{some
early experiments at runtime ownership enforcement}. Compile-time
safety
is \href{https://www.google.com/url?q=https://docs.google.com/document/d/1oVTxJ-4VItkcA7rAMylIW74SOmKsnc4aS6bylr1B8ZY/edit?usp\%3Dsharing\%26resourcekey\%3D0-RNrtKRt8CQ_BgGdGzrDwUA\&sa=D\&source=editors\&ust=1631944129047000\&usg=AOvVaw2FvCsUz7Sn-st7JCiyuSzw}{infeasible} without
fundamental changes to C++ such as new reference types. There is work
for limited safety
in \href{https://www.google.com/url?q=https://youtu.be/80BZxujhY38?t\%3D1096\&sa=D\&source=editors\&ust=1631944129047000\&usg=AOvVaw3izQ8dqJsUTUsFa7E7ou28}{clang
warnings that will catch dangling references} through control flow
analysis, but these will not catch invalid heap pointers by design.

\textbf{Costs:} Runtime costs equivalent to reference counting. Need to
distinguish `owner\_ptr' from `borrowed\_ptr'.

\textbf{Benefits:} UAFs account for
around \href{https://docs.google.com/document/d/e/2PACX-1vRZr-HJcYmf2Y76DhewaiJOhRNpjGHCxliAQTBhFxzv1QTae9o8mhBmDl32CRIuaWZLt5kVeH9e9jXv/pub\#h.eoikp3r0cwlf}{48\%
of high-severity security bugs} (and climbing). This approach is an
alternative to the universal application of checked pointer types.

\section{Use \texttt{-Wdangling-gsl}}

Google has had good results with this internally, finding and fixing UAF
bugs. Chrome should be able to try it out, too. There are some false
positives, but plenty of true positives.

\section{Define All Standard Library Behaviors}

(Where possible.)

\textbf{Problem:} The standard library is riddled with potentially
exploitable undefined behavior. This includes lack of bounds checking
(e.g. \href{https://www.google.com/url?q=https://en.cppreference.com/w/cpp/container/span/operator_at\&sa=D\&source=editors\&ust=1631944129049000\&usg=AOvVaw0ppTtrO7NVxYBnq9-OMQdg}{std::span::operator{[}{]}})
and lack of validity checking
(e.g. \href{https://www.google.com/url?q=https://en.cppreference.com/w/cpp/utility/optional/operator*\&sa=D\&source=editors\&ust=1631944129049000\&usg=AOvVaw23V3p_TRgf2pMCe7iZc9tZ}{std::optional::operator*}). \href{https://www.google.com/url?q=https://github.com/isocpp/CppCoreGuidelines/issues/1038\&sa=D\&source=editors\&ust=1631944129050000\&usg=AOvVaw3BLCBEKIo0Kyb9KcOajrgL}{std::string\_view's
unfortunate affordance for UAF} is a separate problem, though. This is
especially unfortunate in the recent library additions, because
unsafe-but-fast options were already available.

Since std is specified to have lots of UB, we cannot easily be and
remain certain that implementations we use will be fully hardened or
easily hardenable against UB, especially as new features are added.
Instead, we should use a std(-like) replacement whose design and
implementation we can more effectively influence, such as Abseil.
Alternatively, we could dedicate headcount to working with upstream
libcxx to ensure a hardened mode is robust and supported.

\textbf{Solutions:} Add a `hardened' mode (selectable at compile time)
to standard library implementations that allows us to make the undefined
behavior well-defined and safe. This is fairly `easy' for spatial
safety; for temporal safety, see above.

\textbf{Current status:} Abseil team have already added a
spatial safety hardening mode to Abseil. It perhaps could use a
completeness audit, but as of August 2021 it looks pretty
good. {\href{https://www.google.com/url?q=https://source.chromium.org/chromium/chromium/src/\%2B/main:third_party/abseil-cpp/absl_hardening_test.cc;l\%3D19?q\%3Dabsl_harden\%26ss\%3Dchromium\&sa=D\&source=editors\&ust=1631944129051000\&usg=AOvVaw0cHRqJzTxMfueFZGG_xz-J}{We
use }\href{https://www.google.com/url?q=https://source.chromium.org/chromium/chromium/src/\%2B/main:third_party/abseil-cpp/absl_hardening_test.cc;l\%3D19?q\%3Dabsl_harden\%26ss\%3Dchromium\&sa=D\&source=editors\&ust=1631944129052000\&usg=AOvVaw1jFEYauM_cOOMNfTUUADa6}{this
mode in Chromium}. A similar hardening mode for LLVM libcxx is
in \href{https://www.google.com/url?q=https://bugs.chromium.org/p/chromium/issues/detail?id\%3D923166\&sa=D\&source=editors\&ust=1631944129053000\&usg=AOvVaw2CgUfuKxGKgWkZIOQ3Nq9y}{progress} (\href{https://www.google.com/url?q=https://reviews.llvm.org/D89353\&sa=D\&source=editors\&ust=1631944129053000\&usg=AOvVaw3JKAAvS4gV2e6p1iXVACSs}{upstream}).
We have also added spatial hardening to //base (but could use a
completeness
audit). \href{https://www.google.com/url?q=https://source.chromium.org/chromium/chromium/src/\%2B/main:third_party/blink/renderer/platform/wtf/\&sa=D\&source=editors\&ust=1631944129053000\&usg=AOvVaw0kkd5EAIlI1FKFlepB4rBL}{WTF} also
has the same status as //base.

We are
also considering a \href{https://www.google.com/url?q=https://github.com/chromium/libboring\&sa=D\&source=editors\&ust=1631944129054000\&usg=AOvVaw0Wl6dhQfxK9NJixftXrm63}{project
to build a standard-like library with no UB} {[}Google-internal for
now, sorry{]}, since there is not much appetite for making //base
stand-alone. However, Abseil with hardening may obviate that. But if
there is general interest for an open source, std(-like) library that is
specified to have no UB, we could dedicate headcount to that.

\textbf{Costs:} Possible micro-cost in run-time due to increased
checking.

\textbf{Benefits:} \href{https://docs.google.com/document/d/e/2PACX-1vRZr-HJcYmf2Y76DhewaiJOhRNpjGHCxliAQTBhFxzv1QTae9o8mhBmDl32CRIuaWZLt5kVeH9e9jXv/pub\#h.eoikp3r0cwlf}{Spatial
unsafety is 16\% of high severity security bugs; possibly 17.5\%}.

\subsection{Define Undefined Iterator Behaviors}

\textbf{Problem:} In particular, it seems important to mention 2 bug
classes involving UB in iterators. Thanks to Sergei and Mark from P0 for
raising these points:

\begin{minted}{cpp}
for (auto& iter : my_container) {
  MaybeChangeMyContainer();
}
\end{minted}

and

\begin{minted}{cpp}
auto iter = my_container.find(the_thing);
DCHECK(iter != my_container.end());
iter->second->Foo();
\end{minted}

Mark says:

\begin{quote}
It seems like the iterator invalidation problem could be solved
efficiently.

EITHER by having the container track live iterators and "neuter" any
live iterators when an iterator-invalidating operation occurs - it
should be rare that such an operation occurs with many live iterators,
so this should be fairly inexpensive, and would incur zero overhead on
iterator access.

OR using something like a generation tag, which would be checked on
iterator access; this would add iterator access overhead so it might be
too expensive?

OR more API-breakingly, we could simply CHECK on iterator-invalidating
operations when there are live iterators - this would be cheaper but
would likely require significant testing and code changes to ensure that
on-stack iterators are discarded once they're not being used.
\end{quote}

\textbf{Solution:} We
have \href{https://www.google.com/url?q=https://source.chromium.org/chromium/chromium/src/\%2B/master:base/containers/checked_iterators.h\&sa=D\&source=editors\&ust=1631944129057000\&usg=AOvVaw13QYLercw7YJFR6fuENUX8}{a CheckedIterator type
in //base}.

\textbf{Current status:} It had been expensive in practice (due to a
lack of a supported way in libcxx to express that it can be optimized),
but \href{https://www.google.com/url?q=https://bugs.chromium.org/p/chromium/issues/detail?id\%3D994174\%23c15\&sa=D\&source=editors\&ust=1631944129058000\&usg=AOvVaw30V7RnIENr3gQFMdSLCPVt}{that
is fixed now}. We should expand its use now that we can do so
efficiently. For example, it should be possible to create a
well-defined end singleton template that crashes cleanly.

\textbf{Costs:} Hopefully, the run-time overhead is acceptable now.

\textbf{Benefits:} Reduced iterator invalidation UB (including spatial
and temporal unsafety).

\section{Define Integer Semantics}

\textbf{Problem:} \href{https://www.google.com/url?q=https://chromium.googlesource.com/chromium/src/\%2B/master/docs/security/integer-semantics.md\&sa=D\&source=editors\&ust=1631944129059000\&usg=AOvVaw1RJ_W19UD6mSpv0PStkjwY}{C/C++'s
integer semantics are bonkers}: the wrapping, overflow, underflow,
undefined behavior, implicit casting, and silent truncation behaviors
all add up to unsafety and poor ergonomics. As a result, developers have
a hard time correctly calculating sizes, indices, and offsets,
especially when an attacker can control some of the terms. Arithmetic
overflow and underflow often lead to mistakes in memory allocation and
access, and from there to exploitable bugs. Other bug classes arise from
integer overflow too, such
as \href{https://www.google.com/url?q=https://twitter.com/tehjh/status/1045000957678047232?lang\%3Den\&sa=D\&source=editors\&ust=1631944129060000\&usg=AOvVaw3ygs6qYV6oSRM8GtjPepzo}{\uline{reference
counts wrapping}}, or wrapping causing unique IDs to no longer be
unique.

Implicit conversion from integer to floating point hides the fact the
stored value potentially changes. It's insidious as within common ranges
the value does not change, but if an attacker can control the value of
the integer, they can make it large enough to violate the assumption.
Then on conversion back to an
integer, \href{https://www.google.com/url?q=http://crbug.com/278141\&sa=D\&source=editors\&ust=1631944129060000\&usg=AOvVaw0wnyxdxDMCXQ8kLCOWjHyU}{\uline{the
result becomes invalid}}.

\textbf{Solution 1:} Require developers to use the //base/numerics
library or something similar. Specify specific types for intentional
wrapping, saturating, and trapping (as Rust does). The norm should be
that people use reliable arithmetic by default, and leave primitive C
arithmetic behind to the greatest extent possible. In particular, we
should dedicate some headcount to improving the generality and
ergonomics of //base/numerics, and should make it into a stand-alone
dependency. (It already is easily separable from //base, but you have to
copy and paste.)

\textbf{Solution 2:} We could require compiler options to make signed
overflow behave the same as unsigned
(i.e. \href{https://www.google.com/url?q=https://clang.llvm.org/docs/ClangCommandLineReference.html\&sa=D\&source=editors\&ust=1631944129061000\&usg=AOvVaw2PGnWR5j__AwLcxq4JWAHj}{\uline{wrapping}}).
That is, we could standardize on the Java and Go behavior: we could
use -fwrapv in debug and production builds. Alternatively, we could
use -fwrapv in release builds and -ftrapv in debug builds (like Rust).

\textbf{Solution 3:} Clang also
has \href{https://www.google.com/url?q=https://clang.llvm.org/docs/UndefinedBehaviorSanitizer.html\&sa=D\&source=editors\&ust=1631944129062000\&usg=AOvVaw1WcT_P8zjRqDFzV4ZI1TJT}{\uline{sanitizer
options}} --- which can be configured to immediately trap, thus
requiring no run-time support --- to handle division by 0, truncation,
implicit casting, and shifting left too far, casting an integer to an
invalid enum value.

Android already
uses -fsanitize=signed-integer-overflow,unsigned-integer-overflow in
large (and growing) parts of the codebase.

enh@ notes: ``In combination with fuzzing it works quite well to show
you where you need \_\_builtin\_add\_overflow or whatever. Without
fuzzing it's a `good' source of work backporting security fixes as/when
stuff is found in the field.''

\textbf{Solution 4:} Clang provides a warning on implicit int conversion
to float, behind -Wimplicit-int-float-conversion. We should enable this
warning.

\textbf{Current status:} //base/numerics is used in many places
successfully. We just need to use it more. The API needs some ergonomic
improvements.

We do not use either -ftrapv or -fwrapv in any .gn or .gni file.
We \href{https://www.google.com/url?q=https://crbug.com/989932\&sa=D\&source=editors\&ust=1631944129063000\&usg=AOvVaw2EeNXbT-C5RHlx1OGnSA00}{\uline{have
disabled}} the -Wimplicit-int-float-conversion \href{https://www.google.com/url?q=https://source.chromium.org/chromium/chromium/src/\%2B/main:build/config/compiler/BUILD.gn;drc\%3Dab531c265c533cba1c2f6d8240cc0bf7679f605a;l\%3D1611\&sa=D\&source=editors\&ust=1631944129063000\&usg=AOvVaw0y22xRHB_K05IyKVdiN47i}{\uline{warning}}.

\uline{\href{https://www.google.com/url?q=https://source.chromium.org/search?q\%3Dsanitize\%253Dsigned-integer-overflow\%26sq\%3D\%26ss\%3Dchromium\&sa=D\&source=editors\&ust=1631944129064000\&usg=AOvVaw3X2yYb0iYbcUwuPk3SlZUw}{Build
profiles that
use is\_ubsan}\href{https://www.google.com/url?q=https://source.chromium.org/search?q\%3Dsanitize\%253Dsigned-integer-overflow\%26sq\%3D\%26ss\%3Dchromium\&sa=D\&source=editors\&ust=1631944129065000\&usg=AOvVaw1TV5lWdf0gJaEL47oeTXN4}{ sanitize
signed int overflow}},
and \href{https://www.google.com/url?q=https://source.chromium.org/chromium/chromium/src/\%2B/master:tools/ubsan/blacklist.txt?q\%3Dblacklist.txt\%26ss\%3Dchromium\&sa=D\&source=editors\&ust=1631944129065000\&usg=AOvVaw3fCUMhabQSj7I8LGKJMnqC}{\uline{with
a significant block list}}. It does not seem to be on in production
builds of Chrome.

\textbf{Costs:} Training. Migrating code. Some 3rd-party projects (e.g.
Skia) resist systemic solutions. Potential for micro-efficiency
regression if people use checked arithmetic in tight loops. Potential
for binary size increase if we ship UBSan with trapping (which does not
require the UBSan runtime support library and produces small,
coalescable branch targets on failure).

Assuming overflow behavior is a significant change in C/C++ semantics.
(LLVM developers for example try to avoid introducing new semantics with
command-line options; but some already exist out of necessity.) If
developers come to rely on well-defined integer behavior, code can
become buggy if anyone were to turn the option off. (We can, and should
in any case, protect against this with tests.) Using explicit types for
trapping, wrapping, and saturating avoids that, but doesn't easily work
for 3P dependencies and requires explicit changes to call sites.

\textbf{Benefits:} \href{https://docs.google.com/document/d/e/2PACX-1vRZr-HJcYmf2Y76DhewaiJOhRNpjGHCxliAQTBhFxzv1QTae9o8mhBmDl32CRIuaWZLt5kVeH9e9jXv/pub\#h.eoikp3r0cwlf}{\uline{Integer
overflow represents around 2\% of our high severity security bugs}},
though arguably that's reduced in importance if we truly manage to
prevent buffer overflows (see later). Using -fwrapv may and should
be maximally compatible with existing code, and matches most developers'
expectations. Using UBSan with trap-on-failure covers the most problems
but may require some carve-outs and may introduce some speed and binary
size regressions (an empirical question). Call sites that need explicit
checking should continue to use //base/numerics in any case.

There are also logical bugs, such as an expectation that incrementing
numbers will remain unique as identifiers, reference counters wrapping,
and so on. Again, trapping or sanitizing would catch these. With
a Trapping\textless T\textgreater in //base/numerics, we could
statically ensure that.

Having defined behavior and skipping odd optimizations based on
undefined integer behavior might also improve ergonomics.

\section{Set Pointers To Null After Free}

\textbf{Problem:} The contents of a region of memory after free are
undefined. That is confusing and potentially exploitable.

\textbf{Solution:} kcc@ notes: ``Another potential investigation is
nullifying pointers after free (by compiler). After delete
foo-\textgreater bar, add foo-\textgreater bar = nullptr. Obviously,
it'll fix a small portion of cases (guesstimate: 1\% -- 10\%); e.g. it
can't handle delete GetBar();. But it's \textasciitilde{} zero overhead
and relatively easy to implement. LLVM patches have been floating around
(but IDK the current state).''

This will also help make any GC-based approaches more efficient.

\textbf{Current status:} None.

\textbf{Costs:} kcc@ says \textasciitilde{} zero.

\textbf{Benefits:} Detect 1 -- 10\% of UAFs. Improved developer
ergonomics (modulo aliasing, the contents of a region after free and
before it is reused are now defined).

\section{Define Null Pointer Dereferences}

\textbf{Problem:} Null pointer dereferences are UB. This is an issue
because developers (reasonably) expect a null pointer dereference to
crash the process instead of continuing. However, the compiler can and
sometimes will optimize away the null pointer dereference and in some
cases to elide a check for it, even though continuing execution might
result in a much more corrupted state and possibly exploitable behavior.

For example, our smart pointer type WeakPtr was vulnerable to UAFs: if
the pointed-to object was destroyed, WeakPtr::get would return a null
pointer and the subsequent dereference was supposed to crash the
program. However, clang correctly determined that storing a null pointer
and immediately dereferencing that pointer was undefined behavior, and
therefore removed the store of null pointer entirely.
So WeakPtr::get would actually return the stale pointer and the
dereference would instead result in a UAF. There has been at
least \href{https://www.google.com/url?q=https://bugs.chromium.org/p/chromium/issues/detail?id\%3D1133635\&sa=D\&source=editors\&ust=1631944129071000\&usg=AOvVaw0nhje5VI4xjTFlE1-QftQ3}{\uline{1
externally-reported high-severity security bug due to this
issue}} (\href{https://www.google.com/url?q=https://chromium.googlesource.com/chromium/src/\%2B/0b308a0e37b9d14a335c3b487511b7117c98d74b\&sa=D\&source=editors\&ust=1631944129071000\&usg=AOvVaw3Bagjs8VlafIjNTAqrnEZF}{\uline{fixed
with an explicit CHECK}}).

\textbf{Solution:} Clang provides a compiler flag
called -fno-delete-null-pointer-checks (named as such for historical
reasons) that defines null pointer dereferences. With this flag,
dereferences of null are never optimized away.

\textbf{Current
status:} \href{https://www.google.com/url?q=https://chromium-review.googlesource.com/c/chromium/src/\%2B/2481465\&sa=D\&source=editors\&ust=1631944129072000\&usg=AOvVaw1TiZFMeA5sgo2qoM3LqftV}{\uline{Landed}}.

\textbf{Cost: }42 kB Android binary size (at minimum)
and \href{https://www.google.com/url?q=https://crbug.com/1149340\&sa=D\&source=editors\&ust=1631944129072000\&usg=AOvVaw2VVH7EMtibHKBKRTY50Qwe}{\uline{some
microbenchmark regressions in Blink parsing performance}}.

\textbf{Benefits:} With this flag, the compiler behaves the way most
developers expect, making it easier to understand the meaning of code.

\section{Require Coding Patterns To Reduce Lifetime Errors}

\textbf{Benefits:} UAFs account for
around \href{https://docs.google.com/document/d/e/2PACX-1vRZr-HJcYmf2Y76DhewaiJOhRNpjGHCxliAQTBhFxzv1QTae9o8mhBmDl32CRIuaWZLt5kVeH9e9jXv/pub\#h.eoikp3r0cwlf}{\uline{48\%
of high-severity security bugs}} (and climbing). Improved coding
patterns are not a robust solution, since they're still subject to human
error. But they may eliminate some fraction of the bugs. When combined
with a robust solution such as a deterministic MiraclePtr, they may
remove some (non-exploitable) crashes.

\subsection{Use absl::variant Instead Of }enum\textbf{s For State Machines}

\textbf{Problem:} enums are often used for state machines.
Unfortunately, the enum variant is not the only bit of the state ---
there are almost always extra fields which pertain to a subset of the
states. These things get out of sync, causing object lifetime problems
and logic errors.

\begin{minted}{cpp}
struct StateMachine {
   enum {
      CONNECTING,
      CONNECTED,
      DISCONNECTING,
   } state;

   // These fields could get out of sync with `state':
   int thing_relevant_only_when_connected;
   std::string thing_relevant_only_after_connection;
};
\end{minted}

\textbf{Solution:} Use absl::variant, which is a type-safe tagged
union. \href{https://www.google.com/url?q=https://genbattle.bitbucket.io/blog/2016/10/07/Type-Safe-Unions-in-C-and-Rust/\&sa=D\&source=editors\&ust=1631944129075000\&usg=AOvVaw1q7XscklQWyKhOkY9vFTAF}{\uline{All
data which relates only to one of the states should be associated with
that specific variant}}.

\textbf{Current
status:} absl::variant is \href{https://www.google.com/url?q=https://groups.google.com/a/chromium.org/g/chromium-dev/c/sKVWxxSjVFU/m/MWqjjz9CAwAJ\&sa=D\&source=editors\&ust=1631944129076000\&usg=AOvVaw3UMonBEuD7BXE6zG71T5Lw}{\uline{newly
allowed}}. No attempt has been made to retrofit to existing code.

\begin{minted}{cpp}
struct Connecting { int thing; }
struct Connected { std::string thing; }
struct Disconnecting { std::string thing; }

auto state_machine = absl::variant<Connecting, Connected,
  Disconnecting>;
\end{minted}

\textbf{Costs:} Awkward syntax (arguable). Difficulty of identifying
which enums are used for state machines. (Can we simply ban all enums?)

\textbf{Benefits:} Reduced logic errors and object lifetime errors,
currently unquantified.

\subsection{Ban std::unique\_ptr::get; Use Shared Pointers}

\textbf{Problem:} unique\_ptr encourages the notion that there's a
single owner, yet we see such pointers featuring in use-after-free bugs
so this notion is obviously wrong. (unique\_ptr really guarantees a
unique \emph{deleter}, not necessarily a unique \emph{owner}.)

\textbf{Solution:} Prevent any means of getting a raw pointer out of
a unique\_ptr (to the extent possible). Not even a checked pointer: if
developers are getting any extra pointers to something within a unique
pointer, then it's not truly uniquely owned, and they should use a
shared pointer. (Yes, we really do have to incur the costs of reference
counting.) And in most cases where unique\_ptr is used, it might be
better to use base::Optional to obtain composition into a single heap
cell.

\textbf{Current status:} Opposite of current best practice, where shared
pointers are discouraged and unique\_ptr encouraged.

Note that dcheng has a countervailing view: that we instead want clarity
of lifetime and the ability to assert ownership more clearly than
shared\_ptr/reference counting/GC allows:

\begin{quote}
What I think we really need is a safe version of raw pointer to make
lifetime assertions when we believe an object should have single
ownership. There was a previous attempt at this called CheckedPtr
(though that kind of conflicts with MiraclePtr's implementation details
now)... maybe we should seriously consider it though, as this seems to
be a repeated theme.
\end{quote}

\textbf{Costs:} We may find lots of objects need reference counting. But
we need to do that for safety. Reduced clarity about lifetimes and when
an object's destructor would run. It would also be easier to create
reference cycles.

\textbf{Benefits:} Reduced object lifetime errors, currently
unquantified. Fewer heap allocations and dereferences if we use
composition more often than a pointer.

\section{Initialize All Memory}

\textbf{Problem:} When a program uses variables before they have been
initialized, bugs ensue. These can include (possibly exploitable) wild
pointer dereferences and information disclosure bugs. Using
uninitialized memory may also introduce application-semantic bugs that
may or may not be security vulnerabilities.

(Information disclosure bugs can occur when the struct has padding, and
the code memcpys a struct and sends the result to another process. The
padding may be uninitialized, which is to say, whatever data was there
previously --- and perhaps that data is sensitive. We have
had \href{https://www.google.com/url?q=https://bugs.chromium.org/p/chromium/issues/detail?id\%3D765512\&sa=D\&source=editors\&ust=1631944129080000\&usg=AOvVaw1rvWaqATRYqrayvsC28EmW}{\uline{bugs
like this}} in the past. Of course, the proper fix for such information
disclosure bugs is to properly serialize structs, initializing the
whole struct first is a good defense in depth.)

\textbf{Solutions:} In addition to being nice, well-defined behavior
(and hence good for developer ergonomics, as Go has shown), initializing
all memory (either to 0 or some canary value) eliminates wild pointer
and application-semantic bugs arising from the use of uninitialized
memory.

Alternatively, we could configure the compiler to reject variable
declarations that have no initializer.

\textbf{Current status:} vitalybuka@ has
painstakingly \href{https://www.google.com/url?q=https://source.chromium.org/chromium/chromium/src/\%2B/master:build/config/compiler/BUILD.gn;l\%3D2583?q\%3Dtrivial-auto-var-init\%26ss\%3Dchromium\&sa=D\&source=editors\&ust=1631944129081000\&usg=AOvVaw3YUck07c2KvA4COd6OlcrS}{\uline{enabled
stack auto-initialization}},
on \href{https://www.google.com/url?q=https://source.chromium.org/chromium/chromium/src/\%2B/master:build/config/compiler/BUILD.gn;l\%3D135?q\%3Dinit_stack_vars\%26ss\%3Dchromium\&sa=D\&source=editors\&ust=1631944129081000\&usg=AOvVaw3V7aEFva0hRGlT4R3821dt}{\uline{non-Android}},
with \href{https://www.google.com/url?q=https://source.chromium.org/search?q\%3Ddefault_init_stack_vars\%26sq\%3D\%26ss\%3Dchromium\&sa=D\&source=editors\&ust=1631944129082000\&usg=AOvVaw3haxLM_dqlDewzY0dHbkc8}{\uline{some
carve-outs}}, and has worked for months to solve performance regressions
by excluding hot paths.

enh@ says:

\begin{quote}
Android R has stack zero initialization for kernel and userspace. We
found relatively few places in userspace that we needed to annotate for
performance. Folks are looking at heap zero initialization for S. That
will probably be harder.
\end{quote}

Future work:

\begin{itemize}
\item
  \begin{quote}
  Heap auto-init (currently it is stack-only)
  \end{quote}
\item
  \begin{quote}
  Try to push for zero-init again
  \end{quote}
\item
  \begin{quote}
  See if we can switch to \_\_attribute((uninitialized)) instead of
  build config carve-outs
  \end{quote}
\item
  \begin{quote}
  Try V8 again?
  \end{quote}
\end{itemize}

\textbf{Costs:} Auto-initialization has proven to incur noticeable
run-time costs in hot paths. (Vitaly has manually opted those hot paths
out of auto-init.)

Assuming auto-init is a significant change in C/C++ semantics. If
developers come to rely on auto-init, code can become buggy if anyone
were to turn auto-init off. (We can, and should in any case, protect
against this with tests.)

\textbf{Benefits:} Uninitialized memory is a small fraction of our high
severity security bugs, probably no more than 1.5\%. Auto-init reduces
the cognitive load (less need to remember all UB), and may enable
cleaner code paths iff developers can assume autoinit is always
on \href{https://www.google.com/url?q=https://lists.llvm.org/pipermail/cfe-dev/2020-April/065233.html\&sa=D\&source=editors\&ust=1631944129084000\&usg=AOvVaw2T_sqx8Nl21LBxNUvi3dis}{\emph{\uline{if
the program continues}}}.

Since we do need to turn auto-init off in places, such call sites should
be `obvious' and well-documented. (An example of obvious and
well-documented: \uline{\href{https://www.google.com/url?q=https://source.chromium.org/chromium/chromium/src/\%2B/master:sandbox/win/src/sandbox_nt_util.cc;l\%3D63?q\%3Dtrivial-auto-var-init\%26ss\%3Dchromium\&sa=D\&source=editors\&ust=1631944129084000\&usg=AOvVaw3bHIFqDRxdVpYasCoj9_By}{using }\href{https://www.google.com/url?q=https://source.chromium.org/chromium/chromium/src/\%2B/master:sandbox/win/src/sandbox_nt_util.cc;l\%3D63?q\%3Dtrivial-auto-var-init\%26ss\%3Dchromium\&sa=D\&source=editors\&ust=1631944129085000\&usg=AOvVaw0ZvsfgE7IGTXJXGf3A71BS}{\_\_attribute((uninitialized)) or
similar}} in hot spots rather than special-casing directories in the
build.)

\section{Remove Primitive Arrays}

(May be covered
under \href{https://docs.google.com/document/d/e/2PACX-1vRZr-HJcYmf2Y76DhewaiJOhRNpjGHCxliAQTBhFxzv1QTae9o8mhBmDl32CRIuaWZLt5kVeH9e9jXv/pub\#h.c3notccb295u}{\uline{Remove
Raw Pointers}}.)

\textbf{Problem:} Primitive C arrays are not bounds-checked, and thus
tend to exhibit spatial safety bugs. -fsanitize=bounds only works when
the compiler can statically determine the array's size, which is not
always.

\textbf{Solutions:} Require the use of a type like std::array where
C-style arrays are currently in use. It may be possible to automatically
migrate old code to std::array (similar to how we are automatically
migrating code to MiraclePtr). Note that this is only a security win if
we also use a std implementation where all undefined behavior is
defined; for example, std::array::operator{[}{]} does no bounds
checking.
(See \href{https://docs.google.com/document/d/e/2PACX-1vRZr-HJcYmf2Y76DhewaiJOhRNpjGHCxliAQTBhFxzv1QTae9o8mhBmDl32CRIuaWZLt5kVeH9e9jXv/pub\#h.ji6or6kpi3sa}{\uline{Define
All Standard Library Behaviors}}.)

\textbf{Current status:} None.

\textbf{Costs:} Micro-efficiency. Training and socialization. May want
to write a PRESUBMIT check or (preferably) a clang warning, if that's
possible.

\textbf{Benefits:} \href{https://docs.google.com/document/d/e/2PACX-1vRZr-HJcYmf2Y76DhewaiJOhRNpjGHCxliAQTBhFxzv1QTae9o8mhBmDl32CRIuaWZLt5kVeH9e9jXv/pub\#h.eoikp3r0cwlf}{\uline{Spatial
safety is 16\% of high severity security bugs; possibly 17.5\%}}.

\section{Remove Mutable Shared State}

\textbf{Problem:} C++ cannot prevent data races.

\textbf{Solutions:} Implement support in the compiler to enforce this,
and a borrow checker.

\textbf{Current status:} None.

\textbf{Costs:} Unknown. Potentially significant change to developer
expectations. If Chromium makes heavy use of shared mutable state in
places, that code would need to be significantly refactored.

\textbf{Benefits:} Data races
are \href{https://docs.google.com/document/d/e/2PACX-1vRZr-HJcYmf2Y76DhewaiJOhRNpjGHCxliAQTBhFxzv1QTae9o8mhBmDl32CRIuaWZLt5kVeH9e9jXv/pub\#h.eoikp3r0cwlf}{\uline{known
to be \textasciitilde1\% of our high severity security bugs}}; however
they are implicated in lots of bugs that then appear in other categories
(e.g. temporal unsafety) so this likely under-represents the scale of
the problem.

\section{Check Type Casts}

\textbf{Problem:} Type confusion bugs such as this:

\begin{minted}{cpp}
void SomeFunction(Animal* animal) {
  // NOTE: Dog and Cat are subclasses of Animal.
  DCHECK(IsCat(animal));

  Cat* cat = static_cast<Cat*>(animal);
  cat->Meow();    // If animal is really a Dog*, memory unsafety may ensue!
}
\end{minted}

can be exploitable, such as by causing memory corruption when code
incorrectly treats an object of 1 class as if it were an instance of
another.

The static\_cast should be a dynamic\_cast, or otherwise automatically
checked --- not just in debug builds with DCHECK, but in production
builds too. Historically Chrome has avoided dynamic\_cast because the
cost of RTTI is too high (e.g. huge object code size). That allows bugs
that escape detection during debugging and fuzzing to become
vulnerabilities in the wild.

Regarding object code size, davidben@ notes:

\begin{quote}
Playing around with godbolt, looks like the cost of RTTI is each vtable
now also gets a typeinfo with a few quads and a class name. That doesn't
seem like it should be too expensive?

I got an (possibly with the wrong build flags) increase from 171M to
178M in a stripped Linux release build, which is substantial but doesn't
seem huge? Android hit some link error with use\_rtti though I assume
that's fixable.

Assuming it is indeed the type names, maybe we just need a Clang flag to
omit
them? \href{https://www.google.com/url?q=https://en.cppreference.com/w/cpp/types/type_info/name\&sa=D\&source=editors\&ust=1631944129092000\&usg=AOvVaw1kUlFV7YNfYs6EbGGfwfqk}{\uline{std:type\_info::name supposedly
doesn't promise anything}}, so the empty string should be perfectly
compliant...
\end{quote}

\textbf{Solutions:} TODO. dcheng@ says:

\begin{quote}
I'm guessing you're thinking of the casting helpers that
@inferno@google.com originally added to WebKit. These use a handrolled
implementation of RTTI.

I've updated these into a more `modern' template-based C++ solution, but
since we don't actually support RTTI at all today, it doesn't
use dynamic\_cast, even in debug mode.

The way it works today:
\end{quote}

\begin{itemize}
\item
  \begin{quote}
  Blink classes implement IsX
  \end{quote}
\item
  \begin{quote}
  There are traits that tell the cast helpers how to use the
  various IsX methods.
  \end{quote}
\item
  \begin{quote}
  DynamicTo\textless\textgreater behaves
  like dynamic\_cast\textless\textgreater and returns nullptr if the
  type check fails. Used in places where you'd want to
  do IsA\textless X\textgreater followed by To\textless X\textgreater.
  \end{quote}
\item
  \begin{quote}
  To\textless\textgreater behaves
  like static\_cast\textless\textgreater but has
  a DCHECK(IsA\textless X\textgreater);.
  \end{quote}
\end{itemize}

\begin{quote}
My ideal world would be:
\end{quote}

\begin{itemize}
\item
  \begin{quote}
  To\textless\textgreater always does the type check
  \end{quote}
\item
  \begin{quote}
  DynamicTo\textless\textgreater stays the same.
  \end{quote}
\item
  \begin{quote}
  UnsafeTo\textless\textgreater skips the type check in official builds
  when the performance is `necessary'.
  \end{quote}
\item
  \begin{quote}
  \textless Something\textgreater{} enforces that we use these helpers
  for casting rather than static\_cast.
  \end{quote}
\end{itemize}

tsepez@ says:

\begin{quote}
Some years ago, I'd pondered
creating subclass\_cast\textless\textgreater (or down\_cast\textless\textgreater)
which would be defined as dynamic\_cast if RTTI was enabled,
and static\_cast otherwise (to avoid warnings about dynamic cast in
non-RTTI builds). I eventually concluded this was a bad idea, on the
grounds that you make languages better by removing things from them you
shouldn't use rather than adding new things that you should use.
\end{quote}

markbrand@ says: ``AFAIU clang-cfi supports this, not sure whether RTTI
is required.''

kcc@ asks: ``Why
not \href{https://www.google.com/url?q=https://clang.llvm.org/docs/ControlFlowIntegrity.html\%23bad-cast-checking\&sa=D\&source=editors\&ust=1631944129096000\&usg=AOvVaw1hIEhYq5DWiRx3GcY8-Eni}{\uline{cast-cfi}}?
Microsoft is doing it.''

\textbf{Current status:} There is now a convention in Blink to handle
the ``cast to wrong subclass'' problem, which used to be more common in
Blink. Perhaps it could be adopted more widely.

\textbf{Costs:} We can make it cheaper than RTTI and dynamic\_cast, but
there will always be some micro-cost in run-time and possibly some cost
to object code size.

\textbf{Benefits:} \href{https://docs.google.com/document/d/e/2PACX-1vRZr-HJcYmf2Y76DhewaiJOhRNpjGHCxliAQTBhFxzv1QTae9o8mhBmDl32CRIuaWZLt5kVeH9e9jXv/pub\#h.eoikp3r0cwlf}{\uline{Type
confusion is 7\% of our high severity security bugs}}, though a fair
amount of that is within V8 and wouldn't be covered by C++ improvements
(nor by a new general-purpose programming language). Attackers seem
particularly fond of exploiting these bugs disproportionately to their
number.

\section{Make All DCHECKs Into CHECKs}

\textbf{Problem:} DCHECKs exist to check some \emph{static} invariant,
but they are sometimes misused to check
a \emph{dynamic} invariant (sometimes on the assumption that we'll
notice the bug dynamically in debug builds). In release builds,
execution might sail past the DCHECK and lead to some memory safety
problem, or sometimes some logic error. In either case it can be a
security bug.

\textbf{Solutions:} Audit for unnecessary DCHECKs (there are some), and
then turn many remaining DCHECKs into CHECKs. Some DCHECKs really are
checking static invariants, and those should not be converted.

\textbf{Current
status:} \href{https://www.google.com/url?q=https://docs.google.com/document/d/1QY4IbbJ8X6G-6-cMheEkP_mT7ZNPCuUJIW2sr_mEiH4/edit\&sa=D\&source=editors\&ust=1631944129101000\&usg=AOvVaw2BlusRZodOmwWxlt2Ifxxe}{\uline{Albatross}} build.
Proposal probabilistically to enable DCHECKs for some fraction of checks
on some build.

\textbf{Costs:} An extra branch instruction at runtime. The checking
clause on some DCHECKs involves substantial binary bloat and runtime
calculation (e.g. comparing trees of opcodes in V8). We would need some
way of working around this, where possible.

\textbf{Benefits:} Can address memory safety problems and
logic/correctness problems. About 30\% of our high severity security
bugs are \emph{not} memory safety
problems. \href{https://docs.google.com/document/d/e/2PACX-1vRZr-HJcYmf2Y76DhewaiJOhRNpjGHCxliAQTBhFxzv1QTae9o8mhBmDl32CRIuaWZLt5kVeH9e9jXv/pub\#h.eoikp3r0cwlf}{\uline{10\%
of our high severity security bugs are DCHECKs}} where we've determined
that there may be security implications when the release build goes past
the DCHECK site.

\section{Back NOTREACHED With CHECK}

\textbf{Problem:} NOTREACHED implies that the program has entered an
undefined state, and the result will be undefined behaviour. While this
crashes in debug builds, it continues on (un)happily for our users.

\textbf{Solutions:} Make NOTREACHED do a CHECK(false) instead
of DCHECK(false), or equivalent.

\textbf{Current status:} There
is \uline{\href{https://www.google.com/url?q=https://bugs.chromium.org/p/chromium/issues/detail?id\%3D851128\&sa=D\&source=editors\&ust=1631944129105000\&usg=AOvVaw2LrM6i2LECKEjRcO5h20mN}{a
plan to
make NOTREACHED}\href{https://www.google.com/url?q=https://bugs.chromium.org/p/chromium/issues/detail?id\%3D851128\&sa=D\&source=editors\&ust=1631944129106000\&usg=AOvVaw34qgxqGkGdOZeZV-chozbg}{ be {[}{[}noreturn{]}{]}}} which
includes making it IMMEDIATE\_CRASH.

\textbf{Costs:} More binary size for our users, as these checks are
compiled out in release builds today. This should be significantly
smaller and safer than converting DCHECK to CHECK globally,
as NOTREACHED is used to document undefined behaviour and is less
common. It's likely this would take multiple rounds of attempts to land
and stick. Any NOTREACHED that was found happening in production should
be removed and converted to handle that case correctly.

\section{Prevent Use After Move}

\textbf{Problem:} C++ allows programmers to move objects, and then use
the moved-from storage. (We call this use after move, or UAM.) The state
of that storage is undefined and using it is UB.

\textbf{Solution:} Implement a Clang plugin to prevent touching any
moved value.

Existing work:

\begin{itemize}
\item
  \begin{quote}
  Clang \href{https://www.google.com/url?q=https://clang.llvm.org/docs/analyzer/checkers.html\%23alpha-cplusplus-misusedmovedobject-c\&sa=D\&source=editors\&ust=1631944129108000\&usg=AOvVaw0Z8KLx8foXb0CEYmjb384B}{\uline{MisusedMovedObject
  check}}
  \end{quote}
\item
  \begin{quote}
  \href{https://www.google.com/url?q=https://clang.llvm.org/extra/clang-tidy/checks/bugprone-use-after-move.html\&sa=D\&source=editors\&ust=1631944129108000\&usg=AOvVaw3euQN7cJIxz8XBVdcRGC3r}{\uline{Clang-Tidy
  bugprone-use-after-move}} is deployed in
  google3 \href{https://www.google.com/url?q=https://chromium-review.googlesource.com/c/chromium/src/\%2B/2383010\&sa=D\&source=editors\&ust=1631944129108000\&usg=AOvVaw0y_18KwSgPtPRNt-LsDNe_}{\uline{and
  in Chromium}}.
  \end{quote}
\end{itemize}

Clang-tidy's check is helpful but not sufficient. It does not see this
as a UAM:

\begin{quote}
auto consumes = {[}{]}(OnceClosure c) \{\};

auto moves = {[}\&{]}(OnceClosure\& c) \{ consumes(std::move(c)); \};

auto c = BindOnce({[}{]}() \{\});

moves(c);

moves(c); // Use after move.
\end{quote}

Chromium currently relies on things
like OnceCallback and unique\_ptr being readable (and in a null state)
after being moved-from.

\textbf{Current
status:} None. \href{https://www.google.com/url?q=https://crbug.com/1198689\&sa=D\&source=editors\&ust=1631944129110000\&usg=AOvVaw37710zVQHo5HvDBB4mW1Hy}{\uline{Bug
for using an attribute to annotate acceptable uses}}.

\textbf{Costs:} Hopefully none?

\textbf{Benefits:} Reduced object lifetime errors.

\section{{Appendix: }Explore Solutions For Logic Errors}

Chrome Security's primary concern about C++ is memory safety errors.
However, once they're all fixed, attackers will move to exploit logic
errors instead. There are some facilities and idioms provided by other
languages which perhaps make such errors less likely. There are perhaps
some things we could do to C++ to discourage logic bugs in
security-critical areas. These could also potentially serve to reduce
temporal unsafety.

Logic errors are
around \href{https://docs.google.com/document/d/e/2PACX-1vRZr-HJcYmf2Y76DhewaiJOhRNpjGHCxliAQTBhFxzv1QTae9o8mhBmDl32CRIuaWZLt5kVeH9e9jXv/pub\#h.eoikp3r0cwlf}{\uline{10\%
of our high severity security bugs}}.

\subsection{Encourage Type Wrappers For Security Invariants}

\textbf{Problem:} C++ makes it awkward to write type wrappers. In
another language, you might have (for
instance) IpAddress and IpAddressWhichIsNotLoopback (wrapping IpAddress with
zero runtime impact except for an initial check). Some APIs would only
accept the latter type, providing zero runtime impact. None of this is
impossible in C++, it's just not a common pattern because C++ makes it a
bit awkward.

Similarly, it might be useful to
have \href{https://www.google.com/url?q=https://resources.sei.cmu.edu/asset_files/TechnicalNote/2007_004_001_14846.pdf\&sa=D\&source=editors\&ust=1631944129113000\&usg=AOvVaw3rHn2Q-m9cXLF3kGN-M1Hu}{\uline{ranged
integers}}.

For example, there are many web platform features that require the
caller to be in a secure context. Part of that check is checking that
the caller's origin is one of the secure URL schemes. Right now, we pass
in normal GURLs and url::Origins (preferably the latter!), and expect
the callee (the feature) to explicitly check.

Instead, we could bake that check into the constructor of a constrained
class, e.g.:

class SecureGURL : public GURL \{

  // This constructor is intentionally not `explicit`:

  SecureGURL(const GURL\& gurl) \{ CHECK(gurl.IsSecureScheme()); ... \}

\}



std::vector\textless byte\textgreater{} LoadTrustedResource(const
SecureGURL\& gurl);

Thus the callee, in this case the hypothetical LoadTrustedResource, can
only be called with a GURL that has already passed the check, and does
not need to explicitly perform the check.

ellyjones@ notes that we could go further, and  could
make SecureGURL not a subclass of GURL, so that nobody can accidentally
downcast. It would require us to restate the entire interface of GURL,
though.

\textbf{Solution:} Guidance that compile-time type safety can address
higher-level logic errors. Look for cases in the codebase where this can
help. (Origins, URIs, et c. seem like a likely area, for example.)

\textbf{Current status:} None.

\textbf{Costs:} Should be no runtime cost, in fact the opposite if it
means a check can only be performed once or statically, not dynamically
or multiple times. It might or might not be hard to identify cases where
this sort of pattern makes sense in an existing codebase.

\textbf{Benefits:} If any cases can be identified, may eliminate some
logic bugs with no performance penalty (or a micro-improvement, e.g
getting rid of a CHECK).

\section{Appendix: Hardware Support For Detecting Memory Issues}

\subsection{Memory Tagging}

Everything that was old is new again: tagged memory is making a
resurgence. (Between that and the prevalence of JavaScript, Lisp really
did win after all!)
See \href{https://www.google.com/url?q=https://docs.google.com/document/d/1j7S5EZeXesdL2eLC5D6dTAbF8Us9QNUcxAhD9cmWKMg/edit\&sa=D\&source=editors\&ust=1631944129117000\&usg=AOvVaw3aB5nf-FezD0eUwLuRjhjI}{ARM
MTE}. Depending on the specific mechanism, pointers to and/or regions
of memory are `tagged', and if code tries to load or store memory
without using the right tag, the program faults. This helps us detect
bugs and stop exploits with a certain (typically high) probability per
instance.

MTE might appear in high-end devices soon, but will take much longer to
reach most of the world.

Tagging is not a perfect defense. If an attacker can learn or guess the
correct tag
--- \href{https://www.google.com/url?q=https://googleprojectzero.blogspot.com/2019/02/examining-pointer-authentication-on.html\&sa=D\&source=editors\&ust=1631944129118000\&usg=AOvVaw1gThhNgANrziuAi11Pvu4v}{and
they often can} --- they can make use of the correct tag during their
exploit and hence not trigger the hardware's alarm. To get the benefit
of tagging, we will need to make it hard for attackers to learn tag
values, and hard for them to guess. (Such as by, potentially, having the
browser throttle navigations to sites that crash their renderers too
often. Whether that would actually work, I don't know.)

\subsection{Control Flow Integrity}

We are now
shipping \href{https://www.google.com/url?q=https://security.googleblog.com/2021/05/enabling-hardware-enforced-stack.html\&sa=D\&source=editors\&ust=1631944129119000\&usg=AOvVaw2rim68EVFlBXjFhBDe4leE}{support
for Intel Control-flow Enforcement Technology (CET) on Windows}.

Deploying this makes it meaningful to ship forward jump Clang
CFI/Windows CFG, and we are investigating doing so in 2021. TODO:
ENDBRANCH.

We are also going to investigate enabling CET on all OSs, hopefully also
in 2021.

TODO: \href{https://www.google.com/url?q=https://developer.apple.com/documentation/security/preparing_your_app_to_work_with_pointer_authentication\&sa=D\&source=editors\&ust=1631944129120000\&usg=AOvVaw3jurZceu1ptrIB_GQl8_Uy}{ARM
PAC} is shipping today for Apple devices. It would also be good to
explore PAC
and \href{https://www.google.com/url?q=https://developer.arm.com/documentation/ddi0596/2020-12/Base-Instructions/BTI--Branch-Target-Identification-\&sa=D\&source=editors\&ust=1631944129120000\&usg=AOvVaw0EN5YPuPn3O35XJKNmysBM}{BTI} on
other hardware and operating systems.

Appendix: Bug Types

This document talks about the percentage impact of fixing different
types of security bug. Those percentages are based on a manual analysis
of each high severity security bug that has impacted the stable channel
since the start of 2019. Exact root causes are a little approximate.
It's worth noting that the ``temporal safety'' sector seems to be
growing year-by-year.

\includegraphics[width=5.76806in,height=3.60278in]{./media/image2.png}

Appendix: Remove Null Pointers

\textbf{Problem:} While we don't consider null pointer dereferences to
be a security vulnerability (unless the attacker can control an offset
to the pointer, which is rare), they do account for a big fraction of
our stability bugs. It can also take the Security Sheriff substantial
time to ensure that a given bug really is `just' a stability bug
(\href{https://www.google.com/url?q=https://bugs.chromium.org/p/chromium/issues/detail?id\%3D1048473\&sa=D\&source=editors\&ust=1631944129122000\&usg=AOvVaw0fy7uPuy7ezmAwq9hPlixp}{example}).

Removing null pointers may also improve developer ergonomics --- people
can shed the cognitive and code overhead of checking for null.

Note: ASan explicitly reports a crash as null-deref if the faulty
address is within {[}0, 4096). Without ASan one needs to look at the
value of registers, but people make mistakes sometimes. Reproducing as
many crashes as possible on ClusterFuzz would likely reduce the chance
of an error.

\textbf{Solutions:} Modify the smart pointer type to require an explicit
annotation or constructor flag for pointers which may be null. Check and
crash on deference for such pointers. Check on construction for those
pointers which shouldn't be null.

As with integer semantics, it may be possible to get what we want with
compiler
options. \href{https://www.google.com/url?q=https://clang.llvm.org/docs/UndefinedBehaviorSanitizer.html\&sa=D\&source=editors\&ust=1631944129123000\&usg=AOvVaw3Y79vEDZKr7fj2vr_YqYy3}{Several
options are available},
including -fsanitize=null and -fsanitize=nullability-arg.
The nullability-* options work with Clang's \_Nonnull annotation.

\textbf{Current status:} None.

\textbf{Costs:} The micro-cost of the check.

\textbf{Benefits:} We should detect stability bugs sooner, and may save
time in security bug triage.

Appendix: Mitigations

This document focuses on language-level approaches to the problem:
codifying safer usage patterns for C++ to reduce memory unsafety and UB
generally.

Exploit mitigations and enhanced bug-finding techniques can complement
such work. Examples include:

\begin{itemize}
\item
  \begin{quote}
  HWASAN
  \end{quote}
\item
  \begin{quote}
  GWP-ASan
  \end{quote}
\item
  \begin{quote}
  Memory tagging (e.g. MTE)
  \end{quote}
\item
  \begin{quote}
  CFI and stack protection
  \end{quote}
\item
  \begin{quote}
  The classics: W\^{}X/DEP, stack canaries, heap canaries, ASLR
  \end{quote}
\end{itemize}

Appendix: Other Ideas

Others possibly to include:

\begin{itemize}
\item
  \begin{quote}
  Remove base::Unretained --- require developers to decide the
  right ownership model for each object. (WeakPtr is not always a silver
  bullet because it cannot be used across task sequences/threads.)
  \end{quote}
\end{itemize}

\begin{itemize}
\item
  \begin{quote}
  Or, back Unretained with MiraclePtr.
  \end{quote}
\end{itemize}

\begin{itemize}
\item
  \begin{quote}
  Mojom improvements to ensure all objects are stored in suitable
  containers?
  \end{quote}
\item
  \begin{quote}
  Ban design patterns which tend to cause problems (e.g. singletons
  intrinsically mean shared and likely mutable state). (There
  was \href{https://www.google.com/url?q=https://docs.google.com/document/d/1MAuOqSpVaKsjqmw8oPLwwdSywHX9HF4FiPnODHNimmQ/edit\&sa=D\&source=editors\&ust=1631944129127000\&usg=AOvVaw2i9qlTgB1UaL_3zE2nBgKg}{an
  effort to systematically investigate}.)
  \end{quote}
\item
  \begin{quote}
  Re-emphasize composition rather than any kind of pointer in the first
  place. (Can we write a checker to look for objects owned by pointer
  which could instead simply be owned by value?)
  \end{quote}
\item
  \begin{quote}
  Replace base::Bind \emph{and all code which uses it} with a
  straight-line version instead (promises? Some C++ code generator?), to
  simplify the ease of writing asynchronous code and make error cases
  easier to spot.
  \end{quote}
\item
  \begin{quote}
  Design for fuzzing: encourage patterns in mojo \& implementation that
  make it easier to fuzz formats and object lifetimes from a compromised
  renderer.
  \end{quote}
\item
  \begin{quote}
  Design for static analysis: discourage patterns that give static
  analyzers a hard time.
  \end{quote}
\item
  \begin{quote}
  \href{https://www.google.com/url?q=https://docs.microsoft.com/en-us/cpp/code-quality/using-sal-annotations-to-reduce-c-cpp-code-defects\&sa=D\&source=editors\&ust=1631944129129000\&usg=AOvVaw0r_Lm-49kaxSlt0qo4jfhz}{SAL
  annotations}. Valuable at the OS boundary where annotations are
  already available. Basically a new language for base:: developers to
  interface with to take full advantage.
  \end{quote}
\item
  \begin{quote}
  Compile parts of the code with WebAssembly. (We are investigating this
  as of August 2021.)
  \end{quote}
\end{itemize}

\end{document}
