\PassOptionsToPackage{unicode=true}{hyperref} % options for packages loaded elsewhere
\PassOptionsToPackage{hyphens}{url}
\documentclass[a4paper,12pt,notitlepage,twoside,openright]{article}

\usepackage{ifxetex}
\ifxetex{}
\else
\errmessage{Must be built with XeLaTeX}
\fi

\usepackage{amsmath}
\usepackage{fontspec}
\usepackage{fourier-otf} % erewhon-math
\setmonofont{iosevka-type-slab-regular}[
  Path=../common/iosevka-type-slab/,
  Extension=.ttf,
  BoldFont=iosevka-type-slab-bold,
  ItalicFont=iosevka-type-slab-italic,
  BoldItalicFont=iosevka-type-slab-bolditalic,
  Scale=MatchLowercase,
]

% Math
\usepackage[binary-units]{siunitx}

\usepackage{caption}
\usepackage{authblk}
\usepackage{enumitem}
\usepackage{footnote}

% Table
\usepackage{tabu}
\usepackage{longtable}
\usepackage{booktabs}
\usepackage{multirow}

% Verbatim & Source code
\usepackage{fancyvrb}
\usepackage{minted}

% Beauty
\usepackage[protrusion]{microtype}
\usepackage[defaultlines=3]{nowidow}
\usepackage{upquote}
\usepackage{parskip}
\usepackage[strict]{changepage}

\usepackage{hyperref}

% Graph
\usepackage{graphicx}
\usepackage{grffile}
\usepackage{tikz}

\usepackage{wrapfig}
\hypersetup{
  bookmarksnumbered,
  pdfborder={0 0 0},
  pdfpagemode=UseNone,
  pdfstartview=FitH,
  breaklinks=true}
\urlstyle{same}  % don't use monospace font for urls

\usetikzlibrary{arrows.meta,calc,shapes.geometric,shapes.misc}

\setminted{
  autogobble,
  breakbytokenanywhere,
  breaklines,
  fontsize=\footnotesize,
}
\setmintedinline{
  autogobble,
  breakbytokenanywhere,
  breaklines,
  fontsize=\footnotesize,
}

\makeatletter
\def\maxwidth{\ifdim\Gin@nat@width>\linewidth\linewidth\else\Gin@nat@width\fi}
\def\maxheight{\ifdim\Gin@nat@height>\textheight\textheight\else\Gin@nat@height\fi}
\makeatother

% Scale images if necessary, so that they will not overflow the page
% margins by default, and it is still possible to overwrite the defaults
% using explicit options in \includegraphics[width, height, ...]{}
\setkeys{Gin}{width=\maxwidth,height=\maxheight,keepaspectratio}
\setlength{\emergencystretch}{3em}  % prevent overfull lines
\setcounter{secnumdepth}{3}

% Redefines (sub)paragraphs to behave more like sections
\ifx\paragraph\undefined\else
\let\oldparagraph\paragraph{}
\renewcommand{\paragraph}[1]{\oldparagraph{#1}\mbox{}}
\fi
\ifx\subparagraph\undefined\else
\let\oldsubparagraph\subparagraph{}
\renewcommand{\subparagraph}[1]{\oldsubparagraph{#1}\mbox{}}
\fi

% set default figure placement to htbp
\makeatletter
\def\fps@figure{htbp}
\makeatother


\title{LSM-Trees and B-Trees: The Best of Both Worlds}
\author{Varun Jain, James Lennon, Harshita Gupta}
\date{July 5, 2019}

\begin{document}
\maketitle

\begin{abstract}

LSM-Trees and B-Trees are the two primary data structures used as
storage engines in modern key-value (KV) stores. These two structures
are optimal for different workloads; LSM-Trees perform better on update
queries, and B-Trees do on short range lookups. KV stores today use one
or the other. However, for modern applications with increasingly diverse
workloads, utilizing only one of the two designs leads to a significant
loss in performance. We propose a novel method of online transitioning a
KV store from an LSM-Tree to a B-Tree and vice versa. This allows KV
stores to smoothly adapt to changing workloads.

\end{abstract}

\hypertarget{introduction}{%
\section{INTRODUCTION}\label{introduction}}

\textbf{Key-Value (KV) Stores.} NoSQL databases are becoming increasingly popular
due to their scalability and flexible design. A common NoSQL database is
the KV store, which maintains a unique, associative mapping between keys
and data values. LSM-Trees and B-Trees are typically used for
implementing KV stores.

\textbf{LSM-Trees.} Log-Structured Merge Trees {[}13{]} are popular in production
(e.g. RocksDB {[}8{]}, Monkey/Dostoevsky {[}5--7{]}) due to their
superior update performance and space amplification properties. Updates
are logged in an in-memory buffer to minimize disk IO, and this buffer
is periodically flushed into sorted ``runs'' of entries on disk. These
runs are organized in levels of exponentially increasing size. The
smaller levels are merged into the larger levels as they fill up. Each
run in the LSM-Tree contains an in-memory bloom filter {[}4{]} and fence
pointers to minimize disk IO during lookups.

\textbf{B-Trees.} B-Trees {[}3{]} have superior short-range scan performance at
the cost of slower point updates. They are structured as an in-memory
tree of internal nodes with terminal leaf nodes on disk. The internal
nodes of one level of the tree store pointers to nodes of the next
level. Each terminal leaf node stores KV pairs, sorted by key, and
typically has a size of a few disk pages.

\textbf{Problem with Static Data Structures.} We demonstrate in Figure 1 that
neither data structure is optimal across all workloads. We compare a
B-Tree (BerkeleyDB {[}12{]}) to an LSM-Tree (RocksDB {[}8{]}): During
the first 2000 queries, the workload is short-range scan heavy and thus
favors B-Trees. During the next 2000 queries, the workload changes and
becomes write heavy, favoring LSM-Trees.

\begin{figure}
  \centering
  \begin{minipage}[t]{0.9\columnwidth}
    \includegraphics[width=0.55\columnwidth]{./media/image1.jpg}
    \includegraphics[width=0.35\columnwidth]{./media/image2.png}
  \end{minipage}
  \caption{No single design is optimal for diverse workloads. An ideal
design uses elements of both LSM-Tree (bottom right) and B-tree (top
right).}
\end{figure}

Evidently, no single design is optimal across the whole workload {[}1,
2, 10, 11{]}. Both state-of-the-art KV stores face problems once the
workload departs from their respective comfort zones, leading to
significant performance loss. Since applications today are not static,
they must support diverse and changing workloads. New application
features, for example, affect the read and write patterns the underlying
storage system should support. Currently, this requires system designers
to re-evaluate how data is stored and accessed, possibly having to take
their application offline. In many instances, it may be impractical to
change the data storage architecture, leaving a suboptimal data
structure.

\textbf{The Solution: Transitions.} We propose on-the-fly transitions across
different data structure designs. Our KV stores morph their design as
the workload changes. Figure 1 shows that while neither data structure
is optimal, transitioning our KV store from a B-Tree to an LSM-Tree
achieves the best of both worlds (indicated by the dashed black line).
In this paper we provide algorithms for the transitions and initial
results to prove this is a promising direction.

One of the key ideas that makes this possible is recognizing the
structural similarity between these two data structures; we use this
similarity to efficiently change data from one structure to another
in-place {[}9{]}. For example, when transitioning from an LSM-Tree to a
B-Tree, we take the sorted run of an LSM-Tree and reinterpret that data
as an initial leaf level of the B-Tree ``for free''. Another key idea is
that our algorithm is gradual and can be smoothly executed over a
tunable number of steps, allowing us to execute this transition online
without a significant cost.

\hypertarget{fromlsm-treetob-tree}{%
\section{FROMLSM-TREETOB-TREE}\label{fromlsm-treetob-tree}}

To move from an LSM-Tree to B-Tree, we present two strategies, and we
discuss how to use them to holistically schedule the transition.

\textbf{Sort-Merge Transition.} In an LSM-Tree, data is organized into
several sorted runs with some duplicate entries that result in space
amplification. In order to condense this data into the leaf level of a
B-Tree, we force a sort-merge to condense these runs into a dense,
sorted array that becomes the leaf level. This has a performance cost of
reading all the runs' pages, and then writing each of the sorted pages
to disk. This method has optimal space amplification properties, since
the resulting array is compact.

\begin{figure}
  \centering
  \begin{minipage}[t]{0.9\columnwidth}
    \includegraphics[width=0.45\columnwidth]{./media/image3.jpg}
    \includegraphics[width=0.45\columnwidth]{./media/image4.png}
  \end{minipage}
  \caption{Tradeoff Between Merge Sort (Upper Right) and Batch Insert (Lower Right)}
\end{figure}

\textbf{Batch Inserts Transition.} In some cases (e.g. when there is relatively
little data in the levels above the bottom of the LSM-Tree), we can
avoid reading and writing most of the data. We take the bottommost
(largest) run of the LSM-Tree and reinterpret it as the leaf level of a
B-Tree. We then insert the entries from the remaining upper levels of
the LSM-Tree into this B-Tree. By performing these inserts in sorted
order, we can ``batch insert'' the entries into the B-Tree. In
particular, consider a node \(N\) of the leaf layer for which there
are entries in the upper levels to be inserted into \(N\). We load
that node into memory. Then, we iterate through the elements in the
upper levels corresponding to the range of \(N\) --- call this set of
elements \(R\). As we do so, we merge the elements of \(N\) and
\(R\), making new leaf-level nodes in memory if needed. These new
nodes are flushed to disk as they are made. As an example, if the set of
elements \(R\) to be inserted into node \(N\) consists of just one
element, then the above process is equivalent to a B-Tree node split.

\textbf{Choosing the Optimal Transition.} Compared to the sort-merge approach, the
batch inserts method may be more efficient as we may not insert entries
into every node of the B-Tree, saving IO costs for certain nodes. On the
other hand, it has worse space amplification properties because some
pages may be partially empty; this can result in more disk IO. It can be
shown in worst and average case analysis that the batch inserts approach
supersedes the sort-merge approach when the size of the upper runs is
below a certain threshold. To analytically demonstrate this, we present
the results for a worst case analysis of an LSM-Tree with node of size 1
page.

Assume an LSM-Tree with \(L\) levels where \(x_i\) is number of
entries in the \(i\)th level and \(d\) is the size of each entry
in bytes and a machine where \(p\) is the page size in bytes and
\( \phi \) is the ratio of a disk write to a disk read. The cost of a sort
merge transition is \(C_{SM} = \left(\sum^L_{i=1} \left\lceil\frac{d \cdot x_i}{p}\right\rceil \right)(1+\phi)\). The worst-case cost of a batch insert transition is
\(C_{BI} = \left\lceil\frac{d \cdot x_L}{p}\right\rceil + \sum^{L-1}_{i=1} x_i \left( \frac{d}{p} + 1 + 2 \phi \right) \). Let
\(x_{-L}\) be the number of entries in all levels
above \(L\); we choose to do a batch insert transition only when
\(C_{BI} < C_{SM}\), in other terms when
\( \frac{x_{-L}}{x_L} < \frac{d_\phi}{p + (2p - d) \phi } \). See Figure 2 as a visualization of this tradeoff: Using our cost model, we show that as the ratio
\( \frac{x_{-L}}{x_L} \) decreases, the cost of the batch insert method decreases while the
sort-merge method is unaffected. This is because as \(x_L\) becomes
relatively larger, the LSM-Tree is more ``bottom-heavy'', and we can
potentially use fewer disk operations when executing a batch insert
transition.

\textbf{Gradual Transitions} In most situations, a sudden spike in query latency
is undesirable. Transitions should be lightweight, but we should also be
able to control the transition cost with respect to the current system
status. We propose gradual versions of the sort-merge and batch insert
transition algorithms, which can execute over a tunable number of steps.
These algorithms exhibit a tradeoff between total transition cost and
query latency during the transition phase. At any time during the
transition, KV pairs with keys below a ``threshold value'' \( \theta \) have
been transitioned to a B-Tree, whereas those with keys above \( \theta \)
remain in the LSM-Tree.

At each step of the transition, we choose a new threshold value
\(\theta'\), and transition data with keys between
\( \theta \) and \(\theta'\) from the LSM-Tree to the
B-Tree. This data can be transitioned in a sort-merge fashion by reading
pages from each of the runs and writing to a final sorted array. For the
batch-inserts approach, we insert KV pairs from the pages in the upper
runs into the bottom run. While the transition is underway, we dispatch
queries to the B-Tree component, the LSMTree component, or both (for
range queries), based on whether the query keys are less than or greater
than \( \theta \). The total cost of the transition will include the cost of
reading and writing data in the original LSM-Tree and any new data added
to the LSM-Tree during the transition.

\hypertarget{fromb-treetolsm-tree}{%
\section{FROMB-TREETOLSM-TREE}\label{fromb-treetolsm-tree}}

To transition in the opposite direction, we can convert the leaf level
of the B-Tree into a single run in our LSM-Tree. This requires us to
read the contents of the leaf level, to construct the in-memory Bloom
filter and fence pointers, and to write the compact, sorted run to disk.
This transition can be gradually executed by inserting the lowest
\(k\) pages of our B-Tree into our LSM-Tree at each step.

Another approach is to use the B-Tree itself as the lowest level of the
LSM-Tree, ``tricking'' our LSM tree into believing that the data on disk
in the B-Tree is stored contiguously. This can be done by maintaining an
in-memory layer of indirection, consisting of pointers to the fragmented
disk locations of leaf level nodes. From the in-memory internal nodes of
the B-Tree, we can construct this layer of indirection, as well as the
in-memory fence pointers. In our key-value store, we can maintain an
in-memory Bloom filter with the B-Tree, so that the transition itself
would not require reading keys from disk. Consequently, this transition
approach does not need to access the leaf level disk data during the
transition.

Comparing the above approaches, this latter transition approach uses
more main memory. Also, range scans will be more inefficient; the
fragmentation of leaf-level B-Tree nodes will result in more disk
accesses, as well as more random accesses. However, the transition
itself is ``free'', requiring no disk accesses.

\newpage
\hypertarget{efficienttransitions}{%
\section{EFFICIENTTRANSITIONS}\label{efficienttransitions}}

\begin{wrapfigure}{l}{0.5\columnwidth}
\includegraphics[width=0.5\columnwidth]{./media/image5.jpg}
\end{wrapfigure}The
figure on the left demonstrates the potential of on-the-fly transitions.
The workload changes in the same way as the experiment in Figure 1 that
goes from scan-heavy to write-heavy. This figure captures the IO behavior
of our KV stores by measuring the number of disk pages read and
written by the system using analytic IO cost models summarized in this
paper. Thus, overall this demonstrates the potential for data structure
designs that continuously transition to assume the optimal shape without
harming query performance.

\hypertarget{references}{%
\section{REFERENCES}\label{references}}

\begin{enumerate}
\def\labelenumi{\arabic{enumi}.}
\item
  Manos Athanassoulis and Stratos Idreos. 2016. Design Tradeoffs of Data
  Access Methods. In \emph{Proceedings of the 2016 International
  Conference on Management of Data, SIGMOD Conference 2016, San
  Francisco, CA, USA, June 26 - July 01, 2016}.
  2195--2200. \url{https://doi.org/10.1145/2882903.2912569}
\item
  Manos Athanassoulis, Michael S. Kester, Lukas M. Maas, Radu Stoica,
  Stratos Idreos, Anastasia Ailamaki, and Mark Callaghan. 2016.
  Designing Access Methods: The RUM Conjecture. In \emph{Proceedings of
  the 19th International Conference on Extending Database Technology, EDBT 2016, Bordeaux, France, March
  15-16, 2016, Bordeaux, France, March 15-16, 2016.} 461--466.
  \url{https://doi.org/10.5441/002/edbt.2016.42}
\item
  Rudolf Bayer and Edward M. McCreight. 1972. Organization and
  Maintenance of Large Ordered Indices. \emph{Acta Informatica} 1, 3
  (1972), 173--189.
  \url{https://doi.org/10.1007/BF00288683}
\item
  Burton H. Bloom. 1970. Space/Time Trade-offs in Hash Coding with
  Allowable Errors. \emph{Commun. ACM} 13, 7 (1970), 422--426.
  \href{http://dl.acm.org/citation.cfm?id=362686.362692}{http://dl.acm.org/citation.cfm?id=
  362686.362692}
\item
  Niv Dayan, Manos Athanassoulis, and Stratos Idreos. 2017. Monkey:
  Optimal Navigable Key-Value Store. In \emph{Proceedings of the 2017
  ACM International Conference on Management of Data, SIGMOD Conference
  2017, Chicago, IL, USA, May 14-19, 2017}. 79--94.
  \url{https://doi.org/10.1145/3035918.3064054}
\item
  Niv Dayan, Manos Athanassoulis, and Stratos Idreos. 2018. Optimal
  Bloom Filters and Adaptive Merging for LSM-Trees. \emph{ACM Trans.
  Database Syst.} 43, 4 (2018),
  16:1--16:48. \url{https://dl.acm.org/citation.cfm?id=3276980}
\item
  Niv Dayan and Stratos Idreos. 2018. Dostoevsky: Better Space-Time
  TradeOffs for LSM-Tree Based Key-Value Stores via Adaptive Removal of
  Superfluous Merging. In \emph{Proceedings of the 2018 International
  Conference on Management ofData (SIGMOD '18)}. ACM, New York, NY, USA, 505--520.
  \url{https://doi.org/10.1145/3183713.3196927}
\item
  Facebook. 2018. RocksDB. (14 Nov. 2018).
  \url{https://github.com/facebook/rocksdb}
\item
  Stratos Idreos, Niv Dayan, Wilson Qin, Mali Akmanalp, Sophie Hilgard,
  Andrew Ross, James Lennon, Varun Jain, Harshita Gupta, David Li, and
  Zichen Zhu. 2019.
  Design Continuums and the Path Toward Self-Designing Key-Value Stores
  that Know and Learn. In \emph{Biennial Conference on Innovative Data Systems
Research (CIDR)}.
\item
  Stratos Idreos, Kostas Zoumpatianos, Manos Athanassoulis, Niv Dayan,
  Brian
  Hentschel, Michael S. Kester, Demi Guo, Lukas M. Maas, Wilson Qin, Abdul
  Wasay, and Yiyou Sun. 2018. The Periodic Table of Data Structures.
  \emph{IEEE Data Eng. Bull.} 41, 3 (2018), 64--75.
  \url{http://sites.computer.org/debull/A18sept/p64.pdf}
\item
  Stratos Idreos, Kostas Zoumpatianos, Brian Hentschel, Michael S.
  Kester, and Demi Guo. 2018. The Data Calculator: Data Structure Design
  and Cost Synthesis from First Principles and Learned Cost Models. In
  \emph{Proceedings of the 2018 International Conference on Management
  of Data, SIGMOD Conference 2018, Houston, TX, USA, June 10-15, 2018}.
  535--550. \url{https://doi.org/10.1145/3183713.3199671}
\item
  Michael A. Olson, Keith Bostic, and Margo I. Seltzer. 1999. Berkeley
  DB. In \emph{Proceedings of the USENIX Annual Technical Conference
  (ATC)}. 183--191.
  \href{http://www.usenix.org/events/usenix99/olson.html}{http:
  //www.usenix.org/events/usenix99/olson.html}
\item
  Patrick E. O'Neil, Edward Cheng, Dieter Gawlick, and Elizabeth
  J. O'Neil. 1996. The log-structured merge-tree (LSM-tree). \emph{Acta
  Informatica} 33, 4 (1996), 351--385.
  \url{http://dl.acm.org/citation.cfm?id=230823.230826}
\end{enumerate}

\end{document}
