\PassOptionsToPackage{dvipsnames}{xcolor}
\PassOptionsToPackage{unicode=true,colorlinks=true,urlcolor=blue}{hyperref} % options for packages loaded elsewhere
\PassOptionsToPackage{hyphens}{url}
\documentclass[a4paper,12pt,notitlepage,twoside,openright]{article}

\usepackage{ifxetex}
\ifxetex{}
\else
\errmessage{Must be built with XeLaTeX}
\fi

\usepackage{amsmath}
\usepackage{fontspec}
\usepackage{fourier-otf} % erewhon-math
\setmonofont{iosevka-type-slab-regular}[
  Path=../common/iosevka-type-slab/,
  Extension=.ttf,
  BoldFont=iosevka-type-slab-bold,
  ItalicFont=iosevka-type-slab-italic,
  BoldItalicFont=iosevka-type-slab-bolditalic,
  Scale=MatchLowercase,
]

% Math
\usepackage[binary-units]{siunitx}

\usepackage{caption}
\usepackage{authblk}
\usepackage{enumitem}
\usepackage{footnote}

% Table
\usepackage{tabu}
\usepackage{longtable}
\usepackage{booktabs}
\usepackage{multirow}

% Verbatim & Source code
\usepackage{fancyvrb}
\usepackage{minted}

% Beauty
\usepackage[protrusion]{microtype}
\usepackage[defaultlines=3]{nowidow}
\usepackage{upquote}
\usepackage{parskip}
\usepackage[strict]{changepage}

\usepackage{hyperref}

% Graph
\usepackage{graphicx}
\usepackage{grffile}
\usepackage{tikz}

\hypersetup{
  bookmarksnumbered,
  pdfborder={0 0 0},
  pdfpagemode=UseNone,
  pdfstartview=FitH,
  breaklinks=true}
\urlstyle{same}  % don't use monospace font for urls

\usetikzlibrary{arrows.meta,calc,shapes.geometric,shapes.misc}

\setminted{
  autogobble,
  breakbytokenanywhere,
  breaklines,
  fontsize=\footnotesize,
}
\setmintedinline{
  autogobble,
  breakbytokenanywhere,
  breaklines,
  fontsize=\footnotesize,
}

\makeatletter
\def\maxwidth{\ifdim\Gin@nat@width>\linewidth\linewidth\else\Gin@nat@width\fi}
\def\maxheight{\ifdim\Gin@nat@height>\textheight\textheight\else\Gin@nat@height\fi}
\makeatother

% Scale images if necessary, so that they will not overflow the page
% margins by default, and it is still possible to overwrite the defaults
% using explicit options in \includegraphics[width, height, ...]{}
\setkeys{Gin}{width=\maxwidth,height=\maxheight,keepaspectratio}
\setlength{\emergencystretch}{3em}  % prevent overfull lines
\setcounter{secnumdepth}{3}

% Redefines (sub)paragraphs to behave more like sections
\ifx\paragraph\undefined\else
\let\oldparagraph\paragraph{}
\renewcommand{\paragraph}[1]{\oldparagraph{#1}\mbox{}}
\fi
\ifx\subparagraph\undefined\else
\let\oldsubparagraph\subparagraph{}
\renewcommand{\subparagraph}[1]{\oldsubparagraph{#1}\mbox{}}
\fi

% set default figure placement to htbp
\makeatletter
\def\fps@figure{htbp}
\makeatother


\title{State Threads for Internet Applications\\
\footnotesize{\url{http://state-threads.sourceforge.net/docs/st.html}}}
\author{}
\date{}

\begin{document}
\maketitle

\hypertarget{introduction}{%
\section*{Introduction}\label{introduction}}

State Threads is an application library which provides a foundation for
writing fast and highly scalable Internet Applications on UNIX-like
platforms. It combines the simplicity of the multithreaded programming
paradigm, in which one thread supports each simultaneous connection,
with the performance and scalability of an event-driven state machine
architecture.

\hypertarget{definitions}{%
\section{Definitions}\label{definitions}}

\protect\hypertarget{IA}{}{}

\hypertarget{internet-applications}{%
\subsection{Internet Applications}\label{internet-applications}}

An \emph{Internet Application} (IA) is either a server or client network
application that accepts connections from clients and may or may not
connect to servers. In an IA the arrival or departure of network data
often controls processing (that is, IA is a \emph{data-driven}
application). For each connection, an IA does some finite amount of work
involving data exchange with its peer, where its peer may be either a
client or a server. The typical transaction steps of an IA are to accept
a connection, read a request, do some finite and predictable amount of
work to process the request, then write a response to the peer that sent
the request. One example of an IA is a Web server; the most general
example of an IA is a proxy server, because it both accepts connections
from clients and connects to other servers.

We assume that the performance of an IA is constrained by available CPU
cycles rather than network bandwidth or disk I/O (that is, CPU is a
bottleneck resource).

\protect\hypertarget{PS}{}{}

\hypertarget{performance-and-scalability}{%
\subsection{Performance and
Scalability}\label{performance-and-scalability}}

The \emph{performance} of an IA is usually evaluated as its throughput
measured in transactions per second or bytes per second (one can be
converted to the other, given the average transaction size). There are
several benchmarks that can be used to measure throughput of Web serving
applications for specific workloads (such as
\href{http://www.spec.org/osg/web96/}{SPECweb96},
\href{http://www.mindcraft.com/webstone/}{WebStone},
\href{http://www.zdnet.com/zdbop/webbench/}{WebBench}). Although there
is no common definition for \emph{scalability}, in general it expresses
the ability of an application to sustain its performance when some
external condition changes. For IAs this external condition is either
the number of clients (also known as ``users,'' ``simultaneous
connections,'' or ``load generators'') or the underlying hardware system
size (number of CPUs, memory size, and so on). Thus there are two types
of scalability: \emph{load scalability} and \emph{system scalability},
respectively.

The figure below shows how the throughput of an idealized IA changes
with the increasing number of clients (solid blue line). Initially the
throughput grows linearly (the slope represents the maximal throughput
that one client can provide). Within this initial range, the IA is
underutilized and CPUs are partially idle. Further increase in the
number of clients leads to a system saturation, and the throughput
gradually stops growing as all CPUs become fully utilized. After that
point, the throughput stays flat because there are no more CPU cycles
available. In the real world, however, each simultaneous connection
consumes some computational and memory resources, even when idle, and
this overhead grows with the number of clients. Therefore, the
throughput of the real world IA starts dropping after some point (dashed
blue line in the figure below). The rate at which the throughput drops
depends, among other things, on application design.

We say that an application has a good \emph{load scalability} if it can
sustain its throughput over a wide range of loads. Interestingly, the
\href{http://www.spec.org/osg/web99/}{SPECweb99} benchmark somewhat
reflects the Web server's load scalability because it measures the
number of clients (load generators) given a mandatory minimal throughput
per client (that is, it measures the server's \emph{capacity}). This is
unlike \href{http://www.spec.org/osg/web96/}{SPECweb96} and other
benchmarks that use the throughput as their main metric (see the figure
below).

\begin{figure}
\includegraphics[width=0.8\columnwidth]{./6b9304999a1ae86e0e87db0d76523247d33b4220.jpg}
\end{figure}

\emph{System scalability} is the ability of an application to sustain
its performance per hardware unit (such as a CPU) with the increasing
number of these units. In other words, good system scalability means
that doubling the number of processors will roughly double the
application's throughput (dashed green line). We assume here that the
underlying operating system also scales well. Good system scalability
allows you to initially run an application on the smallest system
possible, while retaining the ability to move that application to a
larger system if necessary, without excessive effort or expense. That
is, an application need not be rewritten or even undergo a major porting
effort when changing system size.

Although scalability and performance are more important in the case of
server IAs, they should also be considered for some client applications
(such as benchmark load generators).

\protect\hypertarget{CONC}{}{}

\hypertarget{concurrency}{%
\subsection{Concurrency}\label{concurrency}}

Concurrency reflects the parallelism in a system. The two unrelated
types are \emph{virtual} concurrency and \emph{real} concurrency.

\begin{itemize}

\item
  Virtual (or apparent) concurrency is the number of simultaneous
  connections that a system supports.
\item
  Real concurrency is the number of hardware devices, including CPUs,
  network cards, and disks, that actually allow a system to perform
  tasks in parallel.
\end{itemize}

An IA must provide virtual concurrency in order to serve many users
simultaneously. To achieve maximum performance and scalability in doing
so, the number of programming entities than an IA creates to be
scheduled by the OS kernel should be kept close to (within an order of
magnitude of) the real concurrency found on the system. These
programming entities scheduled by the kernel are known as \emph{kernel
execution vehicles}. Examples of kernel execution vehicles include
Solaris lightweight processes and IRIX kernel threads. In other words,
the number of kernel execution vehicles should be dictated by the system
size and not by the number of simultaneous connections.

\hypertarget{existing-architectures}{%
\section{Existing Architectures}\label{existing-architectures}}

There are a few different architectures that are commonly used by IAs.
These include the \emph{Multi-Process}, \emph{Multi-Threaded}, and
\emph{Event-Driven State Machine} architectures.

\protect\hypertarget{MP}{}{}

\hypertarget{multi-process-architecture}{%
\subsection{Multi-Process
Architecture}\label{multi-process-architecture}}

In the Multi-Process (MP) architecture, an individual process is
dedicated to each simultaneous connection. A process performs all of a
transaction's initialization steps and services a connection completely
before moving on to service a new connection.

User sessions in IAs are relatively independent; therefore, no
synchronization between processes handling different connections is
necessary. Because each process has its own private address space, this
architecture is very robust. If a process serving one of the connections
crashes, the other sessions will not be affected. However, to serve many
concurrent connections, an equal number of processes must be employed.
Because processes are kernel entities (and are in fact the heaviest
ones), the number of kernel entities will be at least as large as the
number of concurrent sessions. On most systems, good performance will
not be achieved when more than a few hundred processes are created
because of the high context-switching overhead. In other words, MP
applications have poor load scalability.

On the other hand, MP applications have very good system scalability,
because no resources are shared among different processes and there is
no synchronization overhead.

The Apache Web Server 1.x (\protect\hyperlink{refs1}{{[}Reference 1{]}})
uses the MP architecture on UNIX systems.

\protect\hypertarget{MT}{}{}

\hypertarget{multi-threaded-architecture}{%
\subsection{Multi-Threaded
Architecture}\label{multi-threaded-architecture}}

In the Multi-Threaded (MT) architecture, multiple independent threads of
control are employed within a single shared address space. Like a
process in the MP architecture, each thread performs all of a
transaction's initialization steps and services a connection completely
before moving on to service a new connection.

Many modern UNIX operating systems implement a \emph{many-to-few} model
when mapping user-level threads to kernel entities. In this model, an
arbitrarily large number of user-level threads is multiplexed onto a
lesser number of kernel execution vehicles. Kernel execution vehicles
are also known as \emph{virtual processors}. Whenever a user-level
thread makes a blocking system call, the kernel execution vehicle it is
using will become blocked in the kernel. If there are no other
non-blocked kernel execution vehicles and there are other runnable
user-level threads, a new kernel execution vehicle will be created
automatically. This prevents the application from blocking when it can
continue to make useful forward progress.

Because IAs are by nature network I/O driven, all concurrent sessions
block on network I/O at various points. As a result, the number of
virtual processors created in the kernel grows close to the number of
user-level threads (or simultaneous connections). When this occurs, the
many-to-few model effectively degenerates to a \emph{one-to-one} model.
Again, like in the MP architecture, the number of kernel execution
vehicles is dictated by the number of simultaneous connections rather
than by number of CPUs. This reduces an application's load scalability.
However, because kernel threads (lightweight processes) use fewer
resources and are more light-weight than traditional UNIX processes, an
MT application should scale better with load than an MP application.

Unexpectedly, the small number of virtual processors sharing the same
address space in the MT architecture destroys an application's system
scalability because of contention among the threads on various locks.
Even if an application itself is carefully optimized to avoid lock
contention around its own global data (a non-trivial task), there are
still standard library functions and system calls that use common
resources hidden from the application. For example, on many platforms
thread safety of memory allocation routines (\texttt{malloc(3)},
\texttt{free(3)}, and so on) is achieved by using a single global lock.
Another example is a per-process file descriptor table. This common
resource table is shared by all kernel execution vehicles within the
same process and must be protected when one modifies it via certain
system calls (such as \texttt{open(2)}, \texttt{close(2)}, and so on).
In addition to that, maintaining the caches coherent among CPUs on
multiprocessor systems hurts performance when different threads running
on different CPUs modify data items on the same cache line.

In order to improve load scalability, some applications employ a
different type of MT architecture: they create one or more thread(s)
\emph{per task} rather than one thread \emph{per connection}. For
example, one small group of threads may be responsible for accepting
client connections, another for request processing, and yet another for
serving responses. The main advantage of this architecture is that it
eliminates the tight coupling between the number of threads and number
of simultaneous connections. However, in this architecture, different
task-specific thread groups must share common work queues that must be
protected by mutual exclusion locks (a typical producer-consumer
problem). This adds synchronization overhead that causes an application
to perform badly on multiprocessor systems. In other words, in this
architecture, the application's system scalability is sacrificed for the
sake of load scalability.

Of course, the usual nightmares of threaded programming, including data
corruption, deadlocks, and race conditions, also make MT architecture
(in any form) non-simplistic to use.

\protect\hypertarget{EDSM}{}{}

\hypertarget{event-driven-state-machine-architecture}{%
\subsection{Event-Driven State Machine
Architecture}\label{event-driven-state-machine-architecture}}

In the Event-Driven State Machine (EDSM) architecture, a single process
is employed to concurrently process multiple connections. The basics of
this architecture are described in Comer and Stevens
\protect\hyperlink{refs2}{{[}Reference 2{]}}. The EDSM architecture
performs one basic data-driven step associated with a particular
connection at a time, thus multiplexing many concurrent connections. The
process operates as a state machine that receives an event and then
reacts to it.

In the idle state the EDSM calls \texttt{select(2)} or \texttt{poll(2)}
to wait for network I/O events. When a particular file descriptor is
ready for I/O, the EDSM completes the corresponding basic step (usually
by invoking a handler function) and starts the next one. This
architecture uses non-blocking system calls to perform asynchronous
network I/O operations. For more details on non-blocking I/O see Stevens
\protect\hyperlink{refs3}{{[}Reference 3{]}}.

To take advantage of hardware parallelism (real concurrency), multiple
identical processes may be created. This is called Symmetric
Multi-Process EDSM and is used, for example, in the Zeus Web Server
(\protect\hyperlink{refs4}{{[}Reference 4{]}}). To more efficiently
multiplex disk I/O, special ``helper'' processes may be created. This is
called Asymmetric Multi-Process EDSM and was proposed for Web servers by
Druschel and others \protect\hyperlink{refs5}{{[}Reference 5{]}}.

EDSM is probably the most scalable architecture for IAs. Because the
number of simultaneous connections (virtual concurrency) is completely
decoupled from the number of kernel execution vehicles (processes), this
architecture has very good load scalability. It requires only minimal
user-level resources to create and maintain additional connection.

Like MP applications, Multi-Process EDSM has very good system
scalability because no resources are shared among different processes
and there is no synchronization overhead.

Unfortunately, the EDSM architecture is monolithic rather than based on
the concept of threads, so new applications generally need to be
implemented from the ground up. In effect, the EDSM architecture
simulates threads and their stacks the hard way.

\protect\hypertarget{ST}{}{}

\hypertarget{state-threads-library}{%
\section{State Threads Library}\label{state-threads-library}}

The State Threads library combines the advantages of all of the above
architectures. The interface preserves the programming simplicity of
thread abstraction, allowing each simultaneous connection to be treated
as a separate thread of execution within a single process. The
underlying implementation is close to the EDSM architecture as the state
of each particular concurrent session is saved in a separate memory
segment.

\hypertarget{state-changes-and-scheduling}{%
\subsection{State Changes and
Scheduling}\label{state-changes-and-scheduling}}

The state of each concurrent session includes its stack environment
(stack pointer, program counter, CPU registers) and its stack.
Conceptually, a thread context switch can be viewed as a process
changing its state. There are no kernel entities involved other than
processes. Unlike other general-purpose threading libraries, the State
Threads library is fully deterministic. The thread context switch
(process state change) can only happen in a well-known set of functions
(at I/O points or at explicit synchronization points). As a result,
process-specific global data does not have to be protected by mutual
exclusion locks in most cases. The entire application is free to use all
the static variables and non-reentrant library functions it wants,
greatly simplifying programming and debugging while increasing
performance. This is somewhat similar to a \emph{co-routine} model
(co-operatively multitasked threads), except that no explicit yield is
needed -\/- sooner or later, a thread performs a blocking I/O operation
and thus surrenders control. All threads of execution (simultaneous
connections) have the same priority, so scheduling is non-preemptive,
like in the EDSM architecture. Because IAs are data-driven (processing
is limited by the size of network buffers and data arrival rates),
scheduling is non-time-slicing.

Only two types of external events are handled by the library's
scheduler, because only these events can be detected by
\texttt{select(2)} or \texttt{poll(2)}: I/O events (a file descriptor is
ready for I/O) and time events (some timeout has expired). However,
other types of events (such as a signal sent to a process) can also be
handled by converting them to I/O events. For example, a signal handling
function can perform a write to a pipe (\texttt{write(2)} is
reentrant/asynchronous-safe), thus converting a signal event to an I/O
event.

To take advantage of hardware parallelism, as in the EDSM architecture,
multiple processes can be created in either a symmetric or asymmetric
manner. Process management is not in the library's scope but instead is
left up to the application.

There are several general-purpose threading libraries that implement a
\emph{many-to-one} model (many user-level threads to one kernel
execution vehicle), using the same basic techniques as the State Threads
library (non-blocking I/O, event-driven scheduler, and so on). For an
example, see GNU Portable Threads
(\protect\hyperlink{refs6}{{[}Reference 6{]}}). Because they are
general-purpose, these libraries have different objectives than the
State Threads library. The State Threads library is \emph{not} a
general-purpose threading library, but rather an application library
that targets only certain types of applications (IAs) in order to
achieve the highest possible performance and scalability for those
applications.

\hypertarget{scalability}{%
\subsection{Scalability}\label{scalability}}

State threads are very lightweight user-level entities, and therefore
creating and maintaining user connections requires minimal resources. An
application using the State Threads library scales very well with the
increasing number of connections.

On multiprocessor systems an application should create multiple
processes to take advantage of hardware parallelism. Using multiple
separate processes is the \emph{only} way to achieve the highest
possible system scalability. This is because duplicating per-process
resources is the only way to avoid significant synchronization overhead
on multiprocessor systems. Creating separate UNIX processes naturally
offers resource duplication. Again, as in the EDSM architecture, there
is no connection between the number of simultaneous connections (which
may be very large and changes within a wide range) and the number of
kernel entities (which is usually small and constant). In other words,
the State Threads library makes it possible to multiplex a large number
of simultaneous connections onto a much smaller number of separate
processes, thus allowing an application to scale well with both the load
and system size.

\hypertarget{performance}{%
\subsection{Performance}\label{performance}}

Performance is one of the library's main objectives. The State Threads
library is implemented to minimize the number of system calls and to
make thread creation and context switching as fast as possible. For
example, per-thread signal mask does not exist (unlike POSIX threads),
so there is no need to save and restore a process's signal mask on every
thread context switch. This eliminates two system calls per context
switch. Signal events can be handled much more efficiently by converting
them to I/O events (see above).

\hypertarget{portability}{%
\subsection{Portability}\label{portability}}

The library uses the same general, underlying concepts as the EDSM
architecture, including non-blocking I/O, file descriptors, and I/O
multiplexing. These concepts are available in some form on most UNIX
platforms, making the library very portable across many flavors of UNIX.
There are only a few platform-dependent sections in the source.

\hypertarget{state-threads-and-nspr}{%
\subsection{State Threads and NSPR}\label{state-threads-and-nspr}}

The State Threads library is a derivative of the Netscape Portable
Runtime library (NSPR) \protect\hyperlink{refs7}{{[}Reference 7{]}}. The
primary goal of NSPR is to provide a platform-independent layer for
system facilities, where system facilities include threads, thread
synchronization, and I/O. Performance and scalability are not the main
concern of NSPR. The State Threads library addresses performance and
scalability while remaining much smaller than NSPR. It is contained in 8
source files as opposed to more than 400, but provides all the
functionality that is needed to write efficient IAs on UNIX-like
platforms.

\begin{longtable}[]{@{}lll@{}}
\toprule
\endhead
\begin{minipage}[t]{0.30\columnwidth}\raggedright
\strut
\end{minipage} & \begin{minipage}[t]{0.30\columnwidth}\raggedright
NSPR\strut
\end{minipage} & \begin{minipage}[t]{0.30\columnwidth}\raggedright
State Threads\strut
\end{minipage}\tabularnewline
\begin{minipage}[t]{0.30\columnwidth}\raggedright
\textbf{Lines of code}\strut
\end{minipage} & \begin{minipage}[t]{0.30\columnwidth}\raggedright
\textasciitilde150,000\strut
\end{minipage} & \begin{minipage}[t]{0.30\columnwidth}\raggedright
\textasciitilde3000\strut
\end{minipage}\tabularnewline
\begin{minipage}[t]{0.30\columnwidth}\raggedright
\textbf{Dynamic library size~~\\
(debug version)}\strut
\end{minipage} & \begin{minipage}[t]{0.30\columnwidth}\raggedright
\strut
\end{minipage} & \begin{minipage}[t]{0.30\columnwidth}\raggedright
\strut
\end{minipage}\tabularnewline
\begin{minipage}[t]{0.30\columnwidth}\raggedright
IRIX\strut
\end{minipage} & \begin{minipage}[t]{0.30\columnwidth}\raggedright
\textasciitilde700 KB\strut
\end{minipage} & \begin{minipage}[t]{0.30\columnwidth}\raggedright
\textasciitilde60 KB\strut
\end{minipage}\tabularnewline
\begin{minipage}[t]{0.30\columnwidth}\raggedright
Linux\strut
\end{minipage} & \begin{minipage}[t]{0.30\columnwidth}\raggedright
\textasciitilde900 KB\strut
\end{minipage} & \begin{minipage}[t]{0.30\columnwidth}\raggedright
\textasciitilde70 KB\strut
\end{minipage}\tabularnewline
\bottomrule
\end{longtable}

\hypertarget{conclusion}{%
\section{Conclusion}\label{conclusion}}

State Threads is an application library which provides a foundation for
writing \protect\hyperlink{IA}{Internet Applications}. To summarize, it
has the following \emph{advantages}:

\begin{itemize}

\item
  It allows the design of fast and highly scalable applications. An
  application will scale well with both load and number of CPUs.
\item
  It greatly simplifies application programming and debugging because,
  as a rule, no mutual exclusion locking is necessary and the entire
  application is free to use static variables and non-reentrant library
  functions.
\end{itemize}

The library's main \emph{limitation}:

\begin{itemize}

\item
  All I/O operations on sockets must use the State Thread library's I/O
  functions because only those functions perform thread scheduling and
  prevent the application's processes from blocking.
\end{itemize}

\hypertarget{references}{%
\section{References}\label{references}}

\begin{enumerate}

\item
  Apache Software Foundation, \url{http://www.apache.org}.
  \protect\hypertarget{refs2}{}{}
\item
  Douglas E. Comer, David L. Stevens, \emph{Internetworking With TCP/IP,
  Vol. III: Client-Server Programming And Applications}, Second Edition,
  Ch. 8, 12. \protect\hypertarget{refs3}{}{}
\item
  W. Richard Stevens, \emph{UNIX Network Programming}, Second Edition,
  Vol. 1, Ch. 15. \protect\hypertarget{refs4}{}{}
\item
  Zeus Technology Limited,
  \href{http://www.zeus.co.uk/}{http://www.zeus.co.uk}.
  \protect\hypertarget{refs5}{}{}
\item
  Peter Druschel, Vivek S. Pai, Willy Zwaenepoel,
  \href{http://www.cs.rice.edu/~druschel/usenix99flash.ps.gz}{Flash: An
  Efficient and Portable Web Server}. In \emph{Proceedings of the USENIX
  1999 Annual Technical Conference}, Monterey, CA, June 1999.
  \protect\hypertarget{refs6}{}{}
\item
  GNU Portable Threads, \url{http://www.gnu.org/software/pth/}.
  \protect\hypertarget{refs7}{}{}
\item
  Netscape Portable Runtime,
  \url{http://www.mozilla.org/docs/refList/refNSPR/}.
\end{enumerate}

\hypertarget{other-resources-covering-various-architectural-issues-in-ias}{%
\section{Other resources covering various architectural issues in
IAs}\label{other-resources-covering-various-architectural-issues-in-ias}}

\begin{enumerate}
\setcounter{enumi}{7}

\item
  Dan Kegel, \emph{The C10K problem},
  \url{http://www.kegel.com/c10k.html}.
\item
  James C. Hu, Douglas C. Schmidt, Irfan Pyarali, \emph{JAWS:
  Understanding High Performance Web Systems},
  \url{http://www.cs.wustl.edu/~jxh/research/research.html}.
\end{enumerate}

\end{document}
