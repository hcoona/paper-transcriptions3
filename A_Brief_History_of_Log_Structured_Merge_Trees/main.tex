\PassOptionsToPackage{dvipsnames}{xcolor}
\PassOptionsToPackage{unicode=true,colorlinks=true,urlcolor=blue}{hyperref} % options for packages loaded elsewhere
\PassOptionsToPackage{hyphens}{url}
\documentclass[a4paper,12pt,notitlepage,twoside,openright]{article}

\usepackage{ifxetex}
\ifxetex{}
\else
\errmessage{Must be built with XeLaTeX}
\fi

\usepackage{amsmath}
\usepackage{fontspec}
\usepackage{fourier-otf} % erewhon-math
\setmonofont{iosevka-type-slab-regular}[
  Path=../common/iosevka-type-slab/,
  Extension=.ttf,
  BoldFont=iosevka-type-slab-bold,
  ItalicFont=iosevka-type-slab-italic,
  BoldItalicFont=iosevka-type-slab-bolditalic,
  Scale=MatchLowercase,
]

% Math
\usepackage[binary-units]{siunitx}

\usepackage{caption}
\usepackage{authblk}
\usepackage{enumitem}
\usepackage{footnote}
\usepackage{bookmark}

% Table
\usepackage{tabu}
\usepackage{longtable}
\usepackage{booktabs}
\usepackage{multirow}

% Verbatim & Source code
\usepackage{fancyvrb}
\usepackage{minted}

% Beauty
\usepackage[protrusion]{microtype}
\usepackage[defaultlines=3]{nowidow}
\usepackage{upquote}
\usepackage{parskip}
\usepackage[strict]{changepage}

\usepackage{hyperref}

% Graph
\usepackage{graphicx}
\usepackage{grffile}
\usepackage{tikz}

\usepackage{endnotes}
\setlist{noitemsep}

\hypersetup{
	bookmarksnumbered,
	colorlinks, % hidelinks,
	pdfpagemode=UseNone,
	pdfstartview=FitH
}

\usetikzlibrary{calc,shapes.multipart,chains,arrows,matrix}
\tikzset{>=latex'}
\tikzstyle{list}=[
	 rectangle split,
	 rectangle split parts=2, draw,
	 rectangle split horizontal,
	 rectangle split part align=base,
	 text width=3ex, text centered
    ]
    
\newcommand{\pFF}[2]{\draw[->] let \p1 = (#1.one), \p2 = (#1.center), \p3 = (#2.one), \p4 = (#2.north) in (\x1,\y2) -- (\x3,\y4)}
\newcommand{\pSF}[2]{\draw[->] let \p1 = (#1.two), \p2 = (#1.center), \p3 = (#2.west), \p4 = (#2.center) in (\x1,\y2) -- (\x3,\y4)}

\newcolumntype{L}{>{\RaggedRight}X}

\newcommand{\SpecialChapter}[1]{%
  \chapter*{#1}%
  \addcontentsline{toc}{chapter}{#1}%
  \markboth{\MakeUppercase{#1}}{\MakeUppercase{#1}}
}

\title{A Brief History of Log Structured Merge Trees \\{\footnotesize \url{https://ristret.com/s/gnd4yr/brief_history_log_structured_merge_trees}}}
\author{Arjun Narayan}
\date{February, 21, 2018}

\begin{document}
\maketitle

LSMs have been around for the past two decades, popularized by their
usage inside Google infrastructure. As Google infrastructure has been
cloned outside Google, the usage of LSMs to store sorted data has as
well. I've been searching for a history of
\href{https://en.wikipedia.org/wiki/Log-structured_merge-tree}{log-structured
merge trees}, for a while. Not finding one, this is my best shot at
writing a start. While I asked people at my workplace for their personal
recollections from their time at Google, all errors are solely my own. I
welcome any and all input you might have, and will make an effort to
keep this updated.

In search of more information on why Google developed LSMs, I posed this
question to \href{https://en.wikipedia.org/wiki/Ben_Darnell}{Ben
Darnell}, who proved to be a wealth of information. One of the first
things he said that the SSTable file format \emph{preceded} the LSM,
which surprised me. SSTables always seemed to me like they were custom
designed to be used as the basic building block of an LSM\endnote{
The best concise explanation of the SSTable file format is found
in Martin Kleppmann's \href{https://dataintensive.net/}{Designing Data
Intensive Applications}.}.
Apparently, MapReduce (\textasciitilde2002) used a \emph{flat}
collection of SSTables. This makes sense to me: if you use MapReduce to
build secondary indexes, then there's no size expectation for the output
of any given reducer, as the distribution of secondary index keys is not
predictable from your distribution of input keys (after all - why are
you building a secondary index in the first place?). SSTables were kept
sorted to facilitate incremental construction, that could then be merged
on the way out: if you don't know how many KVs you're going to receive
as a reducer, you can construct an SSTable up to the size that you can
sort in-memory, and then output it to disk, and start on a new one, in a
memoryless fashion. With \(k\) SSTables, you can then stream your
output with \(k\) iterators. Neat\endnote{As an aside, MapReduce was designed for \emph{index building},
not for computing PageRank. It seems every year there's an SOSP or OSDI
paper that builds some distributed graph processing framework (it's
better now, but good God was 2012 a bleak year) and then beats the dead
horse of comparing PageRank computation versus MapReduce and claiming
10x speedups. But Google never used MapReduce to compute PageRank, but
rather to perform gigantic inverted index construction. Somehow ``Google
= computing PageRank'' got conflated with ``Google runs on MapReduce'',
providing a conveniently sandbagged comparison point for everyone's and
their mother's graph processing framework.}.

The idea to put these SSTables in a log-structured fashion to recover
\(\log(n)\) time search while maintaining \(\log(n)\) iterators
came later, with BigTable (\textasciitilde{}2004).
\href{https://static.googleusercontent.com/media/research.google.com/en//archive/bigtable-osdi06.pdf}{The
BigTable paper} describes this, although it doesn't pay much attention
to the big-O analysis (the section on Compactions does not, for example,
explain what the compaction strategy is).

While we think of LevelDB as the main LSM implementation, LevelDB was
designed after BigTable as an embedded ``BigTable-lite'' implementation,
for storing Key-Value metadata in Google Chrome. This was eventually
open-sourced as ``the'' key-value store, so it's retroactively gained
the status of the defacto LSM implementation.

Throughout all of this history, it wasn't clear to me \emph{why} you
would want to do all of this! By \textasciitilde{}2004, we already had
several mature BTree based database engines, and it seems like Google
reinvented a fairly significant wheel. The reason seems to be that
Google's infrastructure (namely
\href{http://static.googleusercontent.com/media/research.google.com/en//archive/gfs-sosp2003.pdf}{Google
File System}) was designed for appending files, and had atrocious
non-append mutation performance. BTree writes are heavily mutating, and
thus a non-starter if you're going to write your BTree pages to GFS.

So it seems that the engineering efforts to productionize LSMs came out
of an artifact of GFS's design focus on append-only writes at expense of
random writes. I've heard some drunken chatter at SOSPs past that
``LevelDB was optimized for the IO profile of spinning disks'', but I
believe that it's \emph{GFS's} IO profile that's the reason (which is
similar to spinning disks of having much better contiguous read and
write performance to random reads and writes). And from
\href{http://static.googleusercontent.com/media/research.google.com/en//archive/gfs-sosp2003.pdf}{the
Google File System paper}, they say this is because the dominant
paradigm is that their files are \emph{very large}, and for very large
files, appending is the only realistic way to ever modify them.

Facebook forked LevelDB as RocksDB, and invested some heavy resources
into productionizing it to perform better on SSDs. I'm not the most
familiar with RocksDB, but I believe this involved a lot of work on
making the engine lock-free to allow continuous multi-threaded
compactions\endnote{I have no idea \emph{why} continuous multi-thread compactions
results in better SSD performance, and eagerly await your email
explaining this.}. Mark Callaghan from Facebook has a performance
\href{http://smalldatum.blogspot.com/2016/01/summary-of-advantages-of-lsm-vs-b-tree.html}{comparison
of LSMs versus BTrees}, who's conclusion is that RocksDB (the chosen
LSM) has better compression and less write-amplification than InnoDB
(the chosen BTree), at the trade-off of having worse space
amplification, and performing worse on scans (as opposed to point
lookups) due to not being able to use SSTable Bloom Filters. While this
comparison is careful to note the advantages and disadvantages of
\emph{individual} LSM and BTree implementations, I do quibble with the
title. These performance profiles aren't inherent to LSMs or BTrees: as
\href{http://www.cs.utexas.edu/\%7Evijay/papers/sosp17-pebblesdb.pdf}{Vijay
Chidambaram's research group shows}, LSM write/space amplification
performance is all about the specific compaction strategy. PebblesDB
(their LevelDB fork), achieves much better write amplification at the
cost of worse space amplification. This paper is contrasted by
\href{http://cidrdb.org/cidr2017/papers/p82-dong-cidr17.pdf}{some work
from Facebook's RocksDB team} where they explicitly say they tune
RocksDB to keep space amplification down. So perhaps it's merely
quibbling about compaction strategy.

Anyway, we've veered off trail here, but that's how these projects
evolve too! The end result after all of this is that RocksDB has turned
into a rock solid, battle tested key-value storage engine that has been
used in some form or the other at two of the largest internet companies
in the world. So, strangely, if you had to pick a storage engine to use
for your single-node key-value needs, it turns out to be the best
option! There aren't many battle tested BTree based storage engines out
there that aren't tied quite closely to specific database
implementations.

I've come to the conclusion that LSMs and BTrees are essentially
equivalent at this point. For whatever tradeoffs there may be for
particular implementations (e.g. InnoDB vs RocksDB) there's a different
compaction strategy that gets to the other's performance. It just so
happens that the peculiarities of GFS (perhaps it all came down to
spinning disks back around \textasciitilde{}2005?) resulted in Google
investing heavily in LSM based storage infrastructure. Google
open-sourced their work, and that allowed the rest of the world to take
advantage of LSMs, and there are no comparable open-source BTree
libraries with as much battle-testing, so new projects build off
RocksDB, even if a BTree would be just as good.

Edit (2/25/2018): Ben clarifies that there may have been a time when
MapReduce was used to compute PageRank --- mainly because it already
existed, and Pregel wasn't built yet. But again, this was an instance of
using an existing hammer, not any claim that it was the optimal hammer
for the task. So the point stands that everyone knows its a poor tool
for the job of running graph computations and systems builders should
stop comparing their graph computation systems against it.

\enlargethispage{3\baselineskip}
\theendnotes{}

\end{document}
